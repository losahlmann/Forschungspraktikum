\chapter{Examples}\label{sec:example-hp}

To motivate the need of being able to treat Delayed Differential Equations in hybrid programs, we present some examples.

\section{Car Following Model}

    ref

    We want to model a controller for a car
    trying to keep a constant distance between
    considering a reaction time $\tau$, causing a delay in control decision
    controller is ideal continuous, not discrete

    Given the speed pattern of a leading car, the systems models the position and velocity of a following car.

    \begin{equation*}
        \begin{cases}
            \D[2]{x_{n+1}}(t) = \alpha (\D{x_n}(t-\tau)-\D{x_{n+1}}(t-\tau))\\
            \D{x_n}(t) = v(t)
        \end{cases}
    \end{equation*}

    The coefficient $\alpha$ can be seen as a sport or fun factor, describing the strengh of acceleration and deacceleration applied to the following car.

    By introducing $v_{n+1}$ for $x_{n+1}$, we reduce the system to first order.

    Preconditions (for all $t\in\closeddelayinterval{-\tau}$):
    \begin{align*}
        d(t) &= x_n(t)-x_{n+1}(t) \in\compactum{D-m}{D+m}\\
        \D{x_{n+1}}(t) &\in\compactum{V-l}{V+l}\\
        \D{x_n}(t) &= v(t) = 
    \end{align*}
    Both $\alpha$ and $\tau$ are considered to be constant.

    Model
    \begin{equation*}
        \dbox{\hevolve{
            \D{v_{n+1}} = \alpha(\D{x_n}[-\tau]-\D{x_{n+1}}[-\tau])\syssep
            \D{x_{n+1}} = v_{n+1}\syssep
            \D{x_n} = v
        }}{d\geq\delta}
    \end{equation*}

    too strong reactions of the following car lead to damped oscillations

    Of practical interest would be to prove the safety condition that the cars always keep a certain minimal distance, i.e.\ $d\geq\delta$.

\section{Network Induced Delay in Control Loops}

    only an input or measurement delay (in u)