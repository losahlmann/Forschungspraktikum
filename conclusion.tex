\chapter{Conclusion}
    \label{ch:conclusion}

    we have presented ...

    \section{Related Work}
        Verification theory for hybrid systems governed by ordinary differential equations is well established, as can be seen by the exhaustive list of literature to \dL given in the introduction.
        Literature on formal methods with applied to DDEs is rare.
        
        Huang et al.~\cite{Huang16BoundedVerificationNNDS} describe an algorithm allowing bounded invariant verification for dynamical systems with delayed interconnections.
        This is done by numerically computing bounds on the trajectories of a (non-linear) delay differential equation given a set of possible initial states (sensitivity of the solution on initial data).
        
        Zou et al.~\cite{Zou15AutomaticVerifDDEs} have presented an iteration method, combining numerical integration of DDEs with multiple constant delays based on interval Taylor forms
        with real arithmetic constraint solving.
        It automatically analyzes the integration-operator, which yields the Taylor coefficients for the next temporal segment, given by the method of step.
        This procedure allows to compute safety and stability certificates in terms of set constraints.

        To the knowledge of the author, no work combining logics and delay differential equations has been published yet.

    \section{Future Work}
    	hard part, where the magic happens
        find differential invariant
        automatically compute invariants and differential invariants
        as possible for \dL

        prove redution to \dL given by axiom of steps
        completeness

        apply to examples

    	uniform substitution calculus, proof rules, substitutes formula for a predicate symbol
    	US, classical rule in for first-order logic proof calculus
    	have axioms instead of axiom schemata
    	leads to a calculus with finite number od \ddL formulas as axioms
    	to facilitate a sound implementation in a prover

        

    	state dependent delay
    	% TODO: source for state-dependent and distributed delays
        % The definition of a DDE can be extended to state-dependent or distributed delays

        potential implementation in Keymaera X: use modularity, existing axioms, lemmata
        axiomatization, proof rules important for automatization of proofs
        for that as \dL in \cite{Platzer15Uniform} no axiom schemata, but finite number of axioms and proof rules (sets of formulas)
        proof rule for substitution on axiom preserving soundness

        sequent proof calculus

        