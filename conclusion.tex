\chapter{Conclusion}
    \label{ch:conclusion}

    We have presented a combination of first-order and modal logic for the specification and verification of safety and liveness properties of time-delay hybrid systems.
    This logic is essentially an extension of differential dynamic logic (\dL).
    The approach taken in this work has been available for ODEs for several years, but has not yet been applied to DDEs. 
    To the knowledge of the author, no work combining logics and delay differential equations has been published so far.

    \section{Related Work}
        Verification theory for hybrid systems governed by ordinary differential equations is well established, as can be seen by the exhaustive list of literature on \dL given in the introduction.
        Literature on formal methods applied to DDEs however, is rare.
        
        \citeauthor*{Huang16BoundedVerificationNNDS}~\cite{Huang16BoundedVerificationNNDS} describe an algorithm allowing bounded invariant verification for dynamical systems with delayed interconnections.
        This is done by numerically computing bounds on the trajectories of a (non-linear) delay differential equation, given a set of possible initial states (sensitivity of the solution on initial data).
        
        \citeauthor{Zou15AutomaticVerifDDEs}~\cite{Zou15AutomaticVerifDDEs} have presented an iteration method, combining numerical integration of DDEs with multiple constant delays based on interval Taylor forms
        with real arithmetic constraint solving.
        It automatically analyzes the integration-operator, which yields the Taylor coefficients for the next temporal segment, given by the method of step.
        This procedure allows to compute safety and stability certificates in terms of set constraints.

    \section{Future Work}
        There are several avenues for future work. Practical applicability could be improved by combining axioms to rules for a sequent proof calculus, as it is available for \dL.

        Next, we would like to include \ddL into the interactive theorem prover \KeYmaeraX.
        This would require an extension of the presented theory into the \emph{uniform substitution calculus} of~\cite{Platzer15Uniform}.
        In contrast to the axiomatization presented in Chapter~\ref{sec:axiomatization}, this calculus uses a finite number of axioms (instead of axiom schemata) and an additional proof rule to substitute a formula by a predicate symbol, preserving the soundness of the axiom. This approach has the advantage that it can be much easier implemented in a software tool.
        The first step towards this calculus is already done with the definition of \emph{free, bound} and \emph{mustbound variables} in Section~\ref{sec:static-semantics}.

        The actual implementation of \ddL in \KeYmaeraX would benefit from its modularity and abstraction, since most existing axioms and lemmata could be taken from \dL.

        The axiom of steps (\irref{stepsb}) translates a delay differential equation into a ODE inside a loop. This indicates, that \ddL is actually \emph{reducible} to classical \dL. We leave the details, a formal proof and its significance for \emph{completeness} as future work.

        Differential invariants provide a powerful tool for the verification of differential equations, if one has such an invariant formula for the given equation to hand.
        However, finding such is in general not trivial. In the case of ordinary differential equations, different methods to automatically compute differential invariants have been presented~\cite{Platzer12LogicsDynSys}.
        Their adaption to DDEs would mean a tremendous progress towards practical application and would help to demonstrate the utility of \ddL by proving more complex real-world examples.
                
        Moreover, the class of feasible delay differential equations can be extended to include e.g.\ state dependent delays, functional right hand sides or uncertain delays.
        This would allow even more realistic and powerful models of problems, but increases the complexity of the possible dynamics to a large extent.
        It is not sure how far a logical approach can still capture these effects.
    	% TODO: source for state-dependent and distributed delays
        % The definition of a DDE can be extended to state-dependent or distributed delays

        

