\chapter{Delay Differential Equations}

special case of more general functional differential equations (FDEs)

deterministic system
future behaviour depends on present state only

ODE: in $\R^n$
if influence of past by a deviated time argument, functional state on past time-interval
FDS



backward continuation (Richard) -> [\&] limit up to T, state contains entire evolution, can check evo domain constraint on this state

They often arise in automatic control %\cite
, where a controller montitors the state of a system in order to make control decisions to adjust this state.
If there is a delay between the observation and the control action, the differential equation describing the system not only depends on its current state, but also on its past.

need to specify in initial condition
at least for the time of longest delay

neutral type: rhs also depends on deriv of init cond


% proofs similar to

Examples of ... which have been modeled using delay differential equations include epidemics, traffic flow and vibrations/chattering. See \cite{Falbo06FDEs} and there references therein.

Some methods to solve basic DDEs analytically are presented in \cite{Falbo06FDEs}.
numerical methods are not as 
see \cite{Bellen13NumericalDDEs}

\section{Piecewise Continuous Functions}
    \label{sec:piecewise-continuous-functions}
    
    The following definition is motivated by the character of evolution arising from hybrid systems. We define the main functional space of operation for the following chapters.

    % \begin{definition}[Piecewise Continuous]\label{def:piecewise-continuous}
    %     Let $D=[a,b]\subseteq\R$ be a closed interval (this includes the cases when $a=-\infty$ or $b=\infty$, or both). The mapping $x:D\rightarrow\R^n$ is called \emph{piecewise continuous} if and only if there is a finite partition $\{t_i:i=\range{0}{m}\}$ of $D$ (i.e.\ $a=t_0<t_1<\ldots<t_m=b$) such that $x$ is continuous on each interval piece $[t_i,t_{i+1})$ for all $i=\range{0}{m-1}$ and the left sided limits
    %     \begin{equation}
    %         \lim_{\substack{t\upto t_{i+1}\\ t\in[t_i,t_{i+1})}} x(t)
    %     \end{equation}
    %     exist. Hence $x(b)$ can be an isolated point and this right interval limit $b$ is the only spot where such is allowed.

    %     We denote by $\Cnpw[0]{D}{\R^n}$ the set of \emph{piecewise continuous functions} on the compact interval $D$ (this excludes the cases with $\pm\infty$), mapping to $\R^n$.
    % \end{definition}

    \begin{definition}[Piecewise Continuously Differentiable]
    \label{def:pw-cont-diff}
        Let $D=\compactum{a}{b}\subseteq\R$ be a closed interval (this includes the cases when $a=-\infty$ or $b=\infty$, or both). The mapping $x\from D\to\R^n$ is called $n$-times \emph{piecewise continuously differentiable} if and only if there is a finite partition (ordered set) $\partition[t_1]{a=t_0}{t_m=b}$ of $D$ (i.e.\ $a=t_0<t_1<\ldots<t_m=b$) such that $x$ is $n$-times continuously differentiable on each interval $(t_i,t_{i+1})$ with \foreignlanguage{frenchb}{\emph{càdlàg} (\og continue à droite, limite à gauche\fg{})} derivative.

        This means that everywhere on $D$, the function $x$ and each of its derivatives $\D[k]{x}$ are right continuous and have left limits.

        More precisely, for all $i=\range{0}{m-1}$ and for all $k=\range{0}{n}$ exist the left limits
        \begin{equation*}
            \lim_{\substack{t\upto t_{i+1}\\ t\in(t_i,t_{i+1})}} \D[k]{x}(t)
        \end{equation*}
        as well as the right limits
        \begin{equation*}
            \lim_{\substack{t\downto t_{i}\\ t\in(t_i,t_{i+1})}} \D[k]{x}(t) = \D[k]{x}(t_i)
        \end{equation*}
        which additionally coincide with the value of $\D[k]{x}$ at this knot $t_i$.
        Hence $x$ can have an isolated point only in the right interval limit $b$.

        In the case $n=0$, we say $x$ is \emph{piecewise continuous}.
        For a compact interval $D\subset\R$ (this excludes the cases with $\pm\infty$), we denote by $\Cnpw[n]{D}{\R^n}$ the set of \emph{$n$-times piecewise continuously differentiable functions} on $D$ mapping to $\R^n$, and respectively, by $\Cnpw[0]{D}{\R^n}$ the set of \emph{piecewise continuous functions} on $D$.

        The supremum norm $\supnorm{\cdot}$ of the Banach space of continuous functions on the compactum $D$ can be extended to $\Cnpw[n]{D}{\R^n}$, since each element consists of a finite number of continuous parts.
    \end{definition}

    \begin{figure}[t]\centering
        \begin{subfigure}[t]{0.48\textwidth}
            \centering
            \input{figures/plot-pw-allowed.tikz}
            \caption{Admissible piecewise continuous function.}
            \label{fig:allowed}
        \end{subfigure}
        \hfill
        \begin{subfigure}[t]{0.48\textwidth}
            \centering
            \begin{tikzpicture}[line width=0.5pt, scale=(\textwidth-20pt)/2cm, >=Latex]

	% grid
	\draw[help lines, step=0.25, color=gray!30, dashed] (0,-0.5) grid (2,1);
	
	% axis
	\draw[->, thick] (0,0) -- (2,0); % x
	\draw[->, thick] (0,-0.5) -- (0,1); % y

	% ticks
	\draw (0,0) node[below right] {$t_0$};
	\draw[thick] (1,-0.03) -- (1,0.03);
	\draw (1,0) node[below] {$t_1$};

	
	\draw[thick] plot[samples=50, smooth, domain=0:1] (\x, {((\x-2)^2-5*\x^3)/8});
	
	\draw[thick] plot[samples=50, smooth, domain=1:2] (\x, {(5*(\x-2)^2+\x^3)/8});

	\filldraw[leftpoint] (0,0.5) circle (0.025);
	\filldraw[rightpoint] (1,-0.5) circle (0.025);
	
	\filldraw[leftpoint] (1,0.4) circle (0.025);
	
	\filldraw[rightpoint] (1,0.75) circle (0.025);
	

\end{tikzpicture}

            \caption{Not allowed!}
            \label{fig:not-allowed}
        \end{subfigure}
        \caption{Examples to Definition~\ref{def:pw-cont-diff}.}
    \end{figure}

    In the following, when we talk about \emph{piecewise continuous} and \emph{piecewise continuously differentiable}, we refer to it in the sense of Definition~\ref{def:pw-cont-diff}.
    Let us note some basic observations which will be used subsequently.

    % FIXME: where doe I need chain lemma?
    \begin{lemma}\label{lm:comp-pw-cont}
        The composition of a continuous (outer) and a piecewise continuous function (inner) is again piecewise continuous with the same partition.
    \end{lemma}
    \begin{proof}
        The limits exist, because they commute with the continuous function and exist for the piecewise-continuous function.
    \end{proof}

    \begin{lemma}\label{lm:pc-integrable}
        A piecewise continuous function is (Riemann) integrable.
    \end{lemma}
    \begin{proof}
        This proof is usually given in every standard analysis book, see for example Theorem~6.10 in~\cite{Rudin76PrinciplesAnalysis} or Example~11.16b in~\cite{Gathmann12GDM}.
    \end{proof}

    The following lemma generalizes the fundamental theorem of calculus to piecewise continuous derivatives.

    \begin{lemma}\label{lm:pc-hauptsatz}
        Let $F\in\Cn[0]{\compactum{a}{b}}{} \cap \Cnpw[1]{\compactum{a}{b}}{}$
        % with the partition $\partition{a=t_0}{t_m=b}$
        with piecewise derivative $f$. Then
        \begin{equation*}
            F(t)-F(a) = \integral{a}{t} f(s)\dx[s]
            %\sum_{i=0}^k\int\limits_{t_i}^{t_{i+1}}f(t)\dx[t] + \int\limits_{t_k}^s f(t)\dx[t]
        \end{equation*}
        for all $t\in\compactum{a}{b}$.
        %where $t_k\leq s < t_{k+1}$.
        % FIXME: what about a=-inf or b=inf?: By def of C_pw is interval bounded
    \end{lemma}
    \begin{proof}
        On each compact interval $\compactum{t_{i-1}}{t_i}$ of the partition, $f$ is piecewise continuous and hence integrable (Lemma~\ref{lm:pc-integrable}).

        By precondition is $F$ differentiable on $\compactum{t_{i-1}}{\zeta}$ with $\D{F}=f$ for all $\zeta\in\open{t_{i-1}}{t_i}$.
        For that reason, the fundamental theorem of calculus (cf.\ standard analysis literature, e.g.~\cite{Gathmann12GDM,Rudin76PrinciplesAnalysis}) yields
        \begin{equation*}
            \denseintegral{t_{i-1}}{\zeta} f(s)\dx[s] = F(\zeta)-F(t_{i-1})
        \end{equation*}
        and by the continuity of $F$ that
        \begin{equation*}
            \denseintegral{t_{i-1}}{t_i} f(s)\dx[s]
            = \lim_{\zeta\to t_i}\denseintegral{t_{i-1}}{\zeta} f(s)\dx[s]
            = \lim_{\zeta\to t_i} F(\zeta)-F(t_{i-1})
            = F(t_i)-F(t_{i-1})
        \end{equation*}
        For any $t\in\compactum{a}{b}$, there is a $k\in\set{\range{1}{m}}$ such that $t\in\closedopen{t_{k-1}}{t_k}$ (in the case $t=b$, set $k=m$), summation over $i=\range{1}{k}$ yields the telescoping series
        \begin{equation*}
            F(t)-F(a) = \sum_{i=1}^{k} \denseintegral{t_{i-1}}{t_i} f(s)\dx[s] + \integral{t_j}{t} f(s)\dx[s]
        \end{equation*}
        what is by the additivity of the integral equivalent to
        \begin{equation*}
            F(t)-F(a) = \integral{a}{t} f(s)\dx[s].
        \end{equation*}
    \end{proof}


\section{Definition of DDEs}
    \label{sec:definition-dde}

    % FIXME: text DDE
    There are different possibilties to define delay differetial equations, dependent on what 
    We restrict to a class adapted to our needs and often found in literature,  see for example \cite{Roussel04DDEs}

    \begin{definition}[Delay Differential Equation]\label{def:dde}
        Given a function $f\from\deff\to\R^n$ and a set of time delays $\set{\tau_j\with 0<\tau_1<\ldots<\tau_k}$, a functional equation of the form
        \begin{equation}\label{eq:dde}
            \D{x}(t) = f(t,x(t),x(t-\tau_1),\ldots,x(t-\tau_k))
        \end{equation}
        is called (first order) \emph{delay differential equation} (DDE) with \emph{multiple constant, discrete delays} $\tau_j$.
        It is said to be \emph{autonomous} if its right hand side $f$ is time independent and \emph{pure}, if the right hand side only depends on $x(t-\tau_j)$ but not on $x(t)$.
        We define its \emph{maximal} and \emph{minimal delay} as $\taumax\defeq\tau_k$ and $\taumin\defeq\tau_1$, respectively.

        A DDE can be equipped with an \emph{initial condition} $x_{\tzero}\from \compactum{\tzero-\taumax}{\tzero} \to\R^n$. It specifies the initial state, i.e.\ the values of $x$, on which the right hand side depends at $t=\tzero$.
        Such a pair is called \emph{initial value problem} (IVP):
        \begin{equation}\label{eq:ivp}
            \begin{cases}
                \D{x}(t) = f(t,x(t),x(t-\tau_1),\ldots,x(t-\tau_k)) & \text{for } t\geq\tzero\\
                x(t) = x_{\tzero}(t) & \text{for } t\in\compactum{\tzero-\taumax}{\tzero}
            \end{cases}
        \end{equation}
    \end{definition}

    % TODO: put this later

    % TODO: autonomous DDEs is without loss of generality?
    In the following chapters, we will only consider autonomous DDEs, i.e.\ restrict to the case of initial time $\tzero=0$.

    % TODO: source for state-dependent and distributed delays
    The definition of a DDE can be extended to state-dependent or distributed delays (cf.~\cite{}). For simplicity, we restrict here to fixed delays, which already cover a wide range of applications.


%\section{Definition of Solution}
 %   \label{sec:definition-of-solution}

    \begin{definition}[Solution of DDE]\label{def:solution-dde}
        A function $x\from\compactum{\tzero-\taumax}{\tzero+T}\to\R^n$ is called \emph{(local) solution} of the initial value problem~\eqref{eq:ivp}, if and only if there exists a $T>0$ such that
        $x$ obeys the initial condition
        \begin{equation*}
            x(t) = x_{\tzero}(t) \quad\text{for } t\in\compactum{\tzero-\taumax}{\tzero}
        \end{equation*}
        and $x$ is continuous and piecewise continuously differentiable on $\compactum{\tzero}{\tzero+T}$, fulfilling
        \begin{equation*}
            \D{x}(t) = f(t,x(t),x(t-\tau_1),\ldots,x(t-\tau_k))
        \end{equation*}
        on each (open) interval $\open{t_i}{t_{i+1}}$ of its partition $\partition{\tzero=t_0}{t_m=\tzero+T}$.
        
        If the function $x$ is a solution for all $T>0$, it is called \emph{global}.   
    \end{definition}
    
    The piecewise continuity of the derivative means
    \begin{equation*}
        \lim_{s\downto t_i}\D{x}(s) = f(t_i,x(t_i),x(t_i-\tau_1),\ldots,x(t_i-\tau_k))
    \end{equation*}
    in the knots $t_i$ for $i\in\set{\range{0}{m-1}}$.

    % FIXME: local solution is on a single subdiv int only -> cont diffable

    %TODO: Fortsetzbarkeit For example initial condition has jump, this point is limit for local solution.
    % FIXME: is this true? ref to source
    There can be a T such that left limit does not exist. explosion
    The left limits $\lim_{s\upto t_m}\D{x}(s)$ exists not necessarily in the last knot $t_m=\tzero+T$. If it does, the solution is continuable.

    % TODO: for the right-hand derivative?
    in knots: right sided derivative
    the right continuability of derivative is equivalent to right-hand derivative equals f


    \cite{Roussel04DDEs}
    If we limit $T$ by $\taumin$, we can see the solution of a delay differential equation as an operator mapping from functions on $\compactum{t-\taumax}{t}$ to functions on $\compactum{t}{t+\taumin}$.
    Then the solution of the initial value problem is the sequence of these functions.
    derivative not necessarily continuous at knots

% \begin{figure}[h]\centering
%     \includegraphics[width=\textwidth]{figures/multiple.png}
% 	\caption{Illustration of proof to Lemma \ref{lemma-continuity}}
% 	\label{fig:not-allowed}
% \end{figure}

% FIXME: This lemma is wrong. Show instead integrability of f(t,x_t)
%\begin{lemma}
%    \label{lemma-continuity}

    % Let $x:[\tzero-\tau,\tzero+T] \rightarrow \R^n$ be piecewise continuous (as in Definition \ref{definition-piecewise-continuous}) with the partition $\{t_0,\ldots,t_k\}$, i.e. there are $k$ subintervals.
    %
    % Then $t \mapsto x_t = x(t+\theta)$, where $\theta\in[-\tau,0]$, is a piecewise continuous mapping from $[\tzero,\tzero+T]$ into $\statespace[\tau]$.
%\end{lemma}

%\begin{proof}
    % $x$ is piecewise continuous and hence uniformly piecewise continuous on the compact interval $I=[\tzero-\tau,\tzero+T]$.
    % i.e. uniformely continuous on each subinterval with stetiger Fortsetzung in right side.
    % \begin{equation}
    %     \forall\epsilon >0 \exists\delta_i >0 \forall t,s\in I_i: \quad \abs{t-s}<\delta_i \Rightarrow \nnorm{x(t)-x(s)}<\epsilon
    % \end{equation}
    % Let $\epsilon > 0$. $x|_{[t_i,t_{i+1}]}$ (with stetiger fortsetzung in right interval limit) is uniformly continuous, i.e. there is a $\delta_i > 0$ (for the given $\epsilon$), such that $\forall\,t, s \in [t_i,t_{i+1}]$ holds
    % % TODO: can use \leq ?
    % \begin{equation}
    %     \abs{t-s} < \delta_i \Rightarrow \nnorm{x(t)-x(s)} < \epsilon
    % \end{equation}
    %
    % Among the given $\delta_i$, choose the smallest as $\delta = \min_i \delta_i$.
    %
    % For any $i$ and $s,t\in [t_i,t_{i+1})\subset [\tzero,\tzero+T]$ with $\abs{t-s}<\delta$, it holds
    % \begin{equation}
    %     \supnorm{x_t - x_s} = \sup_{\theta\in [-\tau,0]}\nnorm{x(t+\theta) - x(s+\theta)} < \epsilon
    % \end{equation}
    % since $t+\theta, s+\theta \in I$
    % Hence $t \mapsto x_t$ is uniformely continuous on $[t_i,t_{i+1})$.
%\end{proof}


\section{Method of Steps}
    \label{sec:method-of-steps}
    
    If we restrict the IVP (Eq.~\ref{eq:ivp}) onto an interval $\compactum{\tzero}{\tzero+T_1}$ with $T_1\leq\taumin$, then the values of all $x(t-\tau_j)$ are specified by the initial condition and can thus be replaced by $x_{\tzero}(t-\tau_j)$.
    We obtain an \emph{initial value problem} for an \emph{ordinary differential equation}.
    If we can solve this IVP, i.e.\ if we can find a solution of the ODE on $\compactum{\tzero}{\tzero+T_1}$, then we can reapplay this method by plugging the computed solution into the DDE and solving the resulting ODE on the interval $\compactum{\tzero+T_1}{\tzero+T_2}$, where again $T_2\leq\taumin$. As long as one can solve the resulting ODE (for suitable $f$ and $x_{\tzero}$, the existence (and uniqueness) of a solution for the ODE is guaranteed by Picard-Lindelöf's theorem), this step can be iterated.

    This method, which allows to convert DDE into a ODE on a certain interval, eliminating the explicit dependance on the past by inserting the initial condition, is know as \emph{method of steps}.
    See \cite{Falbo06FDEs} for examples.

    % TODO: Why? It shows that solution is sequence of polynomials
    % with incrasing degree, smoothing property


\section{Existence and Uniqueness of Solutions}
    \label{solutions-existence-uniqueness}

    In this section, we show that under certain conditions, we can guarantee the existence of a solution for the DDE-IVP (Eq.~\ref{eq:ivp}) and that in general, it cannot have more than one. 

    % will consider rhs cont and lip
    % $f$ Lipschitz with piecewise continuous initial function have existence and uniqueness ???? smoothing

    \begin{definition}[Lipschitz Continuity]\label{def:lipschitz}
        % similar to \cite{pruesswilke10GewDiffGl,Smith10IntroDDE}
        % FIXME: dont I need xtau in right side ??? 
        A function $f\from\deff\to\R^n$ is called \emph{(locally) Lipschitz continuous} (in its $j$-th argument, refering to $t$ as zeroth argument) if and only if for all $a,b\in\R$ and $M>0$ there is a $L>0$, such that
        \begin{equation*}
            % TODO: is L(\nnorm*{x-x}+\nnorm*{y-y}) better? is equiv, with different L
            % FIXME: or just say Lipschitz continuous with respect to two other arguments, once for x once for y -> compare proof
            \nnorm*{f(t,x_1,\ldots,x_j,\ldots,x_k) - f(t,x_1,\ldots,y_j,\ldots,x_k)} \leq L\nnorm*{x_j-y_j}
        \end{equation*}
        for all $t\in\compactum{a}{b}$ and $x_j,y_j\in\R^n$ with $\nnorm{x_j},\nnorm{y_j},\leq M$.
    \end{definition}

    \begin{lemma}\label{lm:bounded-lipschitz}
        Let $f\from\deff\to\R^n$ be continuous and Lipschitz continuous in all but its zeroth argument.

        For any given compact interval $\compactum{a}{b}\subset\R$ and $M>0$, there exists a bound $K>0$ such that
        \begin{equation}
            \nnorm{f(t,x_1,\ldots,x_k)}\leq K
        \end{equation}
        for all $t\in\compactum{a}{b}$ and $x_j\in\R^n$ with $\nnorm{x_j}\leq M$.
    \end{lemma}
    \begin{proof}
        Let $L_j$ be the Lipschitz constant for the $j$-th argument of $f$ for the given $\compactum{a}{b}$ and $M$. Set $L\defeq\max_j\set{L_j}$.

        Then for all $t\in\compactum{a}{b}$ and $\range{x_1}{x_k}\in\R^n$ with $\range{\nnorm{x_1}}{\nnorm{x_k}}\leq M$
        \begin{multline*}
            \nnorm{f(t,x_1,\ldots,x_k)} \leq \nnorm{f(t,x_1,\ldots,x_k) - f(t,0,\ldots,0)} + \nnorm{f(t,0,\ldots,0)}\\
            \leq L_j\nnorm{x_j-0} + \nnorm{f(t,0,\ldots,0)} \leq LM+P = K
        \end{multline*}
        for every $j\in\set{\range{1}{k}}$. We used the continuity of $f$ on the compact set $\compactum{a}{b}$ for the existence of
        \begin{equation*}
            P = \max_{t\in\compactum{a}{b}}\nnorm{f(t,0,\ldots,0)}.
        \end{equation*}
    \end{proof}

    \begin{lemma}\label{lm:integral-equation}
        %TODO: compare with ODE lecture notes
        Finding a solution of the initial value problem~\eqref{eq:ivp} is equivalent to solving the integral equation
        \begin{equation*}\label{eq:integral-equation}
            \begin{cases}
                x(t) = x_{\tzero}(\tzero) + \integral{\tzero}{t} f(s,x(s),x(s-\tau_1),\ldots,x(s-\tau_k))\dx[s] & \text{for } t\geq\tzero\\
                x(t) = x_{\tzero}(t) & \text{for } t\in [\tzero-\taumax,\tzero]
            \end{cases}
        \end{equation*}
        where $f\from\deff\to\R^n$ is continuous and and Lipschitz continuous in all but its zeroth argument.
        The integral is meant to be componentwise, if $f$ is vector-valued.
    \end{lemma}
    \begin{proof}
        Let $x$ be a solution of the IVP. Thus $x$ is (by definition) piecewise continuous on $\compactum{\tzero-\taumax}{\tzero}$ and continuous and piecewise continuously differentiable on $\compactum{\tzero}{\tzero+T}$ with (piecewise) derivative $t\mapsto f(t,x(t),x(t-\tau_1),\ldots,x(t-\tau_k))$.
        % This composition of a continuous and piecewise continuous function is again piecewise continuous (Lemma~\ref{lm:comp-pw-cont}) and hence by Lemma~\ref{lm:pc-integrable} integrable on $\compactum{\tzero}{\tzero+T}$.
        By Lemma~\ref{lm:pc-hauptsatz} it follows
        \begin{equation*}
            x(t) = x_{\tzero}(\tzero) + \integral{\tzero}{t} f(s,x(s),x(s-\tau_1),\ldots,x(s-\tau_k))\dx[s]
        \end{equation*}
        for $t\geq\tzero$, since $x_{\tzero}(\tzero)=x(\tzero)$.

        Conversely, let $x$ be a solution of the integral equation.
        By the fundamental theorem of calculus, $x$ is continuous on $\compactum{\tzero}{\tzero+T}$.

        From the partition $\partition{t_0}{t_m}$ of $x_{\tzero}$, we define a partition of $\compactum{\tzero}{\tzero+T}$ by
        \begin{equation}\label{eq:partition}
            \mathcal{Z}\defeq \partition{\hat{t}_0}{\hat{t}_p}
            \defeq \set{\tzero,\tzero+T}\cup\bigcup_{j=1}^{k}\bigcup_{\substack{i=1\\t_i\geq\tau_j}}^{m}\set{t_i+\tau_j}
        \end{equation}
        Let $t\in\open{\hat{t}_{l-1}}{\hat{t}_l}$ for any $l\in\set{\range{1}{p}}$. If for any $j\in\set{\range{1}{k}}$ and $i\in\set{\range{0}{m}}$ was $t-\tau_j = t_i$, then $t=t_i+\tau_j=\hat{t}_r$ for a $r\in\set{\range{1}{p}}$, which would be a contradiction to the choice of $t$. Hence $t-\tau_j \neq t_i$ for all $j\in\set{\range{1}{k}}$ and $i\in\set{\range{0}{m}}$, what implies that all $s\mapsto x(s-\tau_j)$ are continuous in $t$. Thus the composition
        \begin{equation*}
            s\mapsto f(s,x(s),x(s-\tau_1),\ldots,x(s-\tau_k))
        \end{equation*}
        is continuous in $t$. The fundamental theorem of calculus states in this case that $x$ is differentiable in $t$ and that $\D{x}(t)=f(t,x(t),x(t-\tau_1),\ldots,x(t-\tau_k))$.

        For the right limits if follows by the continuity of $f$ and $\lim_{t\downto\hat{t}_l} x(t-\tau_j) = x(\hat{t}_l-\tau_j)$, since $t-\tau_j \neq t_i$, that
        \begin{align*}
            \lim_{t\downto\hat{t}_l} \D{x}(t)
            &= \lim_{t\downto\hat{t}_l} f(t,x(t),x(t-\tau_1),\ldots,x(t-\tau_k))\\
            &= f(\hat{t}_l,x(\hat{t}_l),x(\hat{t}_l-\tau_1),\ldots,x(\hat{t}_l-\tau_k))
        \end{align*}
        The left limits
        \begin{equation*}
            \lim_{t\upto\hat{t}_l} \D{x}(t) = \lim_{t\upto\hat{t}_l} f(t,x(s),x(t-\tau_1),\ldots,x(t-\tau_k)) 
        \end{equation*}
        exist for the same reason.
        Summarily, $x$ is continuous and piecewise continuously differentiable on $\compactum{\tzero}{\tzero+T}$ with piecewise derivative $f$ and it obviously obeys the initial condition, i.e. $x(t)=x_{\tzero}(t)$ for all $t\in\compactum{\tzero-\tau}{\tzero}$.
        % FIXME: f cont uberall in Voraussetzung?
    \end{proof}

    The most important result for the considered class of delay differential equations is the following theorem.
    Its proof is an adaption and extension of the existence theorem (Theorem 3.7) given in~\cite{Smith10IntroDDE} and the proof of uniqueness in \cite{PruessWilke10GewDiffGl}.

    \begin{theorem}[Existence of a unique solution]
    \label{thm:solution-existence}
        For a continuous function $f\from\deff\to\R^n$, satisfying the Lipschitz condition (Def.~\ref{def:lipschitz}) in all but its zeroth argument, consider the IVP for a delay differential equation
        \begin{equation}
            \begin{cases}
                \D{x} = f(t,x(t),x(t-\tau_1),\ldots,x(t-\tau_k)) & \text{for } t\geq\tzero\\
                x(t) = x_\tzero(t-\tzero) & \text{for } t\in\compactum{\tzero-\taumax}{\tzero}
            \end{cases}
        \end{equation}
        with zero-aligned initial function.
        
    %TODO: do we need global existence or just local?
        % where $\nnorm{\cdot}$ denotes the Euclidian norm on $\R^n$ and $\supnorm{\cdot}$ the supremum norm of the Banach space of continuous functions on $[-\tau,0]$.

        Then, for each \emph{initial condition} $x_{\tzero}\in\statespace[-\taumax]$ and start time $\tzero\in\R$, there \emph{exists} a \emph{unique local solution} of the IVP on a time interval $\compactum{\tzero-\taumax}{\tzero+T}$.
        %The duration $T>0$ depends on the sup-norm and the partition of the initial condition.
        This solution is continuous and piecewise continuously differentiable on $\compactum{\tzero}{\tzero+T}$ with partition \eqref{eq:partition}.
    \end{theorem}

    
    \begin{proof}
        % FIXME: where sup-norm?
        Let $\partition{-\taumax=t_0}{t_m=0}$ be the partition of $x_{\tzero}$. As a piecewise continuous function, the initial condition can be bounded on $\closeddelayinterval[-\tau]$ by any $M\geq \supnorm{x_\tzero}$.
        
        % FIXME: do I need t_0+tau or is just \tau okay? do I need x not to be pw, just cont in proof?        
        
        % FIXME: M or 2M?
        Let $K>0$ be the upper bound for $f$ on the set
        % TODO: could use any fixed upper bound for time compactum, taumax is arbitrary
        \begin{equation*}
            S\defeq\compactum{\tzero}{\tzero+\taumax} \times \set{x\in\R^n\with \nnorm{x}\leq 2M}^k
            %\times \set{y\in\R^n\with \nnorm{y}\leq 2M}
        \end{equation*}
        given by Lemma~\ref{lm:bounded-lipschitz} and $L>0$ the Lipschitz constant of $f$ for that set.
        We put $T\defeq\min\set{\taumax,\frac{M}{K}}$ to restrict the integrand $f$ to $S$.

        % FIXME: why continuous? its pw cont? cont in tzero
        We construct a series $(x_{(m)})_{m\in\N_0}$ of piecewise continuous functions, which approximates the solution of the initial value problem.
        Set
        \begin{equation*}
            x_{(0)}(t)= \begin{cases}
                x_\tzero(0) & \text{for } t\in\compactum{\tzero}{\tzero+T}\\
                x_\tzero(t-\tzero) & \text{for } t\in\compactum{\tzero-\taumax}{\tzero}
            \end{cases}
        \end{equation*}
        For $m\in\N_{>0}$ define
        \begin{equation*}
            x_{(m)}(t)= \begin{cases}
                x_\tzero(0) + \integral{\tzero}{t}
                f(s,x_{(m-1)}(s),x_{(m-1)}(s-\tau_1),\ldots,x_{(m-1)}(s-\tau_k))\dx[s] & \text{for } t\in\compactum{\tzero}{\tzero+T}\\
                x_\tzero(t-\tzero) & \text{for } t\in\compactum{\tzero-\taumax}{\tzero}
            \end{cases}
        \end{equation*}
        The integral exists, because the integrand is a composition of a continuous and piecewise continuous function, which is again piecewise continuous (Lemma~\ref{lm:comp-pw-cont}) and hence by Lemma~\ref{lm:pc-integrable} integrable on $\compactum{\tzero}{\tzero+T}$.

        It holds for all $m>0$ and $t\in \compactum{\tzero-\taumax}{\tzero}$ by the definition of this sequence that
        \begin{equation*}
            \nnorm*{x_{(m)}(t)-x_{(m-1)}(t)}=0
        \end{equation*}
        We show by induction over $m$ that for all $t\in [\tzero,\tzero+T]$ it holds
        \begin{equation*}
            \nnorm*{x_{(m)}(t)-x_{(m-1)}(t)} \leq \frac{K}{L}\frac{L^m (t-\tzero)^m}{m!}.
        \end{equation*}
        Let $t\in [\tzero,\tzero+T]$. Since for all $s\in [\tzero-\taumax,\tzero+T]$ obviously $\nnorm{x_{(0)}(t)}\leq M$, the statement for $m=0$ follows from the boundedness of $f$ on $S$ and the triangle inequality for integrals:
        \begin{equation*}
            \nnorm{x_{(1)}(t)-x_{(0)}(t)} = \nnorm*{\integral{\tzero}{t} f(s,x_{(0)}(s),x_{(0)}(s-\tau_1),\ldots,,x_{(0)}(s-\tau_k))\dx[s]} \leq K(t-\tzero)
        \end{equation*}
        In the inductive step for any $m>0$, we use that $\nnorm{x_{(m-1)}(t)}\leq 2M$ for all $s\in [\tzero-\taumax,\tzero+T]$ implies
        % FIXME: why x(m)(t) smaller than 2M, such that K holds?
        % TODO: why do integral and norm commute? once integral over vectors, once over scalars
        \begin{align}\label{eq:bounded-xm}
            \nnorm*{x_{(m)}(t)} &\leq \nnorm*{x_\tzero(0)} + \int_\tzero^t \nnorm*{f(s,x_{(m-1)}(s),x_{(m-1)}(s-\tau_1),\ldots,x_{(m-1)}(s-\tau_k))}\dx[s] \nonumber \\
            &\leq M + K(t-\tzero) \leq M+KT \nonumber \\
            &\leq 2M
        \end{align}
        using the triangle inequality and the choice of $T$.

        It follows by the Lipschitz property of $f$ (for its first argument) that
        \begin{multline*}
            \nnorm*{x_{(m+1)}(t)-x_{(m)}(t)}=\\
            = \left\lVert\int_\tzero^t f(s,x_{(m)}(s),x_{(m)}(s-\tau_1),\ldots,x_{(m)}(s-\tau_k))\right.\\ \left.- f(s,x_{(m-1)}(s),x_{(m-1)}(s-\tau_1),\ldots,x_{(m-1)}(s-\tau_k))\dx[s]
            \vphantom{\int_\tzero^t}\right\rVert\\
            \leq L \int_\tzero^t \nnorm*{x_{(m)}(s) - x_{(m-1)}(s)}\dx[s]\\
            \leq \frac{L^m K}{m!} \int_\tzero^t (s-\tzero)^m\dx[s]
            = \frac{L^m K}{(m+1)!}(t-\tzero)^{m+1}
        \end{multline*}
        %We use this bound and the triangle inequality in
        The Cauchy criterion for convergent series (\cite{Gathmann12GDM} 6.13, \cite{Rudin76PrinciplesAnalysis} 3.22) applied to the exponential series states that
        \begin{equation*}
            % "\ " needed for space
            \mforall{\varepsilon>0}\ \mexists{n_0\in\N_0}\ \mforall{m\geq k\geq n_0}\holds \sum_{i=k+1}^m \frac{(LT)^i}{i!} <\varepsilon
        \end{equation*}
        So for any $\varepsilon>0$ exist $k\in\N_0$ and $m\geq k$, such that
        \begin{align*}
            \nnorm*{x_{(m)}(t)-x_{(k)}(t)} \leq{} & \nnorm*{x_{(m)}(t)-x_{(m-1)}(t)} + \nnorm*{x_{(m-1)}(t)-x_{(m-2)}(t)} + {}\\
            & + \ldots + \nnorm*{x^{(k+1)}(t)-x^{(k)}(t)}\\
            \leq{} & \frac{K}{L}\frac{L^m (t-\tzero)^m}{m!} + \frac{K}{L}\frac{L^{m-1} (t-\tzero)^{m-1}}{(m-1)!} + {}\\
            & + \ldots +\frac{K}{L}\frac{L^{k+1} (t-\tzero)^{k+1}}{(k+1)!}\\
            \leq{} & \frac{K}{L}\sum_{i=k+1}^m \frac{(LT)^i}{i!} < \varepsilon
        \end{align*}
        for all $t\in [\tzero,\tzero+T]$, i.e. $(x_{(m)})$ is a Cauchy sequence

    
        % FIXME: show that this a Cauchy series
        %This is the tail of the convergent exponential series and hence it converges to zero for $k\to\infty$ (boundedness and positivity of summands, monotonicity crit).

        % FIXME: why continuous? since integral exists
        Since each $x_{(m)}$ is continuous on $\compactum{\tzero}{\tzero+T}$, this Cauchy sequence admits a limit $x$ in the Banach space $\Cn[0]{\compactum{\tzero}{\tzero+T}}{\R^n}$ with respect to the sup-norm.

        Again, we extend $x$ to $\compactum{\tzero-\taumax}{\tzero}$ with $x_\tzero$, such that $x\in\Cnpw[0]{\compactum{\tzero-\tau}{\tzero+T}}{\R^n}$.


        

        Since by the continuity of the supremum norm it follows from~\eqref{eq:bounded-xm} that
        \begin{equation*}
            \supnorm*{x}=\lim_{m\to\infty}\supnorm*{x_m}\leq 2M
        \end{equation*}
        can apply Lipschitz property of $f$
        \begin{equation*}
            \sup_{t\in\compactum{\tzero}{\tzero+T}}\nnorm*{f(s,x_m(s),x_m(s-\tau))-f(s,x(s),x(s-\tau))} \leq \sup_{t\in\compactum{\tzero}{\tzero+T}}\nnorm*{x_m(t)-x(t)}
        \end{equation*}
        Due to the uniform convergence (conv in sup-norm) of $x_{(m)}\to x$, we get the uniform convergence
        \begin{equation*}
            f(s,x_m(s),x_m(s-\tau)) \xrightarrow{m\to\infty} f(s,x(s),x(s-\tau))
        \end{equation*}
        and hence the integral and the limit process swap and by
        \begin{align*}
            x(t) = \lim_{m\to\infty} x^{(m+1)} &= x_\tzero(0) + \lim_{m\to\infty}\int_\tzero^t f(s,x^{(m)}(s),x^{(m)}(s-\tau))\dx[s]\\
            &= x_\tzero(0) + \int_\tzero^t f(s,x(s),x(s-\tau))\dx[s]
        \end{align*}
        it follows that $x$ solves the integral equation and hence, by Lemma~\ref{lm:integral-equation},
        this proves the existence of a solution to the DDE.
        % TODO: continuous because limit in Banach space, diffable and subdiv see integral equiv lemma

        % TODO: can one solution be on [\tzero, T_2] with T_2<T ?
        It remains to show uniqueness.
        Let $x$ and $\bar{x}$ be two solutions of the DDE on $[\tzero,\tzero+T]$.
        By Lemma \ref{lm:integral-equation} they are equivalent to solutions of the integral equations
        \begin{equation}
            x(t) = x_\tzero(0) + \int_\tzero^t f(s,x(s),x(s-\tau))\dx[s]
        \end{equation}
        and
        \begin{equation}
            \bar{x}(t) = x_\tzero(0) + \int_\tzero^t f(s,\bar{x}(s),\bar{x}(s-\tau))\dx[s]
        \end{equation}
        For $t\in [\tzero,T]$, we set
        \begin{align*}
            \rho(t) &:= \nnorm*{x(t)-\bar{x}(t)} \leq \int_\tzero^t \nnorm*{f(s,x(s),x(s-\tau))-f(s,\bar{x}(s),\bar{x}(s-\tau))}\dx[s]\\
            & \leq L \int_\tzero^t \nnorm*{x(s)-\bar{x}(s)}\dx[s] = L \int_\tzero^t \rho(s)\dx(s)\\
            &= L \int_\tzero^t \e{-\alpha s}\rho(s)\e{\alpha s}\dx[s] \leq L \sup_{s\in [\tzero,\tzero+T]}\left(\e{-\alpha s}\rho(s)\right)\int_\tzero^t \e{\alpha s}\dx[s]\\
            & \leq\frac{L}{\alpha}\e{\alpha t} \sup_{s\in [\tzero,\tzero+T]}\left(\e{-\alpha s}\rho(s)\right)
        \end{align*}
        with $L$ the Lipschitz constant of $f$ on the set ...
        and $\rho$ is continuous, since $x$ continuous
        Choosing $\alpha=2L$ and multiplying with $\e{-\alpha t}>0$ leads to
        \begin{equation}
            \rho(t)\e{-2Lt} \leq \frac{1}{2}\sup_{s\in [\tzero,\tzero+T]}\left(\e{-2L s}\rho(s)\right)
        \end{equation}
        for all $t\in [\tzero,\tzero+T]$
        \begin{equation}
            0 \leq \sup_{t\in [\tzero,\tzero+T]}\left(\rho(t)\e{-2Lt}\right) \leq \frac{1}{2}\sup_{s\in [\tzero,\tzero+T]}\left(\e{-2L s}\rho(s)\right)
        \end{equation}
        That is only possible if $\rho(t)=0$ for all $t\in [\tzero,\tzero+T]$, which means $x(t)=\bar{x}(t)$.

        % TODO: still needed?
        just proof existence/uniqueness on each peace of continuity proof continuity at knots with Lemma of integral equ

    \end{proof}

% TODO: on [\tzero,t_1] DDE equiv to ODE/IntEq
% -> ex unique sol on [\tzero, t_1]
% -> ex unique sol on [\tzero,\tau] (glob Lip of f on [tzero,tau])
% -> ex unique sol on [\tzero,2\tau] (continuous?, diffable?)
% show continuity and pw diffable (nth to show)
    % \begin{lemma}[cont]\label{lm:c}
    %     $x_1$ loc sol on $\compactum{\tzero-\tau}{\tzero+t_1}$ for init cond $x_{\tzero}$
    %     $x_2$ and loc sol on $\compactum{\tzero+t_1-\tau}{\t_1+T}$ for init cond $x_1$
    %     then $x_1(\tzero+t_1)=x_2(\tzero+t_1)$
    %     follows from initial cond $x_1$
    % \end{lemma}
\begin{corollary}
    \label{cor:continuability-of-solution}

    % TODO: What is derivation in randpunkten of interval [] ?
    If in Theorem \ref{theorem-solution-existence} $T=t_1-\tau$, can reapplay theorem with starting point $\tzero=\tzero_{old}+t_1-\tau$. Get existence of unique solution on $[\tzero-\tau,\tzero+S]$ with $S>T$.
\end{corollary}

\begin{corollary}
    \label{corollary}
    If f is polynomial in $t$, $x(t)$ and $x(t-\tau)$ then theorem holds

    polynomial -> continuously differentiable -> locally Lipschitz
% IDEA: can show? init cond bounded by M, and loc sol bounded by M, get glob sol since f glob Lip on set of bounded inputs?


    %TODO: put after uniqueness theorem, need uniqueness and existence so that amap well-defined
    The notion of solution for an autonomous DDE as given above can be lifted to be a trajectory $\trajectory[x]$ in the statespace
    \begin{equation}
        \trajectory[x] \from [0,T] \to \statespace[\tau],\\
        t \mapsto \xbartaut{t}
    \end{equation}

    The \emph{state} at time $t$ is a function which provides a time limited history up to the current time. This is all information needed to determine (using the DDE) to determine the solution for time $\geq t$. It is defined as $\xbartaut{t}(s)\defeq x(t+s)$ for $s\in [-\tau,0]$. In the case of $t=0$, we simplify the notation to $\xbartau \defeq \xbartaut{0}$.
    This notion of solution is a \emph{dynamical systems} point of view which later turns out to be useful.

  Other results know from ordinary differential equations can be adapted to delay differential equations, such as continuous (or even differentiable) dependence of the solution on initial data and, see \cite{Dads06DDEs} % TODO: add other cites

%TODO: can write DDE (eq??) from definition as

\begin{equation}
    \begin{cases}
        \D{x}=f(\xbartaut{t})\defeq g(\xbartaut{t}(0),\xbartaut{t}(-\tau)) &\text{for } t\geq 0\\
        x(t)=x_0(t) & \text{for } t\in[-\tau,0]
    \end{cases}
\end{equation}
\end{corollary}

\begin{proof}

\end{proof}

% TODO: non-autonomous -> autonomous

\begin{example}\label{ex:ode-dde}
    Delay differential equations can often incorporate a much richer behaviour than ordinary differetial equations.
    The basic ODE IVP
    \begin{equation}
        \begin{cases}
            \D{x}(t) = -x(t)\\
            x(0) = x_0
        \end{cases}
    \end{equation}
    has the solution $x(t)=x_0 e^{-t}$. However the similiar DDE
    \begin{equation}
        \begin{cases}
            \D{x}(t) = -x(t-\tau) & t\geq 0\\
            x(t) = x_0(t) & -\tau\leq t\leq 0
        \end{cases}
    \end{equation}
    has a much richer dynamics, but solution (as series) for $x_0\equiv 1$, can compute first solutions by method of steps. \ldots{}    
\end{example}


\begin{figure}[h]\centering
    \includegraphics[width=\textwidth]{figures/piecewise-initial-function.png}
	%\caption{}
	\label{Piecewise continuous initial function.}
\end{figure}

