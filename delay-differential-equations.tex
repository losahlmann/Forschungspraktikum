\chapter{Delay Differential Equations}\label{sec:delay-differential-equations}

\section{Piecewise Continuous Functions}
    \label{sec:piecewise-continuous-functions}
    
    The following definition is motivated by capturing the character evolution arising from hybrid systems. We will see that we can consider such to be piecewise continuous.

    % \begin{definition}[Piecewise Continuous]\label{def:piecewise-continuous}
    %     Let $D=[a,b]\subseteq\R$ be a closed interval (this includes the cases when $a=-\infty$ or $b=\infty$, or both). The mapping $x:D\rightarrow\R^n$ is called \emph{piecewise continuous} if and only if there is a finite subdivision $\{t_i:i=\range{0}{m}\}$ of $D$ (i.e.\ $a=t_0<t_1<\ldots<t_m=b$) such that $x$ is continuous on each interval piece $[t_i,t_{i+1})$ for all $i=\range{0}{m-1}$ and the left sided limits
    %     \begin{equation}
    %         \lim_{\substack{t\upto t_{i+1}\\ t\in[t_i,t_{i+1})}} x(t)
    %     \end{equation}
    %     exist. Hence $x(b)$ can be an isolated point and this right interval limit $b$ is the only spot where such is allowed.

    %     We denote by $\Cnpw[0]{D}{\R^n}$ the set of \emph{piecewise continuous functions} on the compact interval $D$ (this excludes the cases with $\pm\infty$), mapping to $\R^n$.
    % \end{definition}

    \begin{definition}[Piecewise Continuously Differentiable]\label{def:pw-cont-diff}
        % FIXME: find better word for 'subdivision'. partition?
        Let $D=[a,b]\subseteq\R$ be a closed interval (this includes the cases when $a=-\infty$ or $b=\infty$, or both). The mapping $x:D\rightarrow\R^n$ is called $n$-times \emph{piecewise continuously differentiable} if and only if there is a finite subdivision (ordered set) $\{t_i:i=\range{0}{m}\}$ of $D$ (i.e.\ $a=t_0<t_1<\ldots<t_m=b$) such that $x$ is $n$-times continuously differentiable on each interval piece $(t_i,t_{i+1})$ with continuable derivatives on $\compactum{t_i}{t_{i+1}}$.

        This means, for all $i=\range{0}{m-1}$ and for all $k=\range{0}{n}$ exist the left sided limits
        \begin{equation}
            \lim_{\substack{t\upto t_{i+1}\\ t\in(t_i,t_{i+1})}} \D[k]{x}(t)
        \end{equation}
        as well as the right sided limits
        \begin{equation}
            \lim_{\substack{t\downto t_{i}\\ t\in(t_i,t_{i+1})}} \D[k]{x}(t) =: \D[k]{x}(t_i)
        \end{equation}
        which are supposed to coincide with the value of $\D[k]{x}$ at $t_i$.
        Hence $x(b)$ can be an isolated point and this right interval limit $b$ is the only spot where such is allowed.
        In the case $n=0$, we say $x$ is \emph{piecewise continuous}.

        We denote by $\Cnpw[n]{D}{\R^n}$ the set of \emph{$n$-times piecewise continuously differentiable functions} on the compact interval $D$ (this excludes the cases with $\pm\infty$), mapping to $\R^n$, and respectively, by $\Cnpw[0]{D}{\R^n}$ the \emph{piecewise continuous functions}.
    \end{definition}

    % TODO: sup-norm for pw

    \begin{lemma}[]\label{lm:pc-integrable}
        A \emph{piecewise continuous function}, as defined in Definition~\ref{def:pw-cont-diff} is (Riemann) integrable.
    \end{lemma}
    \begin{proof}
        See standard analysis literature, such as \cite{Rudin76PrinciplesAnalysis} (Theorem~6.10) or \cite{Gathmann12GDM} (Example~11.16b).
        % ObdA: one subint with jump at end
    \end{proof}

    The following lemma generalizes the fundamental theorem of calculus to piecewise continuous derivatives.

    \begin{lemma}[]\label{lm:pc-hauptsatz}
        Let $F\in\Cn[0]{\compactum{a}{b}}{} \cap \Cnpw[1]{\compactum{a}{b}}{}$ with the subdivision $\subdivision{a=t_0}{t_m=b}$ and piecewise derivative $f$.
        %of $\compactum{a}{b}\subset\R$. 
        For all $t\in\compactum{a}{b}$ it holds
        \begin{equation*}
            F(t)-F(a) = \integral{a}{t} f(s)\dx[s]
            %\sum_{i=0}^k\int\limits_{t_i}^{t_{i+1}}f(t)\dx[t] + \int\limits_{t_k}^s f(t)\dx[t]
        \end{equation*}
        %where $t_k\leq s < t_{k+1}$.
        % FIXME: what about a=-inf or b=inf?
    \end{lemma}
    \begin{proof}
        On each interval $\compactum{t_{i-1}}{t_i}$ of the subdivision, $f$ is piecewise continuous and hence integrable.

        For all $\zeta\in\open{t_{i-1}}{t_i}$ is $F$ differentiable on $\compactum{t_{i-1}}{\zeta}$ with $\D{F}=f$.
        By the fundamental theorem of calculus (cf.\ standard analysis literature, e.g.~\cite{Gathmann12GDM,Rudin76PrinciplesAnalysis}), it follows
        \begin{equation*}
            \denseintegral{t_{i-1}}{\zeta} f(s)\dx[s] = F(\zeta)-F(t_{i-1})
        \end{equation*}
        and by the continuity of $F$ that
        \begin{equation*}
            \denseintegral{t_{i-1}}{t_i} f(s)\dx[s]
            = \lim_{\zeta\to t_i}\denseintegral{t_{i-1}}{\zeta} f(s)\dx[s]
            = \lim_{\zeta\to t_i} F(\zeta)-F(t_{i-1})
            = F(t_i)-F(t_{i-1})
        \end{equation*}
        For any $t\in\compactum{a}{b}$, there is a $k\in\set{\range{1}{m}}$ such that $t\in\closedopen{t_{k-1}}{t_k}$ (in the case $t=b$, set $k=m$), summation over $i=\range{1}{k}$ yields the telescoping series
        \begin{equation*}
            F(t)-F(a) = \sum_{i=1}^{k} \denseintegral{t_{i-1}}{t_i} f(s)\dx[s] + \integral{t_j}{t} f(s)\dx[s]
        \end{equation*}
        which is by the additivity of the integral
        \begin{equation*}
            F(t)-F(a) = \integral{a}{t} f(s)\dx[s]
        \end{equation*}
    \end{proof}

% \begin{figure*}[h]\centering
%     \begin{subfigure}[t]{0.5\textwidth}\centering
%         \includegraphics[width=\textwidth]{figures/allowed.png}
%         \caption{Admissible piecewise continuous function.}
%         \label{fig:allowed}
%     \end{subfigure}
%     \begin{subfigure}[t]{0.5\textwidth}\centering
%         \includegraphics[width=\textwidth]{figures/not-allowed.png}
% 	    \caption{Not allowed!}
% 	    \label{fig:not-allowed}
%     \end{subfigure}
%     \caption{Examples to Definition \ref{definition-piecewise-continuous}.}
% \end{figure*}


\section{Definition DDE}
    \label{sec:definition-dde}

    \begin{definition}[Delay Differential Equation]\label{def:dde}
        Let $f\from\deff\to\R^n$ and $\tau_j > 0$ for $j=\range{1}{k}$. Put $\taumax\defeq\max_j\set{\tau_j}$.

        A functional equation of the form
        \begin{equation}\label{eq:dde}
            \D{x}(t) = f(t,x(t),x(t-\tau_1),\ldots,x(t-\tau_k))
        \end{equation}
        is called \emph{delay differential equation} (DDE) with \emph{multiple constant, discrete delays $\tau_j$}.
        It is \emph{autonomous} if its right hand side $f$ is time independent and \emph{pure} if the right hand side only depends on $x(t-\tau_i)$ and not on $x(t)$.

        A DDE can be equipped with an \emph{initial condition} $x_{\tzero}$. It specifies the values of $x$ on $[\tzero-\taumax, \tzero]$ on which the right hand side depends.
        Such a pair is called \emph{initial value problem (IVP)}:
        \begin{equation}\label{eq:ivp}
            \begin{cases}
                \D{x}(t) = f(t,x(t),x(t-\tau_1),\ldots,x(t-\tau_k)) & \text{for } t\geq\tzero\\
                x(t) = x_{\tzero}(t) & \text{for } t\in [\tzero-\taumax,\tzero]
            \end{cases}
        \end{equation}
    \end{definition}

    % Since we only consider autonomous DDEs, we can without loss of generality restrict to the case of initial time $t_0=0$.

    % The definition of a DDE can be extended to multiple constant discrete delays. For simplicity, we restrict here to a single delay.


\section{Definition of Solution}
    \label{sec:definition-of-solution}

    \begin{definition}[Solution of DDE]\label{def:solution-dde}
        A function $x\from\compactum{\tzero-\taumax}{\tzero+T}\to\R^n$ is called \emph{(local) solution} of the initial value problem~\eqref{eq:ivp}, if and only if
        $x$ is continuous and piecewise continuously differentiable on $\compactum{\tzero}{\tzero+T}$ (in the sense of Def.~\ref{def:piecewise-continuous}) with subdivision $\Delta$.
        This means, when $\Delta=\subdivision{\tzero=t_0}{t_m=\tzero+T}$, $x$ is continuously differentiable on each interval $(t_i,t_{i+1})$
        %there exists a $T>0$ such that
        % FIXME: local solution is on a single subdiv int only -> cont diffable
        %$\restrict{x}{(t_i,t_{i+1})}\in \Cn[1]{\compactum{\tzero}{\tzero+T}}{\R^n}$ with
        with
        \begin{equation*}
            \D{x}(t) = f(t,x(t),x(t-\tau_1),\ldots,x(t-\tau_k))
        \end{equation*}
        for all $t\in (\tzero,\tzero+T)$ and in $t=t_i$, it holds
        % TODO: for the right-hand derivative?
        % cf ODE sol
        \begin{equation*}
            \lim_{s\downto t_i}\D{x}(s) = f(t_i,x(t_i),x(t_i-\tau_1),\ldots,x(t_i-\tau_k))
        \end{equation*}
        and $x$ obeys the initial condition:
        \begin{equation*}
            x(t) = x_{\tzero}(t) \quad\text{for } t\in [\tzero-\taumax,\tzero].
        \end{equation*}
        % FIXME: global solution should allow knicks
        If the function $x$ is a solution for all $T>0$, it is called \emph{global}.

        %TODO: differentiable in right rand point? need not derivative in right hand point
        %TODO: Fortsetzbarkeit For example initial condition has jump, this point is limit for local solution.
    \end{definition}

    

% \begin{figure}[h]\centering
%     \includegraphics[width=\textwidth]{figures/multiple.png}
% 	\caption{Illustration of proof to Lemma \ref{lemma-continuity}}
% 	\label{fig:not-allowed}
% \end{figure}

% FIXME: This lemma is wrong. Show instead integrability of f(t,x_t)
\begin{lemma}
    \label{lemma-continuity}

    % Let $x:[\tzero-\tau,\tzero+T] \rightarrow \R^n$ be piecewise continuous (as in Definition \ref{definition-piecewise-continuous}) with the subdivision $\{t_0,\ldots,t_k\}$, i.e. there are $k$ subintervals.
    %
    % Then $t \mapsto x_t = x(t+\theta)$, where $\theta\in[-\tau,0]$, is a piecewise continuous mapping from $[\tzero,\tzero+T]$ into $\statespace[\tau]$.
\end{lemma}

\begin{proof}
    % $x$ is piecewise continuous and hence uniformly piecewise continuous on the compact interval $I=[\tzero-\tau,\tzero+T]$.
    % i.e. uniformely continuous on each subinterval with stetiger Fortsetzung in right side.
    % \begin{equation}
    %     \forall\epsilon >0 \exists\delta_i >0 \forall t,s\in I_i: \quad \abs{t-s}<\delta_i \Rightarrow \nnorm{x(t)-x(s)}<\epsilon
    % \end{equation}
    % Let $\epsilon > 0$. $x|_{[t_i,t_{i+1}]}$ (with stetiger fortsetzung in right interval limit) is uniformly continuous, i.e. there is a $\delta_i > 0$ (for the given $\epsilon$), such that $\forall\,t, s \in [t_i,t_{i+1}]$ holds
    % % TODO: can use \leq ?
    % \begin{equation}
    %     \abs{t-s} < \delta_i \Rightarrow \nnorm{x(t)-x(s)} < \epsilon
    % \end{equation}
    %
    % Among the given $\delta_i$, choose the smallest as $\delta = \min_i \delta_i$.
    %
    % For any $i$ and $s,t\in [t_i,t_{i+1})\subset [\tzero,\tzero+T]$ with $\abs{t-s}<\delta$, it holds
    % \begin{equation}
    %     \supnorm{x_t - x_s} = \sup_{\theta\in [-\tau,0]}\nnorm{x(t+\theta) - x(s+\theta)} < \epsilon
    % \end{equation}
    % since $t+\theta, s+\theta \in I$
    % Hence $t \mapsto x_t$ is uniformely continuous on $[t_i,t_{i+1})$.
\end{proof}


\section{Method of Steps}
    \label{sec:method-of-steps}
    
    for $t\in [0,\tau]$, $x$ must satisfy the following ordinary initial value problem obtained by plugging the initial function into equation (??). For suitable $f$ and $x_0$, the existence (and uniqueness) of a solution on $[0,\tau]$ is guaranteed by ODE theory (\ldots{} or Picard-Lindelöf theorems).

    This procedure can then be applied repeatedly to extend the obtained solution by steps of length $\tau$.





\section{Existence and Uniqueness of Solutions}
    \label{solutions-existence-uniqueness}

    will consider rhs cont and lip
    $f$ Lipschitz with piecewise continuous initial function have existence and uniqueness ???? smoothing

    \begin{definition}[Lipschitz Continuity]\label{def:lipschitz}
        % similar to \cite{pruesswilke10GewDiffGl,Smith10IntroDDE}
        % FIXME: dont I need xtau in right side ??? 
        A function $f\from\deff\to\R^n$ is called \emph{(locally) Lipschitz continuous} in its second argument if and only if for all $a,b\in\R$ and $M>0$ there is a $L>0$ such that
        \begin{equation*}
            % TODO: is L(\nnorm*{x-x}+\nnorm*{y-y}) better? is equiv, with different L
            % FIXME: or just say Lipschitz continuous with respect to two other arguments, once for x once for y -> compare proof
            \nnorm*{f(t,x,y) - f(t,\bar{x},y)} \leq L\max\left\{\nnorm*{x - \bar{x}},\nnorm*{y - \bar{y}}\right\}
        \end{equation*}
        for all $t\in [a,b]$ and $x,\bar{x},y\in\R^n$ with $\nnorm{x},\nnorm{\bar{x}},\nnorm{y}\leq M$.
    \end{definition}

    \begin{lemma}\label{lm:bounded-lipschitz}
        Let $f\from\deff\to\R^n$ be continuous and Lipschitz continuous in its second argument.

        For any given compact interval $\compactum{a}{b}$ and $M>0$ there exists a bound $K>0$ such that
        \begin{equation}
            \nnorm{f(t,x,y)}\leq K
        \end{equation}
        for all $t\in\compactum{a}{b}$ and $x,y\in\R^n$ with $\nnorm{x},\nnorm{y}\leq M$.
    \end{lemma}
    \begin{proof}
        Let $L$ be the Lipschitz constant of $f$ for the given $\compactum{a}{b}$ and $M$. Then
        \begin{multline*}
            \nnorm{f(t,x,y)} \leq \nnorm{f(t,x,y) - f(t,0,y)} + \nnorm{f(t,0,y)}\\
            \leq L\nnorm{x-0} + \nnorm{f(t,0,y)} \leq LM+P = K
        \end{multline*}
        for $t\in\compactum{a}{b}$ and $x,y\in\R^n$ with $\nnorm{x},\nnorm{y}\leq M$. We used the continuity of $f$ on the compact set $S=\compactum{a}{b}\times\set{z\in\R^n\with\nnorm{z}\leq M}$ for the existence of
        \begin{equation*}
            P = \max_{(s,z)\in S}\nnorm{f(s,0,z)}
        \end{equation*}
    \end{proof}

    \begin{lemma}\label{lm:integral-equation}
        %TODO: compare with ODE lecture notes
        Finding a solution of the initial value problem~\eqref{eq:ivp} is equivalent to solving the integral equation
        \begin{equation}\label{eq:integral-equation}
            \begin{cases}
                x(t) = x_{\tzero}(\tzero) + \int_{\tzero}^t f(s,x(s),x(s-\tau))\dx[s] & \text{for } t\geq\tzero\\
                x(t) = x_{\tzero}(t) & \text{for } t\in [\tzero-\tau,\tzero]
            \end{cases}
        \end{equation}
        where ... (same as for ivp)
        and is continuous in t.
        integral componentwise, f vector valued
    \end{lemma}
    \begin{proof}
        Let $x$ be a solution of the IVP. Thus $x$ is (by definition) continuous on $\compactum{\tzero}{\tzero+T}$ and piecewise continuous on $\compactum{\tzero-\tau}{\tzero}$. This means that the chain $f(t,x(t),x(t-\tau))$ is piecewise continuous and hence integrable on $\compactum{\tzero}{\tzero+T}$. Furthermore, $x$ is (by definition) piecewise differentiable.
        The fundamental theorem of calculus states
        \begin{equation*}
            x(t) = x_{\tzero}(\tzero) + \int_{\tzero}^t f(s,x(s),x(s-\tau))\dx[s]
        \end{equation*}
        for $t\geq\tzero$.

        Conversely, let $x\from [\tzero-\tau,\tzero+T]$ be a solution of the integral equation~\eqref{eq:integral-equation}, i.e. $x(t)=x_{\tzero}(t)$ for all $t\in\compactum{\tzero-\tau}{\tzero}$ and $x(t) = x_{\tzero}(\tzero) + \int_{\tzero}^t f(s,x(s),x(s-\tau))\dx[s] =: F(t)$ for $t\in\compactum{\tzero}{\tzero+T}$ for a $T>0$.
        HS -> continuous
        % FIXME: f cont uberall in Voraussetzung?
        Since $f$ is by the precondition continuous, x cont -> integrand cont iff xtau cont
        Let $\set{t_0-\tau < \ldots < t_m-\tau}$ be the subdivision of the initial condition $\x_{\tzero}\in\statespace$.
        Since $f$ is integrable on and continuous on ..., the fundamental theorem of calculus states the differntiability of F on $(t_i,t_{i+1})$ with
        \begin{equation}
            x'=F'=f()
        \end{equation}
        % FIXME: t_i is jumppoint of init cond, can be x_sigma or x
        show $\D{F}$ is continuable in jump point $t_i$
        $\lim_{t\downto t_i}\D{F}(t)=\lim_{t\downto t_i}f(t,x(t),x(t-\tau))=f(t_i,x(t_i),x(t_i-\tau))$ by continuity of $f$, $x|(t_i,t_{i+1})$ and def of pw cont $x|(t_i-\tau,t_{i+1}-\tau)$
        same way: limit exists for $t\upto t_{i+1}$



        integrate from discontinuity of $\xbartaut{t}$ to discontinuity and proof stetige fortsetzbarkeit at these points
    \end{proof}

    \begin{theorem}[Existence of unique solution]\label{thm:solution-existence}
        Consider the Delay Differential Equation
    %TODO: do we need global existence or just local?
        \begin{equation}
            \begin{cases}
                \D{x} = f(t,x(t),x(t-\tau)) & \text{for } t\geq\tzero\\
                x(t) = x_\tzero(t-\tzero)   & \text{for } t\in [\tzero-\tau,\tzero]
            \end{cases}
        \end{equation}
        with $f\from\deff\to\R^n$ continuous and satisfying the (local) Lipschitz condition in its second argument (Def.~\ref{def:lipschitz}).

        % where $\nnorm{\cdot}$ denotes the Euclidian norm on $\R^n$ and $\supnorm{\cdot}$ the supremum norm of the Banach space of continuous functions on $[-\tau,0]$.

        Then for each \emph{initial condition} $x_{\tzero}\in\statespace[\tau]$ and start time $\tzero$, there \textbf{exists} a \textbf{unique local solution} of the IVP on a time interval $[\tzero-\tau, \tzero+T]$. The duration $T>0$ depends on the sup-norm and discontinuity points of the initial condition. (?)
        This solution is continuous and piecewise differentiable on $\compactum{\tzero}{\tzero+T}$ with subdivision $t_i+\tau$.
    \end{theorem}

    The proof is smiliar to the proof of the existence theorem (Theorem 3.7) given in~\cite{Smith10IntroDDE}.
    \begin{proof}
        % FIXME: where sup-norm?
        As a piecewise continuous function, the initial condition can bounded by $M\geq \supnorm{x_\tzero}$ on $\delayinterval[\tau]$.
        
        % FIXME: do I need t_0+tau or is just \tau okay? do I need x not to be pw, just cont in proof?
        If $\set{\range{-\tau=t_0}{t_k=0}}$ is the subdivision of $x_{\tzero}$, we choose $T=\min\set{t_0+\tau,\frac{M}{K}}$.
        
        % FIXME: M or 2M?
        Let $K>0$ be the upper bound for $f$ from Lemma~\ref{lm:bounded-lipschitz} on the set $S=[\tzero,\tzero+T] \times \{x\in R^n: \nnorm{x}\leq 2M\}\times \{y\in R^n: \nnorm{y}\leq 2M\}$ and $L>0$ the Lipschitz constant of $f$ for that set.

        % FIXME: why continuous? its pw cont? cont in tzero
        We construct a series $(x_{(m)})_{m\in\N_0}$ of piecewise continuous functions which approximates the solution of the initial value problem.
        Set
        \begin{equation}
            x_{(0)}(t)= \begin{cases}
                x_\tzero(0) & t\in [\tzero,\tzero+T]\\
                x_\tzero(t-\tzero) & t\in [\tzero-\tau,\tzero]
            \end{cases}
        \end{equation}
        For $m\in\N_{>0}$ define
        \begin{equation}
            x_{(m)}(t)= \begin{cases}
                x_\tzero(0) + \int_\tzero^t f(s,x_{(m-1)}(s),x_{(m-1)}(s-\tau))\dx[s] & t\in [\tzero,\tzero+T]\\
                x_\tzero(t-\tzero) & t\in [\tzero-\tau,\tzero]
            \end{cases}
        \end{equation}
        % FIXME: why exists integral? f cont, x in int even cont
        Integral exists
        It holds for all $m>0$ and $t\in \compactum{\tzero-\tau}{\tzero}$ by definition of the series
        \begin{equation}
            \nnorm*{x_{(m)}(t)-x_{(m-1)}(t)}=0
        \end{equation}
        We show by induction over $m$ that for all $t\in [\tzero,\tzero+T]$ it holds
        \begin{equation}
            \nnorm*{x_{(m)}(t)-x_{(m-1)}(t)} \leq \frac{K}{L}\frac{L^m (t-\tzero)^m}{m!}.
        \end{equation}
        Since obviously $\nnorm{x_{(0)}(t)}\leq M$, the statement for $m=0$ follows from the boundedness of $f$ on $S$ and the triangle inequality for integrals:
        \begin{equation}
            \nnorm{x_{(1)}(t)-x_{(0)}(t)} = \nnorm*{\int_\tzero^t f(s,x_{(0)}(s),x_{(0)}(s-\tau))\dx[s]} \leq K(t-\tzero)
        \end{equation}
        In the inductive step we can apply
        Since for any $m>0$, it holds by the triangle inequality and by the choice of $T$
        % FIXME: why x(m)(t) smaller than 2M, such that K holds?
        % TODO: why do integral and norm commute? once integral over vectors, once over scalars
        \begin{align}\label{eq:bounded-xm}
            \nnorm*{x_{(m)}} &\leq \nnorm*{x_\tzero(0)} + \int_\tzero^t \nnorm*{f(s,x_{(m-1)}(s),x_{(m-1)}(s-\tau))}\dx[s]\\
            &\leq M + K(t-\tzero) \leq M+KT\\
            &\leq 2M
        \end{align}
        if $\nnorm{x_{(m-1)}(t)}\leq 2M$.
        It follows by the Lipschitz property of $f$
        \begin{multline*}
            \nnorm*{x_{(m+1)}(t)-x_{(m)}(t)}=\\
            = \nnorm*{\int_\tzero^t f(s,x_{(m)}(s),x_{(m)}(s-\tau)) - f(s,x_{(m-1)}(s),x_{(m-1)}(s-\tau))\dx[s]}\\
            \leq L \int_\tzero^t \nnorm*{x_{(m)}(s) - x_{(m-1)}(s)}\dx[s]\\
            \leq \frac{L^m K}{m!} \int_\tzero^t (s-\tzero)^m\dx[s]
            = \frac{L^m K}{(m+1)!}(t-\tzero)^{m+1}
        \end{multline*}
        %We use this bound and the triangle inequality in
        The Cauchy criterion for convergent series (\cite{Gathmann12GDM} 6.13, \cite{Rudin76PrinciplesAnalysis} 3.22) applied to the exponential series states that
        \begin{equation*}
            % "\ " needed for space
            \mforall{\epsilon>0}\ \mexists{n_0\in\N_0}\ \mforall{m\geq k\geq n_0}\holds \sum_{i=k+1}^m \frac{(LT)^i}{i!} <\epsilon
        \end{equation*}
        So for any $\epsilon>0$ exist $k\in\N_0$ and $m\geq k$, such that
        \begin{align*}
            \nnorm*{x_{(m)}(t)-x_{(k)}(t)} \leq{} & \nnorm*{x_{(m)}(t)-x_{(m-1)}(t)} + \nnorm*{x_{(m-1)}(t)-x_{(m-2)}(t)} + {}\\
            & + \ldots + \nnorm*{x^{(k+1)}(t)-x^{(k)}(t)}\\
            \leq{} & \frac{K}{L}\frac{L^m (t-\tzero)^m}{m!} + \frac{K}{L}\frac{L^{m-1} (t-\tzero)^{m-1}}{(m-1)!} + {}\\
            & + \ldots +\frac{K}{L}\frac{L^{k+1} (t-\tzero)^{k+1}}{(k+1)!}\\
            \leq{} & \frac{K}{L}\sum_{i=k+1}^m \frac{(LT)^i}{i!} < \varepsilon
        \end{align*}
        for all $t\in [\tzero,\tzero+T]$, i.e. $x_{(m)}$ is a Cauchy sequence

    
        % FIXME: show that this a Cauchy series
        %This is the tail of the convergent exponential series and hence it converges to zero for $k\to\infty$ (boundedness and positivity of summands, monotonicity crit).

        % FIXME: why continuous? since integral exists
        Since $x_{(m)}$ is continuous on $[\tzero,\tzero+T]$, this Cauchy
        sequence admits a limit $x$ in the Banach space $\continuouss[0]{[\tzero,\tzero+T]}{\R^n}$ in terms of the supremum-norm.

        Again, we extend $x$ to $[\tzero-\tau,\tzero]$ with $x_\tzero$, such that $x\in\Cnpw[0]{[\tzero-\tau,\tzero]}{\R^n}$.


        

        Since by the continuity of the supremum norm it follows from~\eqref{eq:bounded-xm} that
        \begin{equation*}
            \supnorm*{x}=\lim_{m\to\infty}\supnorm*{x_m}\leq 2M
        \end{equation*}
        can apply Lipschitz property of $f$
        \begin{equation*}
            \sup_{t\in\compactum{\tzero}{\tzero+T}}\nnorm*{f(s,x_m(s),x_m(s-\tau))-f(s,x(s),x(s-\tau))} \leq \sup_{t\in\compactum{\tzero}{\tzero+T}}\nnorm*{x_m(t)-x(t)}
        \end{equation*}
        Due to the uniform convergence (conv in sup-norm) of $x_{(m)}\to x$, we get the uniform convergence
        \begin{equation*}
            f(s,x_m(s),x_m(s-\tau)) \xrightarrow{m\to\infty} f(s,x(s),x(s-\tau))
        \end{equation*}
        and hence the integral and the limit process swap and by
        \begin{align*}
            x(t) = \lim_{m\to\infty} x^{(m+1)} &= x_\tzero(0) + \lim_{m\to\infty}\int_\tzero^t f(s,x^{(m)}(s),x^{(m)}(s-\tau))\dx[s]\\
            &= x_\tzero(0) + \int_\tzero^t f(s,x(s),x(s-\tau))\dx[s]
        \end{align*}
        it follows that $x$ solves the integral equation and hence, by Lemma~\ref{lm:integral-equation},
        this proves the existence of a solution to the DDE.
        % TODO: continuous because limit in Banach space, diffable and subdiv see integral equiv lemma

        % TODO: can one solution be on [\tzero, T_2] with T_2<T ?
        It remains to show uniqueness.
        Let $x$ and $\bar{x}$ be two solutions of the DDE on $[\tzero,\tzero+T]$.
        By Lemma \ref{lm:integral-equation} they are equivalent to solutions of the integral equations
        \begin{equation}
            x(t) = x_\tzero(0) + \int_\tzero^t f(s,x(s),x(s-\tau))\dx[s]
        \end{equation}
        and
        \begin{equation}
            \bar{x}(t) = x_\tzero(0) + \int_\tzero^t f(s,\bar{x}(s),\bar{x}(s-\tau))\dx[s]
        \end{equation}
        For $t\in [\tzero,T]$, we set
        \begin{align*}
            \rho(t) &:= \nnorm*{x(t)-\bar{x}(t)} \leq \int_\tzero^t \nnorm*{f(s,x(s),x(s-\tau))-f(s,\bar{x}(s),\bar{x}(s-\tau))}\dx[s]\\
            & \leq L \int_\tzero^t \nnorm*{x(s)-\bar{x}(s)}\dx[s] = L \int_\tzero^t \rho(s)\dx(s)\\
            &= L \int_\tzero^t \e{-\alpha s}\rho(s)\e{\alpha s}\dx[s] \leq L \sup_{s\in [\tzero,\tzero+T]}\left(\e{-\alpha s}\rho(s)\right)\int_\tzero^t \e{\alpha s}\dx[s]\\
            & \leq\frac{L}{\alpha}\e{\alpha t} \sup_{s\in [\tzero,\tzero+T]}\left(\e{-\alpha s}\rho(s)\right)
        \end{align*}
        with $L$ the Lipschitz constant of $f$ on the set ...
        and $\rho$ is continuous, since $x$ continuous
        Choosing $\alpha=2L$ and multiplying with $\e{-\alpha t}>0$ leads to
        \begin{equation}
            \rho(t)\e{-2Lt} \leq \frac{1}{2}\sup_{s\in [\tzero,\tzero+T]}\left(\e{-2L s}\rho(s)\right)
        \end{equation}
        for all $t\in [\tzero,\tzero+T]$
        \begin{equation}
            0 \leq \sup_{t\in [\tzero,\tzero+T]}\left(\rho(t)\e{-2Lt}\right) \leq \frac{1}{2}\sup_{s\in [\tzero,\tzero+T]}\left(\e{-2L s}\rho(s)\right)
        \end{equation}
        That is only possible if $\rho(t)=0$ for all $t\in [\tzero,\tzero+T]$, which means $x(t)=\bar{x}(t)$.

        % TODO: still needed?
        just proof existence/uniqueness on each peace of continuity proof continuity at knots with Lemma of integral equ

    \end{proof}

% TODO: on [\tzero,t_1] DDE equiv to ODE/IntEq
% -> ex unique sol on [\tzero, t_1]
% -> ex unique sol on [\tzero,\tau] (glob Lip of f on [tzero,tau])
% -> ex unique sol on [\tzero,2\tau] (continuous?, diffable?)
% show continuity and pw diffable (nth to show)
    % \begin{lemma}[cont]\label{lm:c}
    %     $x_1$ loc sol on $\compactum{\tzero-\tau}{\tzero+t_1}$ for init cond $x_{\tzero}$
    %     $x_2$ and loc sol on $\compactum{\tzero+t_1-\tau}{\t_1+T}$ for init cond $x_1$
    %     then $x_1(\tzero+t_1)=x_2(\tzero+t_1)$
    %     follows from initial cond $x_1$
    % \end{lemma}
\begin{corollary}
    \label{cor:continuability-of-solution}

    % TODO: What is derivation in randpunkten of interval [] ?
    If in Theorem \ref{theorem-solution-existence} $T=t_1-\tau$, can reapplay theorem with starting point $\tzero=\tzero_{old}+t_1-\tau$. Get existence of unique solution on $[\tzero-\tau,\tzero+S]$ with $S>T$.
\end{corollary}

\begin{corollary}
    \label{corollary}
    If f is polynomial in $t$, $x(t)$ and $x(t-\tau)$ then theorem holds

    polynomial -> continuously differentiable -> locally Lipschitz
% IDEA: can show? init cond bounded by M, and loc sol bounded by M, get glob sol since f glob Lip on set of bounded inputs?


    %TODO: put after uniqueness theorem, need uniqueness and existence so that amap well-defined
    The notion of solution for an autonomous DDE as given above can be lifted to be a trajectory $\trajectory[x]$ in the statespace
    \begin{equation}
        \trajectory[x] \from [0,T] \to \statespace[\tau],\\
        t \mapsto \xbartaut{t}
    \end{equation}

    The \textbf{state} at time $t$ is a function which provides a time limited history up to the current time. This is all information needed to determine (using the DDE) to determine the solution for time $\geq t$. It is defined as $\xbartaut{t}(s)\defeq x(t+s)$ for $s\in [-\tau,0]$. In the case of $t=0$, we simplify the notation to $\xbartau \defeq \xbartaut{0}$.
    This notion of solution is a \emph{dynamical systems} point of view which later turns out to be useful.

    

%TODO: can write DDE (eq??) from definition as

\begin{equation}
    \begin{cases}
        \D{x}=f(\xbartaut{t})\defeq g(\xbartaut{t}(0),\xbartaut{t}(-\tau)) &\text{for } t\geq 0\\
        x(t)=x_0(t) & \text{for } t\in[-\tau,0]
    \end{cases}
\end{equation}
\end{corollary}

\begin{proof}

\end{proof}

% TODO: non-autonomous -> autonomous

\section{Example}\label{example}
The basic ODE IVP
\begin{equation}
    \begin{cases}
        \D{x}(t) = -x(t)\\
        x(0) = x_0
    \end{cases}
\end{equation}
has the solution $x(t)=x_0 e^{-t}$. However the similiar DDE
\begin{equation}
    \begin{cases}
        \D{x}(t) = -x(t-\tau) & t\geq 0\\
        x(t) = x_0(t) & -\tau\leq t\leq 0
    \end{cases}
\end{equation}
has a much richer dynamics, but solution (as series) for $x_0\equiv 1$, can compute first solutions by method of steps. \ldots{}

\begin{figure}[h]\centering
    \includegraphics[width=\textwidth]{figures/piecewise-initial-function.png}
	%\caption{}
	\label{fig:not-allowed}
\end{figure}

