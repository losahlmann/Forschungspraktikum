% FIXME: use \set instead of {}
% FIXME: proper use of lforall and mforall




% fix bugs: fixes already in kernel now
%\RequirePackage{fixltx2e}

\documentclass[
    final,
    fontsize=10pt,
    a4paper,
    oneside,
    abstract=on,
    toc=bibliographynumbered
]{scrreprt}

% encoding of tex-file
\usepackage[utf8x]{inputenc}

% for propper Umlaute
\usepackage[T1]{fontenc}

% proper hyphenation, last loaded is default: english
\usepackage[frenchb,english]{babel}

% better i18n Postscript version of Knuth's cm fonts, better than cm-super
\usepackage{lmodern}
% Times
%\usepackage{times} % only text
%\usepackage{mathptmx} % text and maths

%\usepackage{textcomp}

% Mathematics
\usepackage{mathtools} % extension and fixes of/in amsmath
\usepackage{amssymb} % provides symbols, loads amsfonts
\usepackage{amsthm} % provides theorem environment
%\usepackage{nicefrac} % better slash fracs in inline

% load one single symbol from stix font
\DeclareFontEncoding{LS1}{}{}
\DeclareFontSubstitution{LS1}{stix}{m}{n}
\DeclareSymbolFont{arrows1}{LS1}{stixsf}{m}{n}
\DeclareMathDelimiter{\steps}{\mathrel}{arrows1}{"EC}{arrows1}{"EC}

% subfigures
\usepackage{subcaption}

%
\usepackage[chapter,ruled]{algorithm}
\usepackage{scrhack} % fix deprecated warning when using algorithm with koma

% For including figures, rotating or scaling text (dont use file extension)
\usepackage{graphicx}

%
\usepackage{tikz}

% rotate figures
\usepackage{rotating}

\usepackage{pdflscape}

% diplay URL
\usepackage{url}

% shows labels and references in margin, good for correction
\usepackage{showkeys}
%\usepackage{showframe}

% LS-Lab auxiliary math commands
\usepackage[Dprime]{math}

% LS-Lab logic commands: includes lcalculus, lmeta, lsemantics, lsyntax
\usepackage[
    varterms, sigmaterms, septerms,
    substopinline,
    modifopinline, % arrow notation for \imodif in semantics
    %longinterpret,
    bracketinterpret,%
    bracketmodalinterpret,
    fixformat,%
    sidenotecalculus,%
    %silentconst,%
    longseqcontext%
    ]{logic}

% LS-Lab differential dynamic logic commands
\usepackage[
    %bracketmodalinterpret,% use [[]] for semantics
    bracketinterpret,%
    bracketmodalinterpret,
    fixformat,%
    %silentconst,% don't show `const' and `algebra'
    precisenames%
    ]{dL}

%%%%%%%%%%%%%%%%%%%%%%%%%%%%%%%%%%%%%%%%%%%%%%%%%%%%%%%%%%%%%%%%%%%%%

% own symbol definitions
%
% Math Environements
%
% italic text
\newtheorem{theorem}{Theorem}[chapter]
\newtheorem{corollary}[theorem]{Corollary}
\newtheorem{proposition}[theorem]{Proposition}
\newtheorem{lemma}[theorem]{Lemma}
% normal text
\theoremstyle{definition}
\newtheorem{definition}[theorem]{Definition}
\newtheorem{example}[theorem]{Example}

%
% General Mathematics
%
% natural numbers
\newcommand{\N}{\mathbb{N}}
% real numbers
\newcommand{\R}{\mathbb{R}}
% rational numbers
\newcommand{\Q}{\mathbb{Q}}
% sets
\DeclarePairedDelimiter{\braces}{\{}{\}} % \braces* uses \left\right, but without space before
\renewcommand{\set}[1]{\braces*{#1}} % override def in malgebra.sty
% integer range
\newcommand{\range}[2]{#1,\ldots,#2}
% functions: f\from\R\to\R
\newcommand*{\from}{\colon}
% define as
% FIXME: define "def" properly
%\newcommand{\defeq}{\,\stackrel{\text{def}}{=}\,}
% defined by
\newcommand\logeq{\mathrel{\vcentcolon\Longleftrightarrow}}
% norm
%\DeclarePairedDelimiter{\abs}{\lvert}{\rvert}
\DeclarePairedDelimiter{\nnorm}{\lVert}{\rVert}
\DeclarePairedDelimiter{\supnorm}{\lVert}{\rVert_{\text{\scriptsize{sup}}}}

% exponential function
\newcommand{\e}[1]{\text{e}^{#1}}

%
% Analysis
%
% derivative: Lagrange style
%\newcommand{\D}[1]{#1'} % ' defined in latex and is same as \prime
% derivative: Leibniz style
\renewcommand{\DD}[2]{\frac{\text{d} #1}{\text{d} #2}}
% differential in integral
\newcommand{\dx}[1][x]{\text{d}#1}

\newcommand{\continuouspws}[3][]{\ensuremath{C^{#1}_\text{pw}\ifthenelse{\equal{#2}{}}{}{\ifthenelse{\equal{#3}{}}{(#2)}{(#2,#3)}}}}

%
% Delay Differential Equation
%
% definition domain of right hand side
\newcommand{\deff}{\R\times\R^n\times\R^n}
% trajectory
\newcommand{\trajectory}[1][]{\gamma_{#1}}

%
% Differential Dynamic Logic
%
% hybrid programs
\newcommand{\dHP}{\text{dHP}\xspace}%
\newcommand{\dHPs}{\text{dHPs}\xspace}%
%% redefine notation for HP
\def\lprogramsname{\dHP}
% ddL
\newcommand{\ddL}{\textsf{dd{\kern-0.1em}$\mathcal{L}$ }}
\newcommand{\ddLformulas}{\lformulasname_{\ddL}}
\newcommand{\ddLterms}{\ltermsname_{\ddL}}
\newcommand{\signature}{\Sigma}
\newcommand{\varsymbols}{V}
\newcommand{\terms}{\lterms{\signature}{\varsymbols}}
\newcommand{\FOLformulas}{\lformulas[\FOL]{\signature}{\varsymbols}}
\newcommand{\FOL}{\text{FOL}}
\newcommand{\FOLR}{\FOL$_\R$}
% FIXME: use \mathcal{M}
\newcommand{\model}{M}
\newcommand{\interpret}[1][]{\ifthenelse{\equal{#1}{}}{I}{I(#1)}}
\newcommand{\universe}{D_{\model}}
\newcommand{\assignment}{\nu}
\newcommand{\ireachability}[2]{\rho\left(#2\right)}
%
% Delay Differential Dynamic Logic
%
\renewcommand{\ivr}{\chi}
\newcommand{\csfml}{\chi}

% formula of first-order real arithmetic
\newcommand{\asfmlfolR}{\chi}

% propositions
\newcommand{\asprop}{p}
\newcommand{\bsprop}{q}

% sets of formulas
\newcommand{\asfmls}{\Gamma}
\newcommand{\bsfmls}{\Delta}
\newcommand{\csfmls}{\Theta}

% states
\newcommand{\states}{\mathcal{S}}
\newcommand{\asstate}{\nu}
\newcommand{\bsstate}{\omega}
\newcommand{\csstate}{\mu}

\newcommand{\delayinterval}[1][T]{[-#1,0]}

\newcommand{\diffvars}{\D{\allvars}}
\newcommand{\delayedvars}{\mathcal{V}_\tau}

\newcommand{\statespace}[1][T]{\continuouspws[0]{\delayinterval[#1]}{\R^n}}
%\newcommand{\xtau}[1][]{\ifthenelse{\equal{#1}{}}{x[\tau]}{x[#1]}}
\newcommand{\x}[1][]{x[#1]}
\newcommand{\xtau}[1][-\tau]{x[#1]}
\newcommand{\Dxtau}[1][-\tau]{\D{x}[#1]}
\newcommand{\holdssince}[3][s]{\lforall{#1\in\delayinterval[#2]}{\left(#3\right)}}
\DeclareMathOperator{\HHtrm}{HH_{\ltermsname}}
\DeclareMathOperator{\HHfml}{HH_{\lformulasname}}
\DeclareMathOperator{\HHprg}{HH_{\lprogramsname}}

\newcommand{\xbartau}{\bar{x}_{\tau}}
\newcommand{\xbartaut}[1]{\bar{x}_{\tau,#1}}


% classic first order logic
\newcommand{\IFOL}{\interpretation[
    algebra=\model,
    const=I,
    assign=\assignment,
    % state=\nu,
    universe=\universe
    ]}

% classic modal logic
\newcommand{\IML}{\interpretation[
    algebra=\model,
    %const=I,
    %assign=\assignment,
    state=\nu,
    worlds=W,
    access=R,
    universe=\universe
    ]}

% ddL
\newcommand{\IddL}{\vdLint[
    const=I, % interpretation
    state=\nu, % state
    assign=\past, % time parameter
    access=\rho % reachability relation
    ]}

% no parentheses around axiom names
\renewcommand*{\irrulename}[1]{\text{#1}}

%% Settings %%%%%%%%%%%%%%%%%%%%%%%%%%%%%%%%%%%%%%%%%%%%%%%%%%%%%%%%%

% serif titles
\addtokomafont{disposition}{\rmfamily}
\setkomafont{descriptionlabel}{\normalfont}

% additional title page variables
\newcommand*{\cdepartment}{}%
\newcommand*{\department}[1]{\gdef\cdepartment{#1}}
\newcommand*{\cinstitution}{}%
\newcommand*{\institution}[1]{\gdef\cinstitution{#1}}
\newcommand*{\csupervisor}{}%
\newcommand*{\supervisor}[1]{\gdef\csupervisor{#1}}
\newcommand*{\csupervisorinst}{}%
\newcommand*{\supervisorinst}[1]{\gdef\csupervisorinst{#1}}
\newcommand*{\cscndsupervisor}{}%
\newcommand*{\scndsupervisor}[1]{\gdef\cscndsupervisor{#1}}
\newcommand*{\cscndsupervisorinst}{}%
\newcommand*{\scndsupervisorinst}[1]{\gdef\cscndsupervisorinst{#1}}
\renewcommand{\maketitle}{\include{Titlepage}}

% Title for algorithm environment
\floatname{algorithm}{Model}

% tikz
\usetikzlibrary{arrows.meta, backgrounds}

\tikzset{%
    curve/.style={thick,dashed},
    deriv/.style={thick,dotted},
    term/.style={thick},
    termderiv/.style={thick,dashdotted},
    leftpoint/.style={color=black},
    rightpoint/.style={color=white, draw=black},
    leftp/.style={
        {Circle[width=4,length=4]}-,
        shorten <=-2},
    rightp/.style={
        -{Circle[width=4,length=4,fill=white]},
        shorten >=-2},
    cadlag/.style={
        {Circle[width=4,length=4]}-{Circle[width=4,length=4,fill=white]},
        shorten >=-2,
        shorten <=-2
    }
}

%%%%%%%%%%%%%%%%%%%%%%%%%%%%%%%%%%%%%%%%%%%%%%%%%%%%%%%%%%%%%%%%%%%%%

\begin{document}

\department{Master Comasic}
\institution{École Polytechnique}
\subject{Stage de Recherche\\ at\\ Carnegie Mellon University}
\title{Delay Differential Logic for Hybrid Systems with Delay}
\author{Lorenz Sahlmann}
\date{March\,--\,August 2016}
% FIXME: proper titles of professors
\supervisor{Prof.~Dr.~André Platzer}
\supervisorinst{Carnegie Mellon University}
\scndsupervisor{Prof.~Eric Goubault}
\scndsupervisorinst{École Polytechnique}

\maketitle

% \begin{abstract}
    Cyber Physical Systems incorporate the connection between the physical world and computing devices.
    This connection is often given by a computer network, which needs hence to be considered in the system model.

    In this work we extend Differential Dynamic Logic with Delay Differential Equations.

    This requires an extension of the syntax, a (partially) redefinition of the semantics and the introduction of additional axioms and proof rules.

    This results in a superset of \dL which we call \emph{Delay Differential Dynamic Logic}.

    the resulting logics extends

    provide a program notation for hybrid systems with delay

    static and dynamic semantics
    differential-forms, to reason in axiomatic way about DDEs
    modular soundness proof
    axiomatization
    proof calculus
\end{abstract}

\begin{otherlanguage}{frenchb}
    \chapter*{Déclaration d’intégrité relative au plagiat}

    Je soussigné SAHLMANN, Lorenz certifie sur l’honneur:
    \begin{enumerate}
        \item Que les résultats décrits dans ce rapport sont l’aboutissement de mon travail.
        \item Que je suis l’auteur de ce rapport.
        \item Que je n’ai pas utilisé des sources ou résultats tiers sans clairement les citer et les référencer selon les régles bibliographiques préconisées.
    \end{enumerate}

    \textbf{Mention à recopier:}
    Je déclare que ce travail ne peut être suspecté de plagiat.

    \vspace{2cm}
    Date:\hfil Signature:

\end{otherlanguage}

\cleardoublepage

% TODO: Acknowledgements
\chapter*{Acknowledgement}
    I would like to thank André Platzer for the very helpful discussions, suggestions and comments concerning this work throu
% to CMU for hosting me
% to Eric for intro me to ddes and to Andre
% to Sylvie
% to fondation ... et chair Thales for supporting me financially

% \cleardoubleoddemptypage
% \tableofcontents

\chapter{Introduction}

	dynamical systems
	mathematical model
	describing evolvement of state of as system over time
	modeling of embedded systems and cyber-physical systems
    discrete dynamical systems: difference equations, discrete state transition relations
    continuous dynamical systems: state evolves continuously, differential equation
    hybrid (dynamical) systems: combine discrete and continuous dynamics
    can capture very complex behaviour


	cyber-physical systems (CPS)
	combine computation, communication, control of physical processes/effects
	discrete and continuous dynamics
	often safety-critical, performance-critical
    safe control choices
    possible states accessible after control choice
	(prone to) software verification
    proof that always chooses safe control

    dynamical systems usually (uncountably) infinite state space
    finite number of tests cannot prove safety
    systematically obtain proofs

    classical safety, liveness, controllability, reactivity, quantified parametrized properties

	hybrid systems: mathematical model to describe CPS
	combine discrete dynamics/computation
	continuous dynamics: differential equations
    conditional switching
    nondeterminism
    repetition
	infinite state space

	examples:
    robotics, medical surgery robots
    electrical circuits
	automotive, self-driving cars (lane controllers for highway car traffic, controllers for intersections)
    speed limit control
	aviation, aircraft collision-avoidance systems (flyable roundabout maneuvers)
	railway, European train control system (ETCS)
    power plants
    chemical, biological processes
    medical models (events which can be seen as discrete with relation to continuous evolution)

	study logic of dynamical systems
    analyze and predict behaviour
    logics and proof principles

    differential dynamic logic (\dL)
    a concise overview and introduction is given in~\cite{Platzer12LogicsDynSys}
    a dynamic logic for hybrid systems
    logic/language for specify, verify safety and liveness properties
    of hybrid systems
    based on first-order modal logic and dynamic logic, first-order real arithmetic
    models of cyber-physical systems
    ordinary differential equations

    (nonlinear) real arithmetic

    transition behaviour as formulas

    differential invariants
    induction principle for differential equations

    theoretical results
    \emph{soundness} (everything provable is true)
    \emph{completeness} (everything true is provable)
    compositionality (denotational semantics, semantics (of models and formuals) functions of their parts, proofs structural decomposition, split complex systems in their parts, completeness: decomposition always successful)
    extendability (rules can be added to proof calculus)
    deductive power

	tutorial with examples modeled in \dL
	\cite{Quesel16Tutorial}

	mechanized proofs
	automatic and interactive theorem proving
	KeYmaeraX
    used successfully

	several formulations of dL
	earliest: sequent calculus \cite{Platzer10HybridSystems}, tuned for automatic proof search, KeYmaera
    automatically find (differential) invariants
	axiomatic formulation \cite{Platzer15Uniform}, implemented in KeYmaeraX

    extensions to \dL
    \emph{differential-algebraic dynamic logic} \DAL: differential-algebraic equations and constraints
    \emph{differential temporal dynamic logic} \dTL: temporal properties, trace semantics


	capture non-deterministic
	not known a priori

	\section{Related Work}
	\cite{Huang16BoundedVerificationNNDS}

% \chapter{Introduction to Logic}
\label{sec:introduction-logic}

    \section{First-Order Logic}
    \label{sec:first-order-logic}

        as defined in \cite{Platzer10HybridSystems} and \cite{Huth04LogicInCS}

        First-order logic (\FOL) defines  a syntax of logical formulas


        define inductively
        set of function and predicate symbols, called signature $\Sigma$. alphabet to built well-formed formulas from

        function: takes value of argument, gives back value, can be any type
        function symbol stand for function
        $f,g,h$

        predicate gives back either true or false, depending on values of arguments
        predicate symbol is either true or false
        $p,q,r$

        arity of function or predicate symbol: number of arguments (can be 0)
        specified by signature $\Sigma$

        set of logical variable symbols $V$, stand for objects
        $x,y,z$

        terms are well-formed/feasible arguments for functions/predicates

        \begin{definition}[Terms]
            well-formed terms: variables, functions applied to terms
            This can alternatively be written as grammar in Backus-Naur form
            \begin{equation}
                \astrm \Coloneqq x \mid c \mid f(\istrm{1},\ldots,\istrm{k})
            \end{equation}

        \end{definition}

        \begin{definition}[First-Order Formulas]


        \end{definition}

        \subsection{First-Order Logic of Real Arithmetic}
        \label{sec:first-order-logic-of-real-arithmetic}

            first-order logic of real arithmetic (\FOLR)
            formula of real arithmetic
            is first order formula
            function/predicate symbols $+,-,\cdot,/,=,<,\leq,>,\geq$
            constant symbols $\Sigma$
            logical Variables $V$


\chapter{Delay Differential Equations}\label{sec:delay-differential-equations}

\section{Piecewise Continuous Functions}
    \label{sec:piecewise-continuous-functions}
    
    The following definition is motivated by capturing the character evolution arising from hybrid systems. We will see that we can consider such to be piecewise continuous.

    % \begin{definition}[Piecewise Continuous]\label{def:piecewise-continuous}
    %     Let $D=[a,b]\subseteq\R$ be a closed interval (this includes the cases when $a=-\infty$ or $b=\infty$, or both). The mapping $x:D\rightarrow\R^n$ is called \emph{piecewise continuous} if and only if there is a finite subdivision $\{t_i:i=\range{0}{m}\}$ of $D$ (i.e.\ $a=t_0<t_1<\ldots<t_m=b$) such that $x$ is continuous on each interval piece $[t_i,t_{i+1})$ for all $i=\range{0}{m-1}$ and the left sided limits
    %     \begin{equation}
    %         \lim_{\substack{t\upto t_{i+1}\\ t\in[t_i,t_{i+1})}} x(t)
    %     \end{equation}
    %     exist. Hence $x(b)$ can be an isolated point and this right interval limit $b$ is the only spot where such is allowed.

    %     We denote by $\Cnpw[0]{D}{\R^n}$ the set of \emph{piecewise continuous functions} on the compact interval $D$ (this excludes the cases with $\pm\infty$), mapping to $\R^n$.
    % \end{definition}

    \begin{definition}[Piecewise Continuously Differentiable]\label{def:pw-cont-diff}
        % FIXME: find better word for 'subdivision'. partition?
        Let $D=[a,b]\subseteq\R$ be a closed interval (this includes the cases when $a=-\infty$ or $b=\infty$, or both). The mapping $x:D\rightarrow\R^n$ is called $n$-times \emph{piecewise continuously differentiable} if and only if there is a finite subdivision (ordered set) $\{t_i:i=\range{0}{m}\}$ of $D$ (i.e.\ $a=t_0<t_1<\ldots<t_m=b$) such that $x$ is $n$-times continuously differentiable on each interval piece $(t_i,t_{i+1})$ with continuable derivatives on $\compactum{t_i}{t_{i+1}}$.

        This means, for all $i=\range{0}{m-1}$ and for all $k=\range{0}{n}$ exist the left sided limits
        \begin{equation}
            \lim_{\substack{t\upto t_{i+1}\\ t\in(t_i,t_{i+1})}} \D[k]{x}(t)
        \end{equation}
        as well as the right sided limits
        \begin{equation}
            \lim_{\substack{t\downto t_{i}\\ t\in(t_i,t_{i+1})}} \D[k]{x}(t) =: \D[k]{x}(t_i)
        \end{equation}
        which are supposed to coincide with the value of $\D[k]{x}$ at $t_i$.
        Hence $x(b)$ can be an isolated point and this right interval limit $b$ is the only spot where such is allowed.
        In the case $n=0$, we say $x$ is \emph{piecewise continuous}.

        We denote by $\Cnpw[n]{D}{\R^n}$ the set of \emph{$n$-times piecewise continuously differentiable functions} on the compact interval $D$ (this excludes the cases with $\pm\infty$), mapping to $\R^n$, and respectively, by $\Cnpw[0]{D}{\R^n}$ the \emph{piecewise continuous functions}.
    \end{definition}

    % TODO: sup-norm for pw

    \begin{lemma}[]\label{lm:pc-integrable}
        A \emph{piecewise continuous function}, as defined in Definition~\ref{def:pw-cont-diff} is (Riemann) integrable.
    \end{lemma}
    \begin{proof}
        See standard analysis literature, such as \cite{Rudin76PrinciplesAnalysis} (Theorem~6.10) or \cite{Gathmann12GDM} (Example~11.16b).
        % ObdA: one subint with jump at end
    \end{proof}

    The following lemma generalizes the fundamental theorem of calculus to piecewise continuous derivatives.

    \begin{lemma}[]\label{lm:pc-hauptsatz}
        Let $F\in\Cn[0]{\compactum{a}{b}}{} \cap \Cnpw[1]{\compactum{a}{b}}{}$ with the subdivision $\subdivision{a=t_0}{t_m=b}$ and piecewise derivative $f$.
        %of $\compactum{a}{b}\subset\R$. 
        For all $t\in\compactum{a}{b}$ it holds
        \begin{equation*}
            F(t)-F(a) = \integral{a}{t} f(s)\dx[s]
            %\sum_{i=0}^k\int\limits_{t_i}^{t_{i+1}}f(t)\dx[t] + \int\limits_{t_k}^s f(t)\dx[t]
        \end{equation*}
        %where $t_k\leq s < t_{k+1}$.
        % FIXME: what about a=-inf or b=inf?
    \end{lemma}
    \begin{proof}
        On each interval $\compactum{t_{i-1}}{t_i}$ of the subdivision, $f$ is piecewise continuous and hence integrable.

        For all $\zeta\in\open{t_{i-1}}{t_i}$ is $F$ differentiable on $\compactum{t_{i-1}}{\zeta}$ with $\D{F}=f$.
        By the fundamental theorem of calculus (cf.\ standard analysis literature, e.g.~\cite{Gathmann12GDM,Rudin76PrinciplesAnalysis}), it follows
        \begin{equation*}
            \denseintegral{t_{i-1}}{\zeta} f(s)\dx[s] = F(\zeta)-F(t_{i-1})
        \end{equation*}
        and by the continuity of $F$ that
        \begin{equation*}
            \denseintegral{t_{i-1}}{t_i} f(s)\dx[s]
            = \lim_{\zeta\to t_i}\denseintegral{t_{i-1}}{\zeta} f(s)\dx[s]
            = \lim_{\zeta\to t_i} F(\zeta)-F(t_{i-1})
            = F(t_i)-F(t_{i-1})
        \end{equation*}
        For any $t\in\compactum{a}{b}$, there is a $k\in\set{\range{1}{m}}$ such that $t\in\closedopen{t_{k-1}}{t_k}$ (in the case $t=b$, set $k=m$), summation over $i=\range{1}{k}$ yields the telescoping series
        \begin{equation*}
            F(t)-F(a) = \sum_{i=1}^{k} \denseintegral{t_{i-1}}{t_i} f(s)\dx[s] + \integral{t_j}{t} f(s)\dx[s]
        \end{equation*}
        which is by the additivity of the integral
        \begin{equation*}
            F(t)-F(a) = \integral{a}{t} f(s)\dx[s]
        \end{equation*}
    \end{proof}

% \begin{figure*}[h]\centering
%     \begin{subfigure}[t]{0.5\textwidth}\centering
%         \includegraphics[width=\textwidth]{figures/allowed.png}
%         \caption{Admissible piecewise continuous function.}
%         \label{fig:allowed}
%     \end{subfigure}
%     \begin{subfigure}[t]{0.5\textwidth}\centering
%         \includegraphics[width=\textwidth]{figures/not-allowed.png}
% 	    \caption{Not allowed!}
% 	    \label{fig:not-allowed}
%     \end{subfigure}
%     \caption{Examples to Definition \ref{definition-piecewise-continuous}.}
% \end{figure*}


\section{Definition DDE}
    \label{sec:definition-dde}

    \begin{definition}[Delay Differential Equation]\label{def:dde}
        Let $f\from\deff\to\R^n$ and $\tau_j > 0$ for $j=\range{1}{k}$. Put $\taumax\defeq\max_j\set{\tau_j}$.

        A functional equation of the form
        \begin{equation}\label{eq:dde}
            \D{x}(t) = f(t,x(t),x(t-\tau_1),\ldots,x(t-\tau_k))
        \end{equation}
        is called \emph{delay differential equation} (DDE) with \emph{multiple constant, discrete delays $\tau_j$}.
        It is \emph{autonomous} if its right hand side $f$ is time independent and \emph{pure} if the right hand side only depends on $x(t-\tau_i)$ and not on $x(t)$.

        A DDE can be equipped with an \emph{initial condition} $x_{\tzero}$. It specifies the values of $x$ on $[\tzero-\taumax, \tzero]$ on which the right hand side depends.
        Such a pair is called \emph{initial value problem (IVP)}:
        \begin{equation}\label{eq:ivp}
            \begin{cases}
                \D{x}(t) = f(t,x(t),x(t-\tau_1),\ldots,x(t-\tau_k)) & \text{for } t\geq\tzero\\
                x(t) = x_{\tzero}(t) & \text{for } t\in [\tzero-\taumax,\tzero]
            \end{cases}
        \end{equation}
    \end{definition}

    % Since we only consider autonomous DDEs, we can without loss of generality restrict to the case of initial time $t_0=0$.

    % The definition of a DDE can be extended to multiple constant discrete delays. For simplicity, we restrict here to a single delay.


\section{Definition of Solution}
    \label{sec:definition-of-solution}

    \begin{definition}[Solution of DDE]\label{def:solution-dde}
        A function $x\from\compactum{\tzero-\taumax}{\tzero+T}\to\R^n$ is called \emph{(local) solution} of the initial value problem~\eqref{eq:ivp}, if and only if
        $x$ is continuous and piecewise continuously differentiable on $\compactum{\tzero}{\tzero+T}$ (in the sense of Def.~\ref{def:piecewise-continuous}) with subdivision $\Delta$.
        This means, when $\Delta=\subdivision{\tzero=t_0}{t_m=\tzero+T}$, $x$ is continuously differentiable on each interval $(t_i,t_{i+1})$
        %there exists a $T>0$ such that
        % FIXME: local solution is on a single subdiv int only -> cont diffable
        %$\restrict{x}{(t_i,t_{i+1})}\in \Cn[1]{\compactum{\tzero}{\tzero+T}}{\R^n}$ with
        with
        \begin{equation*}
            \D{x}(t) = f(t,x(t),x(t-\tau_1),\ldots,x(t-\tau_k))
        \end{equation*}
        for all $t\in (\tzero,\tzero+T)$ and in $t=t_i$, it holds
        % TODO: for the right-hand derivative?
        % cf ODE sol
        \begin{equation*}
            \lim_{s\downto t_i}\D{x}(s) = f(t_i,x(t_i),x(t_i-\tau_1),\ldots,x(t_i-\tau_k))
        \end{equation*}
        and $x$ obeys the initial condition:
        \begin{equation*}
            x(t) = x_{\tzero}(t) \quad\text{for } t\in [\tzero-\taumax,\tzero].
        \end{equation*}
        % FIXME: global solution should allow knicks
        If the function $x$ is a solution for all $T>0$, it is called \emph{global}.

        %TODO: differentiable in right rand point? need not derivative in right hand point
        %TODO: Fortsetzbarkeit For example initial condition has jump, this point is limit for local solution.
    \end{definition}

    

% \begin{figure}[h]\centering
%     \includegraphics[width=\textwidth]{figures/multiple.png}
% 	\caption{Illustration of proof to Lemma \ref{lemma-continuity}}
% 	\label{fig:not-allowed}
% \end{figure}

% FIXME: This lemma is wrong. Show instead integrability of f(t,x_t)
\begin{lemma}
    \label{lemma-continuity}

    % Let $x:[\tzero-\tau,\tzero+T] \rightarrow \R^n$ be piecewise continuous (as in Definition \ref{definition-piecewise-continuous}) with the subdivision $\{t_0,\ldots,t_k\}$, i.e. there are $k$ subintervals.
    %
    % Then $t \mapsto x_t = x(t+\theta)$, where $\theta\in[-\tau,0]$, is a piecewise continuous mapping from $[\tzero,\tzero+T]$ into $\statespace[\tau]$.
\end{lemma}

\begin{proof}
    % $x$ is piecewise continuous and hence uniformly piecewise continuous on the compact interval $I=[\tzero-\tau,\tzero+T]$.
    % i.e. uniformely continuous on each subinterval with stetiger Fortsetzung in right side.
    % \begin{equation}
    %     \forall\epsilon >0 \exists\delta_i >0 \forall t,s\in I_i: \quad \abs{t-s}<\delta_i \Rightarrow \nnorm{x(t)-x(s)}<\epsilon
    % \end{equation}
    % Let $\epsilon > 0$. $x|_{[t_i,t_{i+1}]}$ (with stetiger fortsetzung in right interval limit) is uniformly continuous, i.e. there is a $\delta_i > 0$ (for the given $\epsilon$), such that $\forall\,t, s \in [t_i,t_{i+1}]$ holds
    % % TODO: can use \leq ?
    % \begin{equation}
    %     \abs{t-s} < \delta_i \Rightarrow \nnorm{x(t)-x(s)} < \epsilon
    % \end{equation}
    %
    % Among the given $\delta_i$, choose the smallest as $\delta = \min_i \delta_i$.
    %
    % For any $i$ and $s,t\in [t_i,t_{i+1})\subset [\tzero,\tzero+T]$ with $\abs{t-s}<\delta$, it holds
    % \begin{equation}
    %     \supnorm{x_t - x_s} = \sup_{\theta\in [-\tau,0]}\nnorm{x(t+\theta) - x(s+\theta)} < \epsilon
    % \end{equation}
    % since $t+\theta, s+\theta \in I$
    % Hence $t \mapsto x_t$ is uniformely continuous on $[t_i,t_{i+1})$.
\end{proof}


\section{Method of Steps}
    \label{sec:method-of-steps}
    
    for $t\in [0,\tau]$, $x$ must satisfy the following ordinary initial value problem obtained by plugging the initial function into equation (??). For suitable $f$ and $x_0$, the existence (and uniqueness) of a solution on $[0,\tau]$ is guaranteed by ODE theory (\ldots{} or Picard-Lindelöf theorems).

    This procedure can then be applied repeatedly to extend the obtained solution by steps of length $\tau$.





\section{Existence and Uniqueness of Solutions}
    \label{solutions-existence-uniqueness}

    will consider rhs cont and lip
    $f$ Lipschitz with piecewise continuous initial function have existence and uniqueness ???? smoothing

    \begin{definition}[Lipschitz Continuity]\label{def:lipschitz}
        % similar to \cite{pruesswilke10GewDiffGl,Smith10IntroDDE}
        % FIXME: dont I need xtau in right side ??? 
        A function $f\from\deff\to\R^n$ is called \emph{(locally) Lipschitz continuous} in its second argument if and only if for all $a,b\in\R$ and $M>0$ there is a $L>0$ such that
        \begin{equation*}
            % TODO: is L(\nnorm*{x-x}+\nnorm*{y-y}) better? is equiv, with different L
            % FIXME: or just say Lipschitz continuous with respect to two other arguments, once for x once for y -> compare proof
            \nnorm*{f(t,x,y) - f(t,\bar{x},y)} \leq L\max\left\{\nnorm*{x - \bar{x}},\nnorm*{y - \bar{y}}\right\}
        \end{equation*}
        for all $t\in [a,b]$ and $x,\bar{x},y\in\R^n$ with $\nnorm{x},\nnorm{\bar{x}},\nnorm{y}\leq M$.
    \end{definition}

    \begin{lemma}\label{lm:bounded-lipschitz}
        Let $f\from\deff\to\R^n$ be continuous and Lipschitz continuous in its second argument.

        For any given compact interval $\compactum{a}{b}$ and $M>0$ there exists a bound $K>0$ such that
        \begin{equation}
            \nnorm{f(t,x,y)}\leq K
        \end{equation}
        for all $t\in\compactum{a}{b}$ and $x,y\in\R^n$ with $\nnorm{x},\nnorm{y}\leq M$.
    \end{lemma}
    \begin{proof}
        Let $L$ be the Lipschitz constant of $f$ for the given $\compactum{a}{b}$ and $M$. Then
        \begin{multline*}
            \nnorm{f(t,x,y)} \leq \nnorm{f(t,x,y) - f(t,0,y)} + \nnorm{f(t,0,y)}\\
            \leq L\nnorm{x-0} + \nnorm{f(t,0,y)} \leq LM+P = K
        \end{multline*}
        for $t\in\compactum{a}{b}$ and $x,y\in\R^n$ with $\nnorm{x},\nnorm{y}\leq M$. We used the continuity of $f$ on the compact set $S=\compactum{a}{b}\times\set{z\in\R^n\with\nnorm{z}\leq M}$ for the existence of
        \begin{equation*}
            P = \max_{(s,z)\in S}\nnorm{f(s,0,z)}
        \end{equation*}
    \end{proof}

    \begin{lemma}\label{lm:integral-equation}
        %TODO: compare with ODE lecture notes
        Finding a solution of the initial value problem~\eqref{eq:ivp} is equivalent to solving the integral equation
        \begin{equation}\label{eq:integral-equation}
            \begin{cases}
                x(t) = x_{\tzero}(\tzero) + \int_{\tzero}^t f(s,x(s),x(s-\tau))\dx[s] & \text{for } t\geq\tzero\\
                x(t) = x_{\tzero}(t) & \text{for } t\in [\tzero-\tau,\tzero]
            \end{cases}
        \end{equation}
        where ... (same as for ivp)
        and is continuous in t.
        integral componentwise, f vector valued
    \end{lemma}
    \begin{proof}
        Let $x$ be a solution of the IVP. Thus $x$ is (by definition) continuous on $\compactum{\tzero}{\tzero+T}$ and piecewise continuous on $\compactum{\tzero-\tau}{\tzero}$. This means that the chain $f(t,x(t),x(t-\tau))$ is piecewise continuous and hence integrable on $\compactum{\tzero}{\tzero+T}$. Furthermore, $x$ is (by definition) piecewise differentiable.
        The fundamental theorem of calculus states
        \begin{equation*}
            x(t) = x_{\tzero}(\tzero) + \int_{\tzero}^t f(s,x(s),x(s-\tau))\dx[s]
        \end{equation*}
        for $t\geq\tzero$.

        Conversely, let $x\from [\tzero-\tau,\tzero+T]$ be a solution of the integral equation~\eqref{eq:integral-equation}, i.e. $x(t)=x_{\tzero}(t)$ for all $t\in\compactum{\tzero-\tau}{\tzero}$ and $x(t) = x_{\tzero}(\tzero) + \int_{\tzero}^t f(s,x(s),x(s-\tau))\dx[s] =: F(t)$ for $t\in\compactum{\tzero}{\tzero+T}$ for a $T>0$.
        HS -> continuous
        % FIXME: f cont uberall in Voraussetzung?
        Since $f$ is by the precondition continuous, x cont -> integrand cont iff xtau cont
        Let $\set{t_0-\tau < \ldots < t_m-\tau}$ be the subdivision of the initial condition $\x_{\tzero}\in\statespace$.
        Since $f$ is integrable on and continuous on ..., the fundamental theorem of calculus states the differntiability of F on $(t_i,t_{i+1})$ with
        \begin{equation}
            x'=F'=f()
        \end{equation}
        % FIXME: t_i is jumppoint of init cond, can be x_sigma or x
        show $\D{F}$ is continuable in jump point $t_i$
        $\lim_{t\downto t_i}\D{F}(t)=\lim_{t\downto t_i}f(t,x(t),x(t-\tau))=f(t_i,x(t_i),x(t_i-\tau))$ by continuity of $f$, $x|(t_i,t_{i+1})$ and def of pw cont $x|(t_i-\tau,t_{i+1}-\tau)$
        same way: limit exists for $t\upto t_{i+1}$



        integrate from discontinuity of $\xbartaut{t}$ to discontinuity and proof stetige fortsetzbarkeit at these points
    \end{proof}

    \begin{theorem}[Existence of unique solution]\label{thm:solution-existence}
        Consider the Delay Differential Equation
    %TODO: do we need global existence or just local?
        \begin{equation}
            \begin{cases}
                \D{x} = f(t,x(t),x(t-\tau)) & \text{for } t\geq\tzero\\
                x(t) = x_\tzero(t-\tzero)   & \text{for } t\in [\tzero-\tau,\tzero]
            \end{cases}
        \end{equation}
        with $f\from\deff\to\R^n$ continuous and satisfying the (local) Lipschitz condition in its second argument (Def.~\ref{def:lipschitz}).

        % where $\nnorm{\cdot}$ denotes the Euclidian norm on $\R^n$ and $\supnorm{\cdot}$ the supremum norm of the Banach space of continuous functions on $[-\tau,0]$.

        Then for each \emph{initial condition} $x_{\tzero}\in\statespace[\tau]$ and start time $\tzero$, there \textbf{exists} a \textbf{unique local solution} of the IVP on a time interval $[\tzero-\tau, \tzero+T]$. The duration $T>0$ depends on the sup-norm and discontinuity points of the initial condition. (?)
        This solution is continuous and piecewise differentiable on $\compactum{\tzero}{\tzero+T}$ with subdivision $t_i+\tau$.
    \end{theorem}

    The proof is smiliar to the proof of the existence theorem (Theorem 3.7) given in~\cite{Smith10IntroDDE}.
    \begin{proof}
        % FIXME: where sup-norm?
        As a piecewise continuous function, the initial condition can bounded by $M\geq \supnorm{x_\tzero}$ on $\delayinterval[\tau]$.
        
        % FIXME: do I need t_0+tau or is just \tau okay? do I need x not to be pw, just cont in proof?
        If $\set{\range{-\tau=t_0}{t_k=0}}$ is the subdivision of $x_{\tzero}$, we choose $T=\min\set{t_0+\tau,\frac{M}{K}}$.
        
        % FIXME: M or 2M?
        Let $K>0$ be the upper bound for $f$ from Lemma~\ref{lm:bounded-lipschitz} on the set $S=[\tzero,\tzero+T] \times \{x\in R^n: \nnorm{x}\leq 2M\}\times \{y\in R^n: \nnorm{y}\leq 2M\}$ and $L>0$ the Lipschitz constant of $f$ for that set.

        % FIXME: why continuous? its pw cont? cont in tzero
        We construct a series $(x_{(m)})_{m\in\N_0}$ of piecewise continuous functions which approximates the solution of the initial value problem.
        Set
        \begin{equation}
            x_{(0)}(t)= \begin{cases}
                x_\tzero(0) & t\in [\tzero,\tzero+T]\\
                x_\tzero(t-\tzero) & t\in [\tzero-\tau,\tzero]
            \end{cases}
        \end{equation}
        For $m\in\N_{>0}$ define
        \begin{equation}
            x_{(m)}(t)= \begin{cases}
                x_\tzero(0) + \int_\tzero^t f(s,x_{(m-1)}(s),x_{(m-1)}(s-\tau))\dx[s] & t\in [\tzero,\tzero+T]\\
                x_\tzero(t-\tzero) & t\in [\tzero-\tau,\tzero]
            \end{cases}
        \end{equation}
        % FIXME: why exists integral? f cont, x in int even cont
        Integral exists
        It holds for all $m>0$ and $t\in \compactum{\tzero-\tau}{\tzero}$ by definition of the series
        \begin{equation}
            \nnorm*{x_{(m)}(t)-x_{(m-1)}(t)}=0
        \end{equation}
        We show by induction over $m$ that for all $t\in [\tzero,\tzero+T]$ it holds
        \begin{equation}
            \nnorm*{x_{(m)}(t)-x_{(m-1)}(t)} \leq \frac{K}{L}\frac{L^m (t-\tzero)^m}{m!}.
        \end{equation}
        Since obviously $\nnorm{x_{(0)}(t)}\leq M$, the statement for $m=0$ follows from the boundedness of $f$ on $S$ and the triangle inequality for integrals:
        \begin{equation}
            \nnorm{x_{(1)}(t)-x_{(0)}(t)} = \nnorm*{\int_\tzero^t f(s,x_{(0)}(s),x_{(0)}(s-\tau))\dx[s]} \leq K(t-\tzero)
        \end{equation}
        In the inductive step we can apply
        Since for any $m>0$, it holds by the triangle inequality and by the choice of $T$
        % FIXME: why x(m)(t) smaller than 2M, such that K holds?
        % TODO: why do integral and norm commute? once integral over vectors, once over scalars
        \begin{align}\label{eq:bounded-xm}
            \nnorm*{x_{(m)}} &\leq \nnorm*{x_\tzero(0)} + \int_\tzero^t \nnorm*{f(s,x_{(m-1)}(s),x_{(m-1)}(s-\tau))}\dx[s]\\
            &\leq M + K(t-\tzero) \leq M+KT\\
            &\leq 2M
        \end{align}
        if $\nnorm{x_{(m-1)}(t)}\leq 2M$.
        It follows by the Lipschitz property of $f$
        \begin{multline*}
            \nnorm*{x_{(m+1)}(t)-x_{(m)}(t)}=\\
            = \nnorm*{\int_\tzero^t f(s,x_{(m)}(s),x_{(m)}(s-\tau)) - f(s,x_{(m-1)}(s),x_{(m-1)}(s-\tau))\dx[s]}\\
            \leq L \int_\tzero^t \nnorm*{x_{(m)}(s) - x_{(m-1)}(s)}\dx[s]\\
            \leq \frac{L^m K}{m!} \int_\tzero^t (s-\tzero)^m\dx[s]
            = \frac{L^m K}{(m+1)!}(t-\tzero)^{m+1}
        \end{multline*}
        %We use this bound and the triangle inequality in
        The Cauchy criterion for convergent series (\cite{Gathmann12GDM} 6.13, \cite{Rudin76PrinciplesAnalysis} 3.22) applied to the exponential series states that
        \begin{equation*}
            % "\ " needed for space
            \mforall{\epsilon>0}\ \mexists{n_0\in\N_0}\ \mforall{m\geq k\geq n_0}\holds \sum_{i=k+1}^m \frac{(LT)^i}{i!} <\epsilon
        \end{equation*}
        So for any $\epsilon>0$ exist $k\in\N_0$ and $m\geq k$, such that
        \begin{align*}
            \nnorm*{x_{(m)}(t)-x_{(k)}(t)} \leq{} & \nnorm*{x_{(m)}(t)-x_{(m-1)}(t)} + \nnorm*{x_{(m-1)}(t)-x_{(m-2)}(t)} + {}\\
            & + \ldots + \nnorm*{x^{(k+1)}(t)-x^{(k)}(t)}\\
            \leq{} & \frac{K}{L}\frac{L^m (t-\tzero)^m}{m!} + \frac{K}{L}\frac{L^{m-1} (t-\tzero)^{m-1}}{(m-1)!} + {}\\
            & + \ldots +\frac{K}{L}\frac{L^{k+1} (t-\tzero)^{k+1}}{(k+1)!}\\
            \leq{} & \frac{K}{L}\sum_{i=k+1}^m \frac{(LT)^i}{i!} < \varepsilon
        \end{align*}
        for all $t\in [\tzero,\tzero+T]$, i.e. $x_{(m)}$ is a Cauchy sequence

    
        % FIXME: show that this a Cauchy series
        %This is the tail of the convergent exponential series and hence it converges to zero for $k\to\infty$ (boundedness and positivity of summands, monotonicity crit).

        % FIXME: why continuous? since integral exists
        Since $x_{(m)}$ is continuous on $[\tzero,\tzero+T]$, this Cauchy
        sequence admits a limit $x$ in the Banach space $\continuouss[0]{[\tzero,\tzero+T]}{\R^n}$ in terms of the supremum-norm.

        Again, we extend $x$ to $[\tzero-\tau,\tzero]$ with $x_\tzero$, such that $x\in\Cnpw[0]{[\tzero-\tau,\tzero]}{\R^n}$.


        

        Since by the continuity of the supremum norm it follows from~\eqref{eq:bounded-xm} that
        \begin{equation*}
            \supnorm*{x}=\lim_{m\to\infty}\supnorm*{x_m}\leq 2M
        \end{equation*}
        can apply Lipschitz property of $f$
        \begin{equation*}
            \sup_{t\in\compactum{\tzero}{\tzero+T}}\nnorm*{f(s,x_m(s),x_m(s-\tau))-f(s,x(s),x(s-\tau))} \leq \sup_{t\in\compactum{\tzero}{\tzero+T}}\nnorm*{x_m(t)-x(t)}
        \end{equation*}
        Due to the uniform convergence (conv in sup-norm) of $x_{(m)}\to x$, we get the uniform convergence
        \begin{equation*}
            f(s,x_m(s),x_m(s-\tau)) \xrightarrow{m\to\infty} f(s,x(s),x(s-\tau))
        \end{equation*}
        and hence the integral and the limit process swap and by
        \begin{align*}
            x(t) = \lim_{m\to\infty} x^{(m+1)} &= x_\tzero(0) + \lim_{m\to\infty}\int_\tzero^t f(s,x^{(m)}(s),x^{(m)}(s-\tau))\dx[s]\\
            &= x_\tzero(0) + \int_\tzero^t f(s,x(s),x(s-\tau))\dx[s]
        \end{align*}
        it follows that $x$ solves the integral equation and hence, by Lemma~\ref{lm:integral-equation},
        this proves the existence of a solution to the DDE.
        % TODO: continuous because limit in Banach space, diffable and subdiv see integral equiv lemma

        % TODO: can one solution be on [\tzero, T_2] with T_2<T ?
        It remains to show uniqueness.
        Let $x$ and $\bar{x}$ be two solutions of the DDE on $[\tzero,\tzero+T]$.
        By Lemma \ref{lm:integral-equation} they are equivalent to solutions of the integral equations
        \begin{equation}
            x(t) = x_\tzero(0) + \int_\tzero^t f(s,x(s),x(s-\tau))\dx[s]
        \end{equation}
        and
        \begin{equation}
            \bar{x}(t) = x_\tzero(0) + \int_\tzero^t f(s,\bar{x}(s),\bar{x}(s-\tau))\dx[s]
        \end{equation}
        For $t\in [\tzero,T]$, we set
        \begin{align*}
            \rho(t) &:= \nnorm*{x(t)-\bar{x}(t)} \leq \int_\tzero^t \nnorm*{f(s,x(s),x(s-\tau))-f(s,\bar{x}(s),\bar{x}(s-\tau))}\dx[s]\\
            & \leq L \int_\tzero^t \nnorm*{x(s)-\bar{x}(s)}\dx[s] = L \int_\tzero^t \rho(s)\dx(s)\\
            &= L \int_\tzero^t \e{-\alpha s}\rho(s)\e{\alpha s}\dx[s] \leq L \sup_{s\in [\tzero,\tzero+T]}\left(\e{-\alpha s}\rho(s)\right)\int_\tzero^t \e{\alpha s}\dx[s]\\
            & \leq\frac{L}{\alpha}\e{\alpha t} \sup_{s\in [\tzero,\tzero+T]}\left(\e{-\alpha s}\rho(s)\right)
        \end{align*}
        with $L$ the Lipschitz constant of $f$ on the set ...
        and $\rho$ is continuous, since $x$ continuous
        Choosing $\alpha=2L$ and multiplying with $\e{-\alpha t}>0$ leads to
        \begin{equation}
            \rho(t)\e{-2Lt} \leq \frac{1}{2}\sup_{s\in [\tzero,\tzero+T]}\left(\e{-2L s}\rho(s)\right)
        \end{equation}
        for all $t\in [\tzero,\tzero+T]$
        \begin{equation}
            0 \leq \sup_{t\in [\tzero,\tzero+T]}\left(\rho(t)\e{-2Lt}\right) \leq \frac{1}{2}\sup_{s\in [\tzero,\tzero+T]}\left(\e{-2L s}\rho(s)\right)
        \end{equation}
        That is only possible if $\rho(t)=0$ for all $t\in [\tzero,\tzero+T]$, which means $x(t)=\bar{x}(t)$.

        % TODO: still needed?
        just proof existence/uniqueness on each peace of continuity proof continuity at knots with Lemma of integral equ

    \end{proof}

% TODO: on [\tzero,t_1] DDE equiv to ODE/IntEq
% -> ex unique sol on [\tzero, t_1]
% -> ex unique sol on [\tzero,\tau] (glob Lip of f on [tzero,tau])
% -> ex unique sol on [\tzero,2\tau] (continuous?, diffable?)
% show continuity and pw diffable (nth to show)
    % \begin{lemma}[cont]\label{lm:c}
    %     $x_1$ loc sol on $\compactum{\tzero-\tau}{\tzero+t_1}$ for init cond $x_{\tzero}$
    %     $x_2$ and loc sol on $\compactum{\tzero+t_1-\tau}{\t_1+T}$ for init cond $x_1$
    %     then $x_1(\tzero+t_1)=x_2(\tzero+t_1)$
    %     follows from initial cond $x_1$
    % \end{lemma}
\begin{corollary}
    \label{cor:continuability-of-solution}

    % TODO: What is derivation in randpunkten of interval [] ?
    If in Theorem \ref{theorem-solution-existence} $T=t_1-\tau$, can reapplay theorem with starting point $\tzero=\tzero_{old}+t_1-\tau$. Get existence of unique solution on $[\tzero-\tau,\tzero+S]$ with $S>T$.
\end{corollary}

\begin{corollary}
    \label{corollary}
    If f is polynomial in $t$, $x(t)$ and $x(t-\tau)$ then theorem holds

    polynomial -> continuously differentiable -> locally Lipschitz
% IDEA: can show? init cond bounded by M, and loc sol bounded by M, get glob sol since f glob Lip on set of bounded inputs?


    %TODO: put after uniqueness theorem, need uniqueness and existence so that amap well-defined
    The notion of solution for an autonomous DDE as given above can be lifted to be a trajectory $\trajectory[x]$ in the statespace
    \begin{equation}
        \trajectory[x] \from [0,T] \to \statespace[\tau],\\
        t \mapsto \xbartaut{t}
    \end{equation}

    The \textbf{state} at time $t$ is a function which provides a time limited history up to the current time. This is all information needed to determine (using the DDE) to determine the solution for time $\geq t$. It is defined as $\xbartaut{t}(s)\defeq x(t+s)$ for $s\in [-\tau,0]$. In the case of $t=0$, we simplify the notation to $\xbartau \defeq \xbartaut{0}$.
    This notion of solution is a \emph{dynamical systems} point of view which later turns out to be useful.

    

%TODO: can write DDE (eq??) from definition as

\begin{equation}
    \begin{cases}
        \D{x}=f(\xbartaut{t})\defeq g(\xbartaut{t}(0),\xbartaut{t}(-\tau)) &\text{for } t\geq 0\\
        x(t)=x_0(t) & \text{for } t\in[-\tau,0]
    \end{cases}
\end{equation}
\end{corollary}

\begin{proof}

\end{proof}

% TODO: non-autonomous -> autonomous

\section{Example}\label{example}
The basic ODE IVP
\begin{equation}
    \begin{cases}
        \D{x}(t) = -x(t)\\
        x(0) = x_0
    \end{cases}
\end{equation}
has the solution $x(t)=x_0 e^{-t}$. However the similiar DDE
\begin{equation}
    \begin{cases}
        \D{x}(t) = -x(t-\tau) & t\geq 0\\
        x(t) = x_0(t) & -\tau\leq t\leq 0
    \end{cases}
\end{equation}
has a much richer dynamics, but solution (as series) for $x_0\equiv 1$, can compute first solutions by method of steps. \ldots{}

\begin{figure}[h]\centering
    \includegraphics[width=\textwidth]{figures/piecewise-initial-function.png}
	%\caption{}
	\label{fig:not-allowed}
\end{figure}



\chapter{Delay Differential Dynamic Logic}
\label{ch:delay-differential-dynamic-logic}

We extent classical differential dynamic logic (\dL) (see e.g.~\cite{Platzer12LogicsDynSys}) with syntax, semantics, axiomatization and proof rules to support reasoning about hybrid dynamical systems with delay.

To that purpose, we allow delay differential equations in hybrid programs, which are then called \emph{delay hybrid program} (\dHP).

The definition of \emph{delay differential dynamic logic} (\ddL) provides all operators of first-order logic, as well as modal operators, in order to specify and verify reachability properties about the state of such \dHPs.

In the language of \ddL, we can not only model hybrid programs with DDEs in the continuous part, but also with temporal differences in the discrete fragment.
For example a controller which approximates a derivative by a difference quotient.

The logic \ddL is a superset of \dL, i.e.\ in the absence of any delay, it reduces to classical \emph{differential dynamic logic}.


% Similiar to first-order logic and essentially as an extension of dL we define delay differential logic
% in general an arbitrary unknown initial function, only conditions on it

% time-invariant

% safety and liveness properties specify

% TODO: Abgrenzung zu Trace Semantics
    % The temporal character of delay differential equations (they depend on their own temporal evolution with limited horizon) suggests the introduction of trace semantics.
    %
    % However, we go the way of introducing transition semantics with an augmented state space.

\section{Syntax}
    \label{sec:syntax}

    Terms and formulas in \ddL, as well as \dHPs are defined as \emph{words} of finite length, produced by their corresponding grammars in Backus-Naur-form (BNF).
    % The syntax of is defined inductively by 

    We define by $\allvars$ be the set of \emph{all variables} and by $\diffvars\defeq\Set{\D{x} \with x\in\allvars}$ the corresponding set of \emph{differential symbols}.
    % TODO: call delay or delayed?
    Let $\constants\subset\nonposQ$ be the set of \emph{constant parameters}.
    All three sets are supposed to be finite.
    We denote $\delayvars\defeq\Set{\x[c] \with x\in\allvars,\ c\in\constants}$ as the set of \emph{delay variables} and $\delaydiffvars\defeq\Set{\Dx[c] \with \D{x}\in\diffvars,\ c\in\constants}$ as the set of \emph{delay differentials}.
    % TODO: finite sets, or sets infinite, but only finitely many symbols in one term/formula, Trm()
    % FIXME: parentheses

    We will usually write variables as $x,y,z\in\allvars$ and their differential symbols as $\D{x},\D{y},\D{z}\in\diffvars$.
    % FIXME: just use \Q as set of constants, they have their semantics
    \emph{Function symbols} $f,g,h$ and \emph{constant symbols} $a,b\in\Q$
    % FIXME: do I use predicates?
    % \emph{predicate symbols} $p,q,r$
    are as in first-order logic (cf.\ Section~\ref{sec:first-order-logic}).

    Moreover, we write $\astrm,\bstrm$ for \ddL terms, $\asfml,\bsfml$ for \ddL formulas and $\asprg,\bsprg$ for \dHPs. For formulas of first-order logic of real arithmetic (\FOLR), we use the symbols $\asfmlfolR$ and $\bsfmlfolR$.

    \begin{definition}[s-Terms]\label{def:syntax-terms}
        The syntax of \emph{terms} of \emph{delay differential dynamic logic} is defined by the following grammar:
        \begin{align*}
            \astrm(s),\bstrm(s) \Coloneqq{}&
                \x[s] \mid
                \Dx[s] \mid
                \x[c] \mid
                \Dx[c] \mid
                a \mid\\
                & f(\range{\istrm{1}(s)}{\istrm{k}(s)}) \mid
                \astrm(s) + \bstrm(s) \mid
                \astrm(s) \cdot \bstrm(s) \mid
                \D{(\astrm(s))}
        \end{align*}
        % FIXME: is term' allowed? diff inv is only a FOLR formula? even for dL valid?
        % TODO: line breaks and explination text in syntax
        % \begin{align}
        %     \astrm\Coloneqq{}&\\
        %     &\mid \astrm+\bstrm\hfill\text{test}
        % \end{align}
        where $x\in\allvars,\D{x}\in\diffvars$ and $f$ is a function symbol of arity $k$.
        The symbol $a\in\constants$ stands for a constant value in $\Q$. The constant parameters $c\in\nonposQ$ are not allowed to be positive.
    \end{definition}

    The s-terms listed in the first line are called \emph{atomic}, as opposed to the \emph{composite} s-terms in the second line.
    % FIXME: what is f like?
    %extending \dL with symbols for \emph{delayed variables} and \emph{differentials}
    S-terms generally depend on the time parameter $s\in\nonposR$. This is why we write them as $\astrm(s)$.
    If a s-term $\astrm(s)$ does neither contain $\x[s]$ nor $\Dx[s]$, we say $s\notin\astrm(s)$ and abbreviate its notation to $\astrm$.
    Writing $\astrm(b)$ means that all occurences of $s$ in $\astrm(s)$ have been replaced with $b\in\nonposQ$.
    Moreover, we agree on abbreviating $\x[0]$ to $x$ and $\Dx[0]$ to $\D{x}$.
    % The variable symbol $x$ and the differential symbol $\D{x}$ are the only ones which allow explicitely mentioning this parameter. They are spelled slightly differently using square brackets as $\x[s]$ and $\Dx[s]$, respectively.
    Note that $s\notin\allvars\cup\diffvars\cup\constants$. It is a special variable symbol.

    The \emph{differential} $\D{(\astrm(s))}$ of a term $\astrm(s)$ is its syntactic (total) derivation, obtained by standard differentiation rules.
    Lemma~\ref{lm:derivations} shows the validity of these rules and that the result is again a s-term.

    Subtraction can be defined using addition and multiplication, division would also be possible, if we can exclude any division by zero. The grammar allows in particular the construction of polynomial forms.

    \begin{example}
        Let us consider the s-term
        \begin{equation*}
            \astrm(s) = \x[s] + \x[-\tau].
        \end{equation*}
        Setting $s=-1$ gives the term
        \begin{equation*}
            \astrm(-1) = \x[-1] + \x[-\tau].
        \end{equation*}
    \end{example}

    Delay differential dynamic logic uses hybrid programs with delay differential equations as system model. 
    The grammar defining these \emph{delayed hybrid programs} is the same as for classical \HPs (cf.~\cite{Platzer15Uniform}).

    \begin{definition}[Delay Hybrid Programs]\label{def:syntax-HP}
        The syntax of \emph{delay hybrid programs} (\dHPs) is defined by
        % TODO: first-order logic of real arithmetic formulas include \x?
        \begin{equation*}
            \asprg,\bsprg \Coloneqq
                \hupdate{\humod{x}{\astrm}} \mid
                \Dupdate{\Dumod{\D{x}}{\astrm}} \mid
                % TODO: only FOLR
                \htest{\asfml(s)} \mid
                \hchoice{\asprg}{\bsprg} \mid
                % TODO: replace ; in HPs
                \asprg;\bsprg \mid
                \hrepeat{\asprg} \mid
                \hevolvein{\D{x}=\astrm}{\ivr}
        \end{equation*}
        where $\asprg,\bsprg$ denote \dHPs, $x$ a variable and $\astrm$ a term (possibly containing $x$ or $\x[b]$, but no $\x[s]$).
        The formula $\asfmlfolR$ is of \FOLR, containing only normal variable symbols from $\allvars$.
    \end{definition}
    
    Note that the syntax only allows autonomous DDEs, though with multiple constant delays.

    Atomic \dHPs are given by instantaneous discrete \emph{assignments} $\hupdate{\humod{x}{\astrm(0)}}$ and \emph{differential assignments} $\Dupdate{\Dumod{\D{x}}{\astrm(0)}}$, which change the value of the given variable only at the current time instant, not the past, \emph{tests} $\htest{\asfml(s)}$, which pass only if the current state satisfies the formula $\asfml$
    % first-order formula $\asfmlfolR$ of real arithmetic
    and abords the program execution if not, as well as evolutions along \emph{delay differential equation} systems $\hevolvein{\D{x}=\astrm(0)}{\ivr}$ of an arbitrary amount of time, but restricted by the evolution domain constraint $\ivr$.

    Compound \dHPs combine atomic programs, and comprise \emph{nondeterministic choices} $\hchoice{\asprg}{\bsprg}$, running either $\asprg$ or $\bsprg$, \emph{sequential compositions} $\asprg;\bsprg$, executing $\bsprg$ after $\asprg$ and \emph{nondeterministic repetitions} $\hrepeat{\asprg}$, repeating $\asprg$ any number of times, zero times included.

% Using this, in comparison to \dL extended syntax, we can write down both delay differential equations and ordinary differential equations in the form $\D{x}=\theta$, where $\theta=f(x,\x[-\tau])$ with a polynomial $f$.
    Observe that ODEs are still expressible by this syntax and that hybrid programs are hence only delayed hybrid programs with zero delay.

    % FIXME: where put general dL sources? only once at beginning
    The difference between classical \HPs (as defined in \dL, cf.~\cite{Platzer10HybridSystems,Platzer12LogicsDynSys,Platzer15Uniform}) and \emph{delay hybrid programs} is not syntactical, but only given by their semantics.
    
    \begin{definition}[s-Formulas]\label{def:syntax-formula}
        The syntax for \emph{formulae} of \emph{delay differential dynamic logic} is defined by the grammar
        \begin{align*}
            \asfml(s),\bsfml(s) \Coloneqq{}& % fixes missing space
                % TODO: better abbreviated notation for \holdssince, make forall only appear in semantics rhs
                \astrm(s) = \bstrm(s) \mid
                \astrm(s)\geq\bstrm(s) \mid
                p(\range{\istrm{1}(s)}{\istrm{k}(s)}) \mid
                \hs{\asfml(s)} \mid\\
                &\lnot\asfml(s) \mid
                \asfml(s)\land\bsfml(s) \mid
                \lforall{x}{\asfml(s)} \mid
                \lexists{x}{\asfml(s)} \mid
                \dbox{\asprg}{\asfml(s)} \mid
                \ddiamond{\asprg}{\asfml(s)}
        \end{align*}
        with $\astrm(s),\bstrm(s),\range{\istrm{1}(s)}{\istrm{k}(s)}$ as s-terms, $p$ as predicate symbol, $x$ as variable, and $\asprg$ as \dHP.
    \end{definition}

    These formulae combine connectives of propositional logic with first-order quantifiers (which both have standard meaning) and two modalities, describing \emph{necessary} and \emph{possible} properties.

    The other comparison operators $<,\leq,>$ and logic connectives $\lor,\limply,\lbisubjunct$ can be defined using $=,\land,\lnot$ and are hence not explicitely mentioned in the grammar.
    Analogoulsy is $\lexists{x}{\asfml}$ expressible as $\lnot\lforall{x}{\lnot\asfml}$ and the modal formula $\dbox{\asprg}{\asfml}$ ($\asfml$ holds in the state after all runs of $\asprg$) by its dual $\ddiamond{\asprg}{\asfml}\equiv\lnot\dbox{\asprg}{\lnot\asfml}$ (there is at least one state reachable by $\asprg$ such that $\asfml$ holds).
    The quantifiers $\forall$ and $\exists$ quantify over the state space $\statespace$.
    % FIXME: was state space mentioned before?

    Like the s-terms defined above, the s-formulae depend on a time parameter $s\in\closeddelayinterval$. The symbol $T$ is a symbolic constant related to the length of the domain of the state space, which is induced by the occurence of delay symbols. Its value is defined by the static semantics and set by proof rules.
    % TODO: proof rule/axiom to set T?
    The only way to bind the variable $s$ in a formula $\asfml(s)$ is by using $\hs{\asfml(s)}$, which quantifies $s$ over the domain of the state space, except for the current time point $0$.

    We note $\asfml$ to indicate that $s$ is not a free variable of $\asfml(s)$ and $\asfml(b)$ with $b\in\nonposQ$ to express that each term $\astrm(s)$ in the formula was replaced by its corresponding $\astrm(b)$, even if it was bound by a $\hs{}$.
    % TODO: or only which is not under the scope of a \hs? at the moment I overwrite each \hs
    If we write $\asfml(x)$, we mean entire function $x$, not only the value $\x[0]$ (cf.\ usage for $\forall$).
    
    % TODO: on variables \allvars (signature)
    Formulas of first-order logic of real arithmetic constitute a subset of \ddL, i.e.\ every \FOLR formula is also a formula of delay differential dynamic logic.

    \begin{convention}
        The frequently appearing fact that $\asfml(s)$ is not only supposed to hold for $s\in\delayinterval$ but also in $s=0$
        \begin{equation*}
            \hs{\asfml(s)}\land\asfml(0)
        \end{equation*}
        can also be written as
        \begin{equation*}
            \hsc{\asfml(s)}
        \end{equation*}
        For convenience, we allow the latter, abbreviated notation, which is implicitely replaced by the former, syntactically correct version.
    \end{convention}

    In order to simplify notation by eliminating parentheses, we agree on the following
    \begin{convention}
        The operators in \ddL formulae obey the following binding priorities (from highest to lowest):
        \begin{itemize}
            \item the quantifiers $\forall,\exists$ and the modal operators $\dbox{\cdot}{},\ddiamond{\cdot}{}$ bind strongest
            \item negation $\lnot$ binds stronger than
            \item conjunction $\land$ binds stronger than 
            \item disjunction $\lor$ binds stronger than
            \item implication $\limply$ binds stronger than
            \item equivalence $\lbisubjunct$, which binds weakest.
        \end{itemize}
    \end{convention}

        % TODO: Move to static semantics
    Moreover, when a s-formula does not depend on the quantified parameter $s$, we can drop the quantifier $\hs{}$ in which this formula appears.

    \begin{example}
        Consider the two well-formed \ddL formulae:
        \begin{align*}
            &\hs{(x+\x[s]\geq 0)}\\
            &\hs{(x+\x[-\tau]\geq 0)} 
        \end{align*}
        The quantification over $s$ in the second formula can be dropped, what leads to the equivalent formula
        \begin{equation*}
            x+\x[-\tau]\geq 0.
        \end{equation*}
    \end{example}


\section{Dynamic Semantics}
    \label{sec:dynamic-semantics}

    % TODO: this is called valuation

    In this section, we give meaning to the syntax introduced above, by defining its semantics in a compositional way.

    Following the remark to the solution of a DDE (cf.\ Section~\ref{sec:definition-of-solution}), we define the \emph{state space} in \ddL as $\statespace$, the set of piecewise continuously differentiable functions on $\closeddelayinterval$, as defined in Definition~\ref{def:piecewise-continuous}.
    This means that a variable remembers a limited part of its evolution history, what demands hence an implicit notion of a underlying time.

    We denote by $\states$ the \emph{set of states}. A \emph{state} $\asstate\in\states$ is a mapping
    \begin{equation}
        \asstate \from \allvars\cup\diffvars \to \statespace
    \end{equation}
    which assigns a \emph{history} (function) to each variable and differential symbol.

    By $\modif{\asstate}{x}{y}$ we denote the state which is equal to state $\asstate$, except for the value of the variable $x$, which is set to $y\in\statespace$.

    % FIXME: modification of x in state and general specification of pw function associated to x
    %at the time $t=0$. The value of $x$ for the time before does not change.

    % FIXME: let interpretation $\interpret$ be fixed
    

    \begin{definition}[Semantics of s-terms]\label{def:sematic-terms}
        The \emph{semantics} of a s-term $\astrm(s)$ in the state $\asstate\in\states$ with respect to the time instant $\past\in\closeddelayinterval$
        is a value in $\R$ and defined inductively as follows:
        \begin{enumerate}
            \item $\ivaluation{\IddL}{\x[s]} = \asstate(x)(\past)$ for a variable $x\in\allvars$
            % FIXME: need Cpw for derivatives or only \R?
            \item $\ivaluation{\IddL}{\Dx[s]} = \asstate(\D{x})(\past) \defeq \lim_{t\downto \past} \frac{\asstate(x)(t)-\asstate(x)(\past)}{t-\past}$ (except in $\past=0$)
            % for a differential symbol $\D{x}\in\diffvars$
            \item $\ivaluation{\IddL}{\x[c]} = \asstate(x)(c)$ for a variable $x\in\allvars$ 
            % \item $\ivaluation{\IddL}{\Dx[b]} = \asstate(\D{x})(b)$ for a differential symbol $\D{x}\in\diffvars$ and a constant $b\in\constants$ with $\interpret[b]\in\nonposQ$
            \item $\ivaluation{\IddL}{\Dx[c]} = \asstate(\D{x})(c) \defeq \lim_{t\downto c} \frac{\asstate(x)(t)-\asstate(x)(c)}{t-c}$ (except in $c=0$)
            \item $\ivaluation{\IddL}{a} = \interpret[a]$ for a constant $a\in\constants$
            \item $\ivaluation{\IddL}{f(\range{\istrm{1}(s)}{\istrm{k}(s)})} = \interpret[f](\range{\ivaluation{\IddL}{\istrm{1}(s)}}{\ivaluation{\IddL}{\istrm{k}(s)}})$ for a function symbol $f$
            \item $\ivaluation{\IddL}{\astrm(s)+\bstrm(s)} = \ivaluation{\IddL}{\astrm(s)} + \ivaluation{\IddL}{\bstrm(s)}$
            \item $\ivaluation{\IddL}{\astrm(s)\cdot\bstrm(s)} = \ivaluation{\IddL}{\astrm(s)} \cdot \ivaluation{\IddL}{\bstrm(s)}$
            % FIXME: derivation of terms with x'?
            \item $\ivaluation{\IddL}{\D{(\astrm(s))}} = \displaystyle\sum_{\x[c]\in\delayvars} \asstate(\D{x})(\interpret[c])\frac{\partial\ivaluation{\IddL}{\astrm(s)}}{\partial \x[b]} + \asstate(\D{x})(\past)\frac{\partial\ivaluation{\IddL}{\astrm(s)}}{\partial \x[s]}$
        \end{enumerate} 
        where $c\in\nonposQ$ is a non-positive rational number.
    \end{definition}
        % FIXME: interpretation of $f$ is a smooth function?

    The meaning of the variable and differential symbols is determined by the state. Additionally, the value of a differential symbol has to coincide with the right derivative of the corresponding variable, except for $\past=0$.

    The meaning of the differential of an arbitrary term is the total derivative of its value with respect to the underlying continuous time.
    As a composition of smooth functions is $\ivaluation{\IddL}{\astrm(s)}$ smooth itself and hence these derivatives exist.
    The sum is finite, since each term only mentions finitely many variables.

    % The meaning of a differential symbol: defined by state, 
    % The time derivative of a variable may not be defined in isolated state, because it might have a discontinuity here and we cannot use its history (provided by the state) to define derivative.
    % However, we can define a state local semantics for a differential $\D{(\astrm)}$.

    % Like for ODE case, only use last value given in state. Since in $t=0$ could be incontinuity 

    

    % Along a solution of a DDE $\varphi:[0,r]\rightarrow\states$ (continous differentiable), the differential symbols are interpreted as time derivatives, $r>0$ at any $\zeta\in[0,r]$, meaning of symbol is rate of change of value of $x$ over time
    % \begin{equation}
    %     % TODO: replace \DD with Andres command
    %     \imodel{\IddL}{\D{x}}\varphi(\zeta)
    %         = \varphi(\zeta)(\D{x})(0)
    %         \defeq \DD{\varphi(t)(x)(0)}{t}(\zeta)
    %         = \DD{\imodel{\IddL}{x}\varphi(t)}{t}(\zeta)
    % \end{equation}
   
 
    

    In the precondition, no values are associated to the differential symbols. In general, the initial function is only piecewise continuous.
    Since for later time instances, the values of the differential symbols derive from the DDE, they become (locally) smooth function.

    \begin{definition}[Semantics of s-formulae]\label{def:semantic-formulae}
        The semantics of a \ddL formula $\asfml$ is the subset of all states $\imodel{\IddL}{\asfml}\subseteq\states$ in which $\asfml$ is true. This set is given inductively by
        % TODO: mention interpretation I
        % FIXME: what is difference between \imodel and \ivaluation?
        % terms: ivaluation, formula: imodel, programms: ireachability
        % FIXME: display assign for formula semantics
        \begin{enumerate}
            % TODO: replace : in set by \with
            % FIXME: use notation forall ... : ...
            % FIXME: *version for lforall without parentheses
            % FIXME: forall quantifies over state space or over possible values (reals) for system parameters?
            \item $\imodel{\IddL}{\astrm(s)=\bstrm(s)} = \Set*{\asstate\in\states \with \ivaluation{\IddL}{\astrm(s)} = \ivaluation{\IddL}{\bstrm(s)}}$
            \item $\imodel{\IddL}{\astrm(s)\geq\bstrm(s)} = \Set*{\asstate\in\states \with \ivaluation{\IddL}{\astrm(s)}\geq\ivaluation{\IddL}{\bstrm(s)}}$
            \item $\imodel{\IddL}{p(\range{\istrm{1}(s)}{\istrm{k}(s)})} = \Set*{\asstate\in\states \with \left(\range{\ivaluation{\IddL}{\istrm{1}(s)}}{\ivaluation{\IddL}{\istrm{k}(s)}}\right)\in\interpret[p]}$
            \item $\imodel{\IddL}{\lnot\asfml(s)} = \scomplement{\left(\imodel{\IddL}{\asfml(s)}\right)} = \states\setminus\imodel{\IddL}{\asfml(s)}$
            \item $\imodel{\IddL}{\asfml(s)\land\bsfml(s)} = \imodel{\IddL}{\asfml(s)}\cap\imodel{\IddL}{\bsfml(s)}$
            \item $\imodel{\IddL}{\hs{\asfml(s)}} = \Set*{\asstate\in\states \with \mforall{\tilde{r}\in\delayinterval}\holds\asstate\in\imodel{\iconcat[assign=\tilde{r}]{\IddL}}{\asfml(s)}}$
            \item $\imodel{\IddL}{\lforall{x}{\asfml(s)}} = \Set*{\asstate\in\states \with \modif{\asstate}{x}{y}\in\imodel{\IddL}{\asfml(s)} \text{ for all } y\in\statespace}$
            \item $\imodel{\IddL}{\lexists{x}{\asfml(s)}} = \Set*{\asstate\in\states \with \modif{\asstate}{x}{y}\in\imodel{\IddL}{\asfml(s)} \text{ for some } y\in\statespace}$
            \item $\imodel{\IddL}{\dbox{\asprg}{\asfml(s)}} = \Set*{\asstate\in\states \with \bsstate\in\imodel{\IddL}{\asfml(s)} \text{ for all $\bsstate$ such that} (\asstate,\bsstate)\in\ireachability{\IddL}{\asprg}}$,\\ i.e. $=\Set*{\asstate\in\states \with \mforall{\bsstate\in\states}\holds(\asstate,\bsstate)\in\ireachability{\IddL}{\asprg} \mimply \bsstate\in\imodel{\IddL}{\asfml(s)}}$
            \item $\imodel{\IddL}{\ddiamond{\asprg}{\asfml(s)}} = \Set*{\asstate\in\states \with \bsstate\in\imodel{\IddL}{\asfml(s)} \text{ for some $\bsstate$ such that} (\asstate,\bsstate)\in\ireachability{\IddL}{\asprg}}$,\\ i.e. $=\Set*{\asstate\in\states \with \mexists{\bsstate\in\states}\holds(\asstate,\bsstate)\in\ireachability{\IddL}{\asprg} \land \bsstate\in\imodel{\IddL}{\asfml(s)}}$
        \end{enumerate}
        The fact that formula $\asfml(s)$ is true in state $\asstate$ under the interpretation $\interpret$ at past time instant $\past\in\closeddelayinterval$, i.e. $\asstate\in\imodel{\IddL}{\asfml(s)}$ can also be written as $\interpret,\asstate,\past\models\asfml(s)$.
        A formula $\asfml(s)$ is called valid, written as $\models\asfml(s)$, if and only if $\asfml(s)$ is true in all states, for all $\past\in\closeddelayinterval$ and under all interpretations.
    \end{definition}

    As in classic first-order logic, the interpretation of a predicate symbol of arity n is a relation $\interpret[p]\subseteq\R^n$.

    % Atomic formulas (type 1 and 2) need to be combined with the quantification over s (6) in order to make sense. See Section \ref{sec:well-definedness}.

    % FIXME: text after HP semantics, add waht noted on white board
    % With the semantics of terms if follows for the meaning of $\dbox{\asprg}{\phi}$, that $\phi$ must only hold up to time $\tau$ before leaving the \HP $\asprg$. It is possible, that $\phi$ was not verified before, while \emph{executing} the \HP.

    % However when we apply the Rule of steps, we get the validity of $\phi$ for the entire trace.

    % TODO: is it better to only have xtau and not choice? What if both mentioned?
    % in formulae (such as safety condition or evolution constraint), we have two possibilities: only value at current time instant ($x$) or for entire last $\tau$ time $\x[-\tau]$

    % $\dbox{}{}$ and $\ddiamond{}{}$ only refer to last state and not intermediate states, as \dTL does.

    \begin{lemma}[Barcan formula]
        The box modality and the quantification over $s$ commute
        \begin{equation*}
            \imodel{\IddL}{\hs{\dbox{\asprg}{\asfml(s)}}} = \imodel{\IddL}{\dbox{\asprg}{(\hs{\asfml(s)})}}
        \end{equation*}
    \end{lemma}
    \begin{proof}
        Since $\mforall{x}\holds(p\mimply q(x))\equiv p\mimply\mforall{x}\holds q(x)$, it holds
        \begin{multline*}
            \imodel{\IddL}{\hs{\dbox{\asprg}{\asfml(s)}}} =\\
            \begin{aligned}
                &= \Set*{\asstate\in\states \with \holdssince[\tilde{r}]{-T}\mforall{\bsstate\in\states}\holds \left((\asstate,\bsstate)\in\ireachability{\IddL}{\asprg} \mimply \bsstate\in\imodel{\iconcat[assign=\tilde{r}]{\IddL}}{\asfml(s)}\right)}\\
                &= \Set*{\asstate\in\states \with \mforall{\bsstate\in\states}\holds\holdssince[\tilde{r}]{-T}\left((\asstate,\bsstate)\in\ireachability{\IddL}{\asprg} \mimply \bsstate\in\imodel{\iconcat[assign=\tilde{r}]{\IddL}}{\asfml(s)}\right)}\\
                &= \Set*{\asstate\in\states \with \mforall{\bsstate\in\states}\holds\left((\asstate,\bsstate)\in\ireachability{\IddL}{\asprg} \mimply \holdssince[\tilde{r}]{-T}\bsstate\in\imodel{\iconcat[assign=\tilde{r}]{\IddL}}{\asfml(s)}\right)}\\
                &= \Set*{\asstate\in\states \with \mforall{\bsstate\in\states}\holds\left((\asstate,\bsstate)\in\ireachability{\IddL}{\asprg} \mimply \bsstate\in\imodel{\IddL}{\hs\asfml(s)}\right)}\\
                &= \imodel{\IddL}{\dbox{\asprg}{(\hs{\asfml(s)})}}
            \end{aligned}
        \end{multline*}
    \end{proof}
    
    However, the diamond modality does not commute with the s-quantification.
    % TODO: counterexample for not commuting diamond and forall-s
    % TODO: do we accept that or restrict formula after modality to not having $s$ free? then, since s\notin\asprg commutes also for diamond
    \begin{equation*}
        \imodel{\IddL}{\hs{\ddiamond{\asprg}{\asfml(s)}}} \neq \imodel{\IddL}{\ddiamond{\asprg}{\hs{\asfml(s)}}}
    \end{equation*}

    % Since, with respect to \dL, the state space has been replaced, we need to redefine the semantics.
    
    \begin{definition}[Transition semantics of \dHPs]\label{def:semantic-dHP}
        The interpretation of a \dHP is given by a binary \emph{reachability relation} $\ireachability{\IddL}{\asprg}\subseteq\states\times\states$ between states:
        \begin{enumerate}
            % \item $\ireachability{\IddL}{a} = \interpret[a]$ for a program constant $a$
            % FIXME: astrm(s) or without s
            \item\label{itm:sem-dHP-assgn} $\ireachability{\IddL}{\hupdate{\humod{x}{\astrm}}} =
                \set{(\asstate,\bsstate)\with \bsstate=\asstate \text{ except }
                \bsstate(x) = \left(\past\mapsto\begin{cases}
                    \ivaluation{\IddL}{\astrm(s)} & \past=0\\
                    \asstate(x)(\past) & \past\in\delayinterval
                \end{cases}\right)}$
            \item $\ireachability{\IddL}{\Dupdate{\Dumod{\D{x}}{\astrm}}} =
                \set{(\asstate,\bsstate)\with \bsstate = \asstate \text{ except }
                \bsstate(\D{x}) = \left(\past\mapsto\begin{cases}
                    \ivaluation{\IddL}{\astrm(s)} & \past=0\\
                    \asstate(\D{x})(\past) & \past\in\delayinterval
                \end{cases}\right)}$
            % FIXME: no \assign for FOLR formula \imodel
            \item $\ireachability{\IddL}{\htest{\asfml(s)}} = \set{(\asstate,\asstate) \with \asstate\in\imodel{\IddL}{\asfml(s)}}$
            \item $\ireachability{\IddL}{\hchoice{\asprg}{\bsprg}} = \hchoice{\ireachability{\IddL}{\asprg}}{\ireachability{\IddL}{\bsprg}}$
            \item $\ireachability{\IddL}{\asprg;\bsprg} = \set{(\asstate,\bsstate) \with (\asstate,\csstate)\in\ireachability{\IddL}{\asprg}, (\csstate,\bsstate)\in\ireachability{\IddL}{\bsprg}}$
            \item $\ireachability{\IddL}{\hrepeat{\asprg}} 
                = \displaystyle\cupfold_{n\in\N_0}\ireachability{\IddL}{\asprg^n}$ with $\asprg^{n+1}\equiv (\asprg^n;\asprg)$ and $\asprg^0\equiv (\htest{\ltrue})$
            % TODO: semantics DDE
            \item\label{itm:sem-HP-DDE} $\ireachability{\IddL}{\hevolvein{\D{x}=\astrm}{\ivr}} = \{(\asstate,\bsstate) \;\vert\; \mforall{\zeta\in[0,r]}\holds\trajectory(\zeta)\in\imodel{\IddL}{\hevolve{\D{x}=\astrm}\land\ivr}$ and $\asstate=\trajectory(0)$ on $\scomplement{\Set{\D{x}}}$ and $\bsstate=\trajectory(r)$ for a $\trajectory\from [0,r]\to\states\}$, i.e. there exists a $r\geq 0$ and a trajectory $\trajectory\from [0,r]\to\states$, which fulfills $\trajectory(\zeta)(\D{x})(s) \defeq \DD{\trajectory(t)(x)(s)}{t}(\zeta) \stackrel{!}{=} \ivaluation{\iconcat[state=\trajectory(\zeta+s)]{\IddL}}{\astrm}$ and satisfies $\ivr$ for all $s\in[-\min\Set{\zeta,T},0]$. On $[-T,-\min\Set{\zeta,T})$ it holds $\trajectory(\zeta)(\cdot)(s)=\asstate(\cdot)(s+\zeta)$ for all variables.
            % TODO: case r=0
        \end{enumerate}
    \end{definition}
    The semantics of a delay differential equation is motivated by the definition of a solution for a DDE-IVP (cf.\ Definition~\ref{def:solution-dde}), following the evolution for a nondeterministic period of time, as long as the evolution domain constraint holds.

    % To facilitate notation, we define a \emph{satisfaction relation} with respect to a DDE: the trajectory $\trajectory$ (which is a set of states) is said to \emph{satisfy} $\hevolvein{\D{x}=\astrm}{\ivr}$, written as $\trajectory\models\hevolvein{\D{x}=\astrm}{\ivr}$
    % TODO: :  fulfills DDE 
    % $\trajectory$ solves the DDE and satisfies $\ivr$ in each time instant/state

    % The formula $\hevolve{\D{x}=\astrm(-\tau)}\land\ivr$ has dropped the $\hs{}$, all occurences of $x$ and $\D{x}$ of form with constant

    % TODO: if $r=0$: 
    initial value $\asstate(\D{x})$ may not be compatible with derivative
    final values coincide

    For the \emph{discrete assignment}, we only allow the values at the current time instant to be changed. A functional assignment would essentially allow to rewrite history, which is not permitted.

    The jump behavior caused be discrete assignments is the actual reason why we need to consider piecewise continuous evolutions.

    Time is implicit and usually not revealed. If it is explicitely needed, a clock variable $t$ can be introduced by $\hevolve{\D{t}=1}$.

    %TODO: super dense time: multiple assignments, only consider last assignment

    % in general, we do not write expression explicitely, just give possible intervals in precondition
    
    As a \FOLR formula, $\ivr$ do not contain any delayed variables and thus only depends on the values at the current time instant (and not over the entire interval $\delayinterval$).

    \begin{lemma}[Derivations]\label{lm:derivations}
        Standard analysis derivation rules also hold in the semantics of \ddL terms, i.e.\ the following equations are valid \ddL formulas
        \begin{align}
            \D{(\x[s])} &= \Dx[s]\\
            %\Dx[s] &=\\
            \D{(\x[c])} &= \Dx[c]\\
            %\Dx[b] &=\\
            \D{(a)} &= 0\\
            \D{(f(\range{\istrm{1}}{\istrm{k}}))} &=\\
            \D{(\astrm+\bstrm)} &= \D{(\astrm)}+\D{(\bstrm)}\\
            \D{(\astrm\cdot\bstrm)} &= \D{(\astrm)}\cdot\bstrm + \astrm\cdot\D{(\bstrm)}\\
            % FIXME: is term' allowed? diff inv is only a FOLR formula? even for dL valid?
            %\D{(\astrm)}
        \end{align}
        This allows to apply these rules on a syntactic level, what will be done in the form of axioms (see~\ref{sec:differential-axioms}).
    \end{lemma}
    \begin{proof}
        \begin{align*}
            % FIXME: semantics of x': must coincide with deriv of x
            % FIXME: semantics of Derivation: no I(s) and nu(s)
            \ivaluation{\IddL}{\D{(\x[s])}}
            &= \sum_{\x[c]\in\delayvars} \asstate(\D{x})(c)\frac{\partial\ivaluation{\IddL}{\x[s]}}{\partial \x[c]}+ \asstate(\D{x})(\past)\frac{\partial\ivaluation{\IddL}{\x[s]}}{\partial \x[s]}\\
            &= \asstate(\D{x})(\past)\frac{\partial\ivaluation{\IddL}{\x[s]}}{\partial \x[s]}
            = \asstate(\D{x})(\past) = \ivaluation{\IddL}{\Dx[s]}
        \end{align*}
        \begin{align*}
            % delayed derivative
            \ivaluation{\IddL}{\D{(\x[c])}}
            &= \sum_{\x[d]\in\delayvars} \asstate(\D{x})(d)\frac{\partial\ivaluation{\IddL}{\x[c]}}{\partial \x[d]}+ \asstate(\D{x})(\past)\frac{\partial\ivaluation{\IddL}{\x[c]}}{\partial \x[s]}\\
            &= \asstate(\D{x})(c)\frac{\partial\ivaluation{\IddL}{\x[c]}}{\partial \x[c]}
            = \asstate(\D{x})(c) = \ivaluation{\IddL}{\Dx[b]}
        \end{align*}
        \begin{align*}
            % constant
            \ivaluation{\IddL}{\D{(a)}}
            &= \sum_{\x[c]\in\delayvars} \asstate(\D{x})(c)\frac{\partial\ivaluation{\IddL}{a}}{\partial \x[c]}\\
            % &= \asstate(\D{x})(c)\frac{\partial\interpret[a]}{\partial \x[c] 
            = 0
        \end{align*}
        \begin{align*}
            % chain rule
            % FIXME: derivation of function symbol
            \ivaluation{\IddL}{\D{(f(\range{\istrm{1}(s)}{\istrm{k}(s)}))}} &= \sum_{\x[c]\in\delayvars} \asstate(\D{x})(c)\frac{\partial\ivaluation{\IddL}{f(\range{\istrm{1}(s)}{\istrm{k}(s)})}}{\partial \x[c]}\\ &+ \asstate(\D{x})(\past)\frac{\partial\ivaluation{\IddL}{f(\range{\istrm{1}(s)}{\istrm{k}(s)})}}{\partial \x[s]}
        \end{align*}
        \begin{align*}
            % addition rule
            \ivaluation{\IddL}{\D{(\astrm(s)+\bstrm(s))}}
            &= \sum_{\x[c]\in\delayvars} \asstate(\D{x})(c)\frac{\partial\ivaluation{\IddL}{\astrm(s)+\bstrm(s)}}{\partial \x[c]} + \asstate(\D{x})(\past)\frac{\partial\ivaluation{\IddL}{\astrm(s)+\bstrm(s)}}{\partial \x[s]}\\
            &= \sum_{\x[c]\in\delayvars} \asstate(\D{x})(c)\frac{\partial\ivaluation{\IddL}{\astrm(s)}+\ivaluation{\IddL}{\bstrm(s)}}{\partial \x[c]} + \asstate(\D{x})(\past)\frac{\partial\ivaluation{\IddL}{\astrm(s)}+\ivaluation{\IddL}{\bstrm(s)}}{\partial \x[s]}\\
            &= \sum_{\x[c]\in\delayvars} \asstate(\D{x})(c)\frac{\partial\ivaluation{\IddL}{\astrm(s)}}{\partial \x[c]} + \asstate(\D{x})(\past)\frac{\partial\ivaluation{\IddL}{\astrm(s)}}{\partial \x[s]}\\
            &+ \sum_{\x[c]\in\delayvars} \asstate(\D{x})(c)\frac{\partial\ivaluation{\IddL}{\bstrm(s)}}{\partial \x[c]} + \asstate(\D{x})(\past)\frac{\partial\ivaluation{\IddL}{\bstrm(s)}}{\partial \x[s]}\\
            &= \ivaluation{\IddL}{\D{(\astrm(s))}} + \ivaluation{\IddL}{\D{(\bstrm(s)}}
            = \ivaluation{\IddL}{\D{(\astrm(s))}+\D{(\bstrm(s))}}
        \end{align*}
        \begin{align*}
            % multiplication rule
            \ivaluation{\IddL}{\D{(\astrm(s)\cdot\bstrm(s))}}
            &= \sum_{\x[c]\in\delayvars} \asstate(\D{x})(c)\frac{\partial\ivaluation{\IddL}{\astrm(s)\cdot\bstrm(s)}}{\partial \x[c]} + \asstate(\D{x})(\past)\frac{\partial\ivaluation{\IddL}{\astrm(s)\cdot\bstrm(s)}}{\partial \x[s]}\\
            &= \sum_{\x[c]\in\delayvars} \asstate(\D{x})(c)\frac{\partial\left(\ivaluation{\IddL}{\astrm(s)}\cdot\ivaluation{\IddL}{\bstrm(s)}\right)}{\partial \x[c]} + \asstate(\D{x})(\past)\frac{\partial\left(\ivaluation{\IddL}{\astrm(s)}\cdot\ivaluation{\IddL}{\bstrm(s)}\right)}{\partial \x[s]}\\
            &= \sum_{\x[c]\in\delayvars} \asstate(\D{x})(c)\frac{\partial\ivaluation{\IddL}{\astrm(s)}}{\partial \x[c]}\ivaluation{\IddL}{\bstrm(s)} + \asstate(\D{x})(\past)\frac{\partial\ivaluation{\IddL}{\astrm(s)}}{\partial \x[s]}\ivaluation{\IddL}{\bstrm(s)}\\
            &+ \sum_{\x[c]\in\delayvars} \asstate(\D{x})(c)\frac{\partial\ivaluation{\IddL}{\bstrm(s)}}{\partial \x[c]}\ivaluation{\IddL}{\astrm(s)} + \asstate(\D{x})(\past)\frac{\partial\ivaluation{\IddL}{\bstrm(s)}}{\partial \x[s]}\ivaluation{\IddL}{\astrm(s)}\\
            &= \ivaluation{\IddL}{\D{(\astrm(s))}}\cdot\ivaluation{\IddL}{\bstrm(s)} + \ivaluation{\IddL}{\astrm(s)}\cdot\ivaluation{\IddL}{\D{(\bstrm(s)}}\\
            &= \ivaluation{\IddL}{\D{(\astrm(s))}\cdot\bstrm(s)+\astrm(s)\cdot\D{(\bstrm(s)}}
        \end{align*}
        
    \end{proof}

    \begin{definition}\label{def:termvars}
        We define by
        \begin{equation*}
            % TODO: what if x'[] in term ?
            \constants[\astrm] \defeq \Set{c\in\constants \with \mexists{\x\in\allvars}\holds\x[c]\in\astrm(s)}
        \end{equation*}
        the set of constant parameter symbols occuring in the s-term $\astrm(s)$.

        Note that this set does not contain $s$, since it is, as a special purpose symbol, not in $\constants$.
    \end{definition}

    \begin{definition}[Sampled trajectory]\label{def:sampled-trajectory}
        Since a s-term $\astrm(s)$ only comprises a finite number of atomic terms, its valuation can also be seen as a mapping
        \begin{equation*}
            \ivaluation{\interpretation[const=\interpret]}{\astrm(s)} \from \R^{\abs{\mathcal{K}}} \to \R  
        \end{equation*}
        % TODO: what about x'
        from the concrete values for each element of $\mathcal{K}\defeq \delayvars[\astrm] \cup \delaydiffvars[\astrm] \cup \Set{\x[s], \Dx[s]}$ into the reals, if we assign a fixed $r\in\closeddelayinterval$ to $s$.

        % , instead of using the functional state space as domain.

        This gives rise to the definition of the \emph{sampled trajectory} $\sampledtraj[\past]{\astrm}\from\compactum{0}{\duration}\to\R^{\abs{\mathcal{K}}}$ for a fixed $\past\in\closeddelayinterval$ and s-term $\astrm(s)$ (without loss of generality, considering $\allvars=\Set{x}$)
        % FIXME: differnt vars possible, not just x
        \begin{equation*}
            \sampledtraj[\past]{\astrm}(t) \defeq \colvec{
                \trajectory(t)(x)(c_1)\\
                \vdots\\
                \trajectory(t)(x)(c_n)
            % \\ \vdots\\ \trajectory(t)(x_m)(c_1)\\ \vdots\\ \trajectory(t)(x_m)(c_n)
            }
        \end{equation*}
    \end{definition}
    % TODO: alternatively, could maybe have \trajectory Fréchet-diffable

    The following lemma shows the consistency of the semantics for differentials with the semantics of the evolution of a delay differential equation.
    This means that along a DDE, the values of differential symbols coincide with the time derivative of the value of the corresponding variable.
    % Derivative means right-hand in points of partition.

    \begin{lemma}[Differential Lemma]\label{lm:differential-lemma}
        % FIXME: check if everywhere correctly used: \states=(\statespace)^n
        The value of a s-term $\bstrm(s)$ along a trajectory $\trajectory\from\compactum{0}{\duration}\to\states$ satisfying a DDE for any duration $\duration>0$, i.e.
        % FIXME:doFormatList does not work with iconcat $\imodels{\iconcat[state=\trajectory]{\IddL}}{\D{x}=\asfml\land\ivr}$
        $\interpret,\trajectory\models(\D{x}=\astrm\land\ivr)$,
        % FIXME: do I need case s=0?
        is piecewise continuously differentiable and for all $\zeta\in\compactum{0}{\duration}$ and $\past\in\closeddelayinterval$ it holds:
        \begin{equation*}
            \ivaluation{\iconcat[state=\trajectory(\zeta)]{\IddL}}{\D{(\bstrm(s))}} = \DD{\ivaluation{\iconcat[state=\trajectory(t)]{\IddL}}{\bstrm(s)}}{t}(\zeta)
        \end{equation*}
        As in Definition~\ref{def:pw-cont-diff}, the derivative at a discontinuity point of is to be understood as right derivative.
    \end{lemma}
    \begin{proof}
        \label{prf:differential-lemma}
        % TODO: what if x'[] in term ?
        % TODO: in init cond: x and x' need to match, ie d/ds x[s] = x'[s], same partition, specification of x' in init cond only needed when referenced to it later
        % FIXME: oBdA: t_0=0

        Without loss of generality, we restrict in this proof to a single variable $x$. If $\bstrm(s)$ depends on more variables, consider the union of their partitions in the initial condition.
        Let $\partition{-T=t_0}{t_m=0}$ be the partition of the initial condition $\trajectory(0)(x)\in\statespace$.

        We choose an arbitray but fixed valuation for $s$ from $\closeddelayinterval$, such that the symbol $s$ can be treated as a constant, in the same way as any $c\in\nonposQ$.
        Depending on the s-term $\bstrm(s)$ and the fixed $s$, we define a partition $\mathcal{Z}_\bstrm^s=\partition{\hat{t}_0}{\hat{t}_p}$ of $\closedopen{0}{\infty}$ (which can be limited to $\compactum{0}{R}$) by
        \begin{equation*}
            \mathcal{Z}_\bstrm^s \defeq \set{0}\cup \bigcup_{i=0}^m\bigcup_{\substack{c\in \mathcal{K}\\ t_i\geq c}}\set{t_i-c}\cup \bigcup_{i=0}^m\bigcup_{j=1}^k\bigcup_{\substack{c\in \mathcal{K}\\ t_i+\tau_j\geq c}}\set{t_i+\tau_j-c}
        \end{equation*}
        where $\mathcal{K}\defeq\constants[\bstrm]\cup\set{s}$ is the set of the constants (the interpretations/valuations of the constant symbols) appearing in the term $\bstrm$ and $\tau_j\in\constants[\astrm]$ the delays in the right hand side of the DDE.
        The set $\mathcal{Z}_\bstrm^s$ is finite and non-empty, since it contains at least the value $0$ and $\duration$.
        % FIXME
        % in side of a continuous diffable piece of init cond or solution (cf.)

        We show first that $\trajectory(t)(x)(c)$ is piecewise continuously differentiable in $t$ for all $c\in\mathcal{K}$ with partition $\mathcal{Z}_\bstrm^s$:

        Let $c\in\mathcal{K}$ and $\zeta\in(\hat{t}_l,\hat{t}_{l+1})$.
        Assume that $\zeta+c=t_i$ for some $i$. This implies $\zeta=t_i-c=\hat{t}_k$ for some $k$ by the definition of the partition. This is not possible by the choice of $\zeta$ lying between two consequtive $\hat{t}_l$.
        We apply the same argumentation to the assumption $\zeta+c=t_i+\tau_j$.
        These contradictions show that for $\zeta\in(\hat{t}_j,\hat{t}_{j+1})$, it holds that $\zeta+c\neq t_i$ and $\zeta+c\neq t_i+\tau_j$ for all $c\in\mathcal{K}$ and for all $i\in\Set{\range{0}{m}}$, $j\in\set{\range{1}{k}}$.
        We now distinguish two cases:

        If $\zeta+c<0$, it holds by the definition of the DDE semantics (Definition~\ref{def:semantic-HP}(\ref{itm:sem-HP-DDE})) that $\trajectory(\zeta)(x)(c)=\trajectory(0)(x)(\zeta+c)$, which is continuously differentiable as initial condition, if $\zeta+c\neq t_i$. Hence it follows
        \begin{equation*}
            \DD{\trajectory(t)(x)(c)}{t}(\zeta)=\DD{\trajectory(0)(x)(\past)}{\past}(\zeta+c)=\trajectory(0)(\D{x})(\zeta+c)=\trajectory(\zeta)(\D{x})(c)
        \end{equation*}
        For the right limit it holds
        \begin{align*}
            \lim_{\zeta\downto\hat{t}_j} \DD{\trajectory(t)(x)(c)}{t}(\zeta)
                & = \lim_{\zeta\downto t_i-c} \DD{\trajectory(0)(x)(\past)}{\past}(\zeta+c)\\
                & = \lim_{\zeta\downto t_i} \DD{\trajectory(0)(x)(\past)}{\past}(\zeta)
                %=\lim_{s\downto t_i}\trajectory(0)(\D{x})(s)
                = \trajectory(0)(\D{x})(t_i)
        \end{align*}
        And analogously for the existence of the left limit for $\zeta\upto\hat{t}_{j+1}$

        If $\zeta+c\geq 0$, then $\trajectory(\zeta)(x)(\past)=\trajectory(\zeta+c)(x)(0)$ is differentiable in $\zeta$ with
        \begin{equation*}
            \trajectory(\zeta)(\D{x})(\past) = \DD{\trajectory(t)(x)(\past)}{t}(\zeta)
        \end{equation*}
        by the semantics of the DDEs, if $\zeta+c\neq t_i+\tau_j$.
        % FIXME: jumppoints and pw conitnuity/diffable
        % By Theorem~\ref{thm:solution-existence}: continuous and pw diffable with partition $Z$.

        % By definition, $\hat{t}_j=t_i-\hat{c}$ for some $i\in\set{\range{0}{m}}$ and $\hat{c}\in\mathcal{K}$.
        % If $c\geq\hat{c}$ then $\zeta>\hat{t}_j$ implies $\zeta+c > c+t_i-\hat{c}\geq t_i$.
        % If $c<\hat{c}$, then $\zeta < \hat{t}_{j+1} \leq t_i-c$ (needs explanation: tjp1 is smallest next point, construct one, must be greater or equal) and hence $\zeta+c<t_i$.
        % So whenever $\zeta\in(\hat{t}_j,\hat{t}_{j+1})$ is $\zeta+c\neq t_i$ for all $i=\range{0}{m}$ and $c\in\mathcal{K}$

        %On each $\zeta\in(\hat{t}_k,\hat{t}_{k+1})$ is each $\ivaluation{\iconcat[state={\trajectory(\zeta)},assign=s]{\IddL}}{\x[c]}$ diffable in $\zeta$

        % TODO: f in terms smooth, hence each term on this open interval diffable
        Let $\sampledtraj[s]{\bstrm}$ be the $\bstrm$-sampled trajectory for the considered delay differential equation and the fixed $s$.
        It follows with the above results
        % By the transition semantics of DDEs (Definition~\ref{def:semantic-HP}\,(\ref{itm:sem-HP-DDE})), it holds for $\zeta+c\geq 0$ (along sol of DDE)
        % $\trajectory(\zeta)(\D{x})(c) = \DD{\trajectory(t)(x)(c)}{t}(\zeta)$, and for $\zeta+c\leq 0$ (init cond, match demanded)

        \begin{align*}
            \DD{\ivaluation{\iconcat[state={\sampledtraj[s]{\bstrm}(\zeta)},assign={}]{\IddL}}{\bstrm}}{t}
            &= \D{\left(\ivaluation{\iconcat[state={},assign={}]{\IddL}}{\bstrm}\compose\sampledtraj[s]{\bstrm}(\zeta)\right)}
            = \gradient{\ivaluation{\iconcat[state={},assign={}]{\IddL}}{\bstrm}}(\sampledtraj[s]{\bstrm}(\zeta))\cdot\DD{\sampledtraj[s]{\bstrm}}{t}(\zeta)\\
            &= \sum_{\x[c]\in\delayvars[\bstrm]}\DD{\trajectory(t)(x)(c)}{t}(\zeta) \Dp[{(\x[c])}]{\ivaluation{\iconcat[state={\sampledtraj[s]{\bstrm}(\zeta)},assign=s]{\IddL}}{\bstrm}}\\
            % &= \sum_{\x[c]\in\delayvarswiths}\trajectory(\zeta)(\Dx[c])(s) \Dp[{(\x[c])}]{\ivaluation{\iconcat[state={\sampledtraj[s]{\bstrm}(\zeta)},assign=s]{\IddL}}{\bstrm}}{(s)}\\
            &= \sum_{\x[c]\in\delayvars}\trajectory(\zeta)(\D{x})(c) \Dp[{(\x[c])}]{\ivaluation{\iconcat[state={\sampledtraj[s]{\bstrm}(\zeta)},assign=s]{\IddL}}{\bstrm}}\\
            %&= \sum_{\x[c]\in\delayvarswiths}\asstate(\D{x})(c) \Dp[{(\x[c])}]{\ivaluation{\IddL}{\bstrm}}{(s)}\\
            &= \ivaluation{\iconcat[state={\sampledtraj[s]{\bstrm}(\zeta)},assign={}]{\IddL}}{\D{(\bstrm})}
        \end{align*}
        where each sum only consits of finitely many summands.
        Moreover, it holds for the right limits
        \begin{equation*}
            \lim_{\zeta\downto\hat{t}_j} \DD{\ivaluation{\iconcat[state={\sampledtraj[s]{\bstrm}(\zeta)},assign=s]{\IddL}}{\bstrm}}{t} = \ivaluation{\iconcat[state={\sampledtraj[s]{\bstrm}(\hat{t}_j)},assign={}]{\IddL}}{\D{(\bstrm})}
        \end{equation*}
        and the left limits for $\zeta\upto\hat{t}_{j+1}$ exist.
    \end{proof}

    \begin{figure}[t]
        \centering
        % TODO: 100samples/1=0.01
\begin{tikzpicture}[line width=0.5pt, scale=(\textwidth-20pt)/8.4cm, >=Latex]

    \newcommand{\polytwo}{\ca+\cb*((\x-\ta)/(\tb-\ta))+\cc*((\x-\ta)/(\tb-\ta))^2}
    \newcommand{\polythree}{\ca+\cb*((\x-\ta)/(\tb-\ta))+\cc*((\x-\ta)/(\tb-\ta))^2+\cd*((\x-\ta)/(\tb-\ta))^3}
    \newcommand{\polyfour}{\ca+\cb*((\x-\ta)/(\tb-\ta))+\cc*((\x-\ta)/(\tb-\ta))^2+\cd*((\x-\ta)/(\tb-\ta))^3+\ce*((\x-\ta)/(\tb-\ta))^4}

	% grid
	%\draw[help lines, color=gray!30, dashed] (-4,-3) grid (4,4);
	
	% frame
	\draw (-4.2,-3.2) rectangle (4.2,4.2);
	% \draw[help lines] (-4,0) -- (4,0);
	% \draw[help lines] (0,-3) -- (0,4);
	% \draw[->, thick] (-4.06,0) -- (4,0) node[right] {$t/\zeta$};
	% \draw[->, thick] (-4,-3) -- (-4,4) node[above] {$x$};

	% grid and ticks
	\foreach \y in {-3.0,-2.0,-1.0,0.0,1.0,2.0,3.0,4.0} {
		\draw[help lines, color=gray!30, dashed] (-4.2,\y) -- (4.2,\y);
		\draw[thick] (-4.2,\y) -- (-4.08,\y);
		\draw[thick] (4.2,\y) -- (4.08,\y);
	}
	\foreach \x in {-4.0,-3.25,-2.0,-1.2,0.0} {
		\draw[help lines, color=gray!30, dashed] (\x,-3.2) -- (\x,4.2);
		\draw[thick] (\x,4.2) -- (\x,4.08);
		\draw[thick] (\x,-3.2) -- (\x,-3.08);
	}
    \foreach \x in {0.25,0.6,0.75,1.5,1.8,2.0,2.3,2.55,2.8,3.5,3.8,4.0} {
        \draw[help lines, color=gray!30, dashed] (\x,-3.2) -- (\x,4.2);
        \draw[thick] (\x,4.2) -- (\x,4.08);
        \draw[thick] (\x,-3.2) -- (\x,-3.08);
    }


	\draw (-3.93,4.1)   node[below] {$t_0$};
	\draw (-3.25,4.1)  node[below] {$t_1$};
	\draw (-2.0,4.1)   node[below] {$t_2$};
	\draw (-1.2,4.1)   node[below] {$t_3$};
	\draw (0.0,4.1)    node[below] {$t_4$};
	\draw (0.0,-3.1)   node[above] {$\hat t_0$};
	% \draw (0.75,-3.1)  node[above] {$\hat t_1$};
	% \draw (2.0,-3.1)   node[above] {$\hat t_2$};
	% \draw (2.8,-3.1)   node[above] {$\hat t_3$};
	\draw (3.93,-3.1)   node[above] {$\hat t_p$};
	
	% \draw (-4,-0.3) node[below] {$-T$};
	\draw (-4.1,0.0) node[right] {$0$};

	% \draw (0.5,0) node[below] {$\hat t_1$};
	% \draw[thick] (0.5,-0.06) -- (0.5,0.06);
	% \draw (1,0) node[below] {$\hat t_2$};
	% \draw[thick] (1,-0.06) -- (1,0.06);
	% \draw (2,0) node[below] {$\hat t_3$};
	% \draw[thick] (2,-0.06) -- (2,0.06);
	% \draw (2.5,0) node[below] {$\hat t_4$};
	% \draw[thick] (2.5,-0.06) -- (2.5,0.06);
	% \draw (3,0) node[below] {$\hat t_5$};
	% \draw[thick] (3,-0.06) -- (3,0.06);
	% \draw (4,0) node[below] {$\hat t_6$};
	% \draw (4,-0.3) node[below] {$r$};
	% \draw[thick] (4,-0.06) -- (4,0.06);
	%\draw[thick] (\x,-0.06) -- (\x,0.06);
	

	% initial condition

	% on [-4,-3.25]
	\newcommand{\ta}{-4}
	\newcommand{\tb}{-3.25}
	\newcommand{\ca}{1.3}
	\newcommand{\cb}{1.7}
	\newcommand{\cc}{-5.3}
	\newcommand{\cd}{2.8}
	\draw[curve,cadlag] plot[samples=50, smooth, domain=\ta:\tb] (\x, {\polythree});
	% \filldraw[leftpoint] (-4,1.3) circle (0.04);
	% \filldraw[rightpoint] (-3.25,0.5) circle (0.04);

	% on [-3.25,-2]
	\renewcommand{\ta}{-3.25}
	\renewcommand{\tb}{-2.0}
	\renewcommand{\ca}{-0.2}
	\renewcommand{\cb}{-1.125}
	\renewcommand{\cc}{4.35}
	\renewcommand{\cd}{-2.125}
	\draw[curve,cadlag] plot[samples=50, smooth, domain=\ta:\tb] (\x, {\polythree});
	% \filldraw[leftpoint] (-3.25,-0.2) circle (0.04);
	% \filldraw[rightpoint] (-2,0.9) circle (0.04);
	
	% on [-2,-1.2]
	\renewcommand{\ta}{-2.0}
	\renewcommand{\tb}{-1.2}
	\renewcommand{\ca}{0.4}
	\renewcommand{\cb}{-0.24}
	\renewcommand{\cc}{-0.32}
	\renewcommand{\cd}{0.96}
	\draw[curve,leftp] plot[samples=50, smooth, domain=\ta:\tb] (\x, {\polythree});
	% \filldraw[leftpoint] (-2,0.4) circle (0.04);

	
	% on [-1.2,0]
	\renewcommand{\ta}{-1.2}
	\renewcommand{\tb}{0.0}
	\renewcommand{\ca}{0.8}
	\renewcommand{\cb}{0.2}
	\renewcommand{\cc}{-4.72}
	\renewcommand{\cd}{2.92}
	\draw[curve,cadlag] plot[samples=50, smooth, domain=\ta:\tb] (\x, {\polythree});
	% \filldraw[leftpoint] (-1.2,0.8) circle (0.04);
	% \filldraw[rightpoint] (0,-0.7) circle (0.04);


	% solution of x'=-x(t-4)
	% \filldraw[leftpoint] (0.0,1.5) circle (0.04);


	% on [0,0.75]
	\renewcommand{\ta}{0.0}
	\renewcommand{\tb}{0.75}
	\renewcommand{\ca}{1.5}
	\renewcommand{\cb}{-1.3}
	\renewcommand{\cc}{-0.85}
	\renewcommand{\cd}{1.7666666666666}
	\newcommand{\ce}{-0.7}
	\draw[curve,leftp] plot[samples=50, smooth, domain=\ta:\tb] (\x, {\polyfour});

	% on [0.75,2.0]
	\renewcommand{\ta}{0.75}
	\renewcommand{\tb}{2.0}
	\renewcommand{\ca}{0.4166666666666}
	\renewcommand{\cb}{0.2}
	\renewcommand{\cc}{0.5625}
	\renewcommand{\cd}{-1.45}
	\renewcommand{\ce}{0.53125}
	\draw[curve] plot[samples=50, smooth, domain=\ta:\tb] (\x, {\polyfour});

	% on [2.0,2.8]
	\renewcommand{\ta}{2.0}
	\renewcommand{\tb}{2.8}
	\renewcommand{\ca}{0.2604166666666}
	\renewcommand{\cb}{-0.4}
	\renewcommand{\cc}{0.12}
	\renewcommand{\cd}{0.106666666666666}
	\renewcommand{\ce}{-0.24}
	\draw[curve] plot[samples=50, smooth, domain=\ta:\tb] (\x, {\polyfour});

	% on [2.8,4]
	\renewcommand{\ta}{2.8}
	\renewcommand{\tb}{4.0}
	\renewcommand{\ca}{-0.152916666666}
	\renewcommand{\cb}{-0.8}
	\renewcommand{\cc}{-0.1}
	\renewcommand{\cd}{1.573333333333333}
	\renewcommand{\ce}{-0.73}
	\draw[curve] plot[samples=50, smooth, domain=\ta:\tb] (\x, {\polyfour});


    % derivative of initial condition

    % on [-4.0,-3.25]
    \renewcommand{\ta}{-4.0}
    \renewcommand{\tb}{-3.25}
    \renewcommand{\ca}{1.7}
    \renewcommand{\cb}{-10.6}
    \renewcommand{\cc}{8.4}
    \draw[deriv,cadlag] plot[samples=50, smooth, domain=\ta:\tb] (\x, {\polytwo});

    % on [-3.25,-2]
    \renewcommand{\ta}{-3.25}
    \renewcommand{\tb}{-2.0}
    \renewcommand{\ca}{-1.125}
    \renewcommand{\cb}{8.7}
    \renewcommand{\cc}{-6.375}
    \draw[deriv,cadlag] plot[samples=50, smooth, domain=\ta:\tb] (\x, {\polytwo});

    % on [-2,-1.2]
    \renewcommand{\ta}{-2.0}
    \renewcommand{\tb}{-1.2}
    \renewcommand{\ca}{-0.24}
    \renewcommand{\cb}{-0.64}
    \renewcommand{\cc}{2.88}
    \draw[deriv,cadlag] plot[samples=50, smooth, domain=\ta:\tb] (\x, {\polytwo});

    % on [-1.2,0]
    \renewcommand{\ta}{-1.2}
    \renewcommand{\tb}{0.0}
    \renewcommand{\ca}{0.2}
    \renewcommand{\cb}{-9.44}
    \renewcommand{\cc}{8.76}
    \draw[deriv,cadlag] plot[samples=50, smooth, domain=\ta:\tb] (\x, {\polytwo});

	% term value
    \draw[term,cadlag] (0.0,2.7424) -- (0.01,2.7005) -- (0.02,2.6586) -- (0.03,2.6168) -- (0.04,2.5752) -- (0.05,2.5338) -- (0.06,2.4927) -- (0.07,2.452) -- (0.08,2.4117) -- (0.09,2.3719) -- (0.1,2.3327) -- (0.11,2.2942) -- (0.12,2.2565) -- (0.13,2.2195) -- (0.14,2.1834) -- (0.15,2.1483) -- (0.16,2.1142) -- (0.17,2.0811) -- (0.18,2.0492) -- (0.19,2.0185) -- (0.2,1.9891) -- (0.21,1.9611) -- (0.22,1.9344) -- (0.23,1.9093) -- (0.24,1.8856);

    \draw[term,leftp] (0.25,1.1636) -- (0.26,1.1407) -- (0.27,1.1188) -- (0.28,1.0982) -- (0.29,1.0787) -- (0.3,1.0603) -- (0.31,1.0432) -- (0.32,1.0273) -- (0.33,1.0126) -- (0.34,0.9992) -- (0.35,0.987) -- (0.36,0.9761) -- (0.37,0.9664) -- (0.38,0.9581) -- (0.39,0.951) -- (0.4,0.9452) -- (0.41,0.9408) -- (0.42,0.9376) -- (0.43,0.9358) -- (0.44,0.9353) -- (0.45,0.9362) -- (0.46,0.9384) -- (0.47,0.942) -- (0.48,0.9469) -- (0.49,0.9532) -- (0.5,0.9609) -- (0.51,0.9699) -- (0.52,0.9803) -- (0.53,0.9921) -- (0.54,1.0052) -- (0.55,1.0198) -- (0.56,1.0357) -- (0.57,1.053) -- (0.58,1.0717) -- (0.59,1.0918) -- (0.6,1.1132);
    
    \draw[term,leftp] (0.6,1.1132) -- (0.61,1.112) -- (0.62,1.1106) -- (0.63,1.1092) -- (0.64,1.1078) -- (0.65,1.1062) -- (0.66,1.1045) -- (0.67,1.1027) -- (0.68,1.1008) -- (0.69,1.0988) -- (0.7,1.0966) -- (0.71,1.0943) -- (0.72,1.0919) -- (0.73,1.0893) -- (0.74,1.0865) -- (0.75,1.0836);
    
    \draw[term,cadlag] (0.75,1.0836) -- (0.76,1.0888) -- (0.77,1.0937) -- (0.78,1.0985) -- (0.79,1.103) -- (0.8,1.1074) -- (0.81,1.1115) -- (0.82,1.1154) -- (0.83,1.1192) -- (0.84,1.1227) -- (0.85,1.126) -- (0.86,1.1291) -- (0.87,1.132) -- (0.88,1.1347) -- (0.89,1.1372) -- (0.9,1.1394) -- (0.91,1.1415) -- (0.92,1.1434) -- (0.93,1.145) -- (0.94,1.1465) -- (0.95,1.1477) -- (0.96,1.1487) -- (0.97,1.1496) -- (0.98,1.1502) -- (0.99,1.1506) -- (1.0,1.1508) -- (1.01,1.1508) -- (1.02,1.1506) -- (1.03,1.1502) -- (1.04,1.1495) -- (1.05,1.1487) -- (1.06,1.1477) -- (1.07,1.1465) -- (1.08,1.145) -- (1.09,1.1434) -- (1.1,1.1416) -- (1.11,1.1396) -- (1.12,1.1373) -- (1.13,1.1349) -- (1.14,1.1323) -- (1.15,1.1295) -- (1.16,1.1265) -- (1.17,1.1233) -- (1.18,1.1199) -- (1.19,1.1163) -- (1.2,1.1126) -- (1.21,1.1086) -- (1.22,1.1045) -- (1.23,1.1002) -- (1.24,1.0957) -- (1.25,1.091) -- (1.26,1.0861) -- (1.27,1.0811) -- (1.28,1.0759) -- (1.29,1.0705) -- (1.3,1.065) -- (1.31,1.0593) -- (1.32,1.0534) -- (1.33,1.0473) -- (1.34,1.0411) -- (1.35,1.0348) -- (1.36,1.0282) -- (1.37,1.0216) -- (1.38,1.0147) -- (1.39,1.0077) -- (1.4,1.0006) -- (1.41,0.9933) -- (1.42,0.9859) -- (1.43,0.9783) -- (1.44,0.9706) -- (1.45,0.9628) -- (1.46,0.9548) -- (1.47,0.9467) -- (1.48,0.9385) -- (1.49,0.9302);
    
    \draw[term,cadlag] (1.5,0.4217) -- (1.51,0.4006) -- (1.52,0.3795) -- (1.53,0.3586) -- (1.54,0.3377) -- (1.55,0.317) -- (1.56,0.2964) -- (1.57,0.276) -- (1.58,0.2558) -- (1.59,0.2358) -- (1.6,0.216) -- (1.61,0.1965) -- (1.62,0.1772) -- (1.63,0.1583) -- (1.64,0.1396) -- (1.65,0.1213) -- (1.66,0.1034) -- (1.67,0.0858) -- (1.68,0.0687) -- (1.69,0.0519) -- (1.7,0.0357) -- (1.71,0.0198) -- (1.72,0.0045) -- (1.73,-0.0103) -- (1.74,-0.0246) -- (1.75,-0.0383) -- (1.76,-0.0514) -- (1.77,-0.064) -- (1.78,-0.0759) -- (1.79,-0.0871);
    
    \draw[term,leftp] (1.8,2.2023) -- (1.81,2.1786) -- (1.82,2.1548) -- (1.83,2.1309) -- (1.84,2.1069) -- (1.85,2.0829) -- (1.86,2.0589) -- (1.87,2.035) -- (1.88,2.0111) -- (1.89,1.9874) -- (1.9,1.9638) -- (1.91,1.9403) -- (1.92,1.9171) -- (1.93,1.8941) -- (1.94,1.8714) -- (1.95,1.849) -- (1.96,1.827) -- (1.97,1.8053) -- (1.98,1.784) -- (1.99,1.7631) -- (2.0,1.7426);
    
    \draw[term,leftp] (2.0,1.7426) -- (2.01,1.7249) -- (2.02,1.7079) -- (2.03,1.6914) -- (2.04,1.6757) -- (2.05,1.6606) -- (2.06,1.6463) -- (2.07,1.6327) -- (2.08,1.6199) -- (2.09,1.6078) -- (2.1,1.5966) -- (2.11,1.5862) -- (2.12,1.5766) -- (2.13,1.568) -- (2.14,1.5602) -- (2.15,1.5533) -- (2.16,1.5473) -- (2.17,1.5423) -- (2.18,1.5382) -- (2.19,1.5351) -- (2.2,1.533) -- (2.21,1.5319) -- (2.22,1.5318) -- (2.23,1.5328) -- (2.24,1.5347) -- (2.25,1.5377) -- (2.26,1.5418) -- (2.27,1.547) -- (2.28,1.5532) -- (2.29,1.5605) -- (2.3,1.5689);
    
    \draw[term,leftp] (2.3,1.5689) -- (2.31,1.5544) -- (2.32,1.5395) -- (2.33,1.5242) -- (2.34,1.5086) -- (2.35,1.4927) -- (2.36,1.4765) -- (2.37,1.4599) -- (2.38,1.443) -- (2.39,1.4257) -- (2.4,1.4081) -- (2.41,1.3902) -- (2.42,1.372) -- (2.43,1.3534) -- (2.44,1.3345) -- (2.45,1.3153) -- (2.46,1.2957) -- (2.47,1.2758) -- (2.48,1.2555) -- (2.49,1.2349) -- (2.5,1.2139) -- (2.51,1.1925) -- (2.52,1.1707) -- (2.53,1.1486) -- (2.54,1.126) -- (2.55,1.1031); 
    
    \draw[term,leftp] (2.55,1.1031) -- (2.56,1.0879) -- (2.57,1.0724) -- (2.58,1.0564) -- (2.59,1.04) -- (2.6,1.0231) -- (2.61,1.0058) -- (2.62,0.988) -- (2.63,0.9698) -- (2.64,0.9512) -- (2.65,0.9321) -- (2.66,0.9125) -- (2.67,0.8925) -- (2.68,0.8721) -- (2.69,0.8512) -- (2.7,0.8298) -- (2.71,0.808) -- (2.72,0.7857) -- (2.73,0.763) -- (2.74,0.7398) -- (2.75,0.7161) -- (2.76,0.6919) -- (2.77,0.6673) -- (2.78,0.6422) -- (2.79,0.6167) -- (2.8,0.5906);
    
    \draw[term,cadlag] (2.8,0.5906) -- (2.81,0.5676) -- (2.82,0.5443) -- (2.83,0.521) -- (2.84,0.4974) -- (2.85,0.4738) -- (2.86,0.45) -- (2.87,0.4261) -- (2.88,0.4021) -- (2.89,0.378) -- (2.9,0.3538) -- (2.91,0.3296) -- (2.92,0.3053) -- (2.93,0.281) -- (2.94,0.2566) -- (2.95,0.2323) -- (2.96,0.2079) -- (2.97,0.1836) -- (2.98,0.1592) -- (2.99,0.1349) -- (3.0,0.1107) -- (3.01,0.0865) -- (3.02,0.0624) -- (3.03,0.0383) -- (3.04,0.0144) -- (3.05,-0.0095) -- (3.06,-0.0332) -- (3.07,-0.0568) -- (3.08,-0.0803) -- (3.09,-0.1036) -- (3.1,-0.1267) -- (3.11,-0.1497) -- (3.12,-0.1725) -- (3.13,-0.1951) -- (3.14,-0.2175) -- (3.15,-0.2397) -- (3.16,-0.2616) -- (3.17,-0.2833) -- (3.18,-0.3047) -- (3.19,-0.3259) -- (3.2,-0.3468) -- (3.21,-0.3675) -- (3.22,-0.3878) -- (3.23,-0.4078) -- (3.24,-0.4275) -- (3.25,-0.4468) -- (3.26,-0.4659) -- (3.27,-0.4845) -- (3.28,-0.5028) -- (3.29,-0.5207) -- (3.3,-0.5383) -- (3.31,-0.5554) -- (3.32,-0.5722) -- (3.33,-0.5885) -- (3.34,-0.6044) -- (3.35,-0.6198) -- (3.36,-0.6348) -- (3.37,-0.6493) -- (3.38,-0.6634) -- (3.39,-0.677) -- (3.4,-0.6901) -- (3.41,-0.7027) -- (3.42,-0.7147) -- (3.43,-0.7263) -- (3.44,-0.7373) -- (3.45,-0.7478) -- (3.46,-0.7577) -- (3.47,-0.767) -- (3.48,-0.7758) -- (3.49,-0.784);

    \draw[term,leftp] (3.5,1.5085) -- (3.51,1.4877) -- (3.52,1.4667) -- (3.53,1.4456) -- (3.54,1.4242) -- (3.55,1.4027) -- (3.56,1.381) -- (3.57,1.3592) -- (3.58,1.3374) -- (3.59,1.3154) -- (3.6,1.2934) -- (3.61,1.2714) -- (3.62,1.2494) -- (3.63,1.2273) -- (3.64,1.2053) -- (3.65,1.1834) -- (3.66,1.1615) -- (3.67,1.1396) -- (3.68,1.1179) -- (3.69,1.0962) -- (3.7,1.0747) -- (3.71,1.0533) -- (3.72,1.0321) -- (3.73,1.011) -- (3.74,0.9901) -- (3.75,0.9694) -- (3.76,0.9489) -- (3.77,0.9286) -- (3.78,0.9085) -- (3.79,0.8886) -- (3.8,0.869);

    \draw[term,cadlag] (3.8,0.869) -- (3.81,0.8519) -- (3.82,0.8351) -- (3.83,0.8187) -- (3.84,0.8027) -- (3.85,0.7871) -- (3.86,0.7719) -- (3.87,0.757) -- (3.88,0.7426) -- (3.89,0.7285) -- (3.9,0.7149) -- (3.91,0.7016) -- (3.92,0.6887) -- (3.93,0.6763) -- (3.94,0.6642) -- (3.95,0.6525) -- (3.96,0.6412) -- (3.97,0.6303) -- (3.98,0.6198) -- (3.99,0.6096) -- (4.0,0.5998);

    % term deriv
    \draw[termderiv,cadlag] (0.0,-3.1533) -- (0.01,-3.1551) -- (0.02,-3.1512) -- (0.03,-3.1415) -- (0.04,-3.1263) -- (0.05,-3.1054) -- (0.06,-3.0789) -- (0.07,-3.0469) -- (0.08,-3.0094) -- (0.09,-2.9664) -- (0.1,-2.9181) -- (0.11,-2.8643) -- (0.12,-2.8053) -- (0.13,-2.7409) -- (0.14,-2.6713) -- (0.15,-2.5965) -- (0.16,-2.5165) -- (0.17,-2.4313) -- (0.18,-2.3411) -- (0.19,-2.2458) -- (0.2,-2.1455) -- (0.21,-2.0403) -- (0.22,-1.9301) -- (0.23,-1.815) -- (0.24,-1.695);
    
    \draw[termderiv,cadlag] (0.25,-2.1952) -- (0.26,-2.0807) -- (0.27,-1.9651) -- (0.28,-1.8487) -- (0.29,-1.7315) -- (0.3,-1.6134) -- (0.31,-1.4946) -- (0.32,-1.375) -- (0.33,-1.2547) -- (0.34,-1.1338) -- (0.35,-1.0122) -- (0.36,-0.8901) -- (0.37,-0.7675) -- (0.38,-0.6443) -- (0.39,-0.5207) -- (0.4,-0.3967) -- (0.41,-0.2723) -- (0.42,-0.1476) -- (0.43,-0.0225) -- (0.44,0.1028) -- (0.45,0.2283) -- (0.46,0.3539) -- (0.47,0.4797) -- (0.48,0.6056) -- (0.49,0.7316) -- (0.5,0.8576) -- (0.51,0.9836) -- (0.52,1.1094) -- (0.53,1.2352) -- (0.54,1.3609) -- (0.55,1.4864) -- (0.56,1.6116) -- (0.57,1.7366) -- (0.58,1.8613) -- (0.59,1.9856);
    
    \draw[termderiv,cadlag] (0.6,0.3096) -- (0.61,0.2906) -- (0.62,0.2715) -- (0.63,0.2522) -- (0.64,0.2327) -- (0.65,0.2129) -- (0.66,0.1929) -- (0.67,0.1724) -- (0.68,0.1516) -- (0.69,0.1304) -- (0.7,0.1087) -- (0.71,0.0865) -- (0.72,0.0637) -- (0.73,0.0404) -- (0.74,0.0165);
    
    \draw[termderiv,cadlag] (0.75,0.6919) -- (0.76,0.6692) -- (0.77,0.6463) -- (0.78,0.6234) -- (0.79,0.6002) -- (0.8,0.577) -- (0.81,0.5536) -- (0.82,0.5301) -- (0.83,0.5065) -- (0.84,0.4828) -- (0.85,0.459) -- (0.86,0.4351) -- (0.87,0.4111) -- (0.88,0.3871) -- (0.89,0.3629) -- (0.9,0.3387) -- (0.91,0.3145) -- (0.92,0.2902) -- (0.93,0.2658) -- (0.94,0.2414) -- (0.95,0.217) -- (0.96,0.1926) -- (0.97,0.1681) -- (0.98,0.1436) -- (0.99,0.1191) -- (1.0,0.0947) -- (1.01,0.0702) -- (1.02,0.0457) -- (1.03,0.0213) -- (1.04,-0.0031) -- (1.05,-0.0275) -- (1.06,-0.0519) -- (1.07,-0.0761) -- (1.08,-0.1004) -- (1.09,-0.1246) -- (1.1,-0.1487) -- (1.11,-0.1727) -- (1.12,-0.1967) -- (1.13,-0.2206) -- (1.14,-0.2444) -- (1.15,-0.2681) -- (1.16,-0.2917) -- (1.17,-0.3151) -- (1.18,-0.3385) -- (1.19,-0.3617) -- (1.2,-0.3848) -- (1.21,-0.4078) -- (1.22,-0.4306) -- (1.23,-0.4533) -- (1.24,-0.4758) -- (1.25,-0.4981) -- (1.26,-0.5203) -- (1.27,-0.5423) -- (1.28,-0.5641) -- (1.29,-0.5857) -- (1.3,-0.6072) -- (1.31,-0.6284) -- (1.32,-0.6494) -- (1.33,-0.6702) -- (1.34,-0.6908) -- (1.35,-0.7112) -- (1.36,-0.7313) -- (1.37,-0.7511) -- (1.38,-0.7708) -- (1.39,-0.7901) -- (1.4,-0.8092) -- (1.41,-0.8281) -- (1.42,-0.8467) -- (1.43,-0.8649) -- (1.44,-0.8829) -- (1.45,-0.9006) -- (1.46,-0.918) -- (1.47,-0.9351) -- (1.48,-0.9519) -- (1.49,-0.9684); 
    
    \draw[termderiv,cadlag] (1.5,-2.4245) -- (1.51,-2.415) -- (1.52,-2.4035) -- (1.53,-2.3899) -- (1.54,-2.3743) -- (1.55,-2.3566) -- (1.56,-2.3368) -- (1.57,-2.3149) -- (1.58,-2.2909) -- (1.59,-2.2648) -- (1.6,-2.2366) -- (1.61,-2.2063) -- (1.62,-2.1739) -- (1.63,-2.1393) -- (1.64,-2.1026) -- (1.65,-2.0638) -- (1.66,-2.0228) -- (1.67,-1.9796) -- (1.68,-1.9343) -- (1.69,-1.8869) -- (1.7,-1.8372) -- (1.71,-1.7854) -- (1.72,-1.7314) -- (1.73,-1.6752) -- (1.74,-1.6168) -- (1.75,-1.5562) -- (1.76,-1.4934) -- (1.77,-1.4284) -- (1.78,-1.3611) -- (1.79,-1.2916);
    
    \draw[termderiv,cadlag] (1.8,-2.0399) -- (1.81,-2.0556) -- (1.82,-2.0684) -- (1.83,-2.0783) -- (1.84,-2.0855) -- (1.85,-2.0899) -- (1.86,-2.0915) -- (1.87,-2.0904) -- (1.88,-2.0866) -- (1.89,-2.0802) -- (1.9,-2.0712) -- (1.91,-2.0596) -- (1.92,-2.0455) -- (1.93,-2.0288) -- (1.94,-2.0096) -- (1.95,-1.988) -- (1.96,-1.964) -- (1.97,-1.9377) -- (1.98,-1.9089) -- (1.99,-1.8779);
    
    \draw[termderiv,cadlag] (2.0,-1.3445) -- (2.01,-1.2964) -- (2.02,-1.2463) -- (2.03,-1.1942) -- (2.04,-1.1402) -- (2.05,-1.0842) -- (2.06,-1.0264) -- (2.07,-0.9669) -- (2.08,-0.9056) -- (2.09,-0.8427) -- (2.1,-0.7781) -- (2.11,-0.7119) -- (2.12,-0.6442) -- (2.13,-0.5751) -- (2.14,-0.5045) -- (2.15,-0.4325) -- (2.16,-0.3592) -- (2.17,-0.2847) -- (2.18,-0.2089) -- (2.19,-0.1319) -- (2.2,-0.0539) -- (2.21,0.0252) -- (2.22,0.1054) -- (2.23,0.1865) -- (2.24,0.2685) -- (2.25,0.3514) -- (2.26,0.4351) -- (2.27,0.5196) -- (2.28,0.6047) -- (2.29,0.6905);
    
    \draw[termderiv,cadlag] (2.3,-1.023) -- (2.31,-1.0786) -- (2.32,-1.1333) -- (2.33,-1.1872) -- (2.34,-1.2404) -- (2.35,-1.2929) -- (2.36,-1.3449) -- (2.37,-1.3962) -- (2.38,-1.4471) -- (2.39,-1.4975) -- (2.4,-1.5474) -- (2.41,-1.597) -- (2.42,-1.6463) -- (2.43,-1.6954) -- (2.44,-1.7442) -- (2.45,-1.7929) -- (2.46,-1.8414) -- (2.47,-1.8899) -- (2.48,-1.9385) -- (2.49,-1.987) -- (2.5,-2.0357) -- (2.51,-2.0845) -- (2.52,-2.1335) -- (2.53,-2.1827) -- (2.54,-2.2323);
    
    \draw[termderiv,leftp] (2.55,-1.5822) -- (2.56,-1.6299) -- (2.57,-1.6774) -- (2.58,-1.7248) -- (2.59,-1.7721) -- (2.6,-1.8193) -- (2.61,-1.8664) -- (2.62,-1.9133) -- (2.63,-1.9602) -- (2.64,-2.0069) -- (2.65,-2.0535) -- (2.66,-2.1001) -- (2.67,-2.1465) -- (2.68,-2.1929) -- (2.69,-2.2392) -- (2.7,-2.2854) -- (2.71,-2.3316) -- (2.72,-2.3777) -- (2.73,-2.4237) -- (2.74,-2.4696) -- (2.75,-2.5155) -- (2.76,-2.5614) -- (2.77,-2.6072) -- (2.78,-2.653) -- (2.79,-2.6988) -- (2.8,-2.7445);
    
    \draw[termderiv,cadlag] (2.8,-2.7445) -- (2.81,-2.7661) -- (2.82,-2.7863) -- (2.83,-2.805) -- (2.84,-2.8222) -- (2.85,-2.838) -- (2.86,-2.8523) -- (2.87,-2.8651) -- (2.88,-2.8765) -- (2.89,-2.8865) -- (2.9,-2.895) -- (2.91,-2.902) -- (2.92,-2.9076) -- (2.93,-2.9118) -- (2.94,-2.9146) -- (2.95,-2.9159) -- (2.96,-2.9158) -- (2.97,-2.9143) -- (2.98,-2.9113) -- (2.99,-2.907) -- (3.0,-2.9012) -- (3.01,-2.894) -- (3.02,-2.8854) -- (3.03,-2.8755) -- (3.04,-2.8641) -- (3.05,-2.8513) -- (3.06,-2.8372) -- (3.07,-2.8216) -- (3.08,-2.8047) -- (3.09,-2.7864) -- (3.1,-2.7668) -- (3.11,-2.7457) -- (3.12,-2.7233) -- (3.13,-2.6996) -- (3.14,-2.6744) -- (3.15,-2.6479) -- (3.16,-2.6201) -- (3.17,-2.5909) -- (3.18,-2.5604) -- (3.19,-2.5285) -- (3.2,-2.4953) -- (3.21,-2.4608) -- (3.22,-2.4249) -- (3.23,-2.3877) -- (3.24,-2.3492) -- (3.25,-2.3093) -- (3.26,-2.2682) -- (3.27,-2.2257) -- (3.28,-2.1819) -- (3.29,-2.1368) -- (3.3,-2.0904) -- (3.31,-2.0428) -- (3.32,-1.9938) -- (3.33,-1.9435) -- (3.34,-1.892) -- (3.35,-1.8391) -- (3.36,-1.785) -- (3.37,-1.7296) -- (3.38,-1.6729) -- (3.39,-1.615) -- (3.4,-1.5558) -- (3.41,-1.4953) -- (3.42,-1.4336) -- (3.43,-1.3706) -- (3.44,-1.3064) -- (3.45,-1.2409) -- (3.46,-1.1742) -- (3.47,-1.1062) -- (3.48,-1.037) -- (3.49,-0.9666);
    
    \draw[termderiv,cadlag] (3.5,-1.7149) -- (3.51,-1.7317) -- (3.52,-1.7466) -- (3.53,-1.7597) -- (3.54,-1.7711) -- (3.55,-1.7807) -- (3.56,-1.7887) -- (3.57,-1.795) -- (3.58,-1.7998) -- (3.59,-1.803) -- (3.6,-1.8047) -- (3.61,-1.8049) -- (3.62,-1.8037) -- (3.63,-1.8012) -- (3.64,-1.7973) -- (3.65,-1.7921) -- (3.66,-1.7857) -- (3.67,-1.7781) -- (3.68,-1.7694) -- (3.69,-1.7595) -- (3.7,-1.7485) -- (3.71,-1.7366) -- (3.72,-1.7236) -- (3.73,-1.7097) -- (3.74,-1.695) -- (3.75,-1.6793) -- (3.76,-1.6629) -- (3.77,-1.6457) -- (3.78,-1.6277) -- (3.79,-1.6091);
    
    \draw[termderiv,cadlag] (3.8,-1.0899) -- (3.81,-1.0576) -- (3.82,-1.0248) -- (3.83,-0.9918) -- (3.84,-0.9585) -- (3.85,-0.9251) -- (3.86,-0.8915) -- (3.87,-0.8578) -- (3.88,-0.8242) -- (3.89,-0.7906) -- (3.9,-0.7571) -- (3.91,-0.7238) -- (3.92,-0.6908) -- (3.93,-0.658) -- (3.94,-0.6256) -- (3.95,-0.5937) -- (3.96,-0.5622) -- (3.97,-0.5313) -- (3.98,-0.501) -- (3.99,-0.4713);


\end{tikzpicture}

        \caption{Plot for Example~\ref{ex:differential-lemma}, initial condition and solution of DDE (dashed), derivatives (dotted), value of term (solid) and the derivative of the term (dash-dotted).}
        \label{fig:differential-lemma}
    \end{figure}

    \begin{example}\label{ex:differential-lemma}
        As an example for the construction of the partition in Proof~\ref{prf:differential-lemma}, consider the s-term $\bstrm(s)\equiv x+\x[-3.5]+\x[s]$ together with the DDE $\D{x}=\x[-4]$.
        Let
        \begin{equation*}
            \mathcal{Z} = \Set{-4,-3.25,-2,-1.2,0}
        \end{equation*}
        be the partition of some initial condition.
        Choosing $\past=-1.8$ for $s$, we obtain
        \begin{equation*}
            \mathcal{Z}_\bstrm^\past = \set{0,0.25,0.6,0.75,1.5,1.8,2,2.3,2.55,2.8,3.5,3.8,4}
        \end{equation*}
        when we restrict the evolution to $\compactum{0}{4}$.
        Figure~\ref{fig:differential-lemma} depicts an example for the piecewise continuous differentiability of the term's evolution, given some initial condition.
        % \begin{equation*}
        %     \D{\left(\hs{x+\x[c]+\x[s]\geq 0}\right)} \equiv \hs{\D{x}+\Dx[c]+\Dx[s]\geq 0}
        % \end{equation*}
        
        % then $g_s\in\Cnpw[1]{\compactum{0}{r}}{\R}$ along traj de dde
        % \begin{equation*}
        %     g_s(t)=\ivaluation{\iconcat[state=\trajectory(t),assign=s]{\IddL}}{\bstrm} = \ivaluation{\iconcat[state=\trajectory(t),assign=s]{\IddL}}{\x[0]} + \ivaluation{\iconcat[state=\trajectory(t),assign=s]{\IddL}}{\x[c]} + \ivaluation{\iconcat[state=\trajectory(t),assign=s]{\IddL}}{\x[s]}
        % \end{equation*}
        % each summand diffable in $(\hat{t}_j,\hat{t}_{j+1})$ and
        % \begin{equation*}
        %     \D{g}_s(t)= \ivaluation{\iconcat[state=\trajectory(t),assign=s]{\IddL}}{\Dx[0]} + \ivaluation{\iconcat[state=\trajectory(t),assign=s]{\IddL}}{\Dx[c]} + \ivaluation{\iconcat[state=\trajectory(t),assign=s]{\IddL}}{\Dx[s]}
        % \end{equation*}
        % \begin{equation*}
        %     \lim_{\zeta\downto\hat{t}_j} \D{g}_s(\zeta)=
        % \end{equation*}
    \end{example}

    % FIXME: replace duration r
    \begin{lemma}[Differential assignment]\label{lm:diff-assignment}
        Let $\trajectory\from\compactum{0}{\duration}\to\states$ be a trajectory satisfying a DDE for any duration $\duration\geq 0$, i.e.
        $\interpret,\trajectory\models(\D{x}=\astrm\land\ivr)$.
        Then it holds:
        \begin{equation*}
            \interpret,\trajectory,\past\models\asfml(s) \lbisubjunct \trajectory(\zeta)\in\imodel{\IddL}{\dbox{\Dupdate{\Dumod{\D{x}}{\astrm}}}{\asfml(s)}}  
        \end{equation*}
    \end{lemma}
    \begin{proof}
        Let $\zeta\in\compactum{0}{\duration}$. It is $\trajectory(\zeta)\in\imodel{\IddL}{\Dx[0]=\astrm}$ and $\trajectory(\zeta)\in\imodel{\IddL}{\ivr}$, which means $\trajectory(\zeta)(\D{x})(0)=\ivaluation{\iconcat[state=\trajectory(\zeta)]{\IddL}}{\astrm}$, since $\astrm$ is independent of $s$.
        By Definition~\ref{def:semantic-dHP}(\ref{itm:sem-dHP-assgn}) of the assignment's semantics, this implies $(\trajectory(\zeta),\bsstate)\in\ireachability{\IddL}{\Dupdate{\Dumod{\D{x}}{\astrm}}}$ if and only if $\bsstate=\trajectory(\zeta)$.
        Finally, this implies the equivalence
        \begin{align*}
            \trajectory(\zeta)\in\imodel{\IddL}{\asfml(s)} &\lbisubjunct
            \mforall{\bsstate\in\states}\holds\left((\trajectory(\zeta),\bsstate)\in\ireachability{\IddL}{\Dupdate{\Dumod{\D{x}}{\astrm}}} \limply \bsstate\in\imodel{\IddL}{\asfml(s)}\right)\\
            &\lbisubjunct \trajectory(\zeta)\in\imodel{\IddL}{\dbox{\Dupdate{\Dumod{\D{x}}{\astrm}}}{\asfml(s)}}
        \end{align*}
    \end{proof}

% \section{Static Semantics}
%     \label{sec:static-semantics}

%     \subsection{History Horizon}
%         \label{sec:history-horizon}

%         \begin{definition}[History Horizon]
%             The \emph{history horizon} is a function
%             \begin{equation*}
%                 \HHfml\from \ddLformulas \to \nonposR
%             \end{equation*}
%             which assigns to each \ddL formula the earliest point in time it references to. It limits the time interval of the state space $T\in$.
%             The history horizon depends on all occurences of $\x[s]$ and $\x[c]$ in the formula.

%             It is defined inductively for formulas by:
%             \begin{enumerate}
%                 \item $\HHfml(\hs{\astrm\geq\bstrm}) = \max\set{\HHtrm(\astrm),\HHtrm(\bstrm)}$
%                 \item $\HHfml(\hs{p(\range{\istrm{1}}{\istrm{k}})}) = \max\set{\range{\HHtrm(\istrm{1})}{\HHtrm(\istrm{k})}}$
%                 \item $\HHfml(\contextapp{C}{\asfml}) = $
%                 \item $\HHfml(\lnot\asfml) = \HHfml(\asfml)$
%                 \item $\HHfml(\asfml\land\bsfml) = \max\set{\HHfml(\asfml),\HHfml(\bsfml)}$
%                 \item $\HHfml(\lforall{x}{\asfml}) = \HHfml(\asfml)$
%                 \item $\HHfml(\lexists{x}{\asfml}) = \HHfml(\asfml)$
%                 \item $\HHfml(\dbox{\asprg}{\asfml}) = \max\set{\HHprg(\asprg),\HHfml(\asfml)}$
%                 \item $\HHfml(\ddiamond{\asprg}{\asfml}) = \max\set{\HHprg(\asprg),\HHfml(\asfml)}$
%             \end{enumerate}
%             depending on terms
%             \begin{enumerate}
%                 \item $\HHtrm(\x[s]) = 0$
%                 \item $\HHtrm(\Dx[s]) = 0$
%                 \item $\HHtrm(\x[c]) = \abs{c}$
%                 \item $\HHtrm(\Dx[c]) = \abs{c}$
%                 \item $\HHtrm(c) = 0$
%                 \item $\HHtrm(f(\range{\istrm{1}}{\istrm{k}})) = \max\set{\range{\HHtrm(\istrm{1})}{\HHtrm(\istrm{k})}}$
%                 \item $\HHtrm(\astrm + \bstrm) = \max\set{\HHtrm(\astrm),\HHtrm(\bstrm)}$
%                 \item $\HHtrm(\astrm \cdot \bstrm) = \max\set{\HHtrm(\astrm),\HHtrm(\bstrm)}$
%             \end{enumerate}
%             and \HPs
%             \begin{enumerate}
%                 \item $\HHprg(a) = 0$
%                 \item $\HHprg(\hupdate{\humod{x}{\astrm}}) = 0$
%                 \item $\HHprg(\Dupdate{\Dumod{\D{x}}{\astrm}}) = 0$
%                 \item $\HHprg(\htest{\asfmlfolR}) = 0$
%                 \item $\HHprg(\hchoice{\asprg}{\bsprg}) = \max\set{\HHprg(\asprg),\HHprg(\bsprg)}$
%                 % TODO: replace ; in HPs
%                 \item $\HHprg(\asprg;\bsprg) = \max\set{\HHprg(\asprg),\HHprg(\bsprg)}$
%                 \item $\HHprg(\hrepeat{\asprg}) = \HHprg(\asprg)$
%                 \item $\HHprg(\hevolvein{\D{x}=\astrm(-\tau)}{\ivr}) = \HHtrm(\astrm(-\tau))$
%             \end{enumerate}
%             For formulae as
            
%             T in forall: comp, pred
%             minimum of T in forall parts of subformulas: quantifier, not and
%             forall, exists?
%             modalities: max $\tau$ in DDE and of formula
%         \end{definition}
%         %We need to use $\min$, since $\HH$ is a non-positive number. It is the biggest in absolute value.
    
%     \subsection{Variable Binding}
%         \label{sec:variable-binding}

%         \begin{definition}[Free variable]
%             for terms: $\freevars{\astrm}\subseteq\allvars\cup\diffvars\cup\set{s}$, variables that occur in term
%             \begin{align*}
%                 \freevars{\x[s]} &= \set{x,s}\\
%                 \freevars{\Dx[s]} &= \set{\D{x},s}\\
%                 \freevars{\x[b]} &= \set{x}\\
%                 \freevars{\Dx[b]} &= \set{\D{x}}\\
%                 \freevars{c} &= \emptyset\\
%                 \freevars{f(\range{\istrm{1}}{\istrm{k}})} &= \freevars{\istrm{1}} \cup\cdots\cup \freevars{\istrm{k}}\\
%                 \freevars{\astrm + \bstrm} = \freevars{\astrm \cdot \bstrm} &= \freevars{\astrm}\cup\freevars{\bstrm}\\
%                 \freevars{\D{(\astrm)}} &= \freevars{\astrm}\cup\D{\freevars{\astrm}}
%             \end{align*}
                
%         \end{definition} 

%         only bound variables can change in the execution of a \HP   

%     \subsection{Well-defined Formulae}
%         \label{sec:well-definedness}
    
%         $s$ must not be free
%         and

%         \begin{definition}[Well-defined formula]
%             obeys syntactic definition
%             premisse defines element od statespace for all occuring variables and diffs
%         \end{definition}

%         \begin{example}
%             \begin{equation*}
%                 % FIXME: notation/syntax: ausgeklammertes forall in formulas
%                 \dbox{\hupdate{\humod{x}{\x[-3]^2}};\hevolvein{\D{x}=\x[-\tau]+2x}{(x\geq 0)}}{\hsc{0\leq \x[s] \land \x[s]\leq\x[-5]}}
%             \end{equation*}
%             Hence the history horizon needs to be set to
%             \begin{equation*}
%                 T=\max\set{\max\set{3,\tau},\max\set{0,5}}.
%             \end{equation*}
%         \end{example}

%         % TODO: feasible history horizon for a HP
%         \begin{lemma}
%             choosing the statespace according to history horizon of HP determines
%         \end{lemma}

%         \begin{example}
            
%         \end{example}


\chapter{Axiomatization and Proof Calculus}
\label{ch:axiomatization-proof-calculus}

Formulas of delay differential dynamic logic allow the specification of properties of hybrid programs with delay. Their truth value, whether a such formula is true or false, is determined by the semantics. However, finding the truth value only using the definition of the semantics is inpractical and tedious. As a more powerful means for this verification task, \ddL includes a proof calculus with rules based on axioms. These allow manipulations on a syntcatic level, without falling back on the semantics.
Moreover, they can be implemented in software for computer-aided verification.

\section{Axiomatization}
    \label{sec:axiomatization}

    % axioms: tautologies, syntactic transformation, equivalences

    % FIXME: correct source? \cite{Platzer12Complete,Platzer15Uniform}
    The axiomatization for \ddL presented here is based on the \dL axiomatization, as given in \cite{Platzer12Complete}.
    It is a first-order Hilbert calculus, using \emph{modus ponens} and \emph{$\forall$-generalization} as a basis.

    As opposed to an axiom schemata, which represents an infinte list of axioms by containing placeholders for concrete formulas and terms, we consider here for simplicity an axiom as a concrete formula.
    % FIXME: is s-formula allowed as axiom? or only without s?

    % FOLR or ddL formulas?
    all instances of valid formulas of first-order real arithmetic are allowed as axiom

    Similar to differential forms for \dL \cite{Platzer15Uniform}, we also consider a differential form axiomatization of differential equations.

    The goal of transforming a \ddL formula into another formula by appliying the axioms is to eventually derive a first-order formula of real arithmetic, which is decideable by \emph{quantifier elimination} ($\QE$).

    The axioms listed in Figure~\ref{fig:axioms} are expressed in $\dbox{\cdot}$. Axioms with the dual operator $\ddiamond{\cdot}$ can be obtained using the duality relation (axiom~\irref{diamond}).

    % TODO: what about [a;b]p. is [a][b]p correct?
    \begin{figure}[t]
        \begin{calculuscollections}{\textwidth}
        % axioms
        \begin{calculus}
            \cinferenceRule[diamond|$\didia{\cdot}$]{diamond axiom}{
                \linferenceRule[equiv]{
                    \lnot\dbox{\asprg}{\lnot\asfml(s)}
                }{
                    \ddiamond{\asprg}{\asfml(s)}
                }
            }{}
            % TODO: holds iff semantic of not is complement
            \cinferenceRule[testb|$\dibox{\htest{}}$]{test axiom}{
                \linferenceRule[equiv]{
                    \big(\bsfml(s)\limply\asfml(s)\big)
                }{
                    \dbox{\htest{\bsfml(s)}}{\asfml(s)}
                }
            }{}
            \cinferenceRule[assignb|$\dibox{\assign}$]{assignment/substitution axiom}{
                \linferenceRule[equiv]{
                    \asfml(s,\astrm)
                }{
                    \dbox{\hupdate{\humod{x}{\astrm}}}{\asfml(s,\x[0])}
                }
            }{}
            % FIXME: do I need this axiom, problem of need specify initial condition
            % or need just for ODE
            % \cinferenceRule[evolveb|$\dibox{'}$]{}{
            %     % FIXME: syntax forall t invalid?
            %     \linferenceRule[equiv]{
            %         \lforall{t\geq 0}{\dbox{\hupdate{\humod{x}{y(t)}}}{\asfml(s)}}
            %     }{
            %         \dbox{\hevolve{\D{x}=\astrm}}{\asfml(s)}
            %     }
            % }{$\hevolve{\D{y}(t)=\astrm}$}

            \cinferenceRule[choiceb|$\dibox{\cup}$]{axiom of nondeterministic choice}{
                \linferenceRule[equiv]{
                    \dbox{\asprg}{\asfml(s)}\land\dbox{\bsprg}{\asfml(s)}
                }{
                    \dbox{\hchoice{\asprg}{\bsprg}}{\asfml(s)}
                }
            }{}
            % TODO: name for [&]
            % TODO: [&] for DDEs?
            % \cinferenceRule[constb|$\&$]{}{
            %     \linferenceRule[equiv]{
            %         \lforall{t_0=x_0}{\dbox{\hevolve{\D{x}=\astrm}}{(\dbox{\hevolve{\D{x}=-\astrm}}{(x_0\geq t_0\limply\ivr)}\limply\asfml)}}
            %     }{
            %         \dbox{\hevolvein{\D{x}=\astrm}{\ivr}}{\asfml}
            %     }
            % }{}
            \cinferenceRule[composeb|$\dibox{{;}}$]{composition axiom}{ % double {} for correct spacing around semicolon
                \linferenceRule[equiv]{
                    \dbox{\asprg}{\dbox{\bsprg}{\asfml(s)}}
                }{
                    \dbox{\asprg;\bsprg}{\asfml(s)}
                }
            }{}
            \cinferenceRule[iterateb|$\dibox{*}$]{iteration axiom}{
                \linferenceRule[equiv]{
                    \asfml(s)\land\dbox{\asprg}{\dbox{\hrepeat{\asprg}}{\asfml(s)}}
                }{
                    \dbox{\hrepeat{\asprg}}{\asfml(s)}
                }
            }{}
            \cinferenceRule[K|K]{modal modus ponens}{
                \linferenceRule[impl]{
                    \dbox{\asprg}{\big(\asfml(s)\limply\bsfml(s)\big)}
                }{
                    \big(\dbox{\asprg}{\asfml(s)}\limply\dbox{\asprg}{\bsfml(s)}\big)
                }
            }{}
            % TODO: proper name I-Axiom
            \cinferenceRule[I|I]{loop induction}{
                \linferenceRule[impl]{
                    \dbox{\hrepeat{\asprg}}{\big(\asfml(s)\limply\dbox{\asprg}{\asfml(s)}\big)}
                }{
                    \big(\asfml(s)\limply\dbox{\hrepeat{\asprg}}{\asfml(s)}\big)
                }
            }{}
            % FIXME: do I need C-Axiom ?
            % TODO: C-Axiom holds for DDEs?
            % \cinferenceRule[C|C]{loop convergence}{
            %     \linferenceRule[impl]{
            %         \dbox{\hrepeat{\asprg}}{\lforall{v>0 (\varphi(v)\limply\ddiamond{\asprg}{\varphi(v-1)})}}
            %     }{
            %         \lforall{v}{(\varphi(v)\limply\ddiamond{\hrepeat{\asprg}}{\lexists{v\leq 0}{\varphi(v)}})}\qquad
            %     }
            % }{$v\notin\asprg$}
            % FIXME: do I need B-axiom?
            \cinferenceRule[B|B]{Barcan}{
                \linferenceRule[impl]{
                    \lforall{x}{\dbox{\asprg}{\asfml(s)}}
                }{
                    \dbox{\asprg}{\lforall{x}{\asfml(s)}}
                }
            }{$x\notin\asprg$}
            % TODO: check V axiom with def of FV, BV, can adds to fml?
            \cinferenceRule[V|V]{vacuous}{
                \linferenceRule[impl]{
                    \asfml
                }{
                    \dbox{\asprg}{\asfml}
                }
            }{$\freevars{\asfml}\cap\boundvars{\asprg}=\emptyset$}
        \end{calculus}
        \qquad
        % proof rules
        \begin{calculus}
            \cinferenceRule[G|G]{Gödel's generalization rule}{
                \linferenceRule[sequent]{
                    \asfml
                }{
                    \dbox{\asprg}{\asfml}
                }
            }{}
            \cinferenceRule[MP|MP]{modus ponens rule}{
                \linferenceRule[sequent]{
                    \asfml(s)\limply\bsfml(s) & \asfml(s)
                }{
                    \bsfml(s)
                }
            }{}
            \cinferenceRule[gena|$\forall$]{forall generalization rule}{
                \linferenceRule[sequent]{
                    \asfml(s)
                }{
                    \lforall{x}{\asfml(s)}
                }
            }{}
            % TODO: CT-Axiom, CQ-Axiom, CE-Axiom, US-Axiom
        \end{calculus}
        \end{calculuscollections}
        \caption{Delay differential dynamic logic axioms and proof rules.}
        \label{fig:axioms}
    \end{figure}

    The \emph{test axiom}~\irref{testb}


    The \emph{discrete assignment axiom}~\irref{assignb} substitutes $\x[0]$ by its new value $\astrm$ in the formula.
    % FIXME: admissibility condition for assignb
    This requires $x$ not to be ... (admissibility condition)

    The \emph{axiom of nondeterministic choice}~\irref{choiceb}

    The \emph{composition axiom}~\irref{composeb}

    % The \emph{solution axiom}~\irref{solb} replaces a DDE with the solution of its symbolic initial-value problem (as opposed to a convential IVP, which is numerical), This solution must be expressible as first-order formula of real arithmetic, which is only the case for a small class of delay differential equations.

    The \emph{iteration axiom}~\irref{iterateb} partially unwinds a loop, which can be used for bounded model checking.

    The \emph{induction axioms}~\irref{I}
    % and \irref{Cb}, which is a variant of Harel's convergence rule,
    can be applied when reasoning about loops with unbounded repetitions. 

    The \emph{modal modus ponens} axiom~\irref{K} and the \emph{Barcan formula}~\irref{B} are taken from first-order modal logic.

    axiom V

    The basic \emph{proof rules} for the presented Hilbert calculus are \emph{Gödel's necessitation rule}~\irref{G} of modal logic, as well as \emph{modus ponens}~\irref{MP} and \emph{$\forall$-generalization}~\irref{gena} of first-order logic.


    \subsection{Differential Axioms}
        \label{sec:differential-axioms}

        \begin{figure}[h]
            \begin{calculuscollections}{\coloumnwidth}
            \begin{calculus}
                % replace \D{(x)} with \der{x} 
                \cinferenceRule[Dconst|$c'$]{derive constant}{
                    \linferenceRule[eq]{0}{\der{a}}
                }{}
                \cinferenceRule[Dvar|${x[\cdot]'}$]{derive variable}{
                    \linferenceRule[eq]{\Dx[c]}{\der{\x[c]}}
                }{}
                \cinferenceRule[Dvars|]{derive variable}{
                    \linferenceRule[eq]{\Dx[s]}{\der{\x[s]}}
                }{}
                \cinferenceRule[Dplus|$+'$]{derive sum}{
                    \linferenceRule[eq]{\der{\astrm(s)}+\der{\bstrm(s)}}{\der{\astrm(s)+\bstrm(s)}}
                }{}
                \cinferenceRule[Dmult|$\cdot'$]{derive product}{
                    \linferenceRule[eq]{\der{\astrm(s)}\cdot\bstrm(s)+\astrm(s)\cdot\der{\bstrm(s)}}{\der{\astrm(s)\cdot\bstrm(s)}}
                }{}

                \cinferenceRule[DW|DW]{differential weakening}{
                    \dbox{\hevolvein{\D{x}=\astrm}{\ivr}}{\ivr}
                }{}
                \cinferenceRule[DC|DC]{differential cut}{
                    \linferenceRule[lpmi]{
                        \big(\dbox{\hevolvein{\D{x}=\astrm}{\ivr}}{\asfml(s)}
                        \lbisubjunct
                        \dbox{\hevolvein{\D{x}=\astrm}{\ivr\land\inv}}{\asfml(s)}\big)
                    }{
                        \dbox{\hevolvein{\D{x}=\astrm}{\ivr}}{\inv}
                    }
                }{}
                \cinferenceRule[DE|DE]{differential effect}{
                    \linferenceRule[equiv]{
                        \dbox{\hevolvein{\D{x}=\astrm}{\ivr}}{\dbox{\Dupdate{\Dumod{\D{x}}{\astrm}}}{\asfml(s,x,\D{x})}}
                    }{
                        \dbox{\hevolvein{\D{x}=\astrm}{\ivr}}{\asfml(s,x,\D{x})}
                    }
                }{}
                \cinferenceRule[DI|DI]{differential invariant}{
                    \linferenceRule[lpmi]{
                        \dbox{\hevolvein{\D{x}=\astrm}{\ivr}}{\inv}
                    }{
                        \big(\ivr\limply\inv\land\dbox{\hevolvein{\D{x}=\astrm}{\ivr}}{\D{(\inv)}}\big)
                    }
                }{}
            \end{calculus}
            \end{calculuscollections}
            \caption{Delay differential equation axioms and differential axioms.}
            \label{fig:D-axioms}
        \end{figure}

        $\Dx[c]$ and $\x[c]$ are not allowed in the expression of an differential invariant, because they would lead to discontinuities. Thus differential invariants $\inv$ must be \FOLR formulas.

    % \subsection{History Axiom}
    %     \label{history-axiom}

    %     Just replace symbol by its semantical meaning
    %     The occurence of $\x[-\tau]$ in expressions can be replaced by turning the (implicitely existing) time variable explicit, i.e.\
    %     uniform substitution $\sigma$
    %     allows substitution of $\x[-\tau]$ by, depending on context, $x(t-\tau)$ or $\forall{s\in[0,\tau]}{x(t-\tau)}$
    %     allows substitution of x, +quantifier from semantics in certain contexts

    %     \begin{calculus}
    %         \cinferenceRule[hist|hist]{history axiom}{
    %             \linferenceRule[equiv]{
    %                 \holdssince{-T}{\asfml(s)}
    %             }{
    %                 \hs[-T]{\asfml}
    %             }
    %         }{}
    %     \end{calculus}

    %     and $t\rightarrow t+s$

    %     for a piecewise continuous function $\theta\in\statespace$.

%     \subsection{Solution Axiom}
%         \label{sec:solution-axiom}

%         \begin{calculus}
%             \cinferenceRule[solb|solb]{solution axiom}{
%                 \linferenceRule[equiv]{
%                     (\lforall{0\leq t\leq\taumin}{\dbox{\hupdate{\humod{x}{y(t)}}}{\phi}})
%                     \land
%                     (\holdssince{-T}{x=y(s+\taumin)} \limply \dbox{\hevolvein{\D{x}=\astrm}{\ivr}}{\phi})
%                 }{
%                     \holdssince{-T}{x=\bstrm(s)} \limply \dbox{\hevolvein{\D{x}=\astrm}{\ivr}}{\phi}
%                 }
%             }{}
%         \end{calculus}

%         where $\forall 0\leq t\leq\tau$, $y'(t)=\theta(\theta_0)$, i.e.\ $y$ is a local solution of the symbolic initial value problem. The solution must be expressible in polynomial form so that the axiom leads to decidable arithmetic.
%         However, only a very little class of delay differential equations has such solutions.
%         need to reposition time, so that each step begins at $t=0$, no problem for autonomous ddes
%         (Since the DDE is autonomous, we can emit the time index.)

% it often makes sense to treat the very first initial condition separately, because after it solution is at least $C^1$, at $x(0)$ might be knick

    \subsection{Axiom of Steps}
        \label{sec:axiom-of-steps}

        The \emph{method of steps} presented in Section~\ref{sec:method-of-steps} translates into an axiom.

        By introducing a fresh variable $t$ as a clock, we restrict the evolution of a delay differntial equation starting from a state by a duration not longer than its smallest delay $\taumin$. This evolution is then wrapped in a loop.
        In this case, the right hand side of the differential equation only depends on the initial state of the loop, not on its own solution yet. Hence the differential equation is not longer a DDE, but of \emph{ordinary} type.
        Its right hand side is in general piecewise continuous.
        Theorem~\ref{thm:solution-existence} shows the existence of a unique local solution in this case.
        \begin{equation*} % no blank line before, causes to much space
            \cinferenceRule[stepsb|$\dibox{\steps}$]{method of steps axiom}{
                \linferenceRule[equiv]{
                    \dbox{\Dupdate{\Dumod{\D{x}}{\astrm}};\hrepeat{(\hupdate{\humod{t}{0}};\hevolvein{\D{t}=1\syssep \D{x}=\astrm}{\ivr\land 0\leq t\leq\taumin})}}{\asfml(s)}
                }{
                    \dbox{\hevolvein{\D{x}=\astrm}{\ivr}}{\asfml(s)}
                }
            }{}
        \end{equation*}

        where $\taumin$ is the (by magnitude) smallest delay appearing in $\astrm$.

    % \subsection{Axiom of One Step}
    %     \label{sex:axiom-of-one-step}

    %     Unwind loop in axiom od steps
    %     given an analytic solution on $[0,\tau]$ and given initial condition
    %     useful for bounded model checking

\section{Soundness}
    \label{sec:soundness}

    The following theorem is obviously fundamental for the presented theory in order to make sense.

    \begin{theorem}[Soundness of \ddL]\label{thm:ddL-soundness}
        The \ddL calculus is sound: every formula which is provable from \ddL axioms by \ddL proof rules is \emph{valid} (true in all states), i.e.\ $\infers\asfml$ implies $\models\asfml$.
    \end{theorem}
    \begin{proof}
        The soundness proof of most of the axioms adapted from \dL are independent of the definition of the state space, they only reason about states without considering their structure.
        % The arguments of most proof parts are the same as they were for the classic \dL, since only the definition of the statespace has been replaced. The proofs were independent of this definition, though.
        % FIXME: ref to choiceb
        This is the case for \irref{testb}, \irref{choiceb}
        \irref{composeb}, \irref{iterateb}, \irref{K}, \irref{I},
        %\irref{C},
        \irref{B}, \irref{V} and \irref{G}, whose proof can hence be found in \cite{Platzer12Complete}.

        % TODO: axiom [&] in combination with [steps]
        % backward continuation (Richard) -> [\&] limit up to T, state contains entire evolution, can check evo domain constraint on this state

        % 
    \begin{description}
        % FIXME: assignment symbol
        \item[\irref{assignb}]
        It is $\asstate\in\imodel{\IddL}{\dbox{\hupdate{\humod{x}{\astrm}}}{\asfml(s,\x[0])}}$ iff $\bsstate\in\imodel{\IddL}{\asfml(s,\x[0])}$ for all $(\asstate,\bsstate)\in\ireachability{\IddL}{\hupdate{\humod{x}{\astrm}}}$. There exists only a unique such state $\bsstate\in\states$. For this state it holds $\bsstate=\asstate$ except for the variable $x$, for which
        \begin{equation*}
            \bsstate(x)(\past) = \begin{cases*}
                    \ivaluation{\IddL}{\astrm} & if $\past=0$\\
                    \asstate(x)(\past) & if $\past\in\delayinterval$
                \end{cases*}
        \end{equation*}
        i.e.\ the two states coincide in the values for $\x[s]$, except in $\x[0]$. Hence $\asstate\in\imodel{\IddL}{\asfml(s,\astrm)}$ iff $\bsstate\in\imodel{\IddL}{\asfml(s,\x[0])}$.
        The same holds if $\asfml(\x[0])$ does not depend on $s$.
         (->substitution lemma in book)
        
        \item[\irref{stepsb}] Let $\asstate\in\imodel{\IddL}{\dbox{\hevolvein{\D{x}=\astrm}{\ivr}}{\asfml(s)}}$ and $\trajectory\from\compactum{0}{\duration}\to\states$ be a trajectory of duration $\duration\geq 0$ solving the DDE and having $\asstate$ as initial condition, i.e.\ $\trajectory(0)=\asstate$ on $\scomplement{\set{\D{x}}}$ and $\trajectory(\zeta)\in\imodel{\IddL}{\Dx[0]=\astrm\land\ivr}$ for all $\zeta\in\compactum{0}{\duration}$. By the choice of $\asstate$ it holds $\trajectory(\zeta)\in\imodel{\IddL}{\asfml(s)}$ for all $\zeta\in\compactum{0}{\duration}$.

        We need to show that $\asstate\in\imodel{\IddL}{\dbox{\Dupdate{\Dumod{\D{x}}{\astrm}};\hrepeat{(\hupdate{\humod{t}{0}};\hevolvein{\D{t}=1\syssep \D{x}=\astrm}{\ivr\land 0\leq t\leq\taumin})}}{\asfml(s)}}$.
        If we enter the loop in the right hand side zero times, this holds since $\trajectory(0)\in\imodel{\IddL}{\asfml(s)}$ and $\trajectory(0)=\modif{\asstate}{\Dx[0]}{\astrm}$.
        If we repeated the loop $n$ times, it holds after the last iteration that $\zeta=(n-1)\taumin+t\leq\duration$, since the evolution is restricted by $\ivr$. We know in this case that $\trajectory(\zeta)\in\imodel{\IddL}{\asfml(s)}$ what implies the assertion.

        The converse implication is shown analogously.

        \item[\irref{DW}] Proof as for \dL.
        
        \item[\irref{DC}] For a formula $\bsfmlfolR$ of \FOLR, let $\asstate\in\imodel{\IddL}{\dbox{\hevolvein{\D{x}=\astrm}{\ivr}}{\bsfmlfolR}}$ and $\trajectory\from\compactum{0}{\duration}\to\states$ be an arbitrary trajectory of duration $\duration\geq 0$, solving the DDE and having $\asstate$ as initial condition, i.e.\ $\trajectory(0)=\asstate$ on $\scomplement{\set{\D{x}}}$ and $\trajectory(\xi)\in\imodel{\IddL}{\Dx[0]=\astrm\land\ivr}$ for all $\xi\in\compactum{0}{\duration}$.

        Suppose $\asstate\in\imodel{\IddL}{\dbox{\hevolvein{\D{x}=\astrm}{\ivr}}{\asfml(s)}}$, i.e.\ there exists a $0\leq \bar{r}\leq\duration$, such that $\trajectory(\zeta)\in\imodel{\IddL}{\D{x}=\astrm\land\ivr}$ for all $\zeta\in\compactum{0}{\bar{r}}$ and $\trajectory(\bar{r})\in\imodel{\IddL}{\asfml(s)}$. Since $\zeta\leq\duration$ it is also $\trajectory(\zeta)\in\imodel{\IddL}{\bsfmlfolR}$. This is equivalent to $\trajectory(\zeta)\in\imodel{\IddL}{\D{x}=\astrm\land\ivr\land\bsfmlfolR}$ and $\trajectory(\zeta)\in\imodel{\IddL}{\bsfmlfolR}$ for all $\zeta\in\compactum{0}{\bar{r}}$, which is the same as $\asstate\in\imodel{\IddL}{\dbox{\hevolvein{\D{x}=\astrm}{\ivr\land\bsfmlfolR}}{\asfml(s)}}$.

        \item[\irref{DI}] This proof is an adaption of the \dL proof for DI given in~\cite{Platzer15Uniform}. Without loss of generality we restrict to the case of invariants of the form $\inv\equiv(g(x)\geq 0)$, where $g$ is a term of \FOLR. Then $\D{(\inv)}\equiv(\D{(g(x))}\geq 0)$ (by ??).

        Consider a state $\asstate\in\states$ with $\asstate\in\imodel{\IddL}{\ivr\limply\inv\land\dbox{\hevolvein{\D{x}=\astrm}{\ivr}}{\D{(\inv)}}}$. We need to distinguish two cases. If $\asstate\notin\imodel{\IddL}{\ivr}$, then their is no solution of the DDE and hence $\asstate\in\imodel{\IddL}{\dbox{\hevolvein{\D{x}=\astrm}{\ivr}}{\inv}}$ vacuously.

        If $\asstate\in\imodel{\IddL}{\ivr}$, then $\asstate\in\imodel{\IddL}{\dbox{\hevolvein{\D{x}=\astrm}{\ivr}}{\D{(\inv)}}}$. Let $\trajectory\from\compactum{0}{\duration}\to\states$ be a trajectory solving the DDE for some time $r\geq 0$, i.e.\ $\interpret,\trajectory\models(\hevolvein{\D{x}=\astrm}{\ivr})$.
        % FIXME: r=0 in DI proof
        If $\duration=0$ then $\asstate\in\imodel{\IddL}{\ivr}$ since the only varaible changing its value is $\D{x}$, which is not contained in
        % FIXME: is this freevars{\inv} like in proof by Platzer?
        $\ivr$ ($\freevars{\ivr}\cap\set{\D{x}}=\emptyset$). Hence it follows from the precondition that $\asstate\in\imodel{\IddL}{\inv}$ and for this reason $\asstate\in\imodel{\IddL}{\dbox{\hevolvein{\D{x}=\astrm}{\ivr}}{\inv}}$.
        % FIXME: free vars of inv? x' can be changed? wouldn't influence \ivr

        If $\duration>0$, $\asstate\in\imodel{\IddL}{\dbox{\hevolvein{\D{x}=\astrm}{\ivr}}{\D{(\inv)}}}$ implies $\interpret,\trajectory\models\D{(\inv)}$.
        By the Differential Lemma~\ref{lm:differential-lemma} it holds for all $\zeta\in\compactum{0}{\duration}$
        \begin{equation*}
            0 \leq \ivaluation{\iconcat[state=\trajectory(\zeta)]{\IddL}}{\D{(g(x))}}=\DD{\ivaluation{\iconcat[state=\trajectory(t)]{\IddL}}{g(x)}}{t}(\zeta)
        \end{equation*}
        $\freevars{\inv}\cap\set{\D{x}}=\emptyset$ (means no $\D{x}$ in invariant) implies $\trajectory(0)=\asstate\in\imodel{\IddL}{g(x)\geq 0}$.
        no $\x[c]$ in invariant, hence continuous. 
        Lemma~\ref{lm:pc-integrable} yields for any $z\in\compactum{0}{\duration}$
        \begin{equation*}
            \ivaluation{\iconcat[state=\trajectory(z)]{\IddL}}{g(x)}= \ivaluation{\iconcat[state=\trajectory(0)]{\IddL}}{g(x)} + \integral{0}{z} \DD{\ivaluation{\iconcat[state=\trajectory(t)]{\IddL}}{g(x)}}{t}(\zeta)\dx[\zeta] \geq 0
        \end{equation*}
        hence $\trajectory(z)\in\imodel{\IddL}{\inv}$ and hence $\asstate\in\imodel{\IddL}{\dbox{\hevolvein{\D{x}=\astrm}{\ivr}}{\inv}}$

        \item[\irref{DE}] Let $\asstate\in\imodel{\IddL}{\dbox{\hevolvein{\D{x}=\astrm}{\ivr}}{\asfml(s)}}$ and $\trajectory\from\compactum{0}{\duration}\to\states$ be a trajectory of duration $\duration\geq 0$ solving the DDE and having $\asstate$ as initial condition, i.e.\ $\trajectory(0)=\asstate$ on $\scomplement{\set{\Dx[0]}}$ and $\trajectory(\zeta)\in\imodel{\IddL}{\Dx[0]=\astrm\land\ivr}$ for all $\zeta\in\compactum{0}{\duration}$.
        Since $\asstate\in\imodel{\IddL}{\dbox{\hevolvein{\D{x}=\astrm}{\ivr}}{\asfml(s)}}$, we have $\trajectory(\duration)\in\imodel{\IddL}{\asfml}$ which, by the Differential Assignment Lemma~\ref{lm:diff-assignment}, is equivalent to $\trajectory(\duration)\in\imodel{\IddL}{\dbox{\Dupdate{\Dumod{\D{x}}{\astrm}}}{\asfml(s)}}$. Hence $\asstate\in\dbox{\hevolvein{\D{x}=\astrm}{\ivr}}{\dbox{\Dupdate{\Dumod{\D{x}}{\astrm}}}{\asfml(s)}}$.
        The inverse implication is shown in the same way.
    \end{description}
    \end{proof}

% \section{Proof Rules}
%     \label{sec:proof-rules}

%     In order to avoid small steps in proofs by applying one axiom after the other, several axioms can be combined to a more powerful proof formula.
%     Since we posses for \ddL of most axioms given in \dL, we can also adapt most of its \emph{sequent calculus proof rules}, as given in \cite{cheatsheet}.
%     % TODO: or s-formulas?
%     By $\asfmls,\bsfmls,\csfmls$ we denote sets of formulas.


%         combine proof rule to tactic
    
%     If the \ddL formula $\asfml$ can be derived by \ddL proof rules from \ddL axioms, we say it is provable and write $|-\asfml$
%     (includes first-order axioms and rules)

    % \paragraph{Propositional Sequent Calculus Proof Rules}
    %     \label{sec:propositional-rules}

    %     \par
    %     \begin{calculus}
    %         \cinferenceRule[closeTrue|$\top$R]{close by always true antedecent}{
    %             \linferenceRule[sequent]{
    %                 %
    %             }{
    %                 \lsequent{\asfmls}{\ltrue,\bsfmls}
    %             }
    %         }{}
    %         \cinferenceRule[close|id]{close by identity}{
    %             \linferenceRule[sequent]{
    %                 %
    %             }{
    %                 \lsequent{\asfml,\asfmls}{\asfml,\bsfmls}
    %             }
    %         }{}
    %         \cinferenceRule[andR|$\land$R]{and right proof rule}{
    %             \linferenceRule[sequent]{
    %                 \lsequent{\asfmls}{\asfml,\bsfmls}
    %                 &\lsequent{\asfmls}{\bsfml,\bsfmls}
    %             }{
    %                 \lsequent{\asfmls}{\asfml\land \bsfml,\bsfmls}
    %             }
    %         }{}
    %         \cinferenceRule[andL|$\land$L]{and left proof rule}{
    %             \linferenceRule[sequent]{
    %                 \lsequent{\asfmls, \asfml, \bsfml}{\bsfmls}
    %             }{
    %                 \lsequent{\asfmls, \asfml\land \bsfml}{\bsfmls}
    %             }
    %         }{}
    %         \cinferenceRule[implyR|$\limply$R]{imply right proof rule}{
    %             \linferenceRule[sequent]{
    %                 \lsequent{\asfmls,\asfml}{\bsfml,\bsfmls}
    %             }{
    %                 \lsequent{\asfmls}{\asfml\limply \bsfml,\bsfmls}
    %             }
    %         }{}
    %         % TODO: more rules, or etc.
    %     \end{calculus}


    % \paragraph{Quantifier Sequent Calculus Proof Rules}
    %     \label{sec:quantifier-rules}

    %     \par
    %     \begin{calculus}
    %         \cinferenceRule[allR|$\forall$R]{forall right proof rule}{
    %             \linferenceRule[sequent]{
    %                 \lsequent{\asfmls}{\asfml(y),\bsfmls}
    %             }{
    %                 \lsequent{\asfmls}{\lforall{x}{\asfml(x)},\bsfmls}
    %             }
    %         }{$y\notin\asfmls,\bsfmls$}
    %     \end{calculus}
    %     here $x$ does not mean $\x[0]$.

    % \paragraph{\dL Sequent Calculus Proof Rules}
    %     \label{sec:dL-rules}

    %     \par
    %     $x\notin\asfmls,\bsfmls$ or reduce $\hsc{}$ to $\hs{}$?
    %     \begin{calculus}
    %         % TODO: replace := by command
    %         % FIXME: check this rule, important for proof below
    %         \cinferenceRule[assignb|$\mathrel{{:}{=}}$]{discrete assignment}{
    %             \linferenceRule[sequent]{
    %                 \lsequent{\asfmls,\x[0]=\astrm}{\asfml,\bsfmls}
    %             }{
    %                 \lsequent{\asfmls}{\dbox{\hupdate{\humod{x}{\astrm}}}{\asfml},\bsfmls}
    %             }
    %         }{$x\notin\asfmls,\bsfmls$}

    %         \cinferenceRule[loop|loop]{loop invariant}{
    %             \linferenceRule[sequent]{
    %                 \lsequent{\asfmls}{\bsfml,\bsfmls}
    %                 &\lsequent{\bsfml}{\dbox{\alpha}{\bsfml}}
    %                 &\lsequent{\bsfml}{\asfml}
    %             }{
    %                 \lsequent{\asfmls}{\dbox{\hrepeat{\asprg}}{\asfml},\bsfmls}
    %             }
    %         }{with loop invariant $\bsfml$}
    %     \end{calculus}

    % \paragraph{Differential Equation Sequent Calculus Proof Rules}
    %     \label{sec:diff-rules}

    %     text

    %     \begin{calculus}
    %         \cinferenceRule[DC|DC]{differential cut}{
    %             \linferenceRule[sequent]{
    %                 \lsequent{\asfmls}{\dbox{\hevolvein{\D{x}=\astrm}{\ivr}}{\inv},\bsfmls}
    %                 &\lsequent{\asfmls}{\dbox{\hevolvein{\D{x}=\astrm}{\ivr\land \inv}}{\asfml}}
    %             }{\lsequent{\asfmls}{\dbox{\hevolvein{\D{x}=\astrm}{\ivr}}{\asfml},\bsfmls}}
    %         }{}
    %         \cinferenceRule[dI|dI]{differential invariant}{
    %             \linferenceRule[sequent]{
    %                 \lsequent{\asfmls,\asfml}{\inv,\bsfmls}
    %                 &\lsequent{\asfml}{\dbox{\Dupdate{\Dumod{\D{x}}{\astrm}}}{\der{\inv}}}
    %             }{\lsequent{\asfmls}{\dbox{\hevolvein{\D{x}=\astrm}{\ivr}}{\asfml},\bsfmls}}
    %         }{}
    %         \cinferenceRule[dW|dW]{differential weakening}{
    %             \linferenceRule[sequent]{
    %                 \lsequent{\asfmls}{\lforall{x}{(\ivr\limply\asfml)},\bsfmls}
    %             }{\lsequent{\asfmls}{\dbox{\hevolvein{\D{x}=\astrm}{\ivr}}{\asfml},\bsfmls}}
    %         }{}
    %     \end{calculus}

    % TODO: Rule of Steps
    % \subsection{Rule of Steps}
    %     \label{sec:rule-of-steps}
    %     ODEs don't have notion of \emph{one step}, but DDEs do.
    %     condition valid for initial condition and given condition for a $s\leq t$ then condition holds after dde-evolution of max time tau and safety follows from condition then condition holds after dde with mentioned initial condition
    %     loop induction
    %     truth value of invariant never changes during dde
    %     % \begin{equation}
    %     % \frac{\asfmls(\xbartaut{0})\rightarrow F(\xbartaut{0})\quad F(\xbartaut{s})\rightarrow [\D{x}=\theta(\xbartaut{t})\,\&\,t\leq\tau]F(\xbartaut{t}) \quad F(\xbartaut{t})\rightarrow\phi}{\asfmls(\xbartaut{0}) \rightarrow [\D{x}=\theta(\xbartaut{t})]\phi}
    %     % \end{equation}

    %     %\begin{small}
    %     \begin{calculus}
    %         % FIXME: \landS -> steps
    %         \cinferenceRule[steps|stps]{steps proof rule}{
    %             \linferenceRule[sequent]{
    %                 \lsequent{\asfmls}{\inv,\bsfmls}
    %                 &\lsequent{t=0,\inv(\theta(t-\tau))}{\dbox[]{\hevolvein{\D{t}=1\syssep \D{x}=\rho(x,\theta(t-\tau))}{(\ivr\land 0\leq t\leq\tau)}}{\inv}}
    %                 &\lsequent{\inv}{\asfml}
    %             }{
    %                 \lsequent{\asfmls}{\dbox[]{\hevolvein{\D{x}=\rho(x,\x[-\tau])}{\ivr}}{\asfml},\bsfmls}
    %             }
    %         }{}
    %     \end{calculus}
        %\end{small}

    %     Formulas of the form $\asfmls(\x[-\tau])$ implicitely also include a statement about $x$.

    % \subsection{Delay Differential Induction}
    %     \label{sec:delay-differential-induction}

    %     Like loop+DI, the former for $\lforall{k\geq 0}$, the latter for $\lforall{k\tau\leq t \leq (k+1)\tau}$
    %     The idea behind this proof rule is
    %     the initial condition fulfills a certain condition
    %     evolve a little in time
    %     the values which come out of the dde also fulfill this condition
    %     all runs od dde lead to states satisfying formula
    %     start in safe state
    %     dynamical system only evolve in direction of safe states in $\inv$
    %     direction is given by dde: in state $\omega$ it is $\imodel{}{f(x)}\omega$
    %     only need how system evolves in relation to $\inv$
    %     hence stays safe forever
    %     so the state after the DDE fulfills the condition, parts of the state come from initial condition, parts from dde outcome

    %     invariant of form with $\x[s]$, but without $\x[c]$, i.e. becomes \FOLR formula if I fix $\past=0$.

    %     \begin{calculus}
    %         \cinferenceRule[DDI|DDI]{delay differential induction proof rule}{
    %             % FIXME: \der{} <-> \D{} ?
    %             \linferenceRule[sequent]{
    %                 \lsequent{\asfmls}{\hsc{\bsfml(s)},\bsfmls}
    %                 &\lsequent{\ivr,0\leq t\leq\tau,\inv(\theta)}{\dbox{\hevolve{\Dupdate{\Dumod{\D{x}}{\rho(x,\theta)}}}}{\der{\bsfml(0)}}}
    %                 % FIXME: only hs?
    %                 &\lsequent{\hsc{\bsfml(s)}}{\asfml(s)}
    %             }{
    %                 \lsequent{\asfmls}{\dbox{\hevolvein{\D{x}=\astrm}{\ivr}}{\asfml(s)},\bsfmls}
    %             }
    %         }{}
    %     \end{calculus}

    %     % sidewaysfigure
    %     %\begin{proof}\small
    % \begin{landscape}
        
    %     % \begin{sidewaysfigure}
    %     \footnotesize
    %     % \centering
    %     % TODO: hist axiom earlier? before first DC?
    %     \begin{sequentdeduction}[]
    %         \linfer[stepsb+composeb]{
    %             \linfer[assignb]{
    %                 \linfer[loop]{
    %                     % FIXME: formel zu hoch
    %                     \lsequent{\asfmls,\Dx[0]=\astrm}{\hs{\bsfml(s)},\bsfmls}
    %                     &\linfer[composeb+assignb]{
    %                         \linfer[DC]{
    %                             \linfer[DC]{
    %                                 (1)
    %                             }{
    %                                 \lsequent{\hs{\bsfml(s)},t=0}{\dbox{\hevolvein{\D{t}=1\syssep \D{x}=\astrm}{(\ivr\land 0\leq t\leq\tau\land \inv(x(t-\tau)))}}{\hs{\bsfml(s)}}}
    %                             }
    %                             &\linfer[dW]{
    %                                 (2)
    %                             }{
    %                                 \lsequent{\hs{\bsfml(s)},t=0}{\dbox{\hevolvein{\D{t}=1\syssep \D{x}=\astrm}{(\ivr\land 0\leq t\leq\tau)}}{\inv(x(t-\tau))}}
    %                             }
    %                         }{
    %                             \lsequent{\hs{\bsfml(s)},t=0}{\dbox{\hevolvein{\D{t}=1\syssep \D{x}=\astrm}{(\ivr\land 0\leq t\leq\tau)}}{\hs{\bsfml(s)}}}
    %                             % \lsequent{\lforall{s\in[-\tau,0]}{\inv(x(t+s))},t=0}{\dbox[]{\hevolvein{\D{t}=1\syssep \D{x}=\astrm}{(\ivr\land 0\leq t\leq\tau)}}{\lforall{s\in[-\tau,0]}{\inv(x(t+s))}}}
    %                         }
    %                     }{
    %                         % FIXME: hsc?
    %                         \lsequent{\hs{\bsfml(s)}}{\dbox{\hupdate{\humod{t}{0}}; \hevolvein{\D{t}=1\syssep \D{x}=\astrm}{(\ivr\land 0\leq t\leq\tau)}}{\hs{\bsfml(s)}}}
    %                     }
    %                     &\lsequent{\hs{\bsfml(s)}}{\asfml(s)}
    %                 }{
    %                     % FIXME: Soundness of assignb, and applicaple here?
    %                     % x' not in Gamma, and x not affected since (x[0])' and x'[0] need not to coincide?
    %                     \lsequent{\asfmls,\Dx[0]=\astrm}{\dbox{\hrepeat{(\hupdate{\humod{t}{0}}; \hevolvein{\D{t}=1\syssep \D{x}=\astrm}{(\ivr\land 0\leq t\leq\taumin)})}}{\asfml(s)}},\bsfmls
    %                 }
    %             }{
    %                 \lsequent{\asfmls}{\dbox{\Dupdate{\Dumod{\D{x}}{\astrm}}}{\dbox{\hrepeat{(\hupdate{\humod{t}{0}}; \hevolvein{\D{t}=1\syssep \D{x}=\astrm}{(\ivr\land 0\leq t\leq\taumin)})}}{\asfml(s)}}},\bsfmls
    %             }
    %         }{
    %             \lsequent{\asfmls}{\dbox{\hevolvein{\D{x}=\astrm}{\ivr}}{\asfml(s),\bsfmls}}
    %         }
    %     \end{sequentdeduction}

    %     % TODO: ref to here: (2)
    %     \begin{sequentdeduction}
    %         \linfer[dW]{
    %             \linfer[allR]{
    %                 \linfer[implyR]{
    %                     \linfer[]{
    %                         \lclose
    %                     }{
    %                         \lsequent{\lforall{s\in[-\tau,0]}{\inv(x(s))},t=0,\ivr(r,y),0\leq r\leq\tau}{\inv(x(r-\tau))}
    %                     }
    %                 }{
    %                     \lsequent{\lforall{s\in[-\tau,0]}{\inv(x(s))},t=0}{\ivr(r,y)\land 0\leq r\leq\tau\limply \inv(x(r-\tau))}
    %                 }
    %             }{
    %                 \lsequent{\lforall{s\in[-\tau,0]}{\inv(x(s))},t=0}{\lforall{(t,x)}{(\ivr\land 0\leq t\leq\tau\limply \inv(x(t-\tau)))}}
    %             }
    %         }{
    %             \lsequent{\lforall{s\in[-\tau,0]}{\inv(x(s))},t=0}{\dbox{\hevolvein{\D{t}=1\syssep \D{x}=\astrm}{(\ivr\land 0\leq t\leq\tau)}}{\inv(x(t-\tau))}}
    %         }
    %     \end{sequentdeduction}

    %     % TODO: ref to here: (1)
    %     \begin{sequentdeduction}
    %         \linfer[DC]{
    %             \linfer[dI]{
    %                 \linfer[hist]{
    %                     \linfer[]{
    %                         \lclose
    %                     }{
    %                         \lsequent{\hs{\bsfml(s)},t=0}{\bsfml(0)}
    %                     }
    %                 }{
    %                     \lsequent{\hs{\bsfml(s)},t=0,A, \inv(x(t-\tau))}{\bsfml(0)}
    %                 }
    %                 % TODO: A=\ivr, 0\leq t\leq\tau
    %                 &\lsequent{A, \inv(x(t-\tau)))}{\dbox{\Dupdate{\Dumod{\D{t}}{1},\Dumod{ \D{x}}{\astrm}}}{\der{\inv(x(t))}}}
    %                 %}
    %             }{
    %                 \lsequent{\hs{\bsfml(s)},t=0}{\dbox{\hevolvein{\D{t}=1\syssep \D{x}=\astrm}{(A\land \inv(x(t-\tau)))}}{\bsfml(0)}}
    %             }
    %             &\linfer[dW]{
    %                 (3)
    %             }{
    %                 \lsequent{\lforall{s\in[-\tau,0]}{\inv(x(s))},t=0}{\dbox{\hevolvein{\D{t}=1\syssep \D{x}=\astrm}{(A\land \inv(x(t-\tau))\land \inv(x(t)))}}{\lforall{s\in[-\tau,0]}{\inv(x(t+s))}}}
    %             }
    %         }{
    %             \lsequent{\hs{\bsfml(s)},t=0}{\dbox{\hevolvein{\D{t}=1\syssep \D{x}=\astrm}{(\ivr\land 0\leq t\leq\tau\land \inv(x(t-\tau)))}}{\lforall{s\in[-\tau,0]}{\inv(x(t+s))}}}
    %         }
    %     \end{sequentdeduction}
    %     % TODO: ref here (3)
    %     \begin{sequentdeduction}
    %         \linfer[dW]{
    %             \linfer[allR]{
    %                 \linfer[implyR]{
    %                     \linfer[]{
    %                         \lclose
    %                     }{
    %                         \lsequent{\lforall{s\in[-\tau,0]}{\inv(x(s))}, t=0, \ivr(r,y), 0\leq r\leq\tau, \inv(y(r-\tau)), \inv(y(r))}{\lforall{s\in[-\tau,0]}{\inv(y(r+s))}}
    %                     }
    %                 }{
    %                     \lsequent{\lforall{s\in[-\tau,0]}{\inv(x(s))}, t=0}{((\ivr(r,y)\land 0\leq r\leq\tau\land \inv(y(r-\tau))\land \inv(y(r)))\limply\lforall{s\in[-\tau,0]}{\inv(y(r+s))})}
    %                 }
    %             }{
    %                 \lsequent{\lforall{s\in[-\tau,0]}{\inv(x(s))}, t=0}{\lforall{(t,x)}{((A\land \inv(x(t-\tau))\land \inv(x(t)))\limply\lforall{s\in[-\tau,0]}{\inv(x(t+s))})}}
    %             }
    %         }{
    %             \lsequent{\lforall{s\in[-\tau,0]}{\inv(x(s))}, t=0}{\dbox{\hevolvein{\D{t}=1\syssep \D{x}=\astrm}{(A\land \inv(x(t-\tau))\land \inv(x(t)))}}{(\lforall{s\in[-\tau,0]}{\inv(x(t+s))})}}
    %         }
    %     \end{sequentdeduction}
    %     % \end{sidewaysfigure}
    %     \normalsize
    %         \end{landscape}

        % \end{proof}

    \subsection{Delay Differential Invariant}
        \label{sec:delay-differential-invariant}

        loop and differential invariants are of form $\lforall{s\in[-\tau,0]}{\inv(x(s))}$
        can they have $x$?

        Meaning of derivative $\der{\inv(\x[-\tau])}=\lforall{s\in[-\tau,0]}{\der{\inv(x(t+s))}}$ would lead to occurrence of derivative of init cond, which we don't know

        Mentioning $\x[-\tau]$ in the invariant differential invariant is not permitted, since derivation would lead to the occurrence of the symbol $x_{2\tau}$, whose properties are out of the scope of the current state.

        % TODO: ref to DDI
        As for ODEs in \dL, we cannot have $x(t)$ in the premise in DDI. Would permit to prove wrong statements.

        Invariant for limited time: use with loop unrolling (can it be generalized to unlimited inv?)

    % TODO: Example
    \subsection{Examples}
        \label{sec:examples}

        \subsubsection{Example 1}
            \label{sec:ddi-example-1}

            Consider the non-linear first-order delay differential equation
            \begin{equation}
                \begin{cases}
                    \D{x}(t) = x(t-\tau) & t \geq \tzero\\
                    x(t) = \theta(t)\geq 0 & t \in [\tzero-\tau,\tzero]
                \end{cases}
            \end{equation}
            Using the invariant $F\equiv(x^3\geq 0)$ we prove that the solution stays non-negativ for all time $t$.
            DDE is autonomous, can assume $\tzero=0$.
            \footnotesize
            \begin{sequentdeduction}
                \linfer[DC]{
                    \linfer[DI]{
                        \linfer[DDW]{
                            \linfer[DE+G]{
                                \linfer[assignb]{
                                    \linfer[QE]{
                                        \lclose
                                    }{
                                        (\hsc{\x[s]\geq 0}\limply x^3\geq 0 \land
                                        3x^2\x[-\tau]\geq 0 \land
                                        \hs{\x[s]\geq 0}) \land
                                        (x^3\geq 0 \limply x\geq 0)
                                    }
                                }{
                                    (\hsc{\x[s]\geq 0}
                                    \limply x^3\geq 0
                                    \land
                                    \dbox{\Dupdate{\Dumod{\D{x}}{\x[-\tau]}}}{(3x^2\D{x}\geq 0)}
                                    \land
                                    \hs{\x[s]\geq 0})
                                    \land
                                    (x^3\geq 0 \limply x\geq 0)
                                }
                            }{
                                (\hsc{\x[s]\geq 0}\limply x^3\geq 0
                                \land
                                \dbox{\hevolve{\D{x}=\x[-\tau]}}{(3x^2\D{x}\geq 0)}
                                \land
                                \hs{\x[s]\geq 0})
                                \land
                                (x^3\geq 0 \limply x\geq 0)
                            }
                        }{
                            \hsc{\x[s]\geq 0}
                            \limply
                            x^3\geq 0
                            \land
                            \dbox{\hevolvein{\D{x}=\x[-\tau]}{x^3\geq 0}}{(\hsc{\x[s]\geq 0})}
                            \land
                            \dbox{\hevolve{\D{x}=\x[-\tau]}}{(3x^2\D{x}\geq 0)}
                        }
                    }{
                        \hsc{\x[s]\geq 0}
                        \limply
                        \dbox{\hevolvein{\D{x}=\x[-\tau]}{x^3\geq 0}}{(\hsc{\x[s]\geq 0})}
                        \land
                        \dbox{\hevolve{\D{x}=\x[-\tau]}}{(x^3\geq 0)}
                    }    
                }{
                    \hsc{\x[s]\geq 0}
                    \limply
                    \dbox{\hevolve{\D{x}=\x[-\tau]}}{(\hsc{\x[s]\geq 0})}
                }
            \end{sequentdeduction}
            % \begin{sequentdeduction}
            %     \linfer[DDI]{
            %         \linfer[]{
            %             \lclose
            %         }{
            %             \lsequent{\x[-\tau]\geq 0}{\x[-\tau]^3\geq 0}
            %         }
            %         &\linfer[]{
            %             \linfer[]{
            %                 \linfer[]{
            %                     \lclose
            %                 }{
            %                     \lsequent{\theta^3\geq 0}{3x^2\theta\geq 0}
            %                 }
            %             }{
            %                 \lsequent{0\leq t\leq \tau,\theta^3\geq 0}{\dbox{\Dupdate{\Dumod{\D{x}}{\theta}}}{(3x^2 \D{x}\geq 0)}}
            %             }
            %         }{
            %             %\lclose
            %             \lsequent{0\leq t\leq \tau,\theta^3\geq 0}{\dbox{\Dupdate{\Dumod{\D{t}}{1},\Dumod{\D{x}}{\theta}}}{\der{x^3\geq 0}}}
            %         }
            %         &\linfer[]{
            %             \lclose
            %         }{
            %             \lsequent{\x[-\tau]^3\geq 0}{\x[-\tau]\geq 0}
            %         }
            %     }{
            %         \lsequent{\holdssinceclosed{-\tau}{\x[s]\geq 0}}{\dbox{\D{x}=\x[-\tau]}{(\holdssinceclosed{-\tau}{\x[s]\geq 0})}}
            %     }
            % \end{sequentdeduction}
            \normalsize
            In the same way we can prove that the solution stays negative for all $t$, if the initial condition is non-positive.

        \subsubsection{Example 2}
            \label{sec:ddi-example-2}

            % FIXME: it's not autonomous ?
            Consider the non-linear first-order delay differential equation with explicitely given initial condition
            \begin{equation}
                \begin{cases}
                    \D{x}(t) = -x(t-1)^2 & t \geq 0\\
                    x(t) = t & t \in [-1,0]
                \end{cases}
            \end{equation}
            Using $F\equiv(x^3\leq 0)$ we prove that the solution stays non-positiv.
            \begin{small}
                \begin{sequentdeduction}
                    \linfer[DDI]{
                        \linfer[hist]{
                            \linfer[]{
                                \linfer{\lclose}{
                                    \lsequent{}{\lforall{s\in[-1,0]}{s^3\leq 0}}
                                }
                            }{
                                \lsequent{t=0,\lforall{s\in[-1,0]}{x(t+s)=t+s}}{\lforall{s\in[-1,0]}{x(t+s)^3\leq 0}}
                            }
                        }{
                            \lsequent{t=0,\x[-1]=t}{\x[-1]^3\leq 0}
                        }
                        &\linfer[]{
                            \linfer[]{
                                \linfer[closeTrue]{
                                    \lclose
                                }{
                                    \lsequent{}{-3x^2\theta^2\leq 0}
                                }
                            }{
                                \lsequent{0\leq t\leq 1,\theta^3\leq 0}{\dbox{\Dupdate{\Dumod{\D{x}}{-\theta^2}}}{(-3x^2 \D{x}\leq 0)}}
                            }
                        }{
                            %\lclose
                            \lsequent{0\leq t\leq 1,\theta^3\leq 0}{\dbox{\Dupdate{\Dumod{\D{t}}{1},\Dumod{\D{x}}{-\theta^2}}}{\der{x\leq 0}}}
                        }
                        &\linfer[]{
                            \lclose
                        }{
                            \lsequent{\x[-1]^3\leq 0}{\x[-1]\leq 0}
                        }
                    }{
                        \lsequent{t=0,\x[-1]=t}{\dbox{\D{x}=-\x[-1]^2}{(\x[1]\leq 0)}}
                    }
                \end{sequentdeduction}
            \end{small}

            This proof doesn't even need any premisse about $\x[-\tau]$ in the induction step.

        % \subsubsection{Example 3}
        %     \label{sec:ddi-example-3}

        %     We want to proof the safety condition $\asfml\equiv(-1\leq x\wedge x\leq 1)$ for the continuous program with delay differential equation
        %     \begin{equation}
        %         \forall\,t\in[-\tau,0]:\,-1\leq\xbartaut{0}(t)\wedge\xbartaut{0}(t)\leq 1
        %         \rightarrow
        %         [\D{x}=-\x[-\tau]] (\forall\,s\in[-\tau,0]:\,-1\leq\xbartaut{t}(s)\wedge\xbartaut{t}(s)\leq 1)
        %     \end{equation}
        %     in explicit quantified representation. It can be simplified by using an implicit time variable and a context depending meaning of $\x[-\tau]$
        %     \begin{equation}
        %         -1\leq\x[-\tau]\leq 1 \rightarrow [\D{x}=-\x[-\tau]]\asfml.
        %     \end{equation}

        %     We apply the rule of steps using the safety condition $\phi$ as step condition $F(x)\equiv(\forall\,t\in[-\tau,0]:\,-1\leq x(t)\wedge x(t)\leq 1)$.

        %     The first and third premisses hold. The second by ??? (delay differential invariant)

        %     Use the algebraic differential invariant $F\equiv(-1\leq x^3\wedge x^3\leq1)$, which is valid for the initial condition. Differentiation leads to the inequalities, which needs to be shown $\forall t\in[0,\tau]$
        %     \begin{equation}
        %         0\leq 3\,x(t)^2 \D{x}(t) = -3\,x(t)^2 \x[-\tau](t)
        %     \end{equation}

        %     This holds since



% \chapter{Examples}\label{sec:example-hp}

To motivate the need of being able to treat Delayed Differential Equations in hybrid programs, we present some examples.

\section{Car Following Model}

    ref

    We want to model a controller for a car
    trying to keep a constant distance between
    considering a reaction time $\tau$, causing a delay in control decision
    controller is ideal continuous, not discrete

    Given the speed pattern of a leading car, the systems models the position and velocity of a following car.

    \begin{equation*}
        \begin{cases}
            \D[2]{x_{n+1}}(t) = \alpha (\D{x_n}(t-\tau)-\D{x_{n+1}}(t-\tau))\\
            \D{x_n}(t) = v(t)
        \end{cases}
    \end{equation*}

    The coefficient $\alpha$ can be seen as a sport or fun factor, describing the strengh of acceleration and deacceleration applied to the following car.

    By introducing $v_{n+1}$ for $x_{n+1}$, we reduce the system to first order.

    Preconditions (for all $t\in\closeddelayinterval{-\tau}$):
    \begin{align*}
        d(t) &= x_n(t)-x_{n+1}(t) \in\compactum{D-m}{D+m}\\
        \D{x_{n+1}}(t) &\in\compactum{V-l}{V+l}\\
        \D{x_n}(t) &= v(t) = 
    \end{align*}
    Both $\alpha$ and $\tau$ are considered to be constant.

    Model
    \begin{equation*}
        \dbox{\hevolve{
            \D{v_{n+1}} = \alpha(\D{x_n}[-\tau]-\D{x_{n+1}}[-\tau])\syssep
            \D{x_{n+1}} = v_{n+1}\syssep
            \D{x_n} = v
        }}{d\geq\delta}
    \end{equation*}

    too strong reactions of the following car lead to damped oscillations

    Of practical interest would be to prove the safety condition that the cars always keep a certain minimal distance, i.e.\ $d\geq\delta$.

\section{Network Induced Delay in Control Loops}

    only an input or measurement delay (in u)

    A \textbf{Networked Control Systems (NCS)} is a feedback control system with sensing and control data transmitted on a network.
    Sensors sample the state of a \textit{plant} periodically and send their output as packets to a event-driven \textit{controller}, which calculates a control signal as soon as the sensor data arrives. This is then transmitted to event-driven actuators in the plant, which perform action immediately on reception of the command.

    The plant is considered to be continuous in time by its physical nature, whereas the controller is discrete in time.

    This scenario has some possible issues, such as network induced \textbf{delay} and \textbf{loss} of network packets.
    For that reason, the plant output and the controller input are not delivered at the same time and the controller might not have received all the plant updates when it has to perform its control calculations. This makes NCSs different to conventional sampled-data systems.

    \subsection{Modelling of NCSs with Network-Induced Delay}
        The plant (physical component) is modeled as time continuous and the evolution of its state $x \in\R^{n}$ by
        \begin{align*}
            \dot{x}(t) &= Ax(t) + Bu(t) \\
            y(t) &= Cx(t)
        \end{align*}
        which depends additionally on a control signal $u \in\R^{m}$ provided by the discrete controller
        \begin{equation*}
            u(kh) = -Kx(kh),\quad k\in N_0
        \end{equation*} or \begin{equation*}
            u(kh) = -Ky(kh),\quad k\in N_0
        \end{equation*}
        if not the full state is known to the controller, but only some plant output (sensor data) $y \in\R^{p}$.
        The matrices $A, B, K$ are chosen with suitable dimensions. 

        The network induces delays in the loop, namely $\tau_{sc}$ between sensor and controller, as well as $\tau_{ca}$ between controller and actuator.

        In case of a time-invariant controller, the partial delays can be combined together into a single $\tau_k=\tau_{sc,k}+\tau_{ca,k}+t_{\text{calculation}}$ with the processing time of the controller.

        Assuming that the delay $\tau_k$ of each sample $k$ is less than the sampling period $h$ and that each data sample $x(t)$ fits into a single packet, the system equations can be written as
        \begin{align*}
            \dot{x}(t) &= Ax(t) + Bu(t),\\
            &\quad\quad\quad t\in[kh+\tau_k, (k+1)h+\tau_{k+1}]\\
            y(t) &= Cx(t) \\
            u(t^+) &= -Kx(t-\tau_k)\\
            &\quad\quad\quad t^+\in\{kh+\tau_k, k=0,1,2,...\}
        \end{align*}
        with the piecewise constant control signal $u(t^+)$ in the actuator.

        This plant system can be solved using the standard \textit{variation of constants} method in order to express the new state variables $u(kh)$ and $x((k+1)h)$ as function of their values at the previous sampling instant.

    \subsection{Hybrid Systems}
        A more general class of systems are called \textit{hybrid}, which consist of a continuous dynamics part and discrete events.
        The NCS model above can be written as such what allows applying stability theory for hybrid systems to derive conditions for asymptotical stability depending on the sampling rate $h$ and the network delay $\tau$.

    \subsection{Compensation of Network Induced Delay}
        If the plant and the controller have synchronized clocks, the sensor-controller delay can be determined in the controller.
        Using an estimator to approximate the evolved full plant state at time of reception even if only the partial state measurements $y(t)$ are available, one can try to compensate the sensor-controller delay by an estimator-predictor scheme.

        Having an estimation of the full plant state $\hat{x}(kh)$ for time $kh$, one awaits the reception of the plant output $y(kh)$ for this instant. Receiving this packet at time $kh+\tau_{sc,k}$ one can correct the former prediction $\hat{x}(kh)$ to a better estimation $\bar{x}(kh)$.
        Assuming that this estimation fulfills the equations describing the system, one forwards the estimation to $\bar{x}(kh+\tau_{sc,k})$ which is used to calculate the control command $u(kh+\tau_{sc,k})$.
        In order to prepare the next iteration, $\bar{x}(kh)$ is further forwarded to obtain a prediction of the plant state at time $(k+1)h$.

    \subsection{Modelling of Packet Loss}
        The potential loss of data packets on the network can be modeled as an \textbf{Asynchronous Dynamical System (ADS)}, which comprises continuous dynamics (described by differential/difference equations) and discrete dynamics (governed by finite automata).
        Assuming that the non-networked system is stable, that the network is lossy only between sensor and controller and that the packets contain $x(kh)$ to provide the full current state to the controller, a pair of difference equations
        \begin{align*}
            S_0:&\quad \bar{x}(kh) = \bar{x}((k-1)h)\\
            S_1:&\quad  \bar{x}(kh) = x(kh)
        \end{align*}
        is obtained.

        This system can be interpreted as a switch that closes at a certain rate $r$, indicating if a message is lost ($S_0$) or delivered ($S_1$). 
        In the case of $S_0$ the state in the controller $x(kh)$ is held at its previous value.
        For this system, Lyaponov theory gives conditions for exponential stability.

% \chapter{Conclusions}
    \section{Future Work}
    	compute invariants and differential invariants

    	uniform substitution calculus, proof rules, substitutes formula for a predicate symbol
    	US, classical rule in for first-order logic proof calculus
    	have axioms instead of axiom schemata
    	leads to a calculus with finite number od \ddL formulas as axioms
    	to facilitate a sound implementation in a prover

        completeness

    	state dependent delay
    	    % TODO: source for state-dependent and distributed delays
    % The definition of a DDE can be extended to state-dependent or distributed delays

    potential implementation in Keymaera X: use modularity, existing axioms, lemmata
    axiomatization, proof rules important for automatization of proofs
    for that as \dL in \cite{Platzer15Uniform} no axiom schemata, but finite number of axioms and proof rules (sets of formulas)
proof rule for substitution on axiom preserving soundness

\appendix

% TODO: choose good citation style. Chicago?
\nocite{Fulton16LogicProofs,*}
\bibliographystyle{plain}
\bibliography{Bibliography}

\end{document}
