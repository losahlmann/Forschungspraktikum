% fix bugs: fixes already in kernel now
%\RequirePackage{fixltx2e}


\documentclass[10pt]{report}

% encoding of tex-file
\usepackage[utf8x]{inputenc}

% for propper Umlaute
\usepackage[T1]{fontenc}

% proper hyphenation
\usepackage[english]{babel}

% better i18n Postscript version of Knuth's cm fonts, better than cm-super
\usepackage{lmodern}

% Mathematics
\usepackage{mathtools} % extension and fixes of/in amsmath
\usepackage{amssymb} % provides symbols, loads amsfonts
\usepackage{amsthm} % provides theorem environment
%\usepackage{nicefrac} % better slash fracs in inline

% For including figures, rotating or scaling text (dont use file extension)
\usepackage{graphicx}

% rotate figures
\usepackage{rotating}

% LS-Lab auxiliary math commands
\usepackage{math}

% LS-Lab logic commands: includes lcalculus, lmeta, lsemantics, lsyntax
\usepackage[
    varterms, sigmaterms, septerms,
    substopinline,
    modifopinline, % arrow notation for \imodif in semantics
    longinterpret,
    %bracketinterpret,%
    %fixformat,%
    sidenotecalculus,%
    %silentconst,%
    longseqcontext%
    ]{logic}

% LS-Lab differential dynamic logic commands
\usepackage[
    %bracketmodalinterpret,% use [[]] for semantics
    bracketmodalinterpret,%
    %fixformat,%
    %silentconst,% don't show `const' and `algebra'
    precisenames%
    ]{dL}


% own symbol definitions
%
% Math Environements
%
% italic text
\newtheorem{theorem}{Theorem}[chapter]
\newtheorem{corollary}[theorem]{Corollary}
\newtheorem{proposition}[theorem]{Proposition}
\newtheorem{lemma}[theorem]{Lemma}
% normal text
\theoremstyle{definition}
\newtheorem{definition}[theorem]{Definition}
\newtheorem{example}[theorem]{Example}

%
% General Mathematics
%
% natural numbers
\newcommand{\N}{\mathbb{N}}
% real numbers
\newcommand{\R}{\mathbb{R}}
% rational numbers
\newcommand{\Q}{\mathbb{Q}}
% sets
\DeclarePairedDelimiter{\braces}{\{}{\}} % \braces* uses \left\right, but without space before
\renewcommand{\set}[1]{\braces*{#1}} % override def in malgebra.sty
% integer range
\newcommand{\range}[2]{#1,\ldots,#2}
% functions: f\from\R\to\R
\newcommand*{\from}{\colon}
% define as
% FIXME: define "def" properly
%\newcommand{\defeq}{\,\stackrel{\text{def}}{=}\,}
% defined by
\newcommand\logeq{\mathrel{\vcentcolon\Longleftrightarrow}}
% norm
%\DeclarePairedDelimiter{\abs}{\lvert}{\rvert}
\DeclarePairedDelimiter{\nnorm}{\lVert}{\rVert}
\DeclarePairedDelimiter{\supnorm}{\lVert}{\rVert_{\text{\scriptsize{sup}}}}

% exponential function
\newcommand{\e}[1]{\text{e}^{#1}}

%
% Analysis
%
% derivative: Lagrange style
%\newcommand{\D}[1]{#1'} % ' defined in latex and is same as \prime
% derivative: Leibniz style
\renewcommand{\DD}[2]{\frac{\text{d} #1}{\text{d} #2}}
% differential in integral
\newcommand{\dx}[1][x]{\text{d}#1}

\newcommand{\continuouspws}[3][]{\ensuremath{C^{#1}_\text{pw}\ifthenelse{\equal{#2}{}}{}{\ifthenelse{\equal{#3}{}}{(#2)}{(#2,#3)}}}}

%
% Delay Differential Equation
%
% definition domain of right hand side
\newcommand{\deff}{\R\times\R^n\times\R^n}
% trajectory
\newcommand{\trajectory}[1][]{\gamma_{#1}}

%
% Differential Dynamic Logic
%
% hybrid programs
\newcommand{\dHP}{\text{dHP}\xspace}%
\newcommand{\dHPs}{\text{dHPs}\xspace}%
%% redefine notation for HP
\def\lprogramsname{\dHP}
% ddL
\newcommand{\ddL}{\textsf{dd{\kern-0.1em}$\mathcal{L}$ }}
\newcommand{\ddLformulas}{\lformulasname_{\ddL}}
\newcommand{\ddLterms}{\ltermsname_{\ddL}}
\newcommand{\signature}{\Sigma}
\newcommand{\varsymbols}{V}
\newcommand{\terms}{\lterms{\signature}{\varsymbols}}
\newcommand{\FOLformulas}{\lformulas[\FOL]{\signature}{\varsymbols}}
\newcommand{\FOL}{\text{FOL}}
\newcommand{\FOLR}{\FOL$_\R$}
% FIXME: use \mathcal{M}
\newcommand{\model}{M}
\newcommand{\interpret}[1][]{\ifthenelse{\equal{#1}{}}{I}{I(#1)}}
\newcommand{\universe}{D_{\model}}
\newcommand{\assignment}{\nu}
\newcommand{\ireachability}[2]{\rho\left(#2\right)}
%
% Delay Differential Dynamic Logic
%
\renewcommand{\ivr}{\chi}
\newcommand{\csfml}{\chi}

% formula of first-order real arithmetic
\newcommand{\asfmlfolR}{\chi}

% propositions
\newcommand{\asprop}{p}
\newcommand{\bsprop}{q}

% sets of formulas
\newcommand{\asfmls}{\Gamma}
\newcommand{\bsfmls}{\Delta}
\newcommand{\csfmls}{\Theta}

% states
\newcommand{\states}{\mathcal{S}}
\newcommand{\asstate}{\nu}
\newcommand{\bsstate}{\omega}
\newcommand{\csstate}{\mu}

\newcommand{\delayinterval}[1][T]{[-#1,0]}

\newcommand{\diffvars}{\D{\allvars}}
\newcommand{\delayedvars}{\mathcal{V}_\tau}

\newcommand{\statespace}[1][T]{\continuouspws[0]{\delayinterval[#1]}{\R^n}}
%\newcommand{\xtau}[1][]{\ifthenelse{\equal{#1}{}}{x[\tau]}{x[#1]}}
\newcommand{\x}[1][]{x[#1]}
\newcommand{\xtau}[1][-\tau]{x[#1]}
\newcommand{\Dxtau}[1][-\tau]{\D{x}[#1]}
\newcommand{\holdssince}[3][s]{\lforall{#1\in\delayinterval[#2]}{\left(#3\right)}}
\DeclareMathOperator{\HHtrm}{HH_{\ltermsname}}
\DeclareMathOperator{\HHfml}{HH_{\lformulasname}}
\DeclareMathOperator{\HHprg}{HH_{\lprogramsname}}

\newcommand{\xbartau}{\bar{x}_{\tau}}
\newcommand{\xbartaut}[1]{\bar{x}_{\tau,#1}}


\newcommand{\IFOL}{\interpretation[
    algebra=\model,
    %const=I,
    %assign=\assignment,
    % state=\nu,
    universe=\universe
    ]}

\newcommand{\IML}{\interpretation[
    algebra=\model,
    %const=I,
    %assign=\assignment,
    state=\nu,
    worlds=W,
    access=R,
    universe=\universe
    ]}

\newcommand{\IdL}{\dLint[state=\nu]}
%%%%%%%%%%%%%%%%%%%%%%%%%%%%%%%%%%%%%%%%%%%%%%%%%%%%%%%%%%%%%%%%%%%%%%%%%%%


\begin{document}

\title{Delay Hybrid Systems}

\author{Lorenz Sahlmann\\ École Polytechnique\\ Carnegie Mellon University}
\date{\today}

\maketitle

\begin{abstract}
    In this work we extend Differential Dynamic Logic with Delay Differential Equations.

    This requires an extension of the syntax, a (partially) redefinition of the semantics and the introduction of additional axioms and proof rules.

    This results in a superset of \dL which we call \textbf{Delay Differential Dynamic Logic}.
\end{abstract}

%\chapter{Introduction}

	dynamical systems
	mathematical model
	describing evolvement of state of as system over time
	modeling of embedded systems and cyber-physical systems
    discrete dynamical systems: difference equations, discrete state transition relations
    continuous dynamical systems: state evolves continuously, differential equation
    hybrid (dynamical) systems: combine discrete and continuous dynamics
    can capture very complex behaviour


	cyber-physical systems (CPS)
	combine computation, communication, control of physical processes/effects
	discrete and continuous dynamics
	often safety-critical, performance-critical
    safe control choices
    possible states accessible after control choice
	(prone to) software verification
    proof that always chooses safe control

    dynamical systems usually (uncountably) infinite state space
    finite number of tests cannot prove safety
    systematically obtain proofs

    classical safety, liveness, controllability, reactivity, quantified parametrized properties

	hybrid systems: mathematical model to describe CPS
	combine discrete dynamics/computation
	continuous dynamics: differential equations
    conditional switching
    nondeterminism
    repetition
	infinite state space

	examples:
    robotics, medical surgery robots
    electrical circuits
	automotive, self-driving cars (lane controllers for highway car traffic, controllers for intersections)
    speed limit control
	aviation, aircraft collision-avoidance systems (flyable roundabout maneuvers)
	railway, European train control system (ETCS)
    power plants
    chemical, biological processes
    medical models (events which can be seen as discrete with relation to continuous evolution)

	study logic of dynamical systems
    analyze and predict behaviour
    logics and proof principles

    differential dynamic logic (\dL)
    a concise overview and introduction is given in~\cite{Platzer12LogicsDynSys}
    a dynamic logic for hybrid systems
    logic/language for specify, verify safety and liveness properties
    of hybrid systems
    based on first-order modal logic and dynamic logic, first-order real arithmetic
    models of cyber-physical systems
    ordinary differential equations

    (nonlinear) real arithmetic

    transition behaviour as formulas

    differential invariants
    induction principle for differential equations

    theoretical results
    \emph{soundness} (everything provable is true)
    \emph{completeness} (everything true is provable)
    compositionality (denotational semantics, semantics (of models and formuals) functions of their parts, proofs structural decomposition, split complex systems in their parts, completeness: decomposition always successful)
    extendability (rules can be added to proof calculus)
    deductive power

	tutorial with examples modeled in \dL
	\cite{Quesel16Tutorial}

	mechanized proofs
	automatic and interactive theorem proving
	KeYmaeraX
    used successfully

	several formulations of dL
	earliest: sequent calculus \cite{Platzer10HybridSystems}, tuned for automatic proof search, KeYmaera
    automatically find (differential) invariants
	axiomatic formulation \cite{Platzer15Uniform}, implemented in KeYmaeraX

    extensions to \dL
    \emph{differential-algebraic dynamic logic} \DAL: differential-algebraic equations and constraints
    \emph{differential temporal dynamic logic} \dTL: temporal properties, trace semantics


	capture non-deterministic
	not known a priori

	\section{Related Work}
	\cite{Huang16BoundedVerificationNNDS}

\chapter{Delay Differential Equations}\label{sec:delay-differential-equations}

\section{Piecewise Continuous Functions}
    \label{sec:piecewise-continuous-functions}
    
    The following definition is motivated by capturing the character evolution arising from hybrid systems. We will see that we can consider such to be piecewise continuous.

    % \begin{definition}[Piecewise Continuous]\label{def:piecewise-continuous}
    %     Let $D=[a,b]\subseteq\R$ be a closed interval (this includes the cases when $a=-\infty$ or $b=\infty$, or both). The mapping $x:D\rightarrow\R^n$ is called \emph{piecewise continuous} if and only if there is a finite subdivision $\{t_i:i=\range{0}{m}\}$ of $D$ (i.e.\ $a=t_0<t_1<\ldots<t_m=b$) such that $x$ is continuous on each interval piece $[t_i,t_{i+1})$ for all $i=\range{0}{m-1}$ and the left sided limits
    %     \begin{equation}
    %         \lim_{\substack{t\upto t_{i+1}\\ t\in[t_i,t_{i+1})}} x(t)
    %     \end{equation}
    %     exist. Hence $x(b)$ can be an isolated point and this right interval limit $b$ is the only spot where such is allowed.

    %     We denote by $\Cnpw[0]{D}{\R^n}$ the set of \emph{piecewise continuous functions} on the compact interval $D$ (this excludes the cases with $\pm\infty$), mapping to $\R^n$.
    % \end{definition}

    \begin{definition}[Piecewise Continuously Differentiable]\label{def:pw-cont-diff}
        % FIXME: find better word for 'subdivision'. partition?
        Let $D=[a,b]\subseteq\R$ be a closed interval (this includes the cases when $a=-\infty$ or $b=\infty$, or both). The mapping $x:D\rightarrow\R^n$ is called $n$-times \emph{piecewise continuously differentiable} if and only if there is a finite subdivision (ordered set) $\{t_i:i=\range{0}{m}\}$ of $D$ (i.e.\ $a=t_0<t_1<\ldots<t_m=b$) such that $x$ is $n$-times continuously differentiable on each interval piece $(t_i,t_{i+1})$ with continuable derivatives on $\compactum{t_i}{t_{i+1}}$.

        This means, for all $i=\range{0}{m-1}$ and for all $k=\range{0}{n}$ exist the left sided limits
        \begin{equation}
            \lim_{\substack{t\upto t_{i+1}\\ t\in(t_i,t_{i+1})}} \D[k]{x}(t)
        \end{equation}
        as well as the right sided limits
        \begin{equation}
            \lim_{\substack{t\downto t_{i}\\ t\in(t_i,t_{i+1})}} \D[k]{x}(t) =: \D[k]{x}(t_i)
        \end{equation}
        which are supposed to coincide with the value of $\D[k]{x}$ at $t_i$.
        Hence $x(b)$ can be an isolated point and this right interval limit $b$ is the only spot where such is allowed.
        In the case $n=0$, we say $x$ is \emph{piecewise continuous}.

        We denote by $\Cnpw[n]{D}{\R^n}$ the set of \emph{$n$-times piecewise continuously differentiable functions} on the compact interval $D$ (this excludes the cases with $\pm\infty$), mapping to $\R^n$, and respectively, by $\Cnpw[0]{D}{\R^n}$ the \emph{piecewise continuous functions}.
    \end{definition}

    % TODO: sup-norm for pw

    \begin{lemma}[]\label{lm:pc-integrable}
        A \emph{piecewise continuous function}, as defined in Definition~\ref{def:pw-cont-diff} is (Riemann) integrable.
    \end{lemma}
    \begin{proof}
        See standard analysis literature, such as \cite{Rudin76PrinciplesAnalysis} (Theorem~6.10) or \cite{Gathmann12GDM} (Example~11.16b).
        % ObdA: one subint with jump at end
    \end{proof}

    The following lemma generalizes the fundamental theorem of calculus to piecewise continuous derivatives.

    \begin{lemma}[]\label{lm:pc-hauptsatz}
        Let $F\in\Cn[0]{\compactum{a}{b}}{} \cap \Cnpw[1]{\compactum{a}{b}}{}$ with the subdivision $\subdivision{a=t_0}{t_m=b}$ and piecewise derivative $f$.
        %of $\compactum{a}{b}\subset\R$. 
        For all $t\in\compactum{a}{b}$ it holds
        \begin{equation*}
            F(t)-F(a) = \integral{a}{t} f(s)\dx[s]
            %\sum_{i=0}^k\int\limits_{t_i}^{t_{i+1}}f(t)\dx[t] + \int\limits_{t_k}^s f(t)\dx[t]
        \end{equation*}
        %where $t_k\leq s < t_{k+1}$.
        % FIXME: what about a=-inf or b=inf?
    \end{lemma}
    \begin{proof}
        On each interval $\compactum{t_{i-1}}{t_i}$ of the subdivision, $f$ is piecewise continuous and hence integrable.

        For all $\zeta\in\open{t_{i-1}}{t_i}$ is $F$ differentiable on $\compactum{t_{i-1}}{\zeta}$ with $\D{F}=f$.
        By the fundamental theorem of calculus (cf.\ standard analysis literature, e.g.~\cite{Gathmann12GDM,Rudin76PrinciplesAnalysis}), it follows
        \begin{equation*}
            \denseintegral{t_{i-1}}{\zeta} f(s)\dx[s] = F(\zeta)-F(t_{i-1})
        \end{equation*}
        and by the continuity of $F$ that
        \begin{equation*}
            \denseintegral{t_{i-1}}{t_i} f(s)\dx[s]
            = \lim_{\zeta\to t_i}\denseintegral{t_{i-1}}{\zeta} f(s)\dx[s]
            = \lim_{\zeta\to t_i} F(\zeta)-F(t_{i-1})
            = F(t_i)-F(t_{i-1})
        \end{equation*}
        For any $t\in\compactum{a}{b}$, there is a $k\in\set{\range{1}{m}}$ such that $t\in\closedopen{t_{k-1}}{t_k}$ (in the case $t=b$, set $k=m$), summation over $i=\range{1}{k}$ yields the telescoping series
        \begin{equation*}
            F(t)-F(a) = \sum_{i=1}^{k} \denseintegral{t_{i-1}}{t_i} f(s)\dx[s] + \integral{t_j}{t} f(s)\dx[s]
        \end{equation*}
        which is by the additivity of the integral
        \begin{equation*}
            F(t)-F(a) = \integral{a}{t} f(s)\dx[s]
        \end{equation*}
    \end{proof}

% \begin{figure*}[h]\centering
%     \begin{subfigure}[t]{0.5\textwidth}\centering
%         \includegraphics[width=\textwidth]{figures/allowed.png}
%         \caption{Admissible piecewise continuous function.}
%         \label{fig:allowed}
%     \end{subfigure}
%     \begin{subfigure}[t]{0.5\textwidth}\centering
%         \includegraphics[width=\textwidth]{figures/not-allowed.png}
% 	    \caption{Not allowed!}
% 	    \label{fig:not-allowed}
%     \end{subfigure}
%     \caption{Examples to Definition \ref{definition-piecewise-continuous}.}
% \end{figure*}


\section{Definition DDE}
    \label{sec:definition-dde}

    \begin{definition}[Delay Differential Equation]\label{def:dde}
        Let $f\from\deff\to\R^n$ and $\tau_j > 0$ for $j=\range{1}{k}$. Put $\taumax\defeq\max_j\set{\tau_j}$.

        A functional equation of the form
        \begin{equation}\label{eq:dde}
            \D{x}(t) = f(t,x(t),x(t-\tau_1),\ldots,x(t-\tau_k))
        \end{equation}
        is called \emph{delay differential equation} (DDE) with \emph{multiple constant, discrete delays $\tau_j$}.
        It is \emph{autonomous} if its right hand side $f$ is time independent and \emph{pure} if the right hand side only depends on $x(t-\tau_i)$ and not on $x(t)$.

        A DDE can be equipped with an \emph{initial condition} $x_{\tzero}$. It specifies the values of $x$ on $[\tzero-\taumax, \tzero]$ on which the right hand side depends.
        Such a pair is called \emph{initial value problem (IVP)}:
        \begin{equation}\label{eq:ivp}
            \begin{cases}
                \D{x}(t) = f(t,x(t),x(t-\tau_1),\ldots,x(t-\tau_k)) & \text{for } t\geq\tzero\\
                x(t) = x_{\tzero}(t) & \text{for } t\in [\tzero-\taumax,\tzero]
            \end{cases}
        \end{equation}
    \end{definition}

    % Since we only consider autonomous DDEs, we can without loss of generality restrict to the case of initial time $t_0=0$.

    % The definition of a DDE can be extended to multiple constant discrete delays. For simplicity, we restrict here to a single delay.


\section{Definition of Solution}
    \label{sec:definition-of-solution}

    \begin{definition}[Solution of DDE]\label{def:solution-dde}
        A function $x\from\compactum{\tzero-\taumax}{\tzero+T}\to\R^n$ is called \emph{(local) solution} of the initial value problem~\eqref{eq:ivp}, if and only if
        $x$ is continuous and piecewise continuously differentiable on $\compactum{\tzero}{\tzero+T}$ (in the sense of Def.~\ref{def:piecewise-continuous}) with subdivision $\Delta$.
        This means, when $\Delta=\subdivision{\tzero=t_0}{t_m=\tzero+T}$, $x$ is continuously differentiable on each interval $(t_i,t_{i+1})$
        %there exists a $T>0$ such that
        % FIXME: local solution is on a single subdiv int only -> cont diffable
        %$\restrict{x}{(t_i,t_{i+1})}\in \Cn[1]{\compactum{\tzero}{\tzero+T}}{\R^n}$ with
        with
        \begin{equation*}
            \D{x}(t) = f(t,x(t),x(t-\tau_1),\ldots,x(t-\tau_k))
        \end{equation*}
        for all $t\in (\tzero,\tzero+T)$ and in $t=t_i$, it holds
        % TODO: for the right-hand derivative?
        % cf ODE sol
        \begin{equation*}
            \lim_{s\downto t_i}\D{x}(s) = f(t_i,x(t_i),x(t_i-\tau_1),\ldots,x(t_i-\tau_k))
        \end{equation*}
        and $x$ obeys the initial condition:
        \begin{equation*}
            x(t) = x_{\tzero}(t) \quad\text{for } t\in [\tzero-\taumax,\tzero].
        \end{equation*}
        % FIXME: global solution should allow knicks
        If the function $x$ is a solution for all $T>0$, it is called \emph{global}.

        %TODO: differentiable in right rand point? need not derivative in right hand point
        %TODO: Fortsetzbarkeit For example initial condition has jump, this point is limit for local solution.
    \end{definition}

    

% \begin{figure}[h]\centering
%     \includegraphics[width=\textwidth]{figures/multiple.png}
% 	\caption{Illustration of proof to Lemma \ref{lemma-continuity}}
% 	\label{fig:not-allowed}
% \end{figure}

% FIXME: This lemma is wrong. Show instead integrability of f(t,x_t)
\begin{lemma}
    \label{lemma-continuity}

    % Let $x:[\tzero-\tau,\tzero+T] \rightarrow \R^n$ be piecewise continuous (as in Definition \ref{definition-piecewise-continuous}) with the subdivision $\{t_0,\ldots,t_k\}$, i.e. there are $k$ subintervals.
    %
    % Then $t \mapsto x_t = x(t+\theta)$, where $\theta\in[-\tau,0]$, is a piecewise continuous mapping from $[\tzero,\tzero+T]$ into $\statespace[\tau]$.
\end{lemma}

\begin{proof}
    % $x$ is piecewise continuous and hence uniformly piecewise continuous on the compact interval $I=[\tzero-\tau,\tzero+T]$.
    % i.e. uniformely continuous on each subinterval with stetiger Fortsetzung in right side.
    % \begin{equation}
    %     \forall\epsilon >0 \exists\delta_i >0 \forall t,s\in I_i: \quad \abs{t-s}<\delta_i \Rightarrow \nnorm{x(t)-x(s)}<\epsilon
    % \end{equation}
    % Let $\epsilon > 0$. $x|_{[t_i,t_{i+1}]}$ (with stetiger fortsetzung in right interval limit) is uniformly continuous, i.e. there is a $\delta_i > 0$ (for the given $\epsilon$), such that $\forall\,t, s \in [t_i,t_{i+1}]$ holds
    % % TODO: can use \leq ?
    % \begin{equation}
    %     \abs{t-s} < \delta_i \Rightarrow \nnorm{x(t)-x(s)} < \epsilon
    % \end{equation}
    %
    % Among the given $\delta_i$, choose the smallest as $\delta = \min_i \delta_i$.
    %
    % For any $i$ and $s,t\in [t_i,t_{i+1})\subset [\tzero,\tzero+T]$ with $\abs{t-s}<\delta$, it holds
    % \begin{equation}
    %     \supnorm{x_t - x_s} = \sup_{\theta\in [-\tau,0]}\nnorm{x(t+\theta) - x(s+\theta)} < \epsilon
    % \end{equation}
    % since $t+\theta, s+\theta \in I$
    % Hence $t \mapsto x_t$ is uniformely continuous on $[t_i,t_{i+1})$.
\end{proof}


\section{Method of Steps}
    \label{sec:method-of-steps}
    
    for $t\in [0,\tau]$, $x$ must satisfy the following ordinary initial value problem obtained by plugging the initial function into equation (??). For suitable $f$ and $x_0$, the existence (and uniqueness) of a solution on $[0,\tau]$ is guaranteed by ODE theory (\ldots{} or Picard-Lindelöf theorems).

    This procedure can then be applied repeatedly to extend the obtained solution by steps of length $\tau$.





\section{Existence and Uniqueness of Solutions}
    \label{solutions-existence-uniqueness}

    will consider rhs cont and lip
    $f$ Lipschitz with piecewise continuous initial function have existence and uniqueness ???? smoothing

    \begin{definition}[Lipschitz Continuity]\label{def:lipschitz}
        % similar to \cite{pruesswilke10GewDiffGl,Smith10IntroDDE}
        % FIXME: dont I need xtau in right side ??? 
        A function $f\from\deff\to\R^n$ is called \emph{(locally) Lipschitz continuous} in its second argument if and only if for all $a,b\in\R$ and $M>0$ there is a $L>0$ such that
        \begin{equation*}
            % TODO: is L(\nnorm*{x-x}+\nnorm*{y-y}) better? is equiv, with different L
            % FIXME: or just say Lipschitz continuous with respect to two other arguments, once for x once for y -> compare proof
            \nnorm*{f(t,x,y) - f(t,\bar{x},y)} \leq L\max\left\{\nnorm*{x - \bar{x}},\nnorm*{y - \bar{y}}\right\}
        \end{equation*}
        for all $t\in [a,b]$ and $x,\bar{x},y\in\R^n$ with $\nnorm{x},\nnorm{\bar{x}},\nnorm{y}\leq M$.
    \end{definition}

    \begin{lemma}\label{lm:bounded-lipschitz}
        Let $f\from\deff\to\R^n$ be continuous and Lipschitz continuous in its second argument.

        For any given compact interval $\compactum{a}{b}$ and $M>0$ there exists a bound $K>0$ such that
        \begin{equation}
            \nnorm{f(t,x,y)}\leq K
        \end{equation}
        for all $t\in\compactum{a}{b}$ and $x,y\in\R^n$ with $\nnorm{x},\nnorm{y}\leq M$.
    \end{lemma}
    \begin{proof}
        Let $L$ be the Lipschitz constant of $f$ for the given $\compactum{a}{b}$ and $M$. Then
        \begin{multline*}
            \nnorm{f(t,x,y)} \leq \nnorm{f(t,x,y) - f(t,0,y)} + \nnorm{f(t,0,y)}\\
            \leq L\nnorm{x-0} + \nnorm{f(t,0,y)} \leq LM+P = K
        \end{multline*}
        for $t\in\compactum{a}{b}$ and $x,y\in\R^n$ with $\nnorm{x},\nnorm{y}\leq M$. We used the continuity of $f$ on the compact set $S=\compactum{a}{b}\times\set{z\in\R^n\with\nnorm{z}\leq M}$ for the existence of
        \begin{equation*}
            P = \max_{(s,z)\in S}\nnorm{f(s,0,z)}
        \end{equation*}
    \end{proof}

    \begin{lemma}\label{lm:integral-equation}
        %TODO: compare with ODE lecture notes
        Finding a solution of the initial value problem~\eqref{eq:ivp} is equivalent to solving the integral equation
        \begin{equation}\label{eq:integral-equation}
            \begin{cases}
                x(t) = x_{\tzero}(\tzero) + \int_{\tzero}^t f(s,x(s),x(s-\tau))\dx[s] & \text{for } t\geq\tzero\\
                x(t) = x_{\tzero}(t) & \text{for } t\in [\tzero-\tau,\tzero]
            \end{cases}
        \end{equation}
        where ... (same as for ivp)
        and is continuous in t.
        integral componentwise, f vector valued
    \end{lemma}
    \begin{proof}
        Let $x$ be a solution of the IVP. Thus $x$ is (by definition) continuous on $\compactum{\tzero}{\tzero+T}$ and piecewise continuous on $\compactum{\tzero-\tau}{\tzero}$. This means that the chain $f(t,x(t),x(t-\tau))$ is piecewise continuous and hence integrable on $\compactum{\tzero}{\tzero+T}$. Furthermore, $x$ is (by definition) piecewise differentiable.
        The fundamental theorem of calculus states
        \begin{equation*}
            x(t) = x_{\tzero}(\tzero) + \int_{\tzero}^t f(s,x(s),x(s-\tau))\dx[s]
        \end{equation*}
        for $t\geq\tzero$.

        Conversely, let $x\from [\tzero-\tau,\tzero+T]$ be a solution of the integral equation~\eqref{eq:integral-equation}, i.e. $x(t)=x_{\tzero}(t)$ for all $t\in\compactum{\tzero-\tau}{\tzero}$ and $x(t) = x_{\tzero}(\tzero) + \int_{\tzero}^t f(s,x(s),x(s-\tau))\dx[s] =: F(t)$ for $t\in\compactum{\tzero}{\tzero+T}$ for a $T>0$.
        HS -> continuous
        % FIXME: f cont uberall in Voraussetzung?
        Since $f$ is by the precondition continuous, x cont -> integrand cont iff xtau cont
        Let $\set{t_0-\tau < \ldots < t_m-\tau}$ be the subdivision of the initial condition $\x_{\tzero}\in\statespace$.
        Since $f$ is integrable on and continuous on ..., the fundamental theorem of calculus states the differntiability of F on $(t_i,t_{i+1})$ with
        \begin{equation}
            x'=F'=f()
        \end{equation}
        % FIXME: t_i is jumppoint of init cond, can be x_sigma or x
        show $\D{F}$ is continuable in jump point $t_i$
        $\lim_{t\downto t_i}\D{F}(t)=\lim_{t\downto t_i}f(t,x(t),x(t-\tau))=f(t_i,x(t_i),x(t_i-\tau))$ by continuity of $f$, $x|(t_i,t_{i+1})$ and def of pw cont $x|(t_i-\tau,t_{i+1}-\tau)$
        same way: limit exists for $t\upto t_{i+1}$



        integrate from discontinuity of $\xbartaut{t}$ to discontinuity and proof stetige fortsetzbarkeit at these points
    \end{proof}

    \begin{theorem}[Existence of unique solution]\label{thm:solution-existence}
        Consider the Delay Differential Equation
    %TODO: do we need global existence or just local?
        \begin{equation}
            \begin{cases}
                \D{x} = f(t,x(t),x(t-\tau)) & \text{for } t\geq\tzero\\
                x(t) = x_\tzero(t-\tzero)   & \text{for } t\in [\tzero-\tau,\tzero]
            \end{cases}
        \end{equation}
        with $f\from\deff\to\R^n$ continuous and satisfying the (local) Lipschitz condition in its second argument (Def.~\ref{def:lipschitz}).

        % where $\nnorm{\cdot}$ denotes the Euclidian norm on $\R^n$ and $\supnorm{\cdot}$ the supremum norm of the Banach space of continuous functions on $[-\tau,0]$.

        Then for each \emph{initial condition} $x_{\tzero}\in\statespace[\tau]$ and start time $\tzero$, there \textbf{exists} a \textbf{unique local solution} of the IVP on a time interval $[\tzero-\tau, \tzero+T]$. The duration $T>0$ depends on the sup-norm and discontinuity points of the initial condition. (?)
        This solution is continuous and piecewise differentiable on $\compactum{\tzero}{\tzero+T}$ with subdivision $t_i+\tau$.
    \end{theorem}

    The proof is smiliar to the proof of the existence theorem (Theorem 3.7) given in~\cite{Smith10IntroDDE}.
    \begin{proof}
        % FIXME: where sup-norm?
        As a piecewise continuous function, the initial condition can bounded by $M\geq \supnorm{x_\tzero}$ on $\delayinterval[\tau]$.
        
        % FIXME: do I need t_0+tau or is just \tau okay? do I need x not to be pw, just cont in proof?
        If $\set{\range{-\tau=t_0}{t_k=0}}$ is the subdivision of $x_{\tzero}$, we choose $T=\min\set{t_0+\tau,\frac{M}{K}}$.
        
        % FIXME: M or 2M?
        Let $K>0$ be the upper bound for $f$ from Lemma~\ref{lm:bounded-lipschitz} on the set $S=[\tzero,\tzero+T] \times \{x\in R^n: \nnorm{x}\leq 2M\}\times \{y\in R^n: \nnorm{y}\leq 2M\}$ and $L>0$ the Lipschitz constant of $f$ for that set.

        % FIXME: why continuous? its pw cont? cont in tzero
        We construct a series $(x_{(m)})_{m\in\N_0}$ of piecewise continuous functions which approximates the solution of the initial value problem.
        Set
        \begin{equation}
            x_{(0)}(t)= \begin{cases}
                x_\tzero(0) & t\in [\tzero,\tzero+T]\\
                x_\tzero(t-\tzero) & t\in [\tzero-\tau,\tzero]
            \end{cases}
        \end{equation}
        For $m\in\N_{>0}$ define
        \begin{equation}
            x_{(m)}(t)= \begin{cases}
                x_\tzero(0) + \int_\tzero^t f(s,x_{(m-1)}(s),x_{(m-1)}(s-\tau))\dx[s] & t\in [\tzero,\tzero+T]\\
                x_\tzero(t-\tzero) & t\in [\tzero-\tau,\tzero]
            \end{cases}
        \end{equation}
        % FIXME: why exists integral? f cont, x in int even cont
        Integral exists
        It holds for all $m>0$ and $t\in \compactum{\tzero-\tau}{\tzero}$ by definition of the series
        \begin{equation}
            \nnorm*{x_{(m)}(t)-x_{(m-1)}(t)}=0
        \end{equation}
        We show by induction over $m$ that for all $t\in [\tzero,\tzero+T]$ it holds
        \begin{equation}
            \nnorm*{x_{(m)}(t)-x_{(m-1)}(t)} \leq \frac{K}{L}\frac{L^m (t-\tzero)^m}{m!}.
        \end{equation}
        Since obviously $\nnorm{x_{(0)}(t)}\leq M$, the statement for $m=0$ follows from the boundedness of $f$ on $S$ and the triangle inequality for integrals:
        \begin{equation}
            \nnorm{x_{(1)}(t)-x_{(0)}(t)} = \nnorm*{\int_\tzero^t f(s,x_{(0)}(s),x_{(0)}(s-\tau))\dx[s]} \leq K(t-\tzero)
        \end{equation}
        In the inductive step we can apply
        Since for any $m>0$, it holds by the triangle inequality and by the choice of $T$
        % FIXME: why x(m)(t) smaller than 2M, such that K holds?
        % TODO: why do integral and norm commute? once integral over vectors, once over scalars
        \begin{align}\label{eq:bounded-xm}
            \nnorm*{x_{(m)}} &\leq \nnorm*{x_\tzero(0)} + \int_\tzero^t \nnorm*{f(s,x_{(m-1)}(s),x_{(m-1)}(s-\tau))}\dx[s]\\
            &\leq M + K(t-\tzero) \leq M+KT\\
            &\leq 2M
        \end{align}
        if $\nnorm{x_{(m-1)}(t)}\leq 2M$.
        It follows by the Lipschitz property of $f$
        \begin{multline*}
            \nnorm*{x_{(m+1)}(t)-x_{(m)}(t)}=\\
            = \nnorm*{\int_\tzero^t f(s,x_{(m)}(s),x_{(m)}(s-\tau)) - f(s,x_{(m-1)}(s),x_{(m-1)}(s-\tau))\dx[s]}\\
            \leq L \int_\tzero^t \nnorm*{x_{(m)}(s) - x_{(m-1)}(s)}\dx[s]\\
            \leq \frac{L^m K}{m!} \int_\tzero^t (s-\tzero)^m\dx[s]
            = \frac{L^m K}{(m+1)!}(t-\tzero)^{m+1}
        \end{multline*}
        %We use this bound and the triangle inequality in
        The Cauchy criterion for convergent series (\cite{Gathmann12GDM} 6.13, \cite{Rudin76PrinciplesAnalysis} 3.22) applied to the exponential series states that
        \begin{equation*}
            % "\ " needed for space
            \mforall{\epsilon>0}\ \mexists{n_0\in\N_0}\ \mforall{m\geq k\geq n_0}\holds \sum_{i=k+1}^m \frac{(LT)^i}{i!} <\epsilon
        \end{equation*}
        So for any $\epsilon>0$ exist $k\in\N_0$ and $m\geq k$, such that
        \begin{align*}
            \nnorm*{x_{(m)}(t)-x_{(k)}(t)} \leq{} & \nnorm*{x_{(m)}(t)-x_{(m-1)}(t)} + \nnorm*{x_{(m-1)}(t)-x_{(m-2)}(t)} + {}\\
            & + \ldots + \nnorm*{x^{(k+1)}(t)-x^{(k)}(t)}\\
            \leq{} & \frac{K}{L}\frac{L^m (t-\tzero)^m}{m!} + \frac{K}{L}\frac{L^{m-1} (t-\tzero)^{m-1}}{(m-1)!} + {}\\
            & + \ldots +\frac{K}{L}\frac{L^{k+1} (t-\tzero)^{k+1}}{(k+1)!}\\
            \leq{} & \frac{K}{L}\sum_{i=k+1}^m \frac{(LT)^i}{i!} < \varepsilon
        \end{align*}
        for all $t\in [\tzero,\tzero+T]$, i.e. $x_{(m)}$ is a Cauchy sequence

    
        % FIXME: show that this a Cauchy series
        %This is the tail of the convergent exponential series and hence it converges to zero for $k\to\infty$ (boundedness and positivity of summands, monotonicity crit).

        % FIXME: why continuous? since integral exists
        Since $x_{(m)}$ is continuous on $[\tzero,\tzero+T]$, this Cauchy
        sequence admits a limit $x$ in the Banach space $\continuouss[0]{[\tzero,\tzero+T]}{\R^n}$ in terms of the supremum-norm.

        Again, we extend $x$ to $[\tzero-\tau,\tzero]$ with $x_\tzero$, such that $x\in\Cnpw[0]{[\tzero-\tau,\tzero]}{\R^n}$.


        

        Since by the continuity of the supremum norm it follows from~\eqref{eq:bounded-xm} that
        \begin{equation*}
            \supnorm*{x}=\lim_{m\to\infty}\supnorm*{x_m}\leq 2M
        \end{equation*}
        can apply Lipschitz property of $f$
        \begin{equation*}
            \sup_{t\in\compactum{\tzero}{\tzero+T}}\nnorm*{f(s,x_m(s),x_m(s-\tau))-f(s,x(s),x(s-\tau))} \leq \sup_{t\in\compactum{\tzero}{\tzero+T}}\nnorm*{x_m(t)-x(t)}
        \end{equation*}
        Due to the uniform convergence (conv in sup-norm) of $x_{(m)}\to x$, we get the uniform convergence
        \begin{equation*}
            f(s,x_m(s),x_m(s-\tau)) \xrightarrow{m\to\infty} f(s,x(s),x(s-\tau))
        \end{equation*}
        and hence the integral and the limit process swap and by
        \begin{align*}
            x(t) = \lim_{m\to\infty} x^{(m+1)} &= x_\tzero(0) + \lim_{m\to\infty}\int_\tzero^t f(s,x^{(m)}(s),x^{(m)}(s-\tau))\dx[s]\\
            &= x_\tzero(0) + \int_\tzero^t f(s,x(s),x(s-\tau))\dx[s]
        \end{align*}
        it follows that $x$ solves the integral equation and hence, by Lemma~\ref{lm:integral-equation},
        this proves the existence of a solution to the DDE.
        % TODO: continuous because limit in Banach space, diffable and subdiv see integral equiv lemma

        % TODO: can one solution be on [\tzero, T_2] with T_2<T ?
        It remains to show uniqueness.
        Let $x$ and $\bar{x}$ be two solutions of the DDE on $[\tzero,\tzero+T]$.
        By Lemma \ref{lm:integral-equation} they are equivalent to solutions of the integral equations
        \begin{equation}
            x(t) = x_\tzero(0) + \int_\tzero^t f(s,x(s),x(s-\tau))\dx[s]
        \end{equation}
        and
        \begin{equation}
            \bar{x}(t) = x_\tzero(0) + \int_\tzero^t f(s,\bar{x}(s),\bar{x}(s-\tau))\dx[s]
        \end{equation}
        For $t\in [\tzero,T]$, we set
        \begin{align*}
            \rho(t) &:= \nnorm*{x(t)-\bar{x}(t)} \leq \int_\tzero^t \nnorm*{f(s,x(s),x(s-\tau))-f(s,\bar{x}(s),\bar{x}(s-\tau))}\dx[s]\\
            & \leq L \int_\tzero^t \nnorm*{x(s)-\bar{x}(s)}\dx[s] = L \int_\tzero^t \rho(s)\dx(s)\\
            &= L \int_\tzero^t \e{-\alpha s}\rho(s)\e{\alpha s}\dx[s] \leq L \sup_{s\in [\tzero,\tzero+T]}\left(\e{-\alpha s}\rho(s)\right)\int_\tzero^t \e{\alpha s}\dx[s]\\
            & \leq\frac{L}{\alpha}\e{\alpha t} \sup_{s\in [\tzero,\tzero+T]}\left(\e{-\alpha s}\rho(s)\right)
        \end{align*}
        with $L$ the Lipschitz constant of $f$ on the set ...
        and $\rho$ is continuous, since $x$ continuous
        Choosing $\alpha=2L$ and multiplying with $\e{-\alpha t}>0$ leads to
        \begin{equation}
            \rho(t)\e{-2Lt} \leq \frac{1}{2}\sup_{s\in [\tzero,\tzero+T]}\left(\e{-2L s}\rho(s)\right)
        \end{equation}
        for all $t\in [\tzero,\tzero+T]$
        \begin{equation}
            0 \leq \sup_{t\in [\tzero,\tzero+T]}\left(\rho(t)\e{-2Lt}\right) \leq \frac{1}{2}\sup_{s\in [\tzero,\tzero+T]}\left(\e{-2L s}\rho(s)\right)
        \end{equation}
        That is only possible if $\rho(t)=0$ for all $t\in [\tzero,\tzero+T]$, which means $x(t)=\bar{x}(t)$.

        % TODO: still needed?
        just proof existence/uniqueness on each peace of continuity proof continuity at knots with Lemma of integral equ

    \end{proof}

% TODO: on [\tzero,t_1] DDE equiv to ODE/IntEq
% -> ex unique sol on [\tzero, t_1]
% -> ex unique sol on [\tzero,\tau] (glob Lip of f on [tzero,tau])
% -> ex unique sol on [\tzero,2\tau] (continuous?, diffable?)
% show continuity and pw diffable (nth to show)
    % \begin{lemma}[cont]\label{lm:c}
    %     $x_1$ loc sol on $\compactum{\tzero-\tau}{\tzero+t_1}$ for init cond $x_{\tzero}$
    %     $x_2$ and loc sol on $\compactum{\tzero+t_1-\tau}{\t_1+T}$ for init cond $x_1$
    %     then $x_1(\tzero+t_1)=x_2(\tzero+t_1)$
    %     follows from initial cond $x_1$
    % \end{lemma}
\begin{corollary}
    \label{cor:continuability-of-solution}

    % TODO: What is derivation in randpunkten of interval [] ?
    If in Theorem \ref{theorem-solution-existence} $T=t_1-\tau$, can reapplay theorem with starting point $\tzero=\tzero_{old}+t_1-\tau$. Get existence of unique solution on $[\tzero-\tau,\tzero+S]$ with $S>T$.
\end{corollary}

\begin{corollary}
    \label{corollary}
    If f is polynomial in $t$, $x(t)$ and $x(t-\tau)$ then theorem holds

    polynomial -> continuously differentiable -> locally Lipschitz
% IDEA: can show? init cond bounded by M, and loc sol bounded by M, get glob sol since f glob Lip on set of bounded inputs?


    %TODO: put after uniqueness theorem, need uniqueness and existence so that amap well-defined
    The notion of solution for an autonomous DDE as given above can be lifted to be a trajectory $\trajectory[x]$ in the statespace
    \begin{equation}
        \trajectory[x] \from [0,T] \to \statespace[\tau],\\
        t \mapsto \xbartaut{t}
    \end{equation}

    The \textbf{state} at time $t$ is a function which provides a time limited history up to the current time. This is all information needed to determine (using the DDE) to determine the solution for time $\geq t$. It is defined as $\xbartaut{t}(s)\defeq x(t+s)$ for $s\in [-\tau,0]$. In the case of $t=0$, we simplify the notation to $\xbartau \defeq \xbartaut{0}$.
    This notion of solution is a \emph{dynamical systems} point of view which later turns out to be useful.

    

%TODO: can write DDE (eq??) from definition as

\begin{equation}
    \begin{cases}
        \D{x}=f(\xbartaut{t})\defeq g(\xbartaut{t}(0),\xbartaut{t}(-\tau)) &\text{for } t\geq 0\\
        x(t)=x_0(t) & \text{for } t\in[-\tau,0]
    \end{cases}
\end{equation}
\end{corollary}

\begin{proof}

\end{proof}

% TODO: non-autonomous -> autonomous

\section{Example}\label{example}
The basic ODE IVP
\begin{equation}
    \begin{cases}
        \D{x}(t) = -x(t)\\
        x(0) = x_0
    \end{cases}
\end{equation}
has the solution $x(t)=x_0 e^{-t}$. However the similiar DDE
\begin{equation}
    \begin{cases}
        \D{x}(t) = -x(t-\tau) & t\geq 0\\
        x(t) = x_0(t) & -\tau\leq t\leq 0
    \end{cases}
\end{equation}
has a much richer dynamics, but solution (as series) for $x_0\equiv 1$, can compute first solutions by method of steps. \ldots{}

\begin{figure}[h]\centering
    \includegraphics[width=\textwidth]{figures/piecewise-initial-function.png}
	%\caption{}
	\label{fig:not-allowed}
\end{figure}



\chapter{Introduction to Logic}
\label{sec:introduction-logic}

    \section{First-Order Logic}
    \label{sec:first-order-logic}

        as defined in \cite{Platzer10HybridSystems} and \cite{Huth04LogicInCS}

        First-order logic (\FOL) defines  a syntax of logical formulas


        define inductively
        set of function and predicate symbols, called signature $\Sigma$. alphabet to built well-formed formulas from

        function: takes value of argument, gives back value, can be any type
        function symbol stand for function
        $f,g,h$

        predicate gives back either true or false, depending on values of arguments
        predicate symbol is either true or false
        $p,q,r$

        arity of function or predicate symbol: number of arguments (can be 0)
        specified by signature $\Sigma$

        set of logical variable symbols $V$, stand for objects
        $x,y,z$

        terms are well-formed/feasible arguments for functions/predicates

        \begin{definition}[Terms]
            well-formed terms: variables, functions applied to terms
            This can alternatively be written as grammar in Backus-Naur form
            \begin{equation}
                \astrm \Coloneqq x \mid c \mid f(\istrm{1},\ldots,\istrm{k})
            \end{equation}

        \end{definition}

        \begin{definition}[First-Order Formulas]


        \end{definition}

        \subsection{First-Order Logic of Real Arithmetic}
        \label{sec:first-order-logic-of-real-arithmetic}

            first-order logic of real arithmetic (\FOLR)
            formula of real arithmetic
            is first order formula
            function/predicate symbols $+,-,\cdot,/,=,<,\leq,>,\geq$
            constant symbols $\Sigma$
            logical Variables $V$


\chapter{Hybrid Programs with DDEs}\label{hybrid-programs-with-ddes}

We extent hybrid programs (\HPs) and formulae of classic \dL with syntax, semantics, axiomatization and proof rules which allow to deal with Delay Differential Equations.
%Is a super set, \dL is a fragment

\section{Example}
    \label{example-hp-cars}
    To motivate the need of being able to treat Delayed Differential Equations in hybrid programs, we present some examples.

    \subsection{Leading and Following Car}

    \subsection{Network Induced Delay in Control Loops}

\section{Syntax}
    \label{sec:syntax}

    $\allvars$ set of \textit{all variables} and $\D{\allvars}\defeq\{\D{x}:x\in\allvars\}$ set of \textit{differential symbols}

    variables $x,y,z\in\allvars$ with their differential symbols $\D{x},\D{y},\D{z}\in\D{\allvars}$

    % TODO: meaning of function, predicate, constant
    function symbols $f,g,h$

    predicate symbols $p,q,r$

    program constants $\ausprg, \busprg, c$, may be in $\rationals$

    defined inductively

    \begin{definition}[Terms]
        \label{def:syntax-terms}

        We extent the grammar defining \textbf{terms} with a symbol for a \textbf{delayed variable}
        % FIXME: replace ::=
        % FIXME: replace |
        \begin{equation}
            \theta,\eta ::= x|\xtau|\D{x}|c|\theta+\eta|\theta\cdot\eta
        \end{equation}

        The symbol $x$ is only allowed in the right hand side of a DDE and in evolution domain constraints, not in other formulas. The latter may only contain $\xtau$.

    \end{definition}

    where $x$ and where $\xtau$?
    %The grammars for hybrid programs and \dL formulas remain unchanged, but are shown here for completeness.

    \begin{definition}[Hybrid program]
        \label{def:syntax-HP}

        grammar
        $\asprg, \bsprg$ \HPs, program constants $\ausprg$, variable $x$, term $\astrm$ (possibly containing $x$ or $\xtau$?), formula $\bsfml$ of first-order logic of real arithmetic

        \begin{equation}
            % TODO: replace ;
            \asprg, \bsprg ::= a | \hupdate{\humod{x}{\astrm}} | \Dupdate{\Dumod{\D{x}}{\astrm}} | \htest{\bsfml} | \hchoice{\asprg}{\bsprg} | \asprg;\bsprg | \hrepeat{\asprg} | \hevolvein{\D{x}=\astrm}{\bsfml}
        \end{equation}

        difference to \dL, refer to Section \ref{sec:dynamic-semantics} for meaning of the following, different meaning for formulae

        assignment, discretely at an instant of time
        test of formula in current state
        nondeterministic choice
        sequential composition
        nondeterministic repetition
        delay differential equation with restricted evolution, due to domain constraint $\bsfml$, follow arbitrary amount of time

    \end{definition}

    \begin{definition}[(\dL) formula]
        \label{def:syntax-formula}

        formulas of (differential dynamic logic \dL)
        defined by grammar
        with \dL formulas $\asfml,\bsfml$, terms $\astrm,\bstrm,\istrm{1},\ldots,\istrm{k}$,
        % TODO: replace with command for predicate symbols, quantifier symbol
        predicate symbol $p$, quantifier symbol $C$, variable $x$, \HP $\asprg$

        \begin{equation}
            \asfml,\bsfml ::= \astrm\geq\bstrm | p(\istrm{1},\ldots,\istrm{k}) | \contextapp{C}{\asfml} | \lnot\asfml | \asfml\land\bsfml | \lforall{x}{\asfml} | \lexists{x}{\asfml} | \dbox{\asprg}{\asfml} | \ddiamond{\asprg}{\asfml}
        \end{equation}

        other operators, such as $>,\leq,<,\lor,\limply,\lbisubjunct$ can be derived from $\land,\lnot$

        modal formula $\dbox{\asprg}{\asfml}$ : $\asfml$ holds in the state after all runs of $\asprg$, dual: there is a run
        quantifier symbols $C$, with formula as argument are higher order predicate symbols and bind the variables of $\asfml$

    \end{definition}

\section{Dynamic Semantics}
    \label{sec:dynamic-semantics}
    % TODO: fix display of semantics [[]]
    \begin{equation}
        \imodels[\I]{\phi}{x}
    \end{equation}

    Following the remark to the solution of a DDE, we augment the \textbf{state space} in \dL to $\statespace$, the set of piecewise continuous functions on $[-\tau,0]$, as defined in \ref{definition-piecewise-continuous}.


    We denote by $\states$ the set of states. A state $\omega\in\states$ is a mapping
    \begin{equation}
        \omega : \mathcal{V}\cup\mathcal{V'}\rightarrow\statespace
    \end{equation}
    that assigns a \emph{history} (function) $\xbartau$ to each variable and differential symbol.

    %FIXME: need diff var symbol? can determine derivative from x? need pw diffable?

    rewriting history is not allowed, only values at current time instant can be changed $\modif{\nu}{x}{r}$ denotes state which is equal to state $\nu$ except for the value of variable $x$ at the time $t=0$, which is changed to $r\in\R$. values of $x$ for time before do not change

    % TODO: Abgrenzung zu Trace Semantics
    % The temporal character of delay differential equations (they depend on their own temporal evolution with limited horizon) suggests the introduction of trace semantics.
    %
    % However, we go the way of introducing transition semantics with an augmented state space.

    \begin{definition}[Semantics of terms]
        \label{def:sematic-terms}
        semantics of a term $\astrm$ in a state $\nu\in\statespace$ is a value in $\R$????
        defined inductively as follows

        % The semantics of the variable symbols in terms are given by
        \begin{itemize}
            \item $\imodel{}{x} = \nu(x)(0)$
            \item $\imodel[\nu]{}{\D{x}}=\nu(\D{x})(0)$
            \item $\imodel{}{f(\istrm{1},\ldots,\istrm{k})} = $ for function symbol $f$
            \item $\imodel{}{\astrm+\bstrm} = \imodel{}{\astrm} + \imodel{}{\bstrm}$
            \item $\imodel{}{\astrm\cdot\bstrm} = \imodel{}{\astrm} \cdot \imodel{}{\bstrm}$
            \item $\imodel{}{\D{(\astrm)}} = $
        \end{itemize}

        The semantic of the new symbol $\xtau$ depends on the context in which the term occures:
        \begin{itemize}
            \item in a hybrid program: $\imodel{\I}{\xtau} = \imodel[\nu]{}{x(t-\tau)} = \nu(x)(-\tau)$
            \item in a formula: \begin{equation}
                % FIXME: rechte seite noch nicht korrekt
                \imodel[]{}{\phi(\xtau)} =
                \{\nu\in\states : \lforall{t\in[-\tau,0]}{\phi(\nu(x)(t))} \}
            \end{equation}
        \end{itemize}

        In the precondition, no values are associated to the differential symbols. In general, the initial function is only piecewise continuous.
        Since for later time instances, the values of the differential symbols derive from the DDE, they become (locally) smooth function.

    \end{definition}

    When we write $x$ we mean $x(t)$.

    \begin{definition}[Semantics of (\dL) formulae]
        \label{def:semantic-formulae}

        semantics of (\dL) formula $\asfml$ with set of states $\states$ is the subset of states $\imodel{}{\asfml}\subset\states$ in which $\asfml$ is true. this set is defined inductively by

        \begin{itemize}
            % TODO: replace : in set by \with
            \item $\imodel{}{\asfml\geq\bsfml} = \{\nu\in\states : \imodel{}{\asfml}\geq\imodel{}{\bsfml}\}$
            \item $\imodel{}{p(\istrm{1},\ldots,\istrm{k})} = \{\nu\in\states : \}$
            \item $\imodel{}{\contextapp{C}{\asfml}} = $
            \item $\imodel{}{\lnot\asfml} = \scomplement{(\imodel{}{\asfml})} = \states\setminus\imodel{}{\asfml}$
            \item $\imodel{}{\asfml\land\bsfml} = \imodel{}{\asfml}\cap\imodel{}{\bsfml}$
            \item $\imodel{}{\lforall{x}{\asfml}} = \{\nu\in\states \with \modif{\nu}{x}{r}\in\imodel{}{\asfml} \text{ for all } r\in\R \}$
            \item $\imodel{}{\lexists{x}{\asfml}} = \{\nu\in\states \with \modif{\nu}{x}{r}\in\imodel{}{\asfml} \text{ for some } r\in\R \}$
            \item $\imodel{}{\dbox{\asprg}{\asfml}} = \{\nu\in\states \with \omega\in\imodel{}{\asfml} \text{ for all $\omega$ such that} (\nu,\omega)\in\imodel{}{\asprg}\}$
            \item $\imodel{}{\ddiamond{\asprg}{\asfml}} = \{\nu\in\states \with \omega\in\imodel{}{\asfml} \text{ for some $\omega$ such that} (\nu,\omega)\in\imodel{}{\asprg}\}$
        \end{itemize}
        if formula $\asfml$ is true in state $\nu$ we write $\nu\models\asfml$. It is called valid, written as $\models\asfml$ iff $\asfml$ is true in all states.

    \end{definition}


        With the semantics of terms if follows for the meaning of $\dbox{\asprg}{\phi}$, that $\phi$ must only hold up to time $\tau$ before leaving the \HP $\asprg$. It is possible, that $\phi$ was not verified before, while \textit{executing} the \HP.

        However when we apply the Rule of steps, we get the validity of $\phi$ for the entire trace.

        % TODO: is it better to only have xtau and not choice? What if both mentioned?
        in formulae (such as safety condition or evolution constraint), we have two possibilities: only value at current time instant ($x$) or for entire last $\tau$ time $\xtau$


    \begin{definition}[Transition semantics of \HPs]
        \label{def:semantic-HP}

         reachability relation $\iaccessibility(\asprg)\subseteq\states\times\states$. Since, with respect to \dL, the state space has been replaced, we need to redefine the semantics:
        \begin{itemize}
            % TODO: () for iaccessibility
            \item $\iaccessibility{a} = $
            \item $\iaccessibility{\hupdate{\humod{x}{\astrm}}}$
            \item $\iaccessibility{\Dupdate{\Dumod{\D{x}}{\astrm}}} $
            \item $\iaccessibility{\htest{\bsfml}}$
            \item $\iaccessibility{\hchoice{\asprg}{\bsprg}} = \hchoice{\iaccessibility{\asprg}}{\iaccessibility{\bsprg}}$
            \item $\iaccessibility{\asprg;\bsprg} $
            \item $\iaccessibility{\hrepeat{\asprg}} = \hrepeat{(\iaccessibility{\asprg})} = \cupfold_{n\in\N}\iaccessibility{\asprg^n}$ with $\asprg^{n+1}\equiv (\asprg^n;\asprg)$ and $\asprg^0\equiv \htest{\ltrue}$
            \item $\iaccessibility{\hevolvein{\D{x}=\astrm}{\bsfml}}$
        \end{itemize}

    \end{definition}

        The transition semantic of a hybrid program $\asprg$ is inductively given by a binary

        The \emph{discrete assignment} does not rewrite history, but changes only the value at the current time instant:
        \begin{equation}
        \iaccessibility(\hupdate{\humod{x}{\theta}}) = \left\{(\nu,\omega): \omega = \nu \text{ except } \omega(x)=\left(t\mapsto\begin{cases}\imodel{\I}{\theta} & t=0\\ \nu(t) &t\in[-\tau,0)\end{cases}\right)\qquad\right\}_.
        \end{equation}
        This assignment is the actual reason why we need to consider piecewise continuous evolutions.

        %TODO: super dense time: multiple assignments

        Using the extended syntax, we can write down both a delay differential equation and an ordinary differential equation in the form $\D{x}=\theta$, where $\theta=f(x,\xtau)$ with a polynomial $f$.
        \begin{equation}
            \iaccessibility(\hevolvein{\D{x}=\theta}{\ivr}) = \left\{
                (\varphi(0),\varphi(s))\,:\,\lforall{0\leq t\leq s}{\varphi(t)\models \D{x}=\theta\,\wedge\,\varphi(t)(0)\models\ivr}\text{ for a solution } \varphi:[0,s]\rightarrow\states \right\}
        \end{equation}
        As a solution, $\varphi$ needs to fulfill
        \begin{equation}
            \varphi(t)(\D{x})(0) \defeq \DD{\varphi(\zeta)(x)(0)}{\zeta}(t) \stackrel{!}{=} \imodel[]{\varphi(t)}{\theta}
        \end{equation}
        Remember that a $\xtau$ mentioned in $\theta$ here means $\varphi(t)(x)(-\tau)$.
        And so needs $\ivr$ always just hold at the current time instant (and not over the entire interval $[-\tau,0]$). The same case for $\ptest\psi$.

        % TODO: what about [a;b]p. is [a][b]p correct?


%\include{delay-differential-logic}
% FIXME: put dDL chapter in seperate file
\chapter{Delay Differential Dynamic Logic}
\label{sec:delay-differential-dynamic-logic}

\section{Axioms}
    \label{sec:axioms}

    provide syntactic operations, verify properties without going back to their mathematical semantics
    important for automatization of proofs

    \dL axiomatization, as given in \cite{Platzer12Complete}

    use first-order Hilbert calculus, modus ponens and forall-generalization as basis

    all instances of valid formulas of first-order real arithmetic are allowed as axiom

    first-order real arithmetic is decideable by quantifier elimination ($\QE$)

    axioms expressed in $\dbox{\cdot}$, duality to $\ddiamond{\cdot}$ by $\ddiamond{\asprg}{\asfml}\lbisubjunct\lnot\dbox{\asprg}{\lnot\asfml}$

    % QUESTION: which axioms are needed? in papers differences
    % FIXME: is there a paragraph like calculus environment?
    \begin{calculus}
        % TODO: Diamond-Axiom
        \cinferenceRule[testb|$\htest{}$]{test condition}{
            \linferenceRule[equiv]{
                (\chi\limply\asfml)
            }{
                \dbox{\htest{\chi}}{\asfml}
            }
        }{}
        % TODO: ['] axiom name
        % FIXME: how is relation to DDE solution axiom
        \cinferenceRule[solb|$'$]{}{
            \linferenceRule[equiv]{
                \lforall{t\geq 0}{\dbox{\hupdate{\humod{x}{y(t)}}}{\asfml}}
            }{
                \dbox{\hevolve{\D{x}=\astrm}}{\asfml}
            }
        }{$\hevolve{\D{y}(t)=\astrm}$}
        \cinferenceRule[choiceb|$\hchoice{}{}$]{choice}{
            \linferenceRule[equiv]{
                \dbox{\asprg}{\asfml}\land\dbox{\bsprg}{\asfml}
            }{
                \dbox{\hchoice{\asprg}{\bsprg}}{\asfml}
            }
        }{}
        % TODO: name for [&]
        % TODO: [&] also holds for DDEs?
        \cinferenceRule[constb|$\&$]{}{
            \linferenceRule[equiv]{
                \lforall{t_0=x_0}{\dbox{\hevolve{\D{x}=\astrm}}{(\dbox{\hevolve{\D{x}=-\astrm}}{(x_0\geq t_0\limply\ivr)}\limply\asfml)}}
            }{
                \dbox{\hevolvein{\D{x}=\astrm}{\ivr}}{\asfml}
            }
        }{}
        % TODO: composeb symbol
        \cinferenceRule[composeb|$;$]{composition}{
            \linferenceRule[equiv]{
                \dbox{\asprg}{\dbox{\bsprg}{\asfml}}
            }{
                \dbox{\asprg;\bsprg}{\asfml}
            }
        }{}
        % TODO: iterateb symbol
        \cinferenceRule[iterateb|$*$]{loop unrolling}{
            \linferenceRule[equiv]{
                \asfml\land\dbox{\asprg}{\dbox{\hrepeat{\asprg}}{\asfml}}
            }{
                \dbox{\hrepeat{\asprg}}{\asfml}
            }
        }{}
        % TODO: proper name K-Axiom
        \cinferenceRule[Kb|K]{}{
            \linferenceRule[impl]{
                \dbox{\asprg}{(\asfml\limply\bsfml)}
            }{
                (\dbox{\asprg}{\asfml}\limply\dbox{\asprg}{\bsfml})
            }
        }{}
        % TODO: proper name I-Axiom
        \cinferenceRule[Ib|I]{}{
            \linferenceRule[impl]{
                \dbox{\hrepeat{\asprg}}{(\asfml\limply\dbox{\asprg}{\asfml})}
            }{
                (\asfml\limply\dbox{\hrepeat{\asprg}}{\asfml})
            }
        }{}
        % TODO: proper name C-Axiom
        \cinferenceRule[Cb|C]{}{
            \linferenceRule[impl]{
                \dbox{\hrepeat{\asprg}}{\lforall{v>0 (\varphi(v)\limply\ddiamond{\asprg}{\varphi(v-1)})}}
            }{
                \lforall{v}{(\varphi(v)\limply\ddiamond{\hrepeat{\asprg}}{\lexists{v\leq 0}{\varphi(v)}})}
            }
        }{$v\notin\asprg$}
        % TODO: proper name B-Axiom
        \cinferenceRule[Bb|B]{}{
            \linferenceRule[impl]{
                \lforall{x}{\dbox{\asprg}{\asfml}}
            }{
                \dbox{\asprg}{\lforall{x}{\asfml}}
            }
        }{$x\notin\asprg$}
        % TODO: proper name V-Axiom
        % QUESTION: what is difference between V and G axiom?
        \cinferenceRule[Vb|V]{}{
            \linferenceRule[impl]{
                \asfml
            }{
                \dbox{\asprg}{\asfml}
            }
        }{$\freevars{\asfml}\cap\boundvars{\asprg}=\emptyset$}
        % TODO: proper name G-Axiom
        \cinferenceRule[gen|G]{Gödel's generalization rule}{
            \linferenceRule[sequent]{
                \asfml
            }{
                \dbox{\asprg}{\asfml}
            }
        }{}
        \cinferenceRule[MP|MP]{modus ponens}{
            \linferenceRule[sequent]{
                \asfml\limply\bsfml & \asfml
            }{
                \bsfml
            }
        }{}
        \cinferenceRule[forall|$\forall$]{forall generalization rule}{
            \linferenceRule[sequent]{
                \asfml
            }{
                \lforall{x}{\asfml}
            }
        }{}

        % TODO: CT-Axiom
        % TODO: CQ-Axiom
        % TODO: CE-Axiom
        % TODO: US-Axiom

    \end{calculus}

    % TODO: proofs, do not change, since only mention states?
    \begin{theorem}[Soundness of \dL]
        \label{thm:dL-soundness}
        The \dL calculus adapted to delay differential equations is sound.

    \end{theorem}
    \begin{proof}
        The arguments of most proof parts are the same as they were for the classic \dL, since only the definition of the statespace has been replaced. The proofs were independent of this definition, though. This is the case for [?], [choice], [;], [*], K, I, C, B, V, G and their proof can be found in \cite{platzer2012complete}.
        Exceptions: [:=] (->substitution lemma), ['] (-> new PL)

        [\&]???
    \end{proof}

    \subsection{Box-Assignment Axiom}
        \label{box-assignment-axiom}

        \begin{calculus}
            % TODO: add box in rule names
            \cinferenceRule[assignbb|$\mathrel{{:}{=}}$]{discrete assignment}{
                \linferenceRule[equiv]{
                    \asfml(\astrm) \land \left( \lforall{s\in[-\tau,0)}{\asfml(x(t+s))} \right)
                }{
                    \dbox{\hupdate{\humod{x}{\astrm}}}{\asfml(x)}
                }
            }{}
        \end{calculus}

    \subsection{History Axiom}
        \label{history-axiom}

        Just replace symbol by its semantical meaning
        The occurence of $\xtau$ in expressions can be replaced by turning the (implicitely existing) time variable explicit, i.e.\
        uniform substitution $\sigma$
        allows substitution of $\xtau$ by, depending on context, $x(t-\tau)$ or $\forall{s\in[0,\tau]}{x(t-\tau)}$
        allows substitution of x, +quantifier from semantics in certain contexts

        \begin{calculus}
            \cinferenceRule[hist|hist]{history axiom}{
                \linferenceRule[equiv]{
                    \lforall{s\in[-\tau,0]}{\asfml(x(t+s))}
                }{
                    \asfml(\xtau)
                }
            }{}
        \end{calculus}
        and $t\rightarrow t+s$

        for a piecewise continuous function $\theta\in\statespace$.

    \subsection{Solution Axiom}
        \label{sec:solution-axiom}
        \begin{equation}
            \xbartau = \theta_0 \rightarrow [\D{x}=\theta(\xbartaut{t})]\phi
            \leftrightarrow
            \left(\forall 0\leq t\leq\tau: [x:= y(t)]\phi \right)
            \wedge \xbartaut{\tau} = y \rightarrow [\D{x}=\theta (\xbartaut{t})]\phi
        \end{equation}
        where $\forall 0\leq t\leq\tau$, $y'(t)=\theta(\theta_0)$, i.e.\ $y$ is a local solution of the DDE. The solution must be expressible in polynomial form so that the axiom leads to decidable arithmetic.
        need to reposition time, so that each step begins at $t=0$, no problem for autonomous ddes
        (Since the DDE is autonomous, we can emit the time index.)

it often makes sense to treat the very first initial condition separately, because after it solution is at least $C^1$, at $x(0)$ might be knick

    \subsection{Axiom of Steps}
        \label{sec:axiom-of-steps}

        The \emph{Method of Steps} presented above translates into an axiom.
        %It allows to partially unwind an autonomous DDE given a analytic representation of its solution.

        % Let $\theta_0$ and $\theta$
        \begin{calculus}
            \cinferenceRule[stepsb|stpsb]{method of steps axiom}{
                \linferenceRule[equiv]{
                    \dbox[]{\hrepeat{(\hupdate{\humod{t}{0}};\hevolvein{\D{t}=1\syssep \D{x}=\rho(x,\xtau)}{(\ivr\land 0\leq t\leq\tau)})}}{\phi}
                }{
                    \dbox[]{\hevolvein{\D{x}=\rho(x,\xtau)}{\ivr}}{\phi}
                }
            }{}


        \end{calculus}

        What if $T=t_{max} < \tau$?
        Step character of DDE gives subdivision of continuous time

        \begin{proof}
            apply methods of steps
        \end{proof}

    \subsection{Axiom of One Step}
        \label{sex:axiom-of-one-step}

        Unwind loop in axiom od steps
        given an analytic solution on $[0,\tau]$ and given initial condition
        useful for bounded model checking

\section{Proof Rules}
    \label{sec:proof-rules}

    We keep the standard sequent calculus proof rules, given in \dL.

    \paragraph{Propositional Sequent Calculus Proof Rules}
        \label{sec:propositional-rules}

        some text

        \begin{calculus}
            \cinferenceRule[closeTrue|$\top$R]{close by always true antedecent}{
                \linferenceRule[sequent]{
                %\lsequent{}{}
                }{
                    \lsequent{\Gamma}{\ltrue,\Delta}
                }
            }{}
            \cinferenceRule[close|id]{close by identity}{
                \linferenceRule[sequent]{
                %\lsequent{}{}
                }{
                    \lsequent{P,\Gamma}{P,\Delta}
                }
            }{}
            \cinferenceRule[andR|$\land$R]{and right proof rule}{
                \linferenceRule[sequent]{
                    \lsequent{\Gamma} {P,\Delta}
                    &\lsequent{\Gamma} {Q,\Delta}
                }{\lsequent{\Gamma} {P\land Q,\Delta}}
            }{}
            \cinferenceRule[andL|$\land$L]{and left proof rule}{
                \linferenceRule[sequent]{
                    \lsequent{\Gamma, P, Q} {\Delta}
                }{\lsequent{\Gamma, P\land Q} {\Delta}}
            }{}
            \cinferenceRule[implyR|$\limply$R]{imply right proof rule}{
                \linferenceRule[sequent]{
                    \lsequent{\Gamma,P} {Q,\Delta}}{
                    \lsequent{\Gamma} {P\limply Q,\Delta}}
            }{}

        \end{calculus}


    \paragraph{Quantifier Sequent Calculus Proof Rules}
        \label{sec:quantifier-rules}

        some text

        \begin{calculus}
            \cinferenceRule[allR|$\forall$R]{for all right proof rule}{
                \linferenceRule[sequent]{
                    \lsequent{\Gamma}{p(y),\Delta}
                }{
                    \lsequent{\Gamma}{\lforall{x}{p(x)},\Delta}
                }
            }{$y\notin\Gamma,\Delta$}
        \end{calculus}

    \paragraph{\dL Sequent Calculus Proof Rules}
        \label{sec:dL-rules}

        some text

        \begin{calculus}
            % TODO: replace := by command
            \cinferenceRule[assignb|$\mathrel{{:}{=}}$]{discrete assignment}{
                \linferenceRule[sequent]{
                    \lsequent{\Gamma,x=e}{P,\Delta}
                }{\lsequent{\Gamma}{\dbox{\hupdate{\humod{x}{e}}}{P},\Delta}}
            }{$x\notin\Gamma,\Delta$}
            \cinferenceRule[loop|loop]{loop invariant}{
                \linferenceRule[sequent]{
                    \lsequent{\Gamma}{J,\Delta}
                    &\lsequent{J}{\dbox{\alpha}{J}}
                    &\lsequent{J}{P}
                }{\lsequent{\Gamma}{\dbox{\hrepeat{\alpha}}{P},\Delta}}
            }{}
        \end{calculus}

    \paragraph{Differential Equation Sequent Calculus Proof Rules}
        \label{sec:ode-rules}

        some text

        \begin{calculus}
            \cinferenceRule[DC|DC]{differential cut}{
                \linferenceRule[sequent]{
                    \lsequent{\Gamma}{\dbox{\hevolvein{\D{x}=f(x)}{\ivr}}{r(x)},\Delta}
                    &\lsequent{\Gamma}{\dbox{\hevolvein{\D{x}=f(x)}{\ivr\land r(x)}}{P}}
                }{\lsequent{\Gamma}{\dbox{\hevolvein{\D{x}=f(x)}{\ivr}}{P},\Delta}}
            }{}
            \cinferenceRule[dI|dI]{differential invariant}{
                \linferenceRule[sequent]{
                    \lsequent{\Gamma,Q}{P,\Delta}
                    &\lsequent{Q}{\dbox{\Dupdate{\Dumod{\D{x}}{f(x)}}}{\der{P}}}
                }{\lsequent{\Gamma}{\dbox{\hevolvein{\D{x}=f(x)}{\ivr}}{P},\Delta}}
            }{}
            \cinferenceRule[dW|dW]{differential weakening}{
                \linferenceRule[sequent]{
                    \lsequent{\Gamma}{\lforall{x}{(\ivr\limply P)},\Delta}
                }{\lsequent{\Gamma}{\dbox{\hevolvein{\D{x}=f(x)}{\ivr}}{P},\Delta}}
            }{}
        \end{calculus}

    % TODO: Rule of Steps
    \subsection{Rule of Steps}
        \label{sec:rule-of-steps}
        ODEs don't have notion of \emph{one step}, but DDEs do.
        condition valid for initial condition and given condition for a $s\leq t$ then condition holds after dde-evolution of max time tau and safety follows from condition then condition holds after dde with mentioned initial condition
        loop induction
        truth value of invariant never changes during dde
        % \begin{equation}
        % \frac{\Gamma(\xbartaut{0})\rightarrow F(\xbartaut{0})\quad F(\xbartaut{s})\rightarrow [\D{x}=\theta(\xbartaut{t})\,\&\,t\leq\tau]F(\xbartaut{t}) \quad F(\xbartaut{t})\rightarrow\phi}{\Gamma(\xbartaut{0}) \rightarrow [\D{x}=\theta(\xbartaut{t})]\phi}
        % \end{equation}

        %\begin{small}
        \begin{calculus}
            % FIXME: \landS -> steps
            \cinferenceRule[steps|stps]{steps proof rule}{
                \linferenceRule[sequent]{
                    \lsequent{\Gamma}{\inv,\Delta}
                    &\lsequent{t=0,\inv(\theta(t-\tau))}{\dbox[]{\hevolvein{\D{t}=1\syssep \D{x}=\rho(x,\theta(t-\tau))}{(\ivr\land 0\leq t\leq\tau)}}{\inv}}
                    &\lsequent{\inv}{\asfml}
                }{
                    \lsequent{\Gamma}{\dbox[]{\hevolvein{\D{x}=\rho(x,\xtau)}{\ivr}}{\asfml},\Delta}
                }
            }{}
        \end{calculus}
        %\end{small}

        Formulas of the form $\Gamma(\xtau)$ implicitely also include a statement about $x$.

    \subsection{Delay Differential Induction}
        \label{sec:delay-differential-induction}

        Like loop+DI, the former for $\lforall{k\geq 0}$, the latter for $\lforall{k\tau\leq t \leq (k+1)\tau}$
        The idea behind this proof rule is
        the initial condition fulfills a certain condition
        evolve a little in time
        the values which come out of the dde also fulfill this condition
        all runs od dde lead to states satisfying formula
        start in safe state
        dynamical system only evolve in direction of safe states in $\inv$
        direction is given by dde: in state $\omega$ it is $\imodel{}{f(x)}\omega$
        only need how system evolves in relation to $\inv$
        hence stays safe forever
        so the state after the DDE fulfills the condition, parts of the state come from initial condition, parts from dde outcome

        \begin{calculus}
            \cinferenceRule[DDI|DDI]{delay differential induction proof rule}{
                \linferenceRule[sequent]{
                    \lsequent{\Gamma}{\inv(\xtau),\Delta}
                    &\lsequent{\ivr,0\leq t\leq\tau,\inv(\theta)}{\dbox[]{\hevolve{\Dupdate{\Dumod{\D{x}}{\rho(x,\theta)}}}}{\der{\inv(x)}}}
                    &\lsequent{\inv(\xtau)}{\asfml}
                }{
                    \lsequent{\Gamma}{\dbox[]{\hevolvein{\D{x}=\rho(x,\xtau)}{\ivr}}{\asfml,\Delta}}
                }
            }{}
        \end{calculus}

        %\begin{proof}\small
        \begin{sidewaysfigure}\footnotesize
        \centering
        % TODO: hist axiom earlier? before first DC?
        \begin{sequentdeduction}[]
            \linfer[stepsb]{
                \linfer[loop]{
                    % FIXME: formel zu hoch
                    \lsequent{\Gamma(\xtau)}{\inv(\xtau),\Delta}
                    % FIXME: multiple rules at once
                    &\linfer[assignb+composeb]{
                        \linfer[hist]{
                            \linfer[DC]{
                                \linfer[DC]{
                                    (1)
                                }{
                                    \lsequent{\lforall{s\in[-\tau,0]}{\inv(x(t+s))},t=0}{\dbox{\hevolvein{\D{t}=1\syssep \D{x}=\eta(x,x(t-\tau))}{(\ivr\land 0\leq t\leq\tau\land \inv(x(t-\tau)))}}{\lforall{s\in[-\tau,0]}{\inv(x(t+s))}}}
                                }
                                &\linfer[dW]{
                                    (2)
                                }{
                                    \lsequent{\lforall{s\in[-\tau,0]}{\inv(x(s))},t=0}{\dbox{\hevolvein{\D{t}=1\syssep \D{x}=\eta(x,x(t-\tau))}{(\ivr\land 0\leq t\leq\tau)}}{\inv(x(t-\tau))}}
                                }
                            }{
                                \lsequent{\lforall{s\in[-\tau,0]}{\inv(x(t+s))},t=0}{\dbox[]{\hevolvein{\D{t}=1\syssep \D{x}=\eta(x,x(t-\tau))}{(\ivr\land 0\leq t\leq\tau)}}{\lforall{s\in[-\tau,0]}{\inv(x(t+s))}}}
                            }
                        }{
                            \lsequent{\inv(\xtau),t=0}{\dbox[]{\hevolvein{\D{t}=1\syssep \D{x}=\eta(x,\xtau)}{(\ivr\land 0\leq t\leq\tau)}}{\inv(\xtau)}}
                        }
                    }{
                        \lsequent{\inv(\xtau)}{\dbox[]{\hupdate{\humod{t}{0}}; \hevolvein{\D{t}=1\syssep \D{x}=\eta(x,\xtau)}{(\ivr\land 0\leq t\leq\tau)}}{\inv(\xtau)}}
                    }
                    \lsequent{\inv(\xtau)}{\asfml(\xtau)}
                }{
                \lsequent{\Gamma(\xtau)}{\dbox{\hrepeat{(\hupdate{\humod{t}{0}}; \hevolvein{\D{t}=1\syssep \D{x}=\eta(x,\xtau)}{(\ivr\land 0\leq t\leq\tau))}}}{\asfml(\xtau),\Delta}}}
            }
            {\lsequent{\Gamma(\xtau)}{\dbox{\hevolvein{\D{x}=\eta(x,\xtau)}{\ivr}}{\asfml(\xtau),\Delta}}}
        \end{sequentdeduction}

        % TODO: ref to here: (2)
        \begin{sequentdeduction}
            \linfer[dW]{
                \linfer[allR]{
                    \linfer[implyR]{
                        \linfer[]{
                            \lclose
                        }{
                            \lsequent{\lforall{s\in[-\tau,0]}{\inv(x(s))},t=0,\ivr(r,y),0\leq r\leq\tau}{\inv(x(r-\tau))}
                        }
                    }{
                        \lsequent{\lforall{s\in[-\tau,0]}{\inv(x(s))},t=0}{\ivr(r,y)\land 0\leq r\leq\tau\limply \inv(x(r-\tau))}
                    }
                }{
                    \lsequent{\lforall{s\in[-\tau,0]}{\inv(x(s))},t=0}{\lforall{(t,x)}{(\ivr\land 0\leq t\leq\tau\limply \inv(x(t-\tau)))}}
                }
            }{
                \lsequent{\lforall{s\in[-\tau,0]}{\inv(x(s))},t=0}{\dbox{\hevolvein{\D{t}=1\syssep \D{x}=\eta(x,x(t-\tau))}{(\ivr\land 0\leq t\leq\tau)}}{\inv(x(t-\tau))}}
            }
        \end{sequentdeduction}

        % TODO: ref to here: (1)
        \begin{sequentdeduction}
            \linfer[DC]{
                \linfer[dI]{
                    \linfer[hist]{
                        \linfer[]{
                            \lclose
                        }{
                            \lsequent{\lforall{s\in[-\tau,0]}{\inv(x(s))},t=0}{\inv(x(0))}
                        }
                    }{
                        \lsequent{\lforall{s\in[-\tau,0]}{\inv(x(s))},t=0,A, \inv(x(t-\tau))}{\inv(x(t))}
                    }
                    % TODO: A=\ivr, 0\leq t\leq\tau
                    &\lsequent{A, \inv(x(t-\tau)))}{\dbox{\Dupdate{\Dumod{\D{t}}{1},\Dumod{ \D{x}}{\eta(x(t),x(t-\tau))}}}{\der{\inv(x(t))}}}
                    %}
                }{
                    \lsequent{\lforall{s\in[-\tau,0]}{\inv(x(s))},t=0}{\dbox{\hevolvein{\D{t}=1\syssep \D{x}=\eta(x,x(t-\tau))}{(A\land \inv(x(t-\tau)))}}{\inv(x(t))}}
                }
                &\linfer[dW]{
                    (3)
                }{
                    \lsequent{\lforall{s\in[-\tau,0]}{\inv(x(s))},t=0}{\dbox{\hevolvein{\D{t}=1\syssep \D{x}=\eta(x,x(t-\tau))}{(A\land \inv(x(t-\tau))\land \inv(x(t)))}}{\lforall{s\in[-\tau,0]}{\inv(x(t+s))}}}
                }
            }{
                \lsequent{\lforall{s\in[-\tau,0]}{\inv(x(t+s))},t=0}{\dbox{\hevolvein{\D{t}=1\syssep \D{x}=\eta(x,x(t-\tau))}{(A\land \inv(x(t-\tau)))}}{\lforall{s\in[-\tau,0]}{\inv(x(t+s))}}}
            }
        \end{sequentdeduction}
        % TODO: ref here (3)
        \begin{sequentdeduction}
            \linfer[dW]{
                \linfer[allR]{
                    \linfer[implyR]{
                        \linfer[]{
                            \lclose
                        }{
                            \lsequent{\lforall{s\in[-\tau,0]}{\inv(x(s))}, t=0, \ivr(r,y), 0\leq r\leq\tau, \inv(y(r-\tau)), \inv(y(r))}{\lforall{s\in[-\tau,0]}{\inv(y(r+s))}}
                        }
                    }{
                        \lsequent{\lforall{s\in[-\tau,0]}{\inv(x(s))}, t=0}{((\ivr(r,y)\land 0\leq r\leq\tau\land \inv(y(r-\tau))\land \inv(y(r)))\limply\lforall{s\in[-\tau,0]}{\inv(y(r+s))})}
                    }
                }{
                    \lsequent{\lforall{s\in[-\tau,0]}{\inv(x(s))}, t=0}{\lforall{(t,x)}{((A\land \inv(x(t-\tau))\land \inv(x(t)))\limply\lforall{s\in[-\tau,0]}{\inv(x(t+s))})}}
                }
            }{
                \lsequent{\lforall{s\in[-\tau,0]}{\inv(x(s))}, t=0}{\dbox{\hevolvein{\D{t}=1\syssep \D{x}=\eta(x,x(t-\tau))}{(A\land \inv(x(t-\tau))\land \inv(x(t)))}}{(\lforall{s\in[-\tau,0]}{\inv(x(t+s))})}}
            }
        \end{sequentdeduction}
        \end{sidewaysfigure}
        \normalsize
        %\end{proof}

    \subsection{Delay Differential Invariant}
        \label{sec:delay-differential-invariant}

        loop and differential invariants are of form $\lforall{s\in[-\tau,0]}{\inv(x(s))}$
        can they have $x$?

        Meaning of derivative $\der{\inv(\xtau)}=\lforall{s\in[-\tau,0]}{\der{\inv(x(t+s))}}$ would lead to occurrence of derivative of init cond, which we don't know

        Mentioning $\xtau$ in the invariant differential invariant is not permitted, since derivation would lead to the occurrence of the symbol $x_{2\tau}$, whose properties are out of the scope of the current state.

        % TODO: ref to DDI
        As for ODEs in \dL, we cannot have $x(t)$ in the premise in DDI. Would permit to prove wrong statements.

        Invariant for limited time: use with loop unrolling (can it be generalized to unlimited inv?)

    % TODO: Example
    \subsection{Examples}
        \label{sec:examples}

        \subsubsection{Example 1}
            \label{sec:ddi-example-1}

            Consider the non-linear first-order delay differential equation
            \begin{equation}
                \D{x}(t) = \begin{cases}
                     x(t-\tau) & t \geq \sigma\\
                     \theta(t)\geq 0 & t \in [\sigma-\tau,\sigma]
                \end{cases}
            \end{equation}
            Using the invariant $F\equiv(x^3\geq 0)$ we prove that the solution stays non-negativ for all time $t$.
            \begin{sequentdeduction}
                \linfer[DDI]{
                    \linfer[]{
                        \lclose
                    }{
                        \lsequent{\xtau\geq 0}{\xtau^3\geq 0}
                    }
                    &\linfer[]{
                        \linfer[]{
                            \linfer[]{
                                \lclose
                            }{
                                \lsequent{\theta^3\geq 0}{3x^2\theta\geq 0}
                            }
                        }{
                            \lsequent{0\leq t\leq \tau,\theta^3\geq 0}{\dbox{\Dupdate{\Dumod{\D{x}}{\theta}}}{(3x^2 \D{x}\geq 0)}}
                        }
                    }{
                        %\lclose
                        \lsequent{0\leq t\leq \tau,\theta^3\geq 0}{\dbox{\Dupdate{\Dumod{\D{t}}{1},\Dumod{\D{x}}{\theta}}}{\der{x^3\geq 0}}}
                    }
                    &\linfer[]{
                        \lclose
                    }{
                        \lsequent{\xtau^3\geq 0}{\xtau\geq 0}
                    }
                }{
                    \lsequent{\xtau\geq 0}{\dbox{\D{x}=\xtau}{(\xtau\geq 0)}}
                }
            \end{sequentdeduction}
            In the same way we can prove that the solution stays negative for all $t$, if the initial condition is non-positive.

        \subsubsection{Example 2}
            \label{sec:ddi-example-2}

            Consider the non-linear first-order delay differential equation with explicitely given initial condition
            \begin{equation}
                \D{x}(t) = \begin{cases}
                     -x(t-1)^2 & t \geq 0\\
                     t & t \in [-1,0]
                \end{cases}
            \end{equation}
            Using $F\equiv(x^3\leq 0)$ we prove that the solution stays non-positiv.
            \begin{small}
                \begin{sequentdeduction}
                    \linfer[DDI]{
                        \linfer[hist]{
                            \linfer[]{
                                \linfer{\lclose}{
                                    \lsequent{}{\lforall{s\in[-1,0]}{s^3\leq 0}}
                                }
                            }{
                                \lsequent{t=0,\lforall{s\in[-1,0]}{x(t+s)=t+s}}{\lforall{s\in[-1,0]}{x(t+s)^3\leq 0}}
                            }
                        }{
                            \lsequent{t=0,\xtau[1]=t}{\xtau[1]^3\leq 0}
                        }
                        &\linfer[]{
                            \linfer[]{
                                \linfer[closeTrue]{
                                    \lclose
                                }{
                                    \lsequent{}{-3x^2\theta^2\leq 0}
                                }
                            }{
                                \lsequent{0\leq t\leq 1,\theta^3\leq 0}{\dbox{\Dupdate{\Dumod{\D{x}}{-\theta^2}}}{(-3x^2 \D{x}\leq 0)}}
                            }
                        }{
                            %\lclose
                            \lsequent{0\leq t\leq 1,\theta^3\leq 0}{\dbox{\Dupdate{\Dumod{\D{t}}{1},\Dumod{\D{x}}{-\theta^2}}}{\der{x\leq 0}}}
                        }
                        &\linfer[]{
                            \lclose
                        }{
                            \lsequent{\xtau[1]^3\leq 0}{\xtau[1]\leq 0}
                        }
                    }{
                        \lsequent{t=0,\xtau[1]=t}{\dbox{\D{x}=-\xtau[1]^2}{(\xtau[1]\leq 0)}}
                    }
                \end{sequentdeduction}
            \end{small}

            This proof doesn't even need any premisse about $\xtau$ in the induction step.

        \subsubsection{Example 3}
            \label{sec:ddi-example-3}

            We want to proof the safety condition $\asfml\equiv(-1\leq x\wedge x\leq 1)$ for the continuous program with delay differential equation
            \begin{equation}
                \forall\,t\in[-\tau,0]:\,-1\leq\xbartaut{0}(t)\wedge\xbartaut{0}(t)\leq 1
                \rightarrow
                [\D{x}=-\xtau] (\forall\,s\in[-\tau,0]:\,-1\leq\xbartaut{t}(s)\wedge\xbartaut{t}(s)\leq 1)
            \end{equation}
            in explicit quantified representation. It can be simplified by using an implicit time variable and a context depending meaning of $\xtau$
            \begin{equation}
                -1\leq\xtau\leq 1 \rightarrow [\D{x}=-\xtau]\asfml.
            \end{equation}

            We apply the rule of steps using the safety condition $\phi$ as step condition $F(x)\equiv(\forall\,t\in[-\tau,0]:\,-1\leq x(t)\wedge x(t)\leq 1)$.

            The first and third premisses hold. The second by ??? (delay differential invariant)

            Use the algebraic differential invariant $F\equiv(-1\leq x^3\wedge x^3\leq1)$, which is valid for the initial condition. Differentiation leads to the inequalities, which needs to be shown $\forall t\in[0,\tau]$
            \begin{equation}
                0\leq 3\,x(t)^2 \D{x}(t) = -3\,x(t)^2 \xtau(t)
            \end{equation}

            This holds since


% TODO: choose good citation style. Chicago?
\nocite{*}
\bibliographystyle{plain}
\bibliography{Bibliography}

\end{document}
