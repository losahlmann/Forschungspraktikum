\chapter{Introduction}

	dynamical systems
	mathematical model
	describing evolvement of state of as system over time
	modeling of embedded systems and cyber-physical systems
    discrete dynamical systems: difference equations, discrete state transition relations
    continuous dynamical systems: state evolves continuously, differential equation
    hybrid (dynamical) systems: combine discrete and continuous dynamics
    can capture very complex behaviour


	cyber-physical systems (CPS)
	combine computation, communication, control of physical processes/effects
	discrete and continuous dynamics
	often safety-critical, performance-critical
    safe control choices
    possible states accessible after control choice
	(prone to) software verification
    proof that always chooses safe control

    dynamical systems usually (uncountably) infinite state space
    finite number of tests cannot prove safety
    systematically obtain proofs

    classical safety, liveness, controllability, reactivity, quantified parametrized properties

	hybrid systems: mathematical model to describe CPS
	combine discrete dynamics/computation
	continuous dynamics: differential equations
    conditional switching
    nondeterminism
    repetition
	infinite state space

	examples:
    robotics, medical surgery robots
    electrical circuits
	automotive, self-driving cars (lane controllers for highway car traffic, controllers for intersections)
    speed limit control
	aviation, aircraft collision-avoidance systems (flyable roundabout maneuvers)
	railway, European train control system (ETCS)
    power plants
    chemical, biological processes
    medical models (events which can be seen as discrete with relation to continuous evolution)

	study logic of dynamical systems
    analyze and predict behaviour
    logics and proof principles

    differential dynamic logic (\dL)
    a concise overview and introduction is given in~\cite{Platzer12LogicsDynSys}
    a dynamic logic for hybrid systems
    logic/language for specify, verify safety and liveness properties
    of hybrid systems
    based on first-order modal logic and dynamic logic, first-order real arithmetic
    models of cyber-physical systems
    ordinary differential equations

    (nonlinear) real arithmetic

    transition behaviour as formulas

    differential invariants
    induction principle for differential equations

    theoretical results
    \emph{soundness} (everything provable is true)
    \emph{completeness} (everything true is provable)
    compositionality (denotational semantics, semantics (of models and formuals) functions of their parts, proofs structural decomposition, split complex systems in their parts, completeness: decomposition always successful)
    extendability (rules can be added to proof calculus)
    deductive power

	tutorial with examples modeled in \dL
	\cite{Quesel16Tutorial}

	mechanized proofs
	automatic and interactive theorem proving
	KeYmaeraX
    used successfully

	several formulations of dL
	earliest: sequent calculus \cite{Platzer10HybridSystems}, tuned for automatic proof search, KeYmaera
    automatically find (differential) invariants
	axiomatic formulation \cite{Platzer15Uniform}, implemented in KeYmaeraX

    extensions to \dL
    \emph{differential-algebraic dynamic logic} \DAL: differential-algebraic equations and constraints
    \emph{differential temporal dynamic logic} \dTL: temporal properties, trace semantics


	capture non-deterministic
	not known a priori

	\section{Related Work}
	\cite{Huang16BoundedVerificationNNDS}