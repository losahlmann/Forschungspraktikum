\chapter{Introduction}

    \emph{Dynamical systems} are a mathematical model which describe the evolution of a system's
    % TODO: sense of system?
    state over time.
    
    There are \emph{discrete dynamical systems}, obeying difference equations and discrete state transition relations,
    and \emph{continuous dynamical systems}, whose states evolve continuously, described by a (ordinary) differential equation (ODE).

    A fusion of both are \emph{hybrid (dynamical) systems}, which combine discrete and continuous dynamics with conditional switching, nondeterminism and repetition.

    % modeling of embedded systems and cyber-physical systems
    % interaction between computer and physical world (sensors actuators)
    
    The fashionable notion of \emph{cyber-physical systems} (CPS) is broadly used to describe technical systems which apply the discrete dynamics of digital computation (the cyber part) to control the continuous dynamics of physical processes.

    Hybrid systems well suited for capturing the complex behavior of CPS in a mathematical model.

    With the ongoing emergence of technology, cyber-physical systems are getting perceptibly more present in our surroundings and begin to largely interfere in our everyday life.
    While the complexity of these systems is growing, more and more control and responsibility is handed off to such technical systems.
    Thus the question of their ``safety'' is getting increasingly important.
    
    This demands the definition and specification of what is a reasonable and appropriate behavior for a system.
    This includes not only classical safety properties (something never happens), but also liveness (something eventually happens), controllability and reactivity.

    The task of \emph{verification} is to show that the system satisfies its specification and to establish guarantees for its safety- and performance-critical correctness.

    However, the complexity of dynamical system usually leads to a (uncountably) infinite state space, what means that any finite number of tests cannot cover all possible states reachable by execution of the system. Thus a finite number of tests cannot prove safety!
    
    Formal methods provide a means to systematically obtaining proofs of specifications, if the system can be described as a model in a certain formal language.
    This way, safety certificates can be established.

    
    % safe control choices
    % possible states accessible after control choice
    % (prone to) software verification
    % proof that always chooses safe control

    
    % TODO: references for examples
    Classical examples for CPS include all applications of automatic control, such as in robotics, automotive (self-driving cars), aviation, railway or power plants.
    But also models of electrical circuits, chemical and biological processes
    medical models
    (events which can be seen as discrete with relation to continuous evolution)


    % dL
    A language for the specification and verification of correctness properties, such as safety and liveness properties,
    for hybrid systems is \emph{differential dynamic logic} (\dL)
    (see~\cite{Platzer12LogicsDynSys} for a concise introduction and overview).

    It provides syntax and semantics for hybrid programs and logic formulas as well as a proof calculus to formally reason about hybrid systems.
    This logic is based on first-order modal logic and dynamic logic and relies on first-order real arithmetic.

    % analyze and predict behavior
    % logics and proof principles
    % models of cyber-physical systems
    % sound and complete proof calculus relative to ... diff (ref)
    % and to ... disc (ref)
    % (nonlinear) real arithmetic
    % transition behavior as formulas
    
    With differential forms, \dL provides a powerful tool to reason about ordinary differential equations by differential invariants (an induction principle for differential equations), differential substitutions and ghosts.    

    Some important theoretical results, such as \emph{soundness} (everything provable is true) and \emph{completeness} (everything true is provable) have been established for \dL.

    Another important property of \dL is its compositionality.
    denotational semantics, its semantics (of models and formulas) are functions of their parts.
    This allows a structural decomposition of proofs by splitting complex systems in their parts.
    The completeness property assures that this decomposition is always possible and successful.
    Moreover, this modularity makes \dL extendable. New proof rules can be added to the proof calculus, to improve its deductive power.

    Different formulations of \dL have been presented.
    The earliest, given in \cite{Platzer10HybridSystems}, is a sequent calculus (in Gentzen style), which is tuned for automatic proof search. It is implemented in the proof assistant \KeYmaera.
    automatically find (differential) invariants

    The later, Hilbert-type axiomatic formulation of \dL (cf.~\cite{Platzer15Uniform,Platzer12Complete,Platzer12LogicsDynSys}) is based on uniform substitution and bound variable renaming, which allows a much more concise programmatic implementation. This is done in form of the interactive theorem prover \KeYmaeraX.

    The tutorial \cite{Quesel16Tutorial} shows some examples of systems modeled and proved in \dL.

    Moreover, some extensions to \dL have been presented.
    \emph{Differential-algebraic dynamic logic} (\DAL) adds differential-algebraic equations and constraints to \dL \cite{Platzer10DAL}, whereas
    \emph{differential temporal dynamic logic} (\dTL) is based on a trace semantics, which allows to specify temporal properties of a hybrid system \cite{Platzer07dTL}.
    \emph{Stochastic differential dynamic logic} (\SdL) deals with stochastic hybrid systems, which add stochastic differential equations to hybrid systems \cite{Platzer11StochasticHP}.
    Distributed hybrid systems can be verified using \emph{quantified differential dynamic logic} (\QdL), cf.~\cite{Platzer10QdL}.
    For \emph{differential game logic} (\dGL) see \cite{Platzer15GameLogic}.
    
    Differential dynamic logic and its extensions have successfully demonstrated their usability by application to a number of real-world problems.
    
    Examples include obstacle avoidance in robotics \cite{Mitsch16Robots} and the design of a safe controller for medical surgery robots \cite{Kouskoulas13SurgicalRobot}.
    
    Contributions to the important field of self-driving cars have been made by the verification of cruise controllers \cite{Loos11CruiseControl,Loos13CruiseControllers}, controllers for intersections \cite{Loos11Intersections} and speed limit \cite{Mitsch12SpeedControl}.
    % lane controllers for highway car traffic,
    
    % such as controllers for flyable roundabout maneuvers of
    Safety is paramount in aviation \cite{GhorbalAerospace} and \KeYmaera was used to verify aircraft collision-avoidance systems \cite{Jeannin15ACASX,Jeannin15CollisonAvoidance,Loos13Aircraft}.
    
    For the European Train Control System (ETCS), controllability, safety, liveness, and reactivity properties have been proved \cite{Platzer09ETCS}.

    % capture non-deterministic
    % not known a priori
    

    Models in \dL are limited to ordinary differential equations.
    In this report, we will study a more general class of dynamical systems, called \emph{time-delay systems} (TDS)~\cite{Richard2003TDSs}, which extend dynamical systems by obeying \emph{delay differential equations} (DDEs). Delay differential equations (also called differential-difference equations) belong to the broader class of \emph{functional differential equations} (FDEs).

    Delay is mainly an applied problem. Most real world CPSs incorporate
    sensors and actuators in a feedback loop. This system-intern communication introduces a delay, for example due to computation time, sensing sample-intervals or network transfer, whose effects can often not be ignored.

    Thus, time-delay systems have become of high interest in research and application in the recent decades, especially in the systems and control community. By taking the delay into account, they allow to formulate more realistic models with better performance.
    
    Prototypical examples of time-delay systems include networked control systems and tele-operated systems, general communication networks, robotics, combustion engines and manufacturing processes. See~\cite{Gu03TimeDelaySys} for a number of examples.

    Even small delays can have a complex effect on the system's behavior (cf.\ Example~\ref{ex:ode-dde}), in particular on its stability, which is an important property for many applications.
    The presence of a delay can both be destabilizing and stabilizing.
    This gives rise to the idea of ``control via delay value'', leveraging the stabilizing effect by intentional retardation.
    
    % stability and control issues of time-delay systems
    % Especially important for control engineering
    % can be used to simplify higher order models


    % TODO: check this
    % As functional equations, FDEs have an infinite dimensional state space. In particular and in contrast to ordinary differential equations which only depend on the present value, DDEs extent the state space by remembering the past.
    % In practice however, the application of a controller to a computer demands a discretization 
    % cannot have infinite dimensional state
    One may restrict to point-wise delays, which leads to the special class of delay differential equations with multiple, constant delay.
    In this work, we will restrict to this type of DDE and present a logic to formally reason about time-delay systems.

    We will establish ...
    and demonstrate


    \section{Related Work}
        apart from all literature to \dL (see above)
        \cite{Huang16BoundedVerificationNNDS}
