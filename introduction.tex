\chapter{Introduction}

    \emph{Dynamical systems} are a mathematical model which describe the evolution of a system's
    % TODO: sense of system?
    state over time.
    
    There are \emph{discrete dynamical systems}, obeying difference equations and discrete state transition relations,
    and \emph{continuous dynamical systems}, whose states evolve continuously, described by a (ordinary) differential equation (ODE).

    A fusion of both are \emph{hybrid (dynamical) systems}, which combine discrete and continuous dynamics with conditional switching, nondeterminism and repetition.

    % modeling of embedded systems and cyber-physical systems
    % interaction between computer and physical world (sensors actuators)
    
    The fashionable notion of \emph{cyber-physical systems} (CPS) is broadly used to describe technical systems which apply the discrete dynamics of digital computation (the cyber part) to control the continuous dynamics of physical processes.

    Hybrid systems well suited for capturing the complex behaviour of CPS in a mathematical model.

    With the ongoing emergence of technology, cyber-physical systems are getting perceptibly more present in our surroundings and begin to largely interfere in our everyday life.
    While the complexity of these systems is growing, more and more control and responsability is handed off to such technical systems.
    Thus the question of their ``safety'' is getting increasingly important.
    
    This demands the definition and specification of what is a reasonable and appropriate behaviour for a system.

    This includes not only classical safety properties (something never happens), but also liveness (something eventually happens), controllability and reactivity.

    The task of \emph{verification} is to show that the system satisfies its specification and to establish guarantees for its safety- and performance-critical correctness.

    However, the complexity of dynamical system usually leads to a (uncountably) infinite state space, what means that any finite number of tests cannot cover all possible states reachable by execution of the system. Thus a finite number of tests cannot prove safety!
    
    Formal methods provide a means to systematically obtaining proofs of specifications, if the system can be described as a model in a certain formal language.
    
    This way, safety certificates can be established.

    
    % safe control choices
    % possible states accessible after control choice
    % (prone to) software verification
    % proof that always chooses safe control

    
    % TODO: references for examples
    Classical examples for CPS include all applications of automatic control, such as in robotics, automotive (self-driving cars), aviation, railway or power plants.
    But also models of electrical circuits, chemical and biological processes
    medical models
    (events which can be seen as discrete with relation to continuous evolution)


    % dL
    A language for the specification and verification of correctness properties, such as safety and liveness properties,
    for hybrid systems is \emph{differential dynamic logic} (\dL)
    (see~\cite{Platzer12LogicsDynSys} for a concise introduction and overview).

    It provides syntax and semantics for hybrid programs and logic formulae as well as a proof calculus to formally reason about hybrid systems.
    This logic is based on first-order modal logic and dynamic logic and relies on first-order real arithmetic.

    % analyze and predict behaviour
    % logics and proof principles
    % models of cyber-physical systems
    % sound and complete proof calculus relative to ... diff (ref)
    % and to ... disc (ref)
    % (nonlinear) real arithmetic
    % transition behaviour as formulas
    
    With differential forms, \dL provides a powerful tool to reason about ordinary differential equations by differential invariants (an induction principle for differential equations), differential substitutions and ghosts.    

    Some important theoretical results, such as \emph{soundness} (everything provable is true) and \emph{completeness} (everything true is provable) have been established for \dL.

    Another important property of \dL is its compositionality.
    denotational semantics, its semantics (of models and formuals) are functions of their parts.
    This allows a structural decomposition of proofs by splitting complex systems in their parts.
    The completeness property assures that this decomposition is always possible and successful.
    Moreover, this modularity makes \dL extendable. New proof rules can be added to the proof calculus, to improve its deductive power.

    Different formulations of \dL have been presented.
    The earliest, given in \cite{Platzer10HybridSystems}, is a sequent calculus (in Gentzen style), which is tuned for automatic proof search. It is implemented in the proof assistant \KeYmaera.
    automatically find (differential) invariants

    The later, Hilbert-type axiomatic formulation of \dL (cf.~\cite{Platzer15Uniform,Platzer12Complete,Platzer12LogicsDynSys}) is based on uniform substitution and bound variable renaming, which allows a much more concise programmatic implementation. This is done in form of the interactive theorem prover \KeYmaeraX.

    The tutorial \cite{Quesel16Tutorial} shows some examples of systems modeled and proved in \dL.
    
    real-world examples
    used successfully mechanized proofs
    % EXamples
    robotics, medical surgery robots \cite{Kouskoulas13SurgicalRobot}
    self-driving cars 
    cruise control \cite{Loos11CruiseControl,Loos13CruiseControllers}
    (lane controllers for highway car traffic, controllers for intersections)
    speed limit control
    aviation, aircraft collision-avoidance systems \cite{Jeannin15CollisonAvoidance}
    (flyable roundabout maneuvers)
    railway, European train control system (ETCS) \cite{Platzer09ETCS}

    Some extensions to \dL have been presented
    \emph{Differential-algebraic dynamic logic} \DAL adds differential-algebraic equations and constraints, where as
    \emph{differential temporal dynamic logic} \dTL is based on a trace semantics, which allows to specify temporal properties.
    % TODO: more dL extensions, and sources
    stochastic
    stochastic differential dynamic logic (SdL), stochastic hybrid systems
    distributed: quantified differential dynamic logic (QdL)
    distributed hybrid systems
    differential game logic \dGL \cite{Platzer15GameLogic}


    % capture non-deterministic
    % not known a priori



    % TDS
    Models in \dL are limited to ordinary differential equations
    This report studies a more general class of dynamical systems, called \emph{time-delay systems} (TDS), which extend dynamical systems by obeying \emph{functional differential equations} (FDEs). 

        of high interest in research and application in the recent decades
    especially in systems and control community

    extension of state space by remembering past

    in combination with discrete

    add communication to CPS

    present a logic 
    establish

    \cite{Richard2003TDSs}
    described by functional differential equations (FDEs)
    TDS are governed by \emph{delay differential equations} (DDEs) (also called differential-difference equations), which belong to the broader class of \emph{functional differential equations} (FDEs).
    
    infinite dimensional
    as opposed to ordinary differential equations
    
    delay is an applied problem
    as many real world processes incorporate
    sensors, actuators, feedback loops
    introduce delay


    Allows to formulate more realistic models with better performance
    since these effects can often not be ignored,
    (see example of exp)
    Even small delays can have a complex effect on the system's behaviour, in particular on its stability.
    be both stabilizing and destabilizing
    and are hence of special practial interest
    introduction of delay can benefit the control, can be stabilizing
    control via delay value
    Stability is an important property for many applications.
    stability and control issues of time-delay systems
    Especially important for control engineering

    in application: need discretization for computer
    cannot have infinite dimensional state
    may restrict to point-wise delays, leads to DDEs considered in this work

    Examples
    communication networks
    networked control systems
    teleoperated systems
    robotics
    \cite{Gu03TimeDelaySys} % examples


    constant delays
    time-varying delays

    can be used to simplify higher order models
    

    \section{Related Work}
        apart from all literature to \dL (see above)
        \cite{Huang16BoundedVerificationNNDS}
