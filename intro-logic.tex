\chapter{Introduction to Logic}
    \label{sec:introduction-logic}

    \section{First-Order Logic}
        \label{sec:first-order-logic}

        framework for studying rules of argument (\cite{hodges2001ClassicalLogic})

        as defined in \cite{Platzer10HybridSystems} and \cite{Huth04LogicInCS}
        more subtleties are layed out in \cite{hodges2001ClassicalLogic}, though with a different notation/nomenclature


        First-order logic (\FOL) defines a syntax of logical formulas inductively from a set of function and predicate symbols, called signature $\signature$ of the theory. alphabet to built well-formed formulas from
        a set of symbols and their roles

        % it is predicate logic, not propositional
        propositional: no =, no quantifiers, no functions/predicates, only true/false constant symbols (= predicates of arity 0 = propositions)
        extends propositional logic, which is based on \textit{declarative sentences}/propositions (can be declared as true or false) with quantifiers
        and variables

        declarative sentences made of atomic, indecomposible sentences

        (comprises three subsets: functions, predicates, constants)

        function: takes value of argument, gives back value, can be any type
        function symbol stand for function
        $f,g,h$

        predicate gives back either true or false, depending on values of arguments
        predicate symbol is either true or false
        $p,q,r$

        refer to objects
        arity of function or predicate symbol: number of arguments (can be 0)
        specified by signature $\signature$
        for zero arity $c$ constant symbol=function without argument, so-called nullary function

        set of logical variable symbols $\varsymbols$, stand for objects
        $x,y,z$

        formulas denote truth values
        terms are expressions which denote objects
        terms are well-formed/feasible arguments for functions/predicates



        \subsection{Syntax}
            \label{sec:FOL-syntax}

            for now only syntactic definitions, no specific meaning yet

            \begin{definition}[Terms]
                well-formed terms: variables, functions applied to terms
                $\terms$ set of all terms
                smallest set with
                \begin{enumerate}
                    \item logical variables are terms, $x\in\varsymbols$ then $x\in\terms$
                    \item a nullary function $c\in\signature$ is a term
                    % TODO: command for integer range
                    \item If $f\in\signature$ of arity $k>0$ and $\istrm{i}\in\terms$ for $i=1,\ldots,k$, then $f(\istrm{1},\ldots,\istrm{k})\in\terms$
                \end{enumerate}

                This can alternatively be written as grammar in Backus-Naur form
                \begin{equation}
                    \astrm,\istrm{i} \Coloneqq
                        x \mid
                        c \mid
                        f(\istrm{1},\ldots,\istrm{k})
                \end{equation}
                where $x$ is ranging over the set of logical variables $\varsymbols$, $c$ over the nullary functions in $\signature$ and $f$ over the elements in $\signature$ with arity $k>0$.

            \end{definition}

            \begin{definition}[First-Order Formulas]
                well-formed formulas of a (first-order) logic are
                a formal language over alphabet $\signature\cup\varsymbols$
                words formed by recursive combination of signature-symbols with logical operator symbols
                using the set of terms $\terms$

                $\FOLformulas$ set of formulas of first-order logic
                smallest set with
                \begin{enumerate}
                    \item If predicate symbol $p\in\signature$ of arity $k>=0$ and $\istrm{i}\in\terms$ for $i=1,\ldots,k$ then $p(\istrm{1},\ldots,\istrm{k})\in\FOLformulas$
                    \item If $\asfml,\bsfml\in\FOLformulas$, then $\lnot\asfml,(\asfml\land\bsfml),(\asfml\lor\bsfml),(\asfml\limply\bsfml)\in\FOLformulas$ are formulas
                    \item quantifiers formula $\asfml$, variable $x$ $(\lforall{x}{\asfml}),(\lexists{x}{\asfml})\in\FOLformulas$
                \end{enumerate}

                or again as
                \begin{equation}
                    \asfml,\bsfml \Coloneqq
                        p(\istrm{1},\ldots,\istrm{k}) \mid
                        \lnot\asfml \mid
                        \asfml\land\bsfml \mid
                        \asfml\lor\bsfml \mid
                        \asfml\limply\bsfml \mid
                        \lforall{x}{\asfml} \mid
                        \lexists{x}{\asfml}
                \end{equation}
                where $p\in\signature$ is a predicate symbol (of arity $k\geq 1$), the terms $\istrm{i}\in\terms$ and the variable $x\in\varsymbols$
            \end{definition}

        \subsection{Semantics}
            \label{sec:FOL-semantics}

            \textbf{semantics} specify an interpretation of each symbol of ...
            in language of ...
            given this defines a \textbf{truth value} of a full formula

            \textbf{interpretation} $I$ of function and predicate symbols

            assignment of logical variables
            variables are place holders for any kind of concrete value

            provide seperate characterisation, meaning of the connectives
            truth tables
            to define what is true and false
            semantics is equivalent to proof theory

            formula is satisfieable, valid, or neither
            satisfieable if there is at least one model in which formula evaluates to true
            valid if always evaluates to true, in every model

            relationship
            $\asfml_1,\ldots,\asfml_n\models\bsfml$
            looking at truth values of atomic formulas in premises and conclusion, how manipulated by logical connectives, defined by a table for all possible cases
            calculate truth value of formula from truth value of atomic propositions

            classical logic \cite{reis2014cutelimination}
            set of truth values true,false
            every sentence always true or false
            its sequents calculus is LK
            valuation/model of a formula is an assignment of a truth value to each atomic proposition inside that formula

            valuation: assignment of truth value to all atoms
            infinitely many valuations, called models

            evaluation of formula: computing truth value for it, based on a valuation
            requires to fix a universe of concrete values

            \begin{definition}[Interpretation]
                An interpretation (or model) $\interpret$ assigns concrete elements/function/relation to their corresponding symbols in a given a signature $\signature$. It consists of
                \begin{enumerate}
                    \item The universe, a non-empty set $\universe$ of concrete values/objects
                    \item nullary function symbol
                    % FIXME: proper function definitions, ie. : and arrow
                    % TODO: reference to cartesian product
                    \item To each function symbol $f\in\signature$ of arity $k\geq 1$, $\interpret(f):\universe^k\rightarrow\universe$ is a function with $k$ arguments.
                    \item Each predicate symbol $p$ of arity $k\geq 1$ a relation $\interpret(p)\subseteq\universe^n$ subset
                \end{enumerate}

            \end{definition}

            $\asfml$ is true in $\interpret$/under $\interpret$, $\interpret$ satisfies $\asfml$, is a model of $\asfml$

            interpretation gives information needed to decide on truth value of a sentence
            semantics determine if $\interpret$ makes $\asfml$ true or false

            interpretation of $p$ under $\interpret$
            interpretation of $f$ under $\interpret$

            Interpretation associates $f$ with $\interpret(f)(d_1,\ldots,d_k)\in\universe$ value of function at position $(d_1,\ldots,d_k)\in\universe^k $
            predicate $p$ true at position $(d_1,\ldots,d_k)\in\universe^k$ iff $(d_1,\ldots,d_k)\in\interpret(p)$
            alternative but equivalent formulation: characteristic function $\interpret(p):\universe^k\rightarrow\{\ltrue,\lfalse\}$ with $p$ true at $(d_1,\ldots,d_k)\in\universe^k$ iff $\interpret(p)(d_1,\ldots,d_k)=\ltrue$

            % FIXME: Zielraum in englisch
            difference between functions and predicates is Zielraum

            \begin{definition}[Assignment]
                interpretation of variables
                assignment for a logical variable $x\in\varsymbols$
                map $\assignment:\varsymbols\rightarrow\universe$, assign a value/object from the universe to each variable symbol.
                % TODO: notation for x|->a
            \end{definition}

            formula said to be true in $\interpret$ under $\assignment$

            \subsubsection{Valuation of Terms}
                \label{sec:valuation-of-terms}

                given an interpretation and assignment for a signature, term can be evaluated
                again inductively

                \begin{definition}
                    valuation $\ivaluation{\IFOL}{\phi}$ defined by
                    \begin{enumerate}
                        \item logical variable $\ivaluation{\IFOL}{x} = \assignment(x)$
                        \item constant
                        \item function symbol $f\in\signature$ of arity $k\geq 1$ $\ivaluation{\IFOL}{f(\istrm{1},\ldots,\istrm{k})} = \interpret(f)(\ivaluation{\IFOL}{\istrm{1}},\ldots,\ivaluation{\IFOL}{\istrm{k}})$
                    \end{enumerate}
                \end{definition}

            \subsubsection{Valuation of First-Order Formulas}
                \label{sec:valuation-of-formulas}

                truth table
                translation into natural language?
                \begin{definition}
                    valuation of first-order formulas
                    interpretation $\interpret$
                    assignment $\assignment$
                    \begin{enumerate}
                        \item predicate $\ivaluation{\IFOL}{p(\istrm{1},\ldots,\istrm{k})} = \interpret(p)(\ivaluation{\IFOL}{\istrm{1}},\ldots,\ivaluation{\IFOL}{\istrm{k}})$
                        \item and/conjunction $\ivaluation{\IFOL}{\astrm\land\bstrm} = \ltrue$ iff $\ivaluation{\IFOL}{\astrm}=\ltrue$ and $\ivaluation{\IFOL}{\bstrm}=\ltrue$
                        \item or/disjunction $\ivaluation{\IFOL}{\astrm\lor\bstrm} = \ltrue$ iff $\ivaluation{\IFOL}{\astrm}=\ltrue$ or $\ivaluation{\IFOL}{\bstrm}=\ltrue$
                        \item not $\ivaluation{\IFOL}{\lnot\astrm} = \ltrue$ iff $\ivaluation{\IFOL}{\astrm} \neq \ltrue$
                        \item imply $\ivaluation{\IFOL}{\astrm\limply\bstrm} = \ltrue$ iff $\ivaluation{\IFOL}{\astrm} \neq \ltrue$ or $\ivaluation{\IFOL}{\bstrm}=\ltrue$
                        \item universal quantifier $\ivaluation{\IFOL}{\lforall{x}{\astrm}} = \ltrue$ iff $\ivaluation{\imodif[assign]{\IFOL}{x}{a}}{\astrm} = \ltrue$ for all $a\in\R$
                        \item existential quantifier $\ivaluation{\IFOL}{\lexists{x}{\astrm}} = \ltrue$ iff $\ivaluation{\imodif[assign]{\IFOL}{x}{a}}{\astrm} = \ltrue$ for some $a\in\R$
                    \end{enumerate}

                    With $M=(\interpret,\assignment)$ Also write $\imodels{\IFOL}{\astrm}$ the so-called satisfaction relation if and only if $\ivaluation{\IFOL}{\astrm}=\ltrue$
                    say that $\interpret,\assignment$ satisfies $\astrm$.
                    Formula $\astrm$ is called valid if $\imodels{\IFOL}{\astrm}$ for all possible interpretations and assignments. just write $\models\astrm$
                \end{definition}

        \subsection{Proof Theory}
        \label{sec:FOL-proof-theory}

            % FIXME: when to use vDash instead of models? because of \entails in semantics
            test $\models\vDash $ establish validity by proofs
            semantics: given interpretation,assignment -> compute its truth value, check satisfaction easy
            check validity difficult, need to check for all interpretations, assignments
            proofs shows validity, but not valid difficult, how to show that there is no proof?

            proof for establishing evidence of assertions, like $\lsequent{\Gamma}{\astrm}$ valid
            not useful for $\lsequent{\Gamma}{\astrm}$ not valid

            two characterisations need be equivalent, ie soundness and completeness

            valid formulas are of particular interest, hold under all circumstances, ie for all interpretations and assignments

            need to identify these, check validity

            if not valid, there is a counterexample, an interpretation and assignment for which not true, ie false

            can be hard to find such, but easy to check to be one

            want to easily show validity
            witness for not validity: counterexample
            witness for validity: (formal) proof

            formal proof is a derived formula, which is obviously valid
            form initial formula by valid rules

            can be difficult to find a proof
            but proof can easil be verified by validity of proof rules

            proof rules are derived from axioms or easy valid formulas

            different styles of formal logical argumentation

            deduction systems in which a proof can be represented

            proof of a sequent is a tree, initial sequent as root, axioms as leaves

            $T$ entails $\asfml$, all interpretations satisfying $T$ also satisfy $\asfml$
            $\asfml$ is valid, if every interpretation makes it true

            a sequent is deriveable in proof calculus $C$, $T|-_C\asfml$ if there is a formal proof in the calculus for the sequent

            proof calculus is sound if invalid sequents are not deriveable in C
            is complete if every valid sequent can be derived

            \subsubsection{Axioms}
                \label{sec:FOL-axioms}

                theory of $\interpret$ is set of sentences satisfied by $\interpret$
                generally, theory is set of sentences
                model class of theory $T$: all interpretations which satisfy $T$ (all sentences given in T)
                if $K$ (interpretations) is model class of theory $T$, then $T$ is set of axioms for $K$ (\cite{hodges2001ClassicalLogic})

                zero-premise rule
                \begin{calculus}
                    \cinferenceRule[ax|]{}{
                        \linferenceRule[sequent]{}{\lsequent{\asfml}{\asfml}}
                    }{}
                \end{calculus}

            \subsubsection{Hilbert Calculus}
                \label{sec:hilbert-calculus}

                \cite{hodges2001ClassicalLogic}

                Hilbert Style deduction system/calculi
                every line of a proof is unconditional tautology or theorem

                small number of inference rules
                relying more on axioms

                reach a conclusion by applying deduction rules to axioms

                useful when one wants to add further axioms to first-order base

                judgement is any formula of given (first-order) logic
                theorems: formulae appearing in concluding judgement of a valid proof $\lsequent{}{\bsfml}$

                define class of axioms, ie the set of all formulas which have any of the forms ...

                formal proof is a derivation of the formula (conclusion) from axioms and premises by applying in each step modus ponens or quantifier-generalization

                $\bsfml$ deriveable in C from premises, if exists a derivation with conclusion $\bsfml$

                lemma to be used in another proof


            % TODO: natural deduction calculus
            \subsubsection{Natural Deduction}
                \label{sec:natural-deduction}

                natural deduction calculus
                Gentzen style, few axioms, more rules
                every line is a conditional tautology,
                zero or more conditions on left
                exactly one asserted proposition on right

                inference rules which operate on sequents

                collection of proof rules
                infer formulas from formulas
                deduce a conclusion from a given set of promises by applying rules
                premises: set of formulas $\asfml_1,\ldots,\asfml_n$
                conclusion $\bsfml$ single formula
                umforme premises with rules into conclusion
                this intention, written as
                from premisses conclude conclusion $\bsfml$
                judgement of form
                \begin{equation}
                    \lsequent{\asfml_1,\ldots,\asfml_n}{\bsfml}
                \end{equation}
                called sequent
                sets may be empty
                sequent is valid, if there is a proof

                in contrast to relationship $\asfml_1,\ldots,\asfml_n\models\bsfml$, see semantics

                finding proof can be difficult, because it is not obvious which rules need to be applied in what order

                theorems: formulae appearing in concluding judgement of a valid proof $\lsequent{}{\bsfml}$

                % TODO: add ref to soundness
                rules well chosen, see soundness

            \subsubsection{Sequent Calculus}
                \label{sec:sequent-calculus}

                Gentzen style
                zero or more asserted propositions

                can be regarded as Hilbert-style calculus for deriving finite sequences instead of formulas \cite{hodges2001ClassicalLogic}

                sequent is a standard form for logical formulas
                antecedent $\Gamma$ assumed to be true
                succedent $\Delta$ one of these to be shown
                both finite sets of formulas
                $\lsequent{\Gamma}{\Delta}$, which means
                \begin{equation}
                    \landfold_{\asfml\in\Gamma}\asfml \limply \lorfold_{\bsfml\in\Delta}\bsfml
                \end{equation}

                comma in antecedent like conjunction, in succedent like disjunction
                rules decompose propositional structure of a formula leading to simpler ones

                empty sequent, both side empty, defined to be false
            % or sequent calculus
            % Gentzen, Hilbert
            % TODO: Bruch schreibweise, not iff, only if

            % TODO: free variables
            free variable: needs an object assigned in order to determine a truth valua for formula
            bound variable: no assignment needed
            sentence is formula without free variables, has a determined truth value
            quantifiers bind occurences of variable in formula




            \subsubsection{Proof Rules}
                \label{sec:FOL-proof-rules}


                two groups: logical (operate on formula and sub-formulas) and structural (such as contraction=copy, weakening=erase, cut)

                %prop logic rules + rules for = and quantifiers

                sequent below horizontal line: conclusion
                sequents above horizontal line: premises

                \paragraph{Universal Quantification Rules}
                    % TODO: forall e <-> forall i
                    eliminates $\forall$
                    $\lforall{x}{\asfml}$ is true if we can replace quantified variable by any term (side condition, $\astrm$ needs be free for x in $\asfml$) and this remains true

                    \begin{calculus}
                        \cinferenceRule[FOLforall|$\forall$]{}{
                            \linferenceRule[sequent]{
                                \subst[\asfml]{x}{\astrm}
                            }{
                                \lforall{x}{\asfml}
                            }
                        }{}
                    \end{calculus}

                    soundness self-evident, since true for all, also for more concrete instance

                    \begin{example}
                        necessity for free

                    \end{example}

                \paragraph{Existential Quantification Rules}
                    % TODO: exists e <-> exists i
                    exists introduction

                    \begin{calculus}
                        \cinferenceRule[FOLexists|$\exists$]{exists generalization rule}{
                            \linferenceRule[sequent]{
                                \lexists{x}{\asfml}
                            }{
                                \subst[\asfml]{x}{\astrm}
                            }
                        }{}
                    \end{calculus}

                    $\astrm$ free for $x$ in $\asfml$
                    can deduce $\lexists{x}{\asfml}$ if we know that true for a concrete instance, witness

            \subsubsection{Ground Proving and Free Variable Proving}
                \label{sec:ground-proving-free-variable-proving}



        \subsection{First-Order Logic of Real Arithmetic}
            \label{sec:FOL-R}

            first-order logic of real arithmetic (\FOLR)
            formula of real arithmetic
            is first order formula
            function/predicate symbols $ = {+,-,\cdot,/,=,<,\leq,>,\geq}$
            constant symbols $\Sigma$
            logical Variables $V$

        \subsection{Decideability}
            \label{sec:decideability}

            (\cite{hodges2001ClassicalLogic})
            language (e.g. FOL) decideable if there is an algorithm which returns for any finite sequent if it is valid or not

        \subsection{Correctness}
            \label{sec:correctness}

            $\bsfml$ deriveable from premises, then sequent is valid

        \subsection{Completeness}
            \label{sec:completeness}

            given a valid sequent, then conclusion is deriveable from premises

        % \subsubsection{Substitution}
        % \label{sec:substitution}
        %
        % variables are place holders
        % means of replacing with concrete information
