\chapter{Introduction to Logic}
    \label{sec:introduction-logic}

    \section{First-Order Logic}
        \label{sec:first-order-logic}

        \textit{First-order logic} (\FOL) is a framework for studying rules of argument (\cite{hodges2001ClassicalLogic}) and will be used a means for the specification of properties of hybrid programs later on.

        We introduce first-order logic similiar to the definitions given in \cite{Platzer10HybridSystems} and \cite{Huth04LogicInCS}. More subtleties are adapted from \cite{hodges2001ClassicalLogic}.
        % though with a different notation/nomenclature

        declarative sentences made of atomic, indecomposible sentences


        % it is predicate logic, not propositional
        propositional: no =, no quantifiers, no functions/predicates, only true/false constant symbols (= predicates of arity 0 = propositions)

        % TODO: truth values

        predicate logic

        First-order logic extends propositional logic,
        with quantifiers
        and variables
        and functions/predicates of arity $\geq 1$

        Propositional logic only deals with \textit{declarative sentences}, so-called \textit{propositions}, which can be declared as either true or false. and how these can be connected



        Functions take values as argument and give back a new value, which can be of any type. Function symbols stand for a function and are usually denoted by $f,g,h$.

        % FIXME: Zielraum in englisch
        difference between functions and predicates is Zielraum

        In contrast to functions, predicates return either true or false, depending on the values of their arguments.
        So depending on the context, a predicate symbol, usually written as $p,q,r$, is either true or false.

        The \textit{arity} of a function or predicate symbol is the number of arguments it demands.
        Functions can be of arity zero, i.e. without any argument.
        For clarity, we will treat these so-called \textit{constants} (or nullary functions) seperately and write their symbols as $a,b,c$.
        The arities are specified by the signature $\signature$.

        Variable symbols $x,y,z$ refer to objects

        Precisely define a set of symbols and their roles

        \subsection{Syntax}
            \label{sec:FOL-syntax}

            First-order logic inductively defines a syntax of well-formed \textit{logical formulae} over an alphabet containing different kinds of symbols.

            The syntax only defines rules which specify how the given symbols can be textually combined. At this stage, they have no specific meaning yet.

            The alphabet $\signature\cup\varsymbols$ comprises the \textit{signature} $\signature$ of the theory, which is a set of predicate, function and constant symbols, and the set of logical variable symbols $\varsymbols$.

            Terms are the expressions which denote feasible arguments for functions and predicates.
            \begin{definition}[Terms]
                For a given \textit{signature} $\signature$ and set of variable symbols $\varsymbols$, the well-formed terms are either logical variables, constants or functions applied to terms:
                The set of all \textit{terms} $\terms$ is the smallest set with
                \begin{enumerate}
                    \item If $x\in\varsymbols$ then $x\in\terms$.
                    \item If $c\in\signature$ is a constant symbol, then $c\in\terms$.
                    \item If $f\in\signature$ is a function symbol of arity $n\geq 1$ and $\istrm{i}\in\terms$ for $i=\range{1}{n}$, then $f(\range{\istrm{1}}{\istrm{n}})\in\terms$.
                \end{enumerate}
                This definition can alternatively be written as grammar in Backus-Naur form
                \begin{equation}
                    \astrm,\istrm{i} \Coloneqq
                        x \mid
                        c \mid
                        f(\range{\istrm{1}}{\istrm{n}})
                \end{equation}
                where $x$ is ranging over the set of logical variables $\varsymbols$, $c$ over the nullary functions in $\signature$ and $f$ over the elements in $\signature$ with arity $n\geq 1$.
            \end{definition}

            Formulas are expressions which will admit truth values.
            \begin{definition}[First-Order Formulas]
                For given $\signature$ and $\varsymbols$, the well-formed formulas of a (first-order) logic are words formed by recursive combination of signature-symbols with logical operator symbols:
                The set of formulas $\FOLformulas$ is the smallest set with
                \begin{enumerate}
                    \item If $p\in\signature$ is a predicate symbol of arity $n\geq 0$ and $\istrm{i}\in\terms$ for $i=\range{1}{n}$ then $p(\range{\istrm{1}}{\istrm{n}})\in\FOLformulas$.
                    \item If $\asfml,\bsfml\in\FOLformulas$, then $\lnot\asfml,(\asfml\land\bsfml),(\asfml\lor\bsfml),(\asfml\limply\bsfml)\in\FOLformulas$.
                    \item If $\asfml\in\FOLformulas$ and $x\in\varsymbols$ then $(\lforall{x}{\asfml})\in\FOLformulas$ and  $(\lexists{x}{\asfml})\in\FOLformulas$.
                \end{enumerate}
                We may again write this as
                \begin{equation}
                    \asfml,\bsfml \Coloneqq
                        p(\range{\istrm{1}}{\istrm{n}}) \mid
                        \lnot\asfml \mid
                        \asfml\land\bsfml \mid
                        \asfml\lor\bsfml \mid
                        \asfml\limply\bsfml \mid
                        \lforall{x}{\asfml} \mid
                        \lexists{x}{\asfml}
                \end{equation}
                where $p\in\signature$ ranges over the predicate symbols (of arity $k\geq 1$), $\istrm{i}\in\terms$ over the terms and $x\in\varsymbols$ over the variable symbols.
            \end{definition}

        \subsection{Semantics}
            \label{sec:FOL-semantics}

            The \textbf{semantics} of a first-order logic specify the meaning of each symbol occuring in terms and formulae with the goal to assign a \textbf{truth value} to a full formula.

            need concrete objects for symbols
            based on a valuation
            requires to fix a \textit{universe} of concrete values

            \begin{definition}[Interpretation]
                An \textbf{interpretation} (or model) $\interpretation$ assigns concrete elements/function/relation to their corresponding symbols in a given signature $\signature$. It consists of
                \begin{enumerate}
                    % FIXME: is this part of interpretation, means that assignment only makes sense for a fixed interpretation
                    \item The \textit{universe} $\universe$, a non-empty set of concrete values/objects of discurse.
                    \item For each constant symbol $c\in\signature$ a value $\interpret{c}\in\universe$.
                    % FIXME: proper function definitions, ie. : and arrow, \from\to
                    % TODO: reference to cartesian product
                    \item For each function symbol $f\in\signature$ of arity $k\geq 1$, $\interpret{f}\from\universe^k\to\universe$ is a function with $k$ arguments.
                    \item For each predicate symbol $p$ of arity $k\geq 1$, $\interpret{p}\subseteq\universe^n$ is a relation.
                \end{enumerate}

            \end{definition}

            % Interpretation associates $f$ with $\interpret{f}(\range{d_1}{d_n})\in\universe$ value of function at position $(\range{d_1}{d_n})\in\universe^k $

            The predicate $p$ is \textit{true} at position $(\range{d_1}{d_n})\in\universe^k$ under $\interpretation$ if and only if $(\range{d_1}{d_n})\in\interpret{p}$.
            An alternative but equivalent formulation uses the characteristic function $\interpret{p}\from\universe^n\to\{\ltrue,\lfalse\}$, where $p$ is true at $(\range{d_1}{d_n})\in\universe^n$ if and only if $\interpret{p}(\range{d_1}{d_n})=\ltrue$.

            % provide seperate characterisation, meaning of the connectives
            % truth tables
            % to define what is true and false
            % semantics is equivalent to proof theory


            % relationship
            % $\range{\asfml_1}{\asfml_n}\models\bsfml$
            % looking at truth values of atomic formulas in premises and conclusion, how manipulated by logical connectives, defined by a table for all possible cases
            % calculate truth value of formula from truth value of atomic propositions

            % classical logic \cite{reis2014cutelimination}
            % set of truth values true,false
            % every sentence always true or false
            % its sequent calculus is LK
            % valuation/model of a formula is an assignment of a truth value to each atomic proposition inside that formula


            assignment of logical variables
            Variables are only place holders and need to to refer to a concrete value

            \begin{definition}[Assignment]
                interpretation of variables
                assignment for a logical variable
                An \textbf{assignment} is a map $\assignment\from\varsymbols\to\universe$ which assigns a value/object from the universe to each variable symbol $x\in\varsymbols$.
                % TODO: notation for x|->a
            \end{definition}

            Given the information in an interpretation and assignment, we can finally evaluate a formula, i.e. decide on its truth value.

            given an interpretation and assignment for a signature, term can be evaluated
            again inductively

            formula is satisfieable, valid, or neither
            satisfieable if there is at least one model in which formula evaluates to true
            valid if always evaluates to true, in every model


            valuation: assignment of truth value to all atoms
            % infinitely many valuations, called models

            \begin{definition}[Valuation of Terms]
                For $\terms$ the/a \textit{valuation} $\ivaluation{\IFOL}{\phi}$ is defined by
                \begin{enumerate}
                    \item $\ivaluation{\IFOL}{x} = \assignment(x)$ for each logical variable $x\in\varsymbols$
                    \item constant
                    \item $\ivaluation{\IFOL}{f(\range{\istrm{1}}{\istrm{n}})} = \interpret{f}(\range{\ivaluation{\IFOL}{\istrm{1}}}{\ivaluation{\IFOL}{\istrm{n}}})$ for each function symbol $f\in\signature$ of arity $n\geq 1$.
                \end{enumerate}
            \end{definition}


            meaning of connective symbols, how they preserve a truth value
            defined as truth table
            translation into natural language?
            \begin{definition}[Valuation of First-Order Formulas]
                valuation of first-order formulas $\FOLformulas$
                in the interpretation $\interpretation$
                under the assignment $\assignment$ is given by
                \begin{enumerate}
                    \item predicate $\ivaluation{\IFOL}{p(\range{\istrm{1}}{\istrm{n}})} = \interpret{p}(\range{\ivaluation{\IFOL}{\istrm{1}}}{\ivaluation{\IFOL}{\istrm{n}}})$
                    \item The conjunction $\ivaluation{\IFOL}{\astrm\land\bstrm} = \ltrue$ iff $\ivaluation{\IFOL}{\astrm}=\ltrue$ and $\ivaluation{\IFOL}{\bstrm}=\ltrue$.
                    \item The disjunction $\ivaluation{\IFOL}{\astrm\lor\bstrm} = \ltrue$ iff $\ivaluation{\IFOL}{\astrm}=\ltrue$ or $\ivaluation{\IFOL}{\bstrm}=\ltrue$.
                    \item The negation $\ivaluation{\IFOL}{\lnot\astrm} = \ltrue$ iff $\ivaluation{\IFOL}{\astrm} \neq \ltrue$.
                    \item The implication $\ivaluation{\IFOL}{\astrm\limply\bstrm} = \ltrue$ iff $\ivaluation{\IFOL}{\astrm} \neq \ltrue$ or $\ivaluation{\IFOL}{\bstrm}=\ltrue$.
                    \item The universal quantifier $\ivaluation{\IFOL}{\lforall{x}{\astrm}} = \ltrue$ iff $\ivaluation{\imodif[assign]{\IFOL}{x}{a}}{\astrm} = \ltrue$ for all $a\in\R$.
                    \item The existential quantifier $\ivaluation{\IFOL}{\lexists{x}{\astrm}} = \ltrue$ iff $\ivaluation{\imodif[assign]{\IFOL}{x}{a}}{\astrm} = \ltrue$ for some $a\in\R$.
                \end{enumerate}

                The statement $\ivaluation{\IFOL}{\astrm}=\ltrue$ can also be written as $\imodels{\IFOL}{\astrm}$, the so-called satisfaction relation.

                One says that $\interpretation,\assignment$ satisfies $\asfml$ or that $\asfml$ is true in $\interpretation$ under $\assignment$.

                A formula $\asfml$ is called valid (or a \textit{tautology}) if $\imodels{\IFOL}{\asfml}$ for all possible interpretations $\interpretation$ and assignments $\assignment$. In this case, one writes shortly $\models\asfml$.
            \end{definition}

        \subsection{Proof Theory}
        \label{sec:FOL-proof-theory}

            % FIXME: when to use vDash instead of models? because of \entails in semantics
            test $\models\vDash $ establish validity by proofs
            semantics: given interpretation,assignment -> compute its truth value, check satisfaction easy
            check validity difficult, need to check for all interpretations, assignments
            proofs shows validity, but not valid difficult, how to show that there is no proof?

            proof for establishing evidence of assertions, like $\lsequent{\Gamma}{\astrm}$ valid
            not useful for $\lsequent{\Gamma}{\astrm}$ not valid

            two characterisations need be equivalent, ie soundness and completeness

            valid formulas are of particular interest, hold under all circumstances, ie for all interpretations and assignments

            need to identify these, check validity

            if not valid, there is a counterexample, an interpretation and assignment for which not true, ie false

            can be hard to find such, but easy to check to be one

            want to easily show validity
            witness for not validity: counterexample
            witness for validity: (formal) proof

            formal proof is a derived formula, which is obviously valid
            form initial formula by valid rules

            can be difficult to find a proof
            but proof can easil be verified by validity of proof rules

            proof rules are derived from axioms or easy valid formulas

            different styles of formal logical argumentation

            deduction systems in which a proof can be represented

            proof of a sequent is a tree, initial sequent as root, axioms as leaves

            $T$ entails $\asfml$, all interpretations satisfying $T$ also satisfy $\asfml$
            $\asfml$ is valid, if every interpretation makes it true

            a sequent is deriveable in proof calculus $C$, $T|-_C\asfml$ if there is a formal proof in the calculus for the sequent

            judgement of form
            \begin{equation}
                \lsequent{\range{\asfml_1}{\asfml_n}}{\bsfml}
            \end{equation}
            called sequent
            sets may be empty
            sequent is valid, if there is a proof

            proof calculus is sound if invalid sequents are not deriveable in C
            is complete if every valid sequent can be derived

            \subsubsection{Axioms}
                \label{sec:FOL-axioms}

                theory of $\interpretation$ is set of sentences satisfied by $\interpretation$
                generally, theory is set of sentences
                model class of theory $T$: all interpretations which satisfy $T$ (all sentences given in T)
                if $K$ (interpretations) is model class of theory $T$, then $T$ is set of axioms for $K$ (\cite{hodges2001ClassicalLogic})

                zero-premise rule
                \begin{calculus}
                    \cinferenceRule[ax|]{}{
                        \linferenceRule[sequent]{}{\lsequent{\asfml}{\asfml}}
                    }{}
                \end{calculus}

            \subsubsection{Hilbert Calculus}
                \label{sec:hilbert-calculus}

                \cite{hodges2001ClassicalLogic}

                Hilbert Style deduction system/calculi
                every line of a proof is unconditional tautology or theorem

                small number of inference rules
                relying more on axioms

                reach a conclusion by applying deduction rules to axioms

                useful when one wants to add further axioms to first-order base

                judgement is any formula of given (first-order) logic
                theorems: formulae appearing in concluding judgement of a valid proof $\lsequent{}{\bsfml}$

                define class of axioms, ie the set of all formulas which have any of the forms ...

                formal proof is a derivation of the formula (conclusion) from axioms and premises by applying in each step modus ponens or quantifier-generalization

                $\bsfml$ deriveable in C from premises, if exists a derivation with conclusion $\bsfml$

                lemma to be used in another proof


            % TODO: natural deduction calculus
            \subsubsection{Natural Deduction}
                \label{sec:natural-deduction}

                The \textit{natural deduction calculus} Gentzen style,
                and relies on only a few axioms but an exhaustive collection of inference rules which operate on sequents.

                By applying these rules one infers formulas from formulas and finally deduces a conclusion from the given set of promises.

                Every line of a proof is a conditional tautology, with zero or more conditions on the left side of the sequent and exactly one asserted proposition on the right hand side.

                premises: set of formulas $\range{\asfml_1}{\asfml_n}$
                conclusion $\bsfml$ single formula

                umforme premises with rules into conclusion
                this intention, written as
                from premisses conclude conclusion $\bsfml$


                in contrast to relationship $\range{\asfml_1}{\asfml_n}\models\bsfml$, see semantics

                Finding a proof can be difficult, because it might not be obvious which proof rules need to be applied and in what order.

                theorems: formulae appearing in concluding judgement of a valid proof $\lsequent{}{\bsfml}$

                % TODO: add ref to soundness
                rules well chosen, see soundness

            \subsubsection{Sequent Calculus}
                \label{sec:sequent-calculus}

                Gentzen style
                zero or more asserted propositions

                can be regarded as Hilbert-style calculus for deriving finite sequences instead of formulas \cite{hodges2001ClassicalLogic}

                sequent is a standard form for logical formulas
                antecedent $\Gamma$ assumed to be true
                succedent $\Delta$ one of these to be shown
                both finite sets of formulas
                $\lsequent{\Gamma}{\Delta}$, which means
                \begin{equation}
                    \landfold_{\asfml\in\Gamma}\asfml \limply \lorfold_{\bsfml\in\Delta}\bsfml
                \end{equation}

                comma in antecedent like conjunction, in succedent like disjunction
                rules decompose propositional structure of a formula leading to simpler ones

                empty sequent, both side empty, defined to be false
            % or sequent calculus
            % Gentzen, Hilbert
            % TODO: Bruch schreibweise, not iff, only if

            % TODO: free variables
            free variable: needs an object assigned in order to determine a truth valua for formula
            bound variable: no assignment needed
            sentence is formula without free variables, has a determined truth value
            quantifiers bind occurences of variable in formula




            \subsubsection{Proof Rules}
                \label{sec:FOL-proof-rules}


                two groups: logical (operate on formula and sub-formulas) and structural (such as contraction=copy, weakening=erase, cut)

                %prop logic rules + rules for = and quantifiers

                sequent below horizontal line: conclusion
                sequents above horizontal line: premises

                \paragraph{Universal Quantification Rules}
                    % TODO: forall e <-> forall i
                    eliminates $\forall$
                    $\lforall{x}{\asfml}$ is true if we can replace quantified variable by any term (side condition, $\astrm$ needs be free for x in $\asfml$) and this remains true

                    \begin{calculus}
                        \cinferenceRule[FOLforall|$\forall$]{}{
                            \linferenceRule[sequent]{
                                \subst[\asfml]{x}{\astrm}
                            }{
                                \lforall{x}{\asfml}
                            }
                        }{}
                    \end{calculus}

                    soundness self-evident, since true for all, also for more concrete instance

                    \begin{example}
                        necessity for free

                    \end{example}

                \paragraph{Existential Quantification Rules}
                    % TODO: exists e <-> exists i
                    exists introduction

                    \begin{calculus}
                        \cinferenceRule[FOLexists|$\exists$]{exists generalization rule}{
                            \linferenceRule[sequent]{
                                \lexists{x}{\asfml}
                            }{
                                \subst[\asfml]{x}{\astrm}
                            }
                        }{}
                    \end{calculus}

                    $\astrm$ free for $x$ in $\asfml$
                    can deduce $\lexists{x}{\asfml}$ if we know that true for a concrete instance, witness

            \subsubsection{Ground Proving and Free Variable Proving}
                \label{sec:ground-proving-free-variable-proving}



        \subsection{First-Order Logic of Real Arithmetic}
            \label{sec:FOL-R}

            first-order logic of real arithmetic (\FOLR)
            formula of real arithmetic
            is first-order formula
            function/predicate symbols $ = {+,-,\cdot,/,=,<,\leq,>,\geq}$
            constant symbols $\Sigma$
            logical Variables $V$

        \subsection{Decideability}
            \label{sec:decideability}

            (\cite{hodges2001ClassicalLogic})
            language (e.g. FOL) decideable if there is an algorithm which returns for any finite sequent if it is valid or not

        \subsection{Correctness}
            \label{sec:correctness}

            $\bsfml$ deriveable from premises, then sequent is valid

        \subsection{Completeness}
            \label{sec:completeness}

            given a valid sequent, then conclusion is deriveable from premises

        % \subsubsection{Substitution}
        % \label{sec:substitution}
        %
        % variables are place holders
        % means of replacing with concrete information
