\chapter{Introduction to Logic}
\label{sec:introduction-logic}

    \section{First-Order Logic}
    \label{sec:first-order-logic}

        as defined in \cite{Platzer10HybridSystems} and \cite{Huth04LogicInCS}

        First-order logic (\FOL) defines  a syntax of logical formulas


        define inductively
        set of function and predicate symbols, called signature $\Sigma$. alphabet to built well-formed formulas from

        function: takes value of argument, gives back value, can be any type
        function symbol stand for function
        $f,g,h$

        predicate gives back either true or false, depending on values of arguments
        predicate symbol is either true or false
        $p,q,r$

        arity of function or predicate symbol: number of arguments (can be 0)
        specified by signature $\Sigma$

        set of logical variable symbols $V$, stand for objects
        $x,y,z$

        terms are well-formed/feasible arguments for functions/predicates

        \begin{definition}[Terms]
            well-formed terms: variables, functions applied to terms
            This can alternatively be written as grammar in Backus-Naur form
            \begin{equation}
                \astrm \Coloneqq x \mid c \mid f(\istrm{1},\ldots,\istrm{k})
            \end{equation}

        \end{definition}

        \begin{definition}[First-Order Formulas]


        \end{definition}

        \subsection{First-Order Logic of Real Arithmetic}
        \label{sec:first-order-logic-of-real-arithmetic}

            first-order logic of real arithmetic (\FOLR)
            formula of real arithmetic
            is first order formula
            function/predicate symbols $+,-,\cdot,/,=,<,\leq,>,\geq$
            constant symbols $\Sigma$
            logical Variables $V$
