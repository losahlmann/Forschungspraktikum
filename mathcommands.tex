% Math Environements
\newtheorem{theorem}{Theorem}
\newtheorem{corollary}[theorem]{Corollary}
\newtheorem{proposition}[theorem]{Proposition}
\newtheorem{lemma}[theorem]{Lemma}
\newtheorem{example}[theorem]{Example}
\theoremstyle{definition}
\newtheorem{definition}[theorem]{Definition}

%
% General Mathematics
%
% natural numbers
\newcommand{\N}{\mathbb{N}}
% real numbers
\newcommand{\R}{\mathbb{R}}
% define as
% FIXME: define "def" properly
%\newcommand{\defeq}{\,\stackrel{\text{def}}{=}\,}
% defined by
\newcommand\logeq{\mathrel{\vcentcolon\Longleftrightarrow}}
% norm
%\DeclarePairedDelimiter{\abs}{\lvert}{\rvert}
\DeclarePairedDelimiter{\nnorm}{\lVert}{\rVert}
\DeclarePairedDelimiter{\supnorm}{\lVert}{\rVert_{\text{\scriptsize{sup}}}}

% exponential function
\newcommand{\e}[1]{\text{e}^{#1}}

%
% Analysis
%
% derivative: Lagrange style
%\newcommand{\D}[1]{#1'} % ' defined in latex and is same as \prime
% derivative: Leibniz style
\renewcommand{\DD}[2]{\frac{\text{d} #1}{\text{d} #2}}
% differential in integral
\newcommand{\dx}[1][x]{\text{d}#1}

\newcommand{\continuouspws}[3][]{\ensuremath{C^{#1}_\text{pw}\ifthenelse{\equal{#2}{}}{}{\ifthenelse{\equal{#3}{}}{(#2)}{(#2,#3)}}}}

%
% Delay Differential Equation
%
% definition domain of right hand side
\newcommand{\deff}{\R\times\R^n\times\R^n}

%
% Differential Dynamic Logic
%
% dL
%\newcommand{\dL}{\textsf{d{\kern-0.1em}$\mathcal{L}$ }}
\newcommand{\signature}{\Sigma}
\newcommand{\varsymbols}{V}
\newcommand{\terms}{\lterms{\Sigma}{V}}
\newcommand{\FOLformulas}{\lformulas[\FOL]{\signature}{\varsymbols}}
\newcommand{\FOL}{\text{FOL}}
\newcommand{\FOLR}{\FOL$_\R$}
\newcommand{\interpret}{I}
\newcommand{\universe}{D_{\interpret}}
\newcommand{\assignment}{\eta}
%
% Delay Differential Dynamic Logic
%
\newcommand{\diffvars}{\D{\allvars}}
\newcommand{\delayedvars}{\mathcal{V}_\tau}
\newcommand{\states}{\mathcal{S}}
\newcommand{\statespace}{\continuouspws[0]{[-\tau,0]}{\R^n}}
%\newcommand{\xtau}[1][]{\ifthenelse{\equal{#1}{}}{x[\tau]}{x[#1]}}
\newcommand{\xtau}[1][\tau]{x[#1]}
\newcommand{\xbartau}{\bar{x}_{\tau}}
\newcommand{\xbartaut}[1]{\bar{x}_{\tau,#1}}
