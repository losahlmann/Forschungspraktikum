\section{Hybrid Programs with DDEs}\label{hybrid-programs-with-ddes}

    We extent classic hybrid programs (\HP) and \dL formulae with syntax, semantics, axiomatization and proof rules for DDEs.
    %Is a super set, \dL is a fragment

    \subsection{Example}
        \label{example-hp-cars}
        To motivate the need of being able to treat Delayed Differential Equations in hybrid programs, we present some examples.

        \subsubsection{Leading and Following Car}

        \subsubsection{Network Induced Delay in Control Loops}

    \subsection{Syntax}
        \label{sec:syntax}

        \paragraph{Terms}
            \label{sec:terms}

            We extent the grammar defining \textbf{terms} with a symbol for a \textbf{delayed variable}
            % FIXME: replace ::=
            % FIXME: replace |
            % FIXME: replace x'
            \begin{equation}
                \theta,\eta ::= x|\xtau|x'|c|\theta+\eta|\theta\cdot\eta
            \end{equation}

    The grammars for hybrid programs and \dL formulas remain unchanged.


    \subsection{Semantics}
        \label{sec:semantics}
        \begin{equation}
            \imodels[\I]{\phi}{x}
        \end{equation}

        Following the remark to the solution of a DDE, we augment the \textbf{state space} in \dL to $\statespace$, the set of piecewise continuous functions on $[-\tau,0]$, as defined in \ref{definition-piecewise-continuous}.

        %TODO: need $\xbartau$

        We senote by $\states$ the set of states. A state $\omega\in\states$ is a mapping
        \begin{equation}
            \omega : \mathcal{V}\cup\mathcal{V'}\rightarrow\statespace
        \end{equation}
        that assigns a \emph{history} (function) $\xbartau$ to each variable symbol and

        %FIXME: need diff var symbol? can determine derivative from x? need pw diffable?

        % TODO: Abgrenzung zu Trace Semantics
        % The temporal character of delay differential equations (they depend on their own temporal evolution with limited horizon) suggests the introduction of trace semantics.
        %
        % However, we go the way of introducing transition semantics with an augmented state space.

        \subsubsection{Terms}
            \label{sec:terms-semantic}

            The semantic of the new symbol $\xtau$ depends on the context in which the term occures:
            \begin{itemize}
                \item in a hybrid program: $\imodel{\I}{\xtau} = \imodel[\nu]{}{x(t-\tau)} = \nu(x)(-\tau)$
                \item in a formula: \begin{equation}
                    % FIXME: rechte seite noch nicht korrekt
                    \imodel[]{}{\phi(\xtau)} =
                    \{\nu\in\states : \lforall{t\in[-\tau,0]}{\phi(\nu(x)(t))} \}
                \end{equation}
            \end{itemize}

            The semantics of the variable symbols in terms are given by
            \begin{equation}
                \imodel[\nu]{}{x}=\nu(x)(0)
            \end{equation}
            and
            \begin{equation}
                \imodel[\nu]{}{x'}=\nu(x')(0)
            \end{equation}
            In the precondition, no values are associated to the differential symbols. In general, the initial function is only piecewise continuous.
            Since for later time instances, the values of the differential symbols derive from the DDE, they become (locally) smooth function.


            When we write $x$ we mean $x(t)$.

        \subsubsection{\dL semantics}
            With the semantics of terms if follows for the meaning of $\dbox{\alpha}{\phi}$, that $\phi$ must only hold up to time $\tau$ before leaving the \HP $\alpha$. It is possible, that $\phi$ was not verified before, while \textit{executing} the \HP.

            However when we apply the Rule of steps, we get the validity of $\phi$ for the entire trace.

            % TODO: is it better to only have xtau and not choice? What if both mentioned?
            in formulae (such as safety condition or evolution constraint), we have two possibilities: only value at current time instant ($x$) or for entire last $\tau$ time $\xtau$

        \subsubsection{Hybrid Programs}
            \label{sec:hp-semantics}

            The transition semantic of a hybrid program $\alpha$ is inductively given by a binary reachability relation $\rho(\alpha)\subseteq\states\times\states$. Since the state space has been replaced, we need to redefine the semantics:

            The \emph{discrete assignment} does not rewrite history, but changes only the value at the current time instant:
            \begin{equation}
            \rho(x:=\theta) = \left\{(\nu,\omega): \omega = \nu \text{ except } \omega(x)=\left(t\mapsto\begin{cases}\imodel{\I}{\theta} & t=0\\ \nu(t) &t\in[-\tau,0)\end{cases}\right)\qquad\right\}_.
            \end{equation}
            This assignment is the actual reason why we need to consider piecewise continuous evolutions.

            %TODO: super dense time: multiple assignments

            Using the extended syntax, we can write down both a delay differential equation and an ordinary differential equation in the form $x'=\theta$, where $\theta=f(x,\xtau)$ with a polynomial $f$.
            \begin{equation}
                \rho(x'=\theta\,\&\,\chi) = \left\{
                    (\varphi(0),\varphi(s))\,:\,\varphi(t)\models x'=\theta\,\wedge\,\varphi(t)(0)\models\chi\,\forall\,0\leq t\leq s\text{ for a solution } \varphi:[0,s]\rightarrow\states \right\}
            \end{equation}
            As a solution, $\varphi$ needs to fulfill
            \begin{equation}
                \varphi(t)(x')(0) \defeq \DD{\varphi(\zeta)(x)(0)}{\zeta}(t) \stackrel{!}{=} \imodel[]{\varphi(t)}{\theta}
            \end{equation}
            Remember that a $\xtau$ mentioned in $\theta$ here means $\varphi(t)(x)(-\tau)$.
            And so needs $\chi$ always just hold at the current time instant (and not over the entire interval $[-\tau,0]$). The same case for $\ptest\psi$.

            % TODO: what about [a;b]p. is [a][b]p correct?
