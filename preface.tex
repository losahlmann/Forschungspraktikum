\begin{abstractpage}
    \begin{abstract}{english}
        Cyber physical systems incorporate the connection between the physical world and computing devices.
        This connection (e.g.\ given by a computer network) often needs to be considered in the system model, if it introduces a delay with non-negligible effects.

        In this work we provide a program notation for hybrid systems with delay and introduce a logic to reason about such systems, called \emph{delay differential dynamic logic} (\ddL).
        It extends \emph{differential dynamic logic} (\dL) with syntax, semantics and axiomatization for dealing with delay differential equations.

        We prove the soundness of the axiomatization and give examples demonstrating its usefulness.

        % static and dynamic semantics
        % differential-forms, to reason in axiomatic way about DDEs
        % modular soundness proof
        % proof calculus

        % To this end,
    \end{abstract}

    \begin{abstract}{frenchb}
        Le système cyber-physique est le trait d’union entre le monde physique et le périphérique numérique. Cette connexion (par ex.\ donnée par un réseau numérique) doit souvent être prise en considération dans le modèle du système, si ce dernier introduit un retard avec des effets non négligeables.

        Dans cet oeuvre, une notation de programme sera donnée pour des systèmes hybrides avec un retard, ainsi qu’une logique de raisonnement pour de tels systèmes, appelée \emph{logique dynamique différentielle retardée} (\ddL), sera introduite.
        Elle étend la \emph{logique dynamique différentielle} (\dL) avec une syntaxe, une sémantique et une axiomatisation pour gérer les équations différentielles retardées.

        La preuve de correction de l’axiomatisation est également présentée et quelques exemples sont avancés pour montrer son utilité. 
    \end{abstract}
\end{abstractpage}

% \cleardoublepage
\thispagestyle{empty}

\begin{otherlanguage}{frenchb}
    \section*{Déclaration d’intégrité relative au plagiat}

    Je soussigné SAHLMANN, Lorenz certifie sur l’honneur:
    \begin{enumerate}
        \item Que les résultats décrits dans ce rapport sont l’aboutissement de mon travail.
        \item Que je suis l’auteur de ce rapport.
        \item Que je n’ai pas utilisé des sources ou résultats tiers sans clairement les citer et les référencer selon les régles bibliographiques préconisées.
    \end{enumerate}

    \textbf{Mention à recopier:}
    Je déclare que ce travail ne peut être suspecté de plagiat.

    \vspace{2cm}
    Date:\hfil Signature:

\end{otherlanguage}

\cleardoublepage

\chapter*{Acknowledgement}
    I would like to express my sincere thanks to Prof.~Dr.~André Platzer for supervising this work, the inspirational discussions and his very helpful suggestions and comments.

    Also, I would like to thank him and the Carnegie-Mellon-University for having accepted me as a research intern. 

    My thanks go to my professor Eric Goubault, who introduced me to delay differential equations and who presented me to Prof.~Platzer.

    A grateful word of thanks to the \emph{Chair Thales} and \foreignlanguage{frenchb}{\emph{La Fondation de l'Ecole Polytechnique}} for supporting me financially throughout the period of the internship.
